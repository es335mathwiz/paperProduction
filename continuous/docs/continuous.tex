\documentclass[1p]{elsarticle}
%%%%%%%%%%%%%%%%%%%%%%%%%%%%%%%%%%%%%%%%%%%%%%%%%%%%%%%%%%%%%%%%%%%%%%%%%%%%%%%%%%%%%%%%%%%%%%%%%%%%%%
%%%%%%%%%%%%%%%%%%%%%%%%%%%%%%%%%%%%%%%%%%%%%%%%%%%%%%%%%%%%%%
\usepackage[english]{babel}
\usepackage{array}
\usepackage{dcolumn}
\usepackage{datetime}
\usepackage{theorem}
 \newtheorem{theorem}{Theorem}
 \newtheorem{definition}[theorem]{Definition}
 \newtheorem{conjecture}[theorem]{Conjecture}

\usepackage{algorithm}
\usepackage{algpseudocode}


\settimeformat{ampmtime}
% \usepackage{fancyhdr}
% \pagestyle{fancy}

%  \lhead{\today}  \rhead{\currenttime}
%
%\lfoot[lf-even]{lf-odd}
% \chead[ch-even]{ch-odd}   \cfoot[cf-even]{cf-odd}
% \rhead[rh-even]{rh-odd}   \rfoot[rf-even]{rf-odd}
%\usepackage[final]{changes}
\usepackage[draft]{changes}

\usepackage{amssymb}
\usepackage{moreverb}
\usepackage{graphicx}
\usepackage{amsfonts}
\usepackage{amsmath}
\usepackage{pdfpages}

%\definechangesauthor[Gary]{GSA}{red}
%\definechangesauthor[Tack/Gary]{TG}{purple}

\setcounter{MaxMatrixCols}{10}


\usepackage{amssymb}
\usepackage{amsmath}
\usepackage{amscd}

%\usepackage{pstricks,pst-node,pstcol,pst-tree}
%\usepackage{program}
%\normalbaroutside
\usepackage{draftcopy}
%\usepackage{newcent}
\usepackage{bookman}
%\usepackage{lucid}
%\usepackage{avant}
%\usepackage{authordate1-4}
\usepackage{epsfig}
\usepackage{notebook}
\textheight9.0in
\textwidth6.5in
\topmargin0.0in
\oddsidemargin0.0in

%\usepackage{epsfig}

\setcounter{MaxMatrixCols}{15}

\newcommand{ \us }{_}
\newcommand{ \aw }{^}

\newcommand{ \myamp }{&}
\newcommand{ \anAmp }{&}
\newcommand{ \unds }{_}
\newcommand{\matob}[2]{\mbox{$\left [ ~ #1 ~ \ldots ~ #2~ \right ] $}} 
%\epsfig{file=aGraph.ps}


%\bibliographystyle{authordate1}


\begin{document}
\begin{frontmatter}
\title{Continuous Time Application of the Anderson-Moore(AIM) Algorithm for
Imposing the Saddle Point Property in Dynamic Models}
\author{Gary S. Anderson}
%EndAName
\address{Federal Reserve Board}
\fntext[fn1]{The author appreciates comments by
Chris Sims.
The views in this paper are solely the responsibility of the authors
and should not be interpreted as reflecting the views of the Board
of Governors of the Federal Reserve System or any other person
associated with the Federal Reserve System.}
 

\begin{abstract}
  \begin{description}
  \item[Use reliablity paper notation and machinery] 
  \item[Apply to Infinite Horizon Linear Quadratic Control]
  \item[Handle nonlinear models] 
  \item[Use Symbolic] Perhaps projection extension
  \item[Continuous Time to Discrete Time Mapping] For MPS, 
Jumping off period handled better
\item[Find stochastic]
\item[Find higher order derivatives]
\item[Non optimal control, Find complicated Exogenous] 
  \end{description}

\end{abstract}

\begin{keyword}
Continuous Time \sep Dynamic Models \sep    Projection Methods \sep
C61 \sep  C63 
\end{keyword}
\end{frontmatter}




% \bibliographystyle{ifac}
% \setlength{\unitlength}{2em}

% \author{Gary S. Anderson}
% \begin{frontmatter}
% \title{Continuous Time Application of the Anderson-Moore(AIM) Algorithm for
% Imposing the Saddle Point Property in Dynamic Models}
% \author{Gary S. Anderson\thanksref{Gary}}\\
% \address{Board of Governors\\Federal Reserve System\\Washington, DC 20551\\Voice: 202 452 2687\\Fax: 202 452 6496\\ganderson\char64frb.gov}
% \thanks[Gary]{
% I wish to thank Christiopher Sims
% for helpful comments.
% I am responsible for
% any remaining errors.
% The views expressed herein are mine and 
% do not necessarily represent thshe views of the Board of Governors of the Federal
% Reserve System.
% }
% \begin{abstract}
% \cite{ANDER:AIM1,ANDER:AIM2} describe a powerful method for solving discrete
% time linear saddle point models.
% This paper shows how one can apply the technique to continuous time models.
% \end{abstract}
% \end{frontmatter}

% \date{\today}



% %\begin{keyword}
% %  JEL classification:  C63--Computational Techniques Subfields Applied
% % Econometrics, Applied Macroeconomics, Other Topics in Mathematical Economics
% %\end{keyword}

\newpage

\section{Introduction and Summary}
\label{sec:intro}



\section{The Algorithm}
\label{sec:algorithm}

Perfect foresight models with solutions determined by saddle point property.

\begin{gather*}
  \begin{bmatrix}
    H_{-\tau}&\ldots&H_{\theta}
  \end{bmatrix}
  \begin{bmatrix}
    \frac{d^0x}{dt^0}(0)\\
\vdots\\
    \frac{d^{\theta-1}x}{dt^{\theta}}(0)\\
  \end{bmatrix}=0
\end{gather*}


With initial conditions:
\begin{gather*}
  Z 
  \begin{bmatrix}
    \frac{d^0x}{dt^0}(0)\\
\vdots\\
    \frac{d^{\theta-1}x}{dt^{\theta-1}}(0)\\
  \end{bmatrix}= \xi
\end{gather*}
To deal with inhomogeneous system one can characterize the system in terms of
deviations from steady state.

We will endeavor to write the system in the form:\cite{bellman70}
\begin{gather*}
  \frac{dx}{dt}(t) = A x(t) \\ \intertext{ If A has distinct eigenvalues then }
x(t)= \exp^{A t} x(0)
\end{gather*}
One can investigate the stability properties of the system by
analyzing $A$ even when $A$ has repeated roots.

The AIM algorithm transition matrix computation 
produces A and, when necessary,
 generates auxiliary conditions which are important for
establishing a full set of initial conditions for the solutions.\cite{ANDER:AIM97}



To apply the technique, one need only 
compute the dominant invariant space vectors  spanning space assoicated with all
positive roots.
\begin{gather*}
  V A = M V
\end{gather*}
But to motivate the solution, recall that one can always write
\begin{gather*}
  \begin{bmatrix}
    V\\W
  \end{bmatrix} A = 
  \begin{bmatrix}
    M&\\&B
  \end{bmatrix}  \begin{bmatrix}
    V\\W
  \end{bmatrix}\\ \intertext{ with all the eigenvalues of $M$ positive and
all the eigenvalues of $B$ zero or negative so that}
 A   = 
\begin{bmatrix}
    V\\W
  \end{bmatrix}^{-1}  \begin{bmatrix}
    M&\\&B
  \end{bmatrix}   \begin{bmatrix}
    V\\W
  \end{bmatrix}\\ \intertext{ so that }
\begin{bmatrix}
    V\\W
  \end{bmatrix}\frac{dx}{dt} = \begin{bmatrix}
    M&\\&B
  \end{bmatrix} \begin{bmatrix}
    V\\W
  \end{bmatrix} x \\ \intertext{ so that by choosing $x$ so that}
V x = 0\\ \intertext{ one has }
\frac{d\begin{bmatrix}
    V x\\W x
  \end{bmatrix}
}{dt} = 
\begin{bmatrix}
  0\\B W x
\end{bmatrix}\\ \intertext{ and one can rest assured $Wx$ converges }
\frac{d (W x)}{dt} = 
 B(W x)
\end{gather*}
Combine results
\begin{gather*}
  \begin{bmatrix}
    V\\W
  \end{bmatrix}x(t) = 
  \begin{bmatrix}
    0\\ \exp^{B t}
  \end{bmatrix}\begin{bmatrix}
    V\\W
  \end{bmatrix} x(0) \\ \intertext{ so that }
x(t)=\begin{bmatrix}
    V\\W
  \end{bmatrix}^{-1}  \begin{bmatrix}
    0\\ \exp^{B t}
  \end{bmatrix} \begin{bmatrix}
    V\\W
  \end{bmatrix}x(0) 
\end{gather*}
One need only choose initial conditions guaranteeing that the initial part of
the trajectory is orthogonal to the left invarinat space associated with 
positive roots, that the initial part of the trajectory not violate any constraints
uncovered in computing the transition matrix, and the other original
initial conditions.

\begin{gather*}
Q=  \begin{bmatrix}
    Z\\Z^\#\\V
  \end{bmatrix}
\end{gather*}
For most economic models, one  will want
an adequate number of constraints to identify a single trajectory.
\begin{gather*}
  Q 
  \begin{bmatrix}
    \frac{d^0x}{dt^0}(0)\\
\vdots\\
    \frac{d^{\theta-1}x}{dt^{\theta-1}}(0)\\
  \end{bmatrix}= 
  \begin{bmatrix}
\xi\\0\\0    
  \end{bmatrix}
\end{gather*}

\section{An Example}
A recent paper by Sims\cite{sims96} presents a stochastic version of
the following  example model.
\begin{gather*}
  w(t) =  \rho \left ( \int_{s=0}^\infty \exp^{-\rho s}W(t+s) ds \right ) - \alpha (u(t) -u_n)\\
W(t) = \rho \left ( \int_{s=0}^\infty \exp^{-\rho s}w(t-s) ds \right )\\
\dot{u}(t) = -\theta u(t) + \gamma W(t) + \mu
\end{gather*}
With initial conditions 
\begin{gather*}
  W(0)=W_0\\u(0)=u_0
\end{gather*}
One can rewrite the system as:
\begin{gather*}
\dot{u}(t) = -\theta u(t) + \gamma W(t) + \mu\\
\dot{w}(t) = \rho (w(t) - W(t) ) - \alpha \dot{u} +\rho \alpha(u(t) - u_n)\\
\dot{W}(t) = \rho(w(t) - W(t))
\end{gather*}
Or:
\begin{gather*}
  \begin{bmatrix}
    -\rho&\rho&-  \alpha \rho  &1&0&\alpha\\
  -\rho&\rho&0&0&1&0 \\
 0&-\gamma&\theta&0&0&1 
  \end{bmatrix}  \begin{bmatrix}
    \frac{d^0w}{dt^0}(0)\\
    \frac{d^0W}{dt^0}(0)\\
    \frac{d^0u}{dt^0}(0)\\
    \frac{dw}{dt}(0)\\
    \frac{dW}{dt}(0)\\
    \frac{du}{dt}(0)
  \end{bmatrix}
\end{gather*}

With
\begin{gather*}
  Z 
  \begin{bmatrix}
    \frac{d^0x}{dt^0}(0)\\
\vdots\\
    \frac{d^{\theta-1}x}{dt^{\theta-1}}(0)\\
  \end{bmatrix}= 
  \begin{bmatrix}
    0&1&0\\0&0&1
  \end{bmatrix} 
\begin{bmatrix}
w(0)\\W(0)\\u(0)
\end{bmatrix}= 
\begin{bmatrix}
  W_0\\u_0
\end{bmatrix}
\end{gather*}


%\input{contTransMat.tex}




\nocite{sims96}
\nocite{berry}
\nocite{golub89}
\nocite{blanchard80}
\nocite{krishnamurthy89}


\bibliographystyle{authordate2}

\bibliography{files,anderson}

\end{document}
