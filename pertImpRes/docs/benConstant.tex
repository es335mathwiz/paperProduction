\documentclass[12pt]{article}
\usepackage{amsmath}
\newcommand{\fv}{\mathcal{N}}

\begin{document}


\subsection{First Order(Linear) Rational Expectations Solution}
\label{sec:firstlinrat}
The variables ordered so that the innovations  in the state vector come
after the  non-innovation variables.


The length of $nzCols$ corresponds to $\fv_-=s$,
the length of $leadsNeeded$ corresponds to $\fv_+$,
the routine generates a matrix conformable with the first order derivatives
of the model equations with respect to each of the $\fv$ variables.

We assume that all innovations have mean zero.\footnote{
On could handle the impact of nonzero mean  can be handled a couple of ways
\begin{gather*}
\Delta x^\ast=(I-\sum_{i=-\tau}^{-1} B_i)(\sum_{i=-\tau}^\theta H_i)^{-1} \mu= (\sum_{s=0}^\infty F^s)\phi \mu
\Delta x^\ast=(I-\sum_{i=-\tau}^{-1} B_i)(\sum_{i=-\tau}^\theta H_i)^{-1} \mu= (\sum_{s=0}^\infty F^s)\phi \mu
\sigma \mu_A \intertext{since}
[(I-\sum_{i=-\tau}^{-1} B_i)(\sum_{i=-\tau}^\theta H_i)^{-1}  -\phi ]
\end{gather*}
But to match certainty equivalence with time t errors one should take
\begin{gather*}
\Delta x^\ast= (\sum_{s=1}^\infty F^s)\phi \mu
\end{gather*}

}



\end{document}
