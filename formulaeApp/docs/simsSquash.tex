Collecting the  auxiliary constraints generated by the 
auto regression phase of the algorithm for the example model one has:
%\begin{figure*}[htbp]
%\begin{center}
%  \leavevmode
  \begin{gather*}
    \begin{bmatrix}
Z^\sharp_\ast\\Z^\flat_\ast
    \end{bmatrix}
=  \begin{bmatrix}
0 \anAmp 0 \anAmp 0 \anAmp -{\frac{1}{2}} \anAmp 0 \anAmp 0 \anAmp 0 \anAmp 0
  \anAmp -{\frac{1}{2}} \anAmp 1\\
0 \anAmp 0 \anAmp -\theta \anAmp 0 \anAmp 0 \anAmp -1 \anAmp 0 \anAmp 1 \anAmp
  0 \anAmp -\gamma\\
0 \anAmp 0 \anAmp 0 \anAmp 0 \anAmp 0 \anAmp 1 \anAmp 0 \anAmp 0 \anAmp 0
  \anAmp 0\\
0 \anAmp 0 \anAmp 0 \anAmp 0 \anAmp 0 \anAmp 0 \anAmp 1 \anAmp 0 \anAmp 0
  \anAmp 0\\
\hline
0 \anAmp -1 \anAmp \alpha \anAmp 1 \anAmp -{\frac{1}{2}} \anAmp 0 \anAmp 0
  \anAmp 0 \anAmp 0 \anAmp -{\frac{1}{2}}\\
1 \anAmp 0 \anAmp 0 \anAmp 0 \anAmp 0 \anAmp 0 \anAmp 0 \anAmp 0 \anAmp 0
  \anAmp 0\\
0 \anAmp 1 \anAmp 0 \anAmp 0 \anAmp 0 \anAmp 0 \anAmp 0 \anAmp 0 \anAmp 0
  \anAmp 0
  \end{bmatrix}
  \end{gather*}
%  \caption{Auxiliary Initial Conditions}
%  \label{zrows}
%\end{center}
%\end{figure*}
One can extend  the basis to get a non singular
matrix.\footnote{The top rows are just the row-echelon form of the $Z^{\sharp,\ast},Z^{\flat,\ast}$ vectors.}
%\begin{figure*}[htbp]
%\begin{center}
%  \leavevmode
  \begin{gather*}
    \begin{bmatrix}
Z\\\bar{Z}
    \end{bmatrix}
=  \begin{bmatrix}
1 \anAmp 0 \anAmp 0 \anAmp 0 \anAmp 0 \anAmp 0 \anAmp 0 \anAmp 0 \anAmp 0
  \anAmp 0\\
0 \anAmp 1 \anAmp 0 \anAmp 0 \anAmp 0 \anAmp 0 \anAmp 0 \anAmp 0 \anAmp 0
  \anAmp 0\\
0 \anAmp 0 \anAmp 1 \anAmp 0 \anAmp 0 \anAmp 0 \anAmp 0 \anAmp
  -{\frac{1}{\theta}} \anAmp 0 \anAmp {\frac{\gamma}{\theta}}\\
0 \anAmp 0 \anAmp 0 \anAmp 1 \anAmp 0 \anAmp 0 \anAmp 0 \anAmp 0 \anAmp 1
  \anAmp -2\\
0 \anAmp 0 \anAmp 0 \anAmp 0 \anAmp 1 \anAmp 0 \anAmp 0 \anAmp
  {\frac{-2\alpha}{\theta}} \anAmp 2 \anAmp {\frac{2\alpha\gamma -
  3\theta}{\theta}}\\
0 \anAmp 0 \anAmp 0 \anAmp 0 \anAmp 0 \anAmp 1 \anAmp 0 \anAmp 0 \anAmp 0
  \anAmp 0\\
0 \anAmp 0 \anAmp 0 \anAmp 0 \anAmp 0 \anAmp 0 \anAmp 1 \anAmp 0 \anAmp 0
  \anAmp 0\\
\hline
0 \anAmp 0 \anAmp 0 \anAmp 0 \anAmp 0 \anAmp 0 \anAmp 0 \anAmp 1 \anAmp 0
  \anAmp 0\\
0 \anAmp 0 \anAmp 0 \anAmp 0 \anAmp 0 \anAmp 0 \anAmp 0 \anAmp 0 \anAmp 1
  \anAmp 0\\
0 \anAmp 0 \anAmp 0 \anAmp 0 \anAmp 0 \anAmp 0 \anAmp 0 \anAmp 0 \anAmp 0
  \anAmp 1
  \end{bmatrix}
  \end{gather*}
%  \caption{$Z$}
%  \label{zmat}
%\end{center}
%\end{figure*}
%\begin{figure*}[htbp]
%\begin{center}
%  \leavevmode
  \begin{gather*}
    A=  \begin{bmatrix}
\rho \anAmp 2\gamma \anAmp -\gamma\\
4\alpha \anAmp 3 \anAmp -2\\
2\alpha \anAmp 2 \anAmp -1
  \end{bmatrix}
  \end{gather*}
%  \caption{Minimal Dimension Transition Matrix}
%  \label{lila}
%\end{center}
%\end{figure*}
For this model, we expect one large root. Consequently, $M= 
\begin{bmatrix}
  \lambda_L
\end{bmatrix}$, a $1\times 1$ matrix.
%\begin{figure*}[htbp]
%\begin{center}
%  \leavevmode
  \begin{gather*}
    \Pi=  \begin{bmatrix}
0 \anAmp 0 \anAmp 0 \anAmp 0 \anAmp 0 \anAmp 0 \anAmp -2\gamma\\
0 \anAmp 0 \anAmp 0 \anAmp 0 \anAmp 0 \anAmp 0 \anAmp -4\\
0 \anAmp 0 \anAmp 0 \anAmp 0 \anAmp 0 \anAmp 0 \anAmp -2
  \end{bmatrix}
  \end{gather*}
%  \caption{$\Pi$}
%  \label{pimat}
%\end{center}
%\end{figure*}
%\begin{figure*}[htbp]
%\begin{center}
%  \leavevmode
  \begin{gather*}
    J_0=  \begin{bmatrix}
0 \anAmp 0 \anAmp 0 \anAmp 0 \anAmp 0 \anAmp 1 \anAmp 0\\
0 \anAmp 0 \anAmp 0 \anAmp 0 \anAmp 0 \anAmp 0 \anAmp 1\\
0 \anAmp 0 \anAmp 0 \anAmp 0 \anAmp 0 \anAmp 0 \anAmp 0\\
0 \anAmp 0 \anAmp 0 \anAmp 0 \anAmp 0 \anAmp 0 \anAmp 0\\
0 \anAmp 0 \anAmp 0 \anAmp 0 \anAmp 0 \anAmp 0 \anAmp -8 +
  {\frac{4\alpha\gamma}{\theta}} - {\frac{2\left( 2\alpha\gamma - 3\theta
  \right) }{\theta}}\\
0 \anAmp 0 \anAmp 0 \anAmp 0 \anAmp 0 \anAmp 0 \anAmp 0\\
0 \anAmp 0 \anAmp 0 \anAmp 0 \anAmp 0 \anAmp 0 \anAmp 0
  \end{bmatrix}
  \end{gather*}
%  \caption{$J_0$}
%  \label{j0mat}
%\end{center}
%\end{figure*}
%\begin{figure*}[htbp]
%\begin{center}
%  \leavevmode
  \begin{gather*}
    ((I \otimes M) - (J_0^T \otimes I))^{-1}(\Pi^T \otimes I)=  \begin{bmatrix}
0 \anAmp 0 \anAmp 0\\
0 \anAmp 0 \anAmp 0\\
0 \anAmp 0 \anAmp 0\\
0 \anAmp 0 \anAmp 0\\
0 \anAmp 0 \anAmp 0\\
0 \anAmp 0 \anAmp 0\\
{\frac{-2\gamma}{{{\lambda} \us L}}} \anAmp {\frac{-4}{{{\lambda} \us L}}}
  \anAmp {\frac{-2}{{{\lambda} \us L}}}
  \end{bmatrix}
  \end{gather*}
%  \caption{$((I \otimes M) - (J_0^T \otimes I))^{-1}(\Pi^T \otimes I)$}
%  \label{mapper}
%\end{center}
%\end{figure*}
