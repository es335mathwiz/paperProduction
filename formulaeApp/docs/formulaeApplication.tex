\documentclass{article}
\usepackage[authoryear]{natbib}
\usepackage{amsmath}

\makeatletter
\def\fullpath{\begingroup\everyeof{\noexpand}\@sanitize
  \edef\x{\@@input|"find `pwd` -name \jobname.tex" }%
  \edef\x{\endgroup\noexpand\zap@space\x\noexpand\@empty}\x}
\makeatother

\begin{document}
\title{Using Formulae From ``Reliable and Computationally Efficient ... '' Paper}

\author{Gary Anderson}

\maketitle


file located in \footnote{\fullpath}

The paper \cite{anderson10} shows how to solve models of the form\footnote{The paper provides formulae for models with multiple leads and lags.}

\begin{gather*}
\sum_{-1}^1 H_i  x_{t+i} =\psi z_{t}.  \intertext{ Given the homogeneous solution ( $\psi=0$ ) }
x_{t}=B x_{t-1}\intertext{ we can compute matrices $\phi$ and $F$ }
\phi=(H_0 + H_1 B)^{-1}, \,\,  F=-\phi H_1 \intertext{ such that }
x_t = B x_{t-1} + \sum_{s=0}^\infty( F^s \phi \psi z_{t+s}) \intertext{furthermore, one can show that if $z_{t+1}= \Upsilon z_t$}
x_t=B x_{t-1} + \vartheta z_t\intertext{satisfies the inhomogenous system,  where }
vec [\vartheta] = [ I-\Upsilon^T \otimes F ]^{-1}  vec[ \phi \psi]
\end{gather*}

This system would not require simulation 
to get the  expected impact of an AR(1) shock.

For a constant we have
\begin{gather*}
  x_t=B x_{t-1} + (\sum_{s=0}^\infty  F^s) \phi \psi= B x_{t-1} + (I-  F)^{-1} \phi \psi
\end{gather*}
\newtheorem{cnj}{Conjecture}

\begin{cnj}
We can construct an exogenous process, $z_{t}(\epsilon_t,x_{t-1}), z_{t+1}(\epsilon_t,x_{t-1}), \ldots , z_{t+K}(\epsilon_t,x_{t-1})$ which   
keeps the $x_t, x_{t+1} \ldots x_{t+K}$ from violating
occasionally binding constraints where K can be arbitrarily large.
\end{cnj}

\begin{gather*}
x_t = B x_{t-1} + \sum_{s=0}^\infty( F^s \phi \psi z_{t+s}(\epsilon_t,x_{t-1})) \\
x_{t+k} = B x_{t+k-1} + \sum_{s=0}^\infty( F^s \phi \psi z_{t+k+s}(\epsilon_{t+k},x_{t+k-1})) \\
g(x_t)\le 0\\
\end{gather*}
\bibliographystyle{plainnat}
\bibliography{files,anderson}
\end{document}

