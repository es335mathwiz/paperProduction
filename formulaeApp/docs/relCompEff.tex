\listfiles
%gv -style "/opt/matlab-7.0/sys/ghostscript/ps_files/gs_init.ps" relCompEff.ps &
%how to pdflatex pstricks http://rp.lip6.fr/~benbadis/tips/presentation.pdf
%latex relCompEff;dvips -o relCompEff.ps relCompEff;ps2pdf relCompEff.ps relCompEff-pics.pdf ;pdflatex relCompEff;pdflatex relCompEff;pdflatex relCompEff
%latex relCompEff;dvips -o relCompEff.ps relCompEff;ps2pdf relCompEff.ps

%pdflatex relCompEff;pst2pdf relCompEff;pdflatex relCompEff
%this should work but doesn't seem to
%pdflatex -shell-escape relCompEff

\newcommand{\myamp}{&}
\newcommand{\mywedge}{^}


%$Id: relCompEff.tex,v 1.20 2007/08/03 15:26:16 m1gsa00 Exp $
\documentclass{elsart}
%\usepackage{indentfirst}
%\usepackage[dcucite]{harvard}
%\usepackage{harvard}
 \usepackage[yyyymmdd,hhmmss]{datetime}
% \usepackage{datetime}

 \usepackage{dcolumn}
 \usepackage{fancyhdr}
\pagestyle{fancy}
\rfoot{Compiled on \today\ at \currenttime}
\cfoot{}
\lfoot{Page \thepage}
\lhead{}
\chead{}
\rhead{}
\renewcommand{\headrulewidth}{0pt}
 \usepackage{pifont}
 \usepackage{hyperref}
\usepackage{color}
\usepackage{ifpdf}
\usepackage{floatfig}
\newlength{\matfieldwidth}
\usepackage{multicol}
%\usepackage[margin=1in]{geometry}
\usepackage{wrapfig}
\usepackage{theorem}
\usepackage{amssymb}
\usepackage{amsmath}
\usepackage{amscd}
\usepackage{authordate1-4}

\usepackage{pdftricks}
\ifpdf
\begin{psinputs}
   \usepackage{pstricks}
   \usepackage{pstcol}
   \usepackage{pst-tree}
   \usepackage{pst-node}
   \usepackage{pspicture}
\end{psinputs}
\else
   \usepackage{pstricks}
   \usepackage{pstcol}
   \usepackage{pst-tree}
   \usepackage{pst-node}
   \usepackage{pspicture}
\fi





 \usepackage{program}
 \normalbaroutside%
 \newsavebox{\appbox}


\setcounter{MaxMatrixCols}{20}
\newcommand{\ugB}[1]{\mathcal{B}_{#1}}
\newcommand{\ugBR}[1]{\mathcal{B}_{R#1}}
\newcommand{\ugBL}[1]{\mathcal{B}_{L#1}}
\newcommand{\sTransA}{\mathcal{A}}
\newcommand{\sTransB}{\mathcal{B}}

\newcommand{\us}{_}
\newcommand{\mH}{\mathcal{H}}
\newcommand{\mQ}{\mathcal{Q}}
\newcommand{\mZ}{\mathcal{Z}}


\newcommand{\longExpH}{\begin{bmatrix}
0&U^k_Z \mH^k_\tau&\ldots&U^k_Z \mH^k_{\theta-1}\\
U^k_N \mH^k_\tau&\multicolumn{2}{c}{\ldots}&U^k_N \mH^k_\theta%
\end{bmatrix}}
\newcommand{\longExpQ}{\begin{bmatrix}\multicolumn{3}{c}{\mZ^k}\\U^k_Z \mH^k_\tau&\ldots&U^k_Z \mH^k_{\theta-1}\end{bmatrix}}

\newcommand{\anAmp}{&}
\newcommand{\unds}{_}
\newcommand{\matob}[2]{\mbox{$\left [  #1  \ldots  #2 \right ] $}} 

\newenvironment{prf}{
  \begin{quote}
{\bf Proof   }
}
{{\hspace*{\fill}\nolinebreak[1]\hspace*{\fill}$\blacksquare$}
  \end{quote}
}
%\newenvironment{exmpl}{}{}

\newenvironment{exmpl}{
%  \color{blue}\em%
{ An example:   }
\/}
{
}


\newcommand{\boththrm}[2]{
\begin{thrm}
{#1    }
\end{thrm}
}
\newtheorem{algrthm}{Algorithm}

{\theoremstyle{plain}\theorembodyfont{\bfseries}
\newtheorem{thrm}{Theorem}
\newtheorem{crrlry}{Corollary}
\newtheorem{rslt}{Algorithm}
\newtheorem{pblm}{Problem Statement}
\newtheorem{dfntn}{Definition}
}
\newcommand{\hmats}{H_{-\tau}&\hspace{0.25in}&\dots&\hspace{0.25in}&H_{\theta}}
\newcommand{\hsh}{H^{\sharp\ast}_{-\tau}&\hspace{0.25in}&\dots&\hspace{0.25in}&H^{\sharp\ast}_{\theta}}
\newcommand{\hshShort}{H^{\sharp\ast}_{-\tau}&\hspace{0.25in}&\dots&\hspace{0.25in}&H^{\sharp\ast}_{\theta-1}}
\newcommand{\hShort}{H_{-\tau}&\hspace{0.25in}&\dots&\hspace{0.25in}&H_{\theta-1}}
\newcommand{\hsht}{
  \begin{bmatrix}
    I&0
  \end{bmatrix}
H^{\sharp\ast}_{-\tau}&\hspace{0.25in}&\dots&\hspace{0.25in}&  \begin{bmatrix}
    I&0
  \end{bmatrix}H^{\sharp\ast}_{\theta}}
\newcommand{\zsh}{Z^{\sharp\ast}_{-\tau}&\hspace{0.25in}&\dots&Z^{\sharp\ast}_{\theta-1}}
\newcommand{\hfl}{H^{\flat\ast}_{-\tau}&\hspace{0.25in}&\dots&\hspace{0.25in}&H^{\flat\ast}_{\theta}}

\definecolor{anewcolor}{rgb}{.30,.20,.60}


\begin{document}
\initfloatingfigs
%\bibliographystyle{econometrica}
%\bibliographystyle{plain}
\bibliographystyle{authordate1}
\setlength{\unitlength}{2em}

\begin{frontmatter}
  



\title{ A Reliable and Computationally Efficient Algorithm for 
Imposing the Saddle Point Property in Dynamic Models
%\footnote{
%$Author: m1gsa00 $  $Date: 2007/08/03 15:26:16 $ $Revision: 1.20 $\newline $Source: /msu/res2/m1gsa00/cvsroot/algDesc/relCompEff.tex,v $    
%}
}



\author{Gary S. Anderson}
\thanks{First, I would like to thank George Moore, my now deceased mentor,
friend and coauthor of\cite{ANDER:AIM2}.
I also wish to thank Brian Madigan, Robert Tetlow, Andrew Levin,
Jeff Fuhrer and Hoyt Bleakley for helpful comments.
I am responsible for
any remaining errors.
The views expressed herein are mine and 
do not necessarily represent the views of the Board of Governors of the Federal
Reserve System.
}

\begin{abstract}
This paper describes a set of 
algorithms for quickly and reliably solving linear rational expectations 
models.
The utility, reliability and speed of these algorithms are a consequence of
1) the algorithm for computing the minimal dimension state space 
transition matrix for models with arbitrary numbers of  lags or leads,
2) the availability of a simple modeling language for characterizing
a linear model and
3) the use of the QR Decomposition and 
 Arnoldi type eigenspace calculations.
The paper also presents new formulae  for
computing and manipulating solutions for
arbitrary exogenous processes.











\end{abstract}


\end{frontmatter}
%\maketitle







%\pagestyle{myheadings}

%\markboth{\today \hfill }{\today \hfill }


% \newpage
% \tableofcontents
% \newpage

\section{Introduction and Summary}
\label{sec:intro}
\nocite{anderson06}


% George Moore and I developed this group of algorithms to 
% facilitate research and analysis activities   at the Federal 
% Reserve.
Economists at the Board have an operational need for tools
that are useful for building, estimating and simulating moderate to large
scale rational 
expectations models.
This context dictates a need for careful 
attention to computational efficiency and
numerical stability of the algorithms.


These algorithms have proved very durable and 
useful for staff at the central bank.
Many economists at the Federal Reserve Board 
have used the algorithms in their daily 
work and their research.\footnote{
See, 
for example,\cite{bomfim96,persist,realrate,stab,zerobnd,fwlb,optpol,longrate,gmmml,learn,optsac,cr96,orphanides97,andrew98,orphanides98,orphanides98a,edge03,orphanides02}.}
With the exception of researchers at other European
central banks\cite{zagaglia02},
 few  economists outside the US central bank seem to know about the 
method.
This paper attempts to make the method and approach more widely 
available by describing the underlying theory along with a number of
improvements on the original algorithm.\footnote{ At the Board, economists commonly refer to this family of
algorithms as the 
AIM algorithm. A metaphor relating our approach to the ``shooting method'' 
inspired the name. }
%The ``C'', Matlab and Mathematica code are available from the author at {\small\url{http://www.federalreserve.gov/pubs/oss/oss4/aimindex.html}.}

The most distinctive features of this approach are:
\begin{itemize}
\item its algorithm for computing the minimal dimension state space 
transition matrix 
\item its
use of bi-orthogonality to characterize the asymptotic constraints 
that guarantee stability (See Section~\ref{sec:invariantSpace}).
\item It's reliance on QR Decomposition and the real Schur Decomposition
for speed and accuracy.
\end{itemize}


This unique combination of features 
makes the algorithm especially effective 
for large models.
See~\cite{anderson06} for a systematic comparison of this algorithm with
the alternatives procedures.

The remainder of the paper is organized as follows.
Section~\ref{sec:problem} states the saddle point problem.
Section~\ref{sec:basic} describes the algorithms for solving
the homogeneous and inhomogeneous 
 versions of the problem and describes several implementations.
Section~\ref{sec:useful} shows how to compute matrices often found useful
for manipulating rational expectations model solutions:
the observable structure(Section~\ref{sec:errorCalc}) and
stochastic transition matrices(Section~\ref{sec:stochtrans}).
 The Appendices contain  proofs for the
linear algebra underlying the 
algorithm and the solution of a simple example model.





\section{Saddle Point Problem Statement}




Consider linear models of the form:


\begin{gather}
\sum_{i=-\tau}^\theta{H_i x_{t+i}}= \Psi%\nonumber
z_{t}, \,\, t = 0,\ldots,\infty\label{eq:canonical}\\ \intertext{with initial conditions, if any, given by constraints of the form}%\nonumber
x_i  =  x^{data}_i,  i =  - \tau, \ldots, -1\label{eq:init}\\ \intertext{where both $\tau$ and $\theta$ are non-negative, and $x_t$ is an L dimensional vector 
of endogenous variables with}%\nonumber
\lim_{ t \rightarrow\infty} \|x_t\|   < \infty\label{eq:limit} %\nonumber
\end{gather}
{ and $z_t$ is a $k$ dimensional vector of exogenous variables.}

    

Section~\ref{alg} describes computationally efficient algorithms for
determining the existence and uniqueness of solutions to this problem.


\label{sec:problem}  

\section{The Algorithms}

\label{alg}

\label{sec:basic}

The uniqueness of solutions to 
system~\ref{eq:canonical} requires that
any transition matrix characterizing the dynamics of the 
linear system have an appropriate
number of explosive and stable eigenvalues\cite{blanchard80},
and that a certain set of  asymptotic linear constraints 
are linearly independent of explicit and certain other auxiliary initial 
conditions\cite{ANDER:AIM2}.

The solution methodology entails 
\begin{enumerate}
\item Manipulating the left hand side of equation~\ref{eq:canonical} to obtain
 a state space transition matrix, $A$, along with
a set of auxiliary initial conditions, $Z$ for the homogeneous solution.
\begin{gather}
  Z
  \begin{bmatrix}
    x_{-\tau}\\ \vdots \\ x_{\theta}
  \end{bmatrix}=0 \,\,\,\,\text{and}\,   \begin{bmatrix}
    x_{-\tau+1}\\ \vdots \\ x_{\theta}
  \end{bmatrix}
=A   \begin{bmatrix}
    x_{-\tau}\\ \vdots \\ x_{\theta-1}
  \end{bmatrix}
\end{gather}
See Section~\ref{sec:arzgev}.
\item Computing the eigenvalues and vectors spanning 
the left invariant space associated with
large eigenvalues. See Section~\ref{asymQ}.
\begin{gather}
 V A =   \mathcal{M}  V 
\end{gather}
with the eigenvalues of $ \mathcal{M}$ all greater than one in absolute value.
\item Assembling asymptotic
constraints, $Q$, (See Section~\ref{asymQ}.)  by combining the:
  \begin{enumerate}
\item  auxiliary initial conditions identified in the computation of the transition matrix and 
\item the invariant space vectors
  \end{enumerate}
\begin{gather}
  Q= 
  \begin{bmatrix}
    Z\\V
  \end{bmatrix}
\end{gather}
\item Investigating the rank of the linear space spanned by these asymptotic
constraints and,  when a unique solution exists, 
\begin{enumerate}
\item Computing the auto-regressive 
representation, $B$. See Section~\ref{conar}.
\item Computing matrices, $\phi, F, \vartheta$ 
for characterizing the impact of the inhomogeneous
right hand side term. See Section~\ref{sec:inhomog}.
\end{enumerate}
\end{enumerate}

Figure \ref{fig:components} presents a  flowchart of the algorithm.




\begin{figure*}[!ht]
  \begin{center}
    \leavevmode
%\fbox{
\begin{picture} (10,23)
  \begin{picture}(11.000000,22.799999)(-5.500000,-22.799999)
% picture environment flowchart generated by flow 0.99b
\put(0.0000,-2.0000){\oval(4.0000,2.0000)}
\put(-2.0000,-3.0000){\makebox(4.0000,2.0000)[c]{\shortstack[c]{
Begin
}}}
\put(0.0000,-3.0000){\vector(0,-1){1.0000}}
\put(-5.0000,-7.4000){\framebox(10.0000,3.4000)[c]{\shortstack[c]{
Compute Unconstrained Autoregressive\\
Representation($H^{\sharp,\ast}$) and Auxiliary \\
Initial Conditions($Z^{\sharp,\ast}$),\\
(See Section \ref{sec:arzgev})\\
$\begin{CD}H @>>>\{H^{\sharp,\ast},Z^{\sharp,\ast}\}\end{CD}$
}}}
\put(0.0000,-7.4000){\vector(0,-1){1.0000}}
\put(-5.0000,-11.8000){\framebox(10.0000,3.4000)[c]{\shortstack[c]{
Compute Convergence Constraint($V$)\\
(See Section \ref{sec:invariantSpace})\\
$\begin{CD}\{ H^{\sharp,\ast},Z^{\sharp,\ast} \}@>>>V\end{CD}$
}}}
\put(0.0000,-11.8000){\vector(0,-1){1.0000}}
\put(-5.5000,-15.8000){\framebox(11.0000,3.0000)[c]{\shortstack[c]{
Compute Asymptotic Constraints ($Q$),\\
(See Section \ref{sec:invariantSpace})\\
$\begin{CD}\{Z^{\sharp,\ast},V\}@>>>Q\end{CD}$
}}}
\put(0.0000,-15.8000){\vector(0,-1){1.0000}}
\put(-5.5000,-19.8000){\framebox(11.0000,3.0000)[c]{\shortstack[c]{
Investigate Rank of $Q_R$\\
(See Section \ref{sec:existunique})\\
$\begin{CD}\{Q\}@>>>\{B,\phi,F,\vartheta\}\end{CD}$
}}}
\put(0.0000,-19.8000){\vector(0,-1){1.0000}}
\put(0.0000,-21.8000){\oval(4.0000,2.0000)}
\put(-2.0000,-22.8000){\makebox(4.0000,2.0000)[c]{\shortstack[c]{
End
}}}
\end{picture}

\end{picture}
%}
    \caption{Algorithm  Components~}
    \label{fig:components}
  \end{center}
\end{figure*}




\subsection{Homogeneous Solution}
\label{sec:homo}


Suppose, for now, that $\Psi=0$:\footnote{Note that there is no unique steady state requirement.
Steady state solutions, $x^\ast$ satisfying
\begin{gather}
 \sum_{i= - \tau}^\theta( H_i ) x^\ast= 0%
\end{gather}
lie in a linear subspace of $\mathcal{R}^L$. 
We will develop conditions that guarantee solutions 
that evolve from a given set of initial conditions to a single point in this
subspace. As a result, one can apply these routines to
models with unit roots, seasonal factors, cointegrating vectors
and error correction terms.
}
\begin{gather}
\sum_{i= - \tau}^\theta{ H_i  x_{ t + i } }= 0, t \geq0\label{eq:homo}\\
\lim_{ t \rightarrow\infty} \|x_t\|   < \infty%\nonumber
%\lim_{ t \rightarrow\infty} x_t  =  x^\ast \intertext{with}
%\sum_{i= - \tau}^\theta{ H_i  x^\ast}= 0%
\end{gather}

The homogeneous specification~\ref{eq:homo} is not restrictive.
Since the procedure can handle 
inhomogeneous versions  of equation~\ref{eq:canonical}
by recasting the problem in terms of  deviations from a steady state value.
However,
the next section provides a  more intuitive, flexible 
and computationally efficient alternative for 
computing inhomogeneous solutions.\footnote{The original algorithmic description and
software implementation of these algorithms
developed homogeneous solutions.
Researchers obtained solutions for models with inhomogeneous systems
by adding an equation of the form $\mathbf{i}_{t} =  \mathbf{i}_{t-1}$ with initial condition 
$\mathbf{i}_{t-1}=1$ to 
the system.}



%\subsubsection{Unit Roots and  Cointegrating Factors}






\subsubsection{State Space Transition Matrix and Auxiliary Initial Conditions: $A, Z$}
\label{sec:arzgev}

This section describes how to determine a first-order state space
representation of the equation system~\ref{eq:homo}. The method is an extension
of the shuffling algorithm developed in\cite{luenberger78,luenberger77}.
If $H_\theta$ is non-singular, we can immediately obtain $x_{t+\theta}$ 
in terms of $x_{t-\tau} \ldots x_{t+\theta-1}$
\begin{gather}
-  H_{\theta}^{-1}\begin{bmatrix}
    \hShort
  \end{bmatrix}  
\end{gather}
However, the natural specification of many economic models has singular
 $H_\theta$.

This, first,  algorithm
applies full rank linear transformations 
to equations from the original linear system in order to express
$x_{t+\theta}$ in terms of $x_{t-\tau} \ldots x_{t+\theta-1}$.
It produces an unconstrained, typically explosive, 
autoregressive representation for the evolution of the components of
the state space vectors and a set of vectors 
that provide important auxiliary initial conditions. 






%\section{Graphical Characterization of Unconstrained Autoregression Computation}
\label{sec:gfrr}
%\placeWrap{}
%\begin{multicols}{2}
%\begin{wrapfigure}[31]{r}{3.5in}

%\begin{floatingfigure}{6cm}
\begin{wrapfigure}[29]{r}{7cm}
%\begin{wrapfigure}{r}{8cm}
%\fbox{
%    \begin{minipage}{1.0\linewidth}
 \caption{
 \label{tableau}
 Initial Tableau 
 }
  \begin{center}
%\fbox{
  \begin{pdfpic}
\input pst-eps
\TeXtoEPS
\begin{pspicture} (6.75,3.75)
%\psgrid
\psset{xunit=0.45,yunit=.35cm}

%\psframe[fillstyle=solid,fillcolor=gray] (5,5) (11,4)
%\psframe[fillstyle=solid,fillcolor=gray] (9,0) (15,1)
\psframe[fillstyle=solid,fillcolor=gray] (5,8) (11,7)
\psframe[fillstyle=solid,fillcolor=gray] (9,0) (15,1)


\psline[linestyle=dashed]{<->}(9,6.5)(11,1.5)
\psset{arrows=->}
\rput(7.5,9.5){\rnode{A}{$\tau+\theta$}}
\rput(5,9.5){\rnode{B}{}}
\rput(10,9.5){\rnode{C}{}}
\ncline{A}{B}
\ncline{A}{C}
% \psline[linestyle=dashed]{<->}(9,3.5)(11,1.5)
% \psset{arrows=->}
% \rput(7.5,8.0){\rnode{A}{$\tau+\theta$}}
% \rput(5,8.0){\rnode{B}{}}
% \rput(10,8.0){\rnode{C}{}}
% \ncline{A}{B}
% \ncline{A}{C}

\rput(10.5,8.0){\rnode{D}{}}
\rput(12.5,9.5){\rnode{E}{$H^{\sharp,0}$}}
\nccurve[angleB=90,angleA=270]{E}{D}

\psset{arrows=-}
\psline[linewidth=0.6mm] (0,8) (15,8)
\psline[linewidth=0.6mm] (5,0) (5,10)
\psline[linewidth=0.6mm] (10,0) (10,10)


\end{pspicture}
\endTeXtoEPS
  \end{pdfpic}
%}
  \end{center}

%  \end{wrapfigure}

%  \begin{wrapfigure}{r}{8cm}
 \caption{Forward Row Annihilation 
 \label{annihilation}}

  \begin{center}
    \leavevmode
    
%\fbox{

  \begin{pdfpic}
\input pst-eps
\TeXtoEPS
\begin{pspicture} (6.75,3.75)
%\psgrid
\psset{xunit=0.45,yunit=.35cm}

\psframe[fillstyle=solid,fillcolor=gray] (5,8) (11,7)
\psframe[fillstyle=solid,fillcolor=white] (10,7.5) (11,8)

\psframe[fillstyle=solid,fillcolor=gray] (6,7) (12,6)
\psframe[fillstyle=solid,fillcolor=white] (11,6.5) (12,7)

\psframe[linestyle=dashed,linewidth=0.7mm] (5,6.5) (11,7.5)

\psframe[fillstyle=solid,fillcolor=gray] (9,0) (15,1)
\psframe[fillstyle=solid,fillcolor=white] (14,0.5) (15,1)




% \psframe[fillstyle=solid,fillcolor=gray] (5,5) (11,4)
% \psframe[fillstyle=solid,fillcolor=white] (10,4.5) (11,5)

% \psframe[fillstyle=solid,fillcolor=gray] (6,4) (12,3)
% \psframe[fillstyle=solid,fillcolor=white] (11,3.5) (12,4)

% \psframe[linestyle=dashed,linewidth=0.7mm] (5,3.5) (11,4.5)

% \psframe[fillstyle=solid,fillcolor=gray] (9,0) (15,1)
% \psframe[fillstyle=solid,fillcolor=white] (14,0.5) (15,1)





\psline[linestyle=dashed]{<->}(9,5.5)(11,1.5)

% \psline(10,7.5) (11,7.5)
% \psline(10,7.0) (11,7.0)
% \psline(11,6.5) (12,6.5)
% \psline(11,7.0) (11,7.0)
% \psline(14,3.5) (15,3.5)
% \psline(14,4.0) (14,4.0)
\psset{arrows=->}
\rput(10.5,7.5){\rnode{D}{}}
\rput(12.5,9.5){\rnode{E}{$H^{\sharp,1}$}}
\nccurve[angleB=90,angleA=270]{E}{D}
\rput(5.0,7.75){\rnode{F}{}}
\rput(2.5,8.5){\rnode{G}{$Z^{\sharp,1}=F^{\sharp,1}$}}
\nccurve[angleB=180]{G}{F}

\psset{arrows=-}
\psline[linewidth=0.6mm] (0,8) (15,8)
\psline[linewidth=0.6mm] (5,0) (5,10)
\psline[linewidth=0.6mm] (10,0) (10,10)
%above fixed%
\end{pspicture}
\endTeXtoEPS
  \end{pdfpic}

%}


  \end{center}

%  \end{wrapfigure}

%  \begin{wrapfigure}{r}{8cm}
  \begin{center}
 \caption{Full Rank Leading Block 
  \label{fullRankFor}}
%    \leavevmode
    
%\fbox{
  \begin{pdfpic}
\input pst-eps
\TeXtoEPS
\begin{pspicture} (6.75,3.75)
%\psgrid
\psset{xunit=0.45cm,yunit=.35cm}

\psframe[fillstyle=solid,fillcolor=gray] (5,8) (10,5.5)%zapped

\psframe[fillstyle=solid,fillcolor=gray] (5,6.0) (11,5.0)%mid front
\psframe[fillstyle=solid,fillcolor=white] (10,5.5) (11,6.0)

\psframe[fillstyle=solid,fillcolor=gray] (6,5.0) (12,4.0)
\psframe[fillstyle=solid,fillcolor=white] (11,4.5) (12,5.0)

\psframe[linestyle=dashed,linewidth=0.7mm] (5,4.5) (11,5.5)
\psframe[fillstyle=solid,fillcolor=black] (10,4.5) (11,5.5)


\psframe[fillstyle=solid,fillcolor=gray] (9,0) (15,1)
\psframe[fillstyle=solid,fillcolor=white] (14,0.5) (15,1)


\psset{arrows=->}
\rput(11.0,5.0){\rnode{D}{}}
\rput(10.5,8.75){\rnode{E}{$H^{\sharp,k}$ with $H^{\sharp,k}_\theta$ non-singular}}
\nccurve[angleB=0,angleA=270]{E}{D}
\rput(5.0,7.0){\rnode{F}{}}
\rput(2.5,7.5){\rnode{G}{$Z^{\sharp,k}$}}
\nccurve[angleB=180]{G}{F}
\rput(5.0,5.75){\rnode{H}{}}
\rput(2.5,4.5){\rnode{G}{$F^{\sharp,k}$}}
\nccurve[angleB=180]{G}{H}
\psset{arrows=-}
\psline[linewidth=0.6mm] (0,8) (15,8)%vertical left
\psline[linewidth=0.6mm] (5,0) (5,10)%midline
\psline[linewidth=0.6mm] (10,0) (10,10)%vertical right
\psline[linestyle=dashed]{<->}(9,3.5)(11,1.5)
\end{pspicture}
\endTeXtoEPS
  \end{pdfpic}
%}
  \end{center}
%   \end{minipage}
%}
%\end{wrapfigure}
%  \end{floatingfigure}

\end{wrapfigure}

Section~\ref{sec:uarproof} presents a proof that repeating 
this process of annihilating and regrouping rows ultimately  produces 
an $H^{\sharp,k}=H^{\sharp,\ast}$ with $H^{\sharp,\ast}_\theta$ non-singular.
The proof identifies a benign rank condition 
that guarantees that the algorithm will successfully compute
the unconstrained autoregression and the auxiliary initial conditions.

Figures~\ref{tableau}-\ref{fullRankFor} provide a graphical characterization
of the linear algebraic transformations characterizing the algorithm.
Figure~\ref{tableau}
% on page~\pageref{tableau} 
presents a graphical characterization of the relevant
set of linear constraints for $t=0 \ldots (\tau{}+\theta{})$.
The figure represents the regions where the coefficients 
are potentially  non-zero as shaded gray. 
If $H^{\sharp,0}_\theta$ is singular, one can find a linear 
combination of its rows which  preserves the rank of $H^{\sharp,0}_\theta$ but
which annihilates one or more of its rows.  


Consequently, one can pre-multiply
the matrices presented in Figure~\ref{tableau} by a unitary matrix 
to get the distribution of
zeros displayed in Figure~\ref{annihilation}.      Since the matrices
repeat over time, one need
only investigate the rank of the square matrix $H^{\sharp,0}_\theta$.
$U_1 H^{\sharp,0}=\begin{bmatrix} F^{\sharp,1} U_1 H^{\sharp,0}_\theta\end{bmatrix}$.
With some of the rows of $U_1 H^{\sharp,0}_\theta$ all zeros.
One can  regroup the rows in the new tableau to get $H^{\sharp,1}$.
By construction, the rank of $H^{\sharp,1}_\theta$ can be no less than
the rank of $H^{\sharp,0}_\theta$.

Note that  changing the order of the equations in
$H^{\sharp,0}_\theta$ will not affect the rank of $H^{\sharp,0}_\theta$
or the space spanned by the nonzero rows that will enter
$H^{\sharp,1}_\theta$.  Since  $H^{\sharp,0}_\theta$ is not full rank, 
a subset of the rows of $H^{\sharp,1}_\theta$ will
span the same space as $H^{\sharp,0}_\theta$.  The other rows in
 $H^{\sharp,1}_\theta$  will come
from linear combinations of the original system of equations in effect
``shifted forward'' in time.  

One can think of
the ``Full Rank Leading Block''
matrix as the result of premultiplications of the
``Initial Tableau'' by a sequence of unitary matrices. Implementations of
the algorithm can take advantage of the fact that the rows of the
matrices repeat. The regrouping can be done by ``shifting equations forward''
in time in an $L \times L(\tau + 1 +\theta)$ version of the tableau.




Section~\ref{sec:uarproof} presents a proof that repeating 
this process of annihilating and regrouping rows ultimately  produces 
an $H^{\sharp,k}=H^{\sharp,\ast}$ with $H^{\sharp,\ast}_\theta$ non-singular.
Algorithm \ref{alg:unconstrainedAR} presents pseudo code for 
an algorithm for computing the 
components of the state space transition matrix and the auxiliary initial
conditions.
%\begin{figure}
%  \centering
  \label{sec:ucarflowpcode}
\NumberProgramstrue
\sfvariables
\begin{algrthm}
\label{alg:unconstrainedAR}
\begin{program}
\mbox{Given $H$,}
\mbox{ compute the unconstrained autoregression.} 
\FUNCT \mathcal{F}_{\ref{alg:unconstrainedAR}} (H) \BODY
k:=0
\mathcal{Z}^0:=\varnothing
\mathcal{H}^0:=H
\Gamma=\varnothing
\WHILE \mathcal{H}^k_\theta \text{ is singular } \cap rows (\mathcal{Z}^k) < L (\tau+\theta) 
\DO
\mbox{{\em {\small Determine a Non-singular matrix that annihilates $ L-r(\mH^k_\theta) $ Rows of $\mH^k_\theta$}}}
U^k=\begin{bmatrix}U^k_Z\\U^k_N\end{bmatrix}:=|rowAnnihilator| (\mH^k_\theta) 
\mH^{k+1}:= \longExpH
\mZ^{k+1}:= \longExpQ
k:=k+1
\OD
\Gamma=- H_{\theta}^{-1}\begin{bmatrix} H_{-\tau}&\hspace{0.25in}&\dots&\hspace{0.25in}&H_{\theta-1}  \end{bmatrix}
A= \begin{bmatrix}  \begin{matrix}    0&I  \end{matrix}\\ \Gamma\end{bmatrix}
|return| \{ \begin{bmatrix}\mH^k_-\tau&\ldots&\mH^k_{\theta}\end{bmatrix},A,\mZ^k \}
\ENDFUNCT
\end{program}
%  \caption{Unconstrained Autoregression Pseudocode}
\end{algrthm}



%  \label{fig:ucar}
%\end{figure}

The algorithm terminates with:
\begin{gather}\label{eq:easy}
  \begin{bmatrix}
    \hsh
  \end{bmatrix} 
  \begin{bmatrix}
    x_{-\tau}\\ \vdots \\x_{\theta}
  \end{bmatrix} =0%\nonumber
   \\ \intertext{ with $H^{\sharp\ast}_{\theta}$ non singular. Let} 
\Gamma^\sharp=-  (H^{\sharp\ast}_{\theta})^{-1}\begin{bmatrix}
    \hshShort
  \end{bmatrix} \\ \intertext{ Then }
x_{\theta} = 
 \Gamma^\sharp
  \begin{bmatrix}
    x_{-\tau}\\ \vdots \\x_{\theta-1}
  \end{bmatrix}  %\nonumber
\end{gather}

This unconstrained auto-regression in $x_t$ provides exactly what one needs to
construct the state space transition matrix. 



\begin{gather}
A^\sharp= 
\begin{bmatrix}
  \begin{matrix}
    0&I
  \end{matrix}\\ \Gamma^\sharp
\end{bmatrix}\\ \intertext{ so that }
  \begin{bmatrix}
    x_{-\tau+1}\\ \vdots \\x_{\theta}
  \end{bmatrix}  = A
  \begin{bmatrix}
    x_{-\tau}\\ \vdots \\x_{\theta-1}
  \end{bmatrix} 
\end{gather}







\subsubsection{Asymptotic Linear Constraint Matrix: $Q$}
\label{asymQ}

\label{sec:invariantSpace}
In order to compute solutions to equation~\ref{eq:homo} that converge, 
one must rule out explosive trajectories. Blanchard and Kahn\cite{blanchard80} 
used eigenvalue and eigenvector calculations to characterize the space in 
which the solutions must lie. In contrast, our approach uses
an orthogonality constraint to characterize regions which the solutions 
must avoid.



Each left  eigenvector associated with a given eigenvalue
is orthogonal to each right eigenvector associated 
with roots associated with  different eigenvalues.
Since vectors in the left invariant space
associated with roots outside the unit circle are orthogonal to right eigenvectors associated with roots 
inside the
unit circle, a given state vector that is part of a convergent trajectory
must be orthogonal to each of these left invariant space vectors. See theorem
\ref{leftInv} on page \pageref{leftInv}.
Thus, the algorithm can exploit bi-orthogonality and a
less burdensome computation of vectors spanning the 
left invariant space  in order to rule out explosive
trajectories. 



If the vectors in V span the invariant space associated with explosive
roots,  trajectories satisfying equation~\ref{eq:homo} 
are non-explosive if and only if

\begin{gather}
 V A =   \mathcal{M}  V 
\end{gather}
with the eigenvalues of $ \mathcal{M}$ similar to the 
Jordan blocks of A associated with all eigenvalues
greater than one in absolute value.
\begin{gather}
V 
\begin{bmatrix}
  x_{t-\tau}\\
\vdots\\
  x_{t+\theta-1}
\end{bmatrix}=0
\end{gather}
for some $t$.\footnote{
If $A$ has roots with magnitude $1$ then trajectories can converge to 
either a limit cycle or a non-zero fixed point. Otherwise,
non-explosive trajectories will converge to the origin.}





Combining V and $Z^\sharp$  completely characterizes the space of 
stable solutions satisfying the linear system~\ref{eq:homo}.
\begin{gather}
  Q= 
  \begin{bmatrix}
    Z^{\sharp}\\V
  \end{bmatrix}
\end{gather}


The first set of equations come from the equations in equation system 
\ref{eq:homo} which do not appear in the transformed system 
of Equation~\ref{eq:easy}
but must nevertheless be satisfied. The second set of equations come from the
constraint that the solutions are non-explosive.
Algorithm \ref{alg:asymptoticConstraints} provides pseudo code for computing Q.
\begin{algrthm}
\label{alg:asymptoticConstraints}
\begin{program}
\mbox{Given $A,Z^{\sharp,\ast}$,}
\FUNCT \mathcal{F}_{\ref{alg:asymptoticConstraints}} (A,Z^{\sharp,\ast})
\text{Compute $V$, the vectors spanning the left}
\text{ invariant space of $A$ associated with eigenvalues }
\text{ greater than one in magnitude}
Q:=\begin{bmatrix}Z^{\sharp,\ast}\\V\end{bmatrix}
|return| \, Q
\ENDFUNCT
\end{program}
\end{algrthm}

  




\subsubsection{Convergent Autoregressive Representation: $B$}
\label{conar}

The first two algorithms  together  produce a matrix $Q$ characterizing
constraints guaranteeing that trajectories are not explosive.  See theorem
\ref{nonExp} and corollary \ref{cnvrg} for a proof.
\cite{anderson99} describes how to use 
the matrix $Q$ from Section~\ref{asymQ} to impose
saddle point stability in non linear perfect foresight models.
However, for linear models with unique  saddle point
solutions it is often useful to 
employ an autoregressive representation of the solution.
Theorem \ref{qzthm} in Section \ref{sec:existunique} 
provides a fully general characterization
of the existence and uniqueness of
a saddle point solution.  

  A summary for typical applications of the algorithm follows.
Partition $Q=
\begin{bmatrix}
  Q_L & Q_R
\end{bmatrix}$ where $Q_L$ has $L\tau$ columns.
When $\eta =L \theta$, $Q_R$ is square.
If  $Q_R$ is non-singular, the system has a unique solution\footnote{
  If $Q_R$ is singular, the system has an infinity of solutions.
  When $\eta <L \theta$,
The system has an infinity of solutions.

  When $Q$ has more than $L \theta$ rows,
The system has a unique nontrivial 
solution only for specific values of $x_{data}$.
}
\begin{gather}
    \begin{bmatrix}
    B\\\underset{2}{B}\\ \vdots \\ \underset{\theta}{B}  
  \end{bmatrix}
= Q_R^{-1} Q_L \,\,\text{ and solutions are of the form }
\begin{matrix}
x_t=B 
\begin{bmatrix}
  x_{t-\tau}\\
\vdots\\
  x_{t-1}
\end{bmatrix},&
x_{t+k}=\underset{k}{B} 
\begin{bmatrix}
  x_{t-\tau}\\
\vdots\\
  x_{t-1}
\end{bmatrix}
\end{matrix} 
\end{gather}

Algorithm \ref{alg:bmat} provides pseudo code for computing B.
\begin{algrthm}
\label{alg:bmat}
\begin{program}
\mbox{Given $Q$,}
\FUNCT \mathcal{F}_{\ref{alg:bmat}} (Q)
|cnt|=noRows (Q)
|return|\begin{cases}
\{Q,\infty\} &|cnt| < L\theta 
\{Q,0\} &|cnt| > L\theta 
\{Q,\infty\}& (Q_R singular) 
\{B=-Q_R^{-1} Q_L,1\} &otherwise
\end{cases}
\ENDFUNCT
\end{program}
\end{algrthm}

\subsection{Inhomogeneous Solution}

Now, suppose
\begin{gather}\label{eq:inhomog}
\sum_{i= - \tau}^\theta{ H_i  x_{ t + i } }= \Psi{} z_{t}, t \geq0\\
\lim_{ t \rightarrow\infty} \|x_t\|   < \infty
\end{gather}





\subsubsection{Inhomogeneous Factor Matrices: $\phi, F$}
\label{sec:inhomofactor}
\label{sec:inhomog}

\begin{thrm}\label{inHomo}
Given structural model matrices, $H_i, i=-\tau,\ldots,\theta$ and $\Psi$,
 convergent autoregression matrix $B$
there exist 
{\em inhomogeneous factor matrices}, $\phi$ and $F$ such that with 
\begin{gather}
\begin{bmatrix}
  B_{-\tau}&\dots&B_{-1}
\end{bmatrix}=B\\
 \phi= \left (  H_0 +   \begin{bmatrix}
    H_{1}\ldots H_\theta
  \end{bmatrix} \begin{bmatrix}
      B\\\vdots\\\ugB{\theta}
    \end{bmatrix} \right )^{-1} \\
 x_t=\sum_{i=-\tau}^{-1} B_i x_{t+i} + 
  \begin{bmatrix}
  0&\dots 0&I
  \end{bmatrix}
\sum_{s=0}^\infty%
 (F^{s} 
\begin{bmatrix}
0\\
\phi\Psi{}z_{t+s}  
\end{bmatrix}) 
\end{gather}
will satisfy 
the linear inhomogeneous system equation~\ref{eq:inhomog}.
\end{thrm}

See Section~\ref{sec:inhomoproof} for  derivations and formulae.
Algorithm \ref{alg:inhomog} provides pseudo code for computing $\phi$ and $F$.
\begin{algrthm}
\label{alg:inhomog}
\begin{program}
\mbox{Given $H,  Q$}
\FUNCT \mathcal{F}_{\ref{alg:inhomog}} (H,Q)
Where
%\begin{gather}
B=  \begin{bmatrix}B_L&B_R
\vdots&\vdots
B_L^\theta&B_R^\theta  \end{bmatrix}= Q_R^{-1} Q_L
%\end{gather}

\phi= (H_0 + H_+  \begin{bmatrix}B_R
\vdots
\ugBR{\theta} \end{bmatrix})^{-1}
F=\begin{bmatrix}0&I
\vdots&&\ddots
0&&&I
-\phi H_+\begin{bmatrix}0
 \vdots 
 0
I  \end{bmatrix}&-\phi H_+\begin{bmatrix}0
 \vdots
I
B_R  \end{bmatrix}&\ldots&-\phi H_+\begin{bmatrix}I
B_R
\vdots
\ugB{\theta-1}  \end{bmatrix}\end{bmatrix}
|return| (\phi,F)
\ENDFUNCT
\end{program}
\end{algrthm}


\subsubsection{Exogenous VAR Impact Matrix, $\vartheta{}$}

Modelers can augment the homogeneous linear perfect foresight solutions with
particular solutions characterizing  the impact of exogenous vector
autoregressive variables.

\begin{thrm}\label{var}
  

When 
\begin{gather}
  z_{t+1} = \Upsilon{} z_t\\ \intertext{one can show that}
 x_{t} = \begin{bmatrix}
B_L& B_R  
\end{bmatrix}
 \begin{bmatrix}
%     (x_{t-\tau}-x^\ast)\\\vdots\\ (x_{t-1}-x^\ast)
     x_{t-\tau}\\\vdots\\ x_{t-1}
  \end{bmatrix} + \vartheta{} z_t
\\
\intertext{ where }
vec(\vartheta{}) =   \begin{bmatrix}
  0&\dots 0&I
  \end{bmatrix}
 (I - \Upsilon^T\otimes{} F)^{-1} vec
\begin{bmatrix}
0\\ \vdots \\ 0 \\
\phi\Psi%
\end{bmatrix}
\end{gather}
\end{thrm}

See Section~\ref{sec:exogproof} for  derivations and formulae.


\subsection{Implementations}
\label{sec:imple}

This set of algorithms has been implemented in a wide variety of languages.
Three implementations, a  Matlab, a ``C'', and a Mathematica implementation,
 are available from the author.%\footnote{\url{http://www.federalreserve.gov/pubs/oss/oss4/aimindex.html}.} 
Each implementation avoids using the large tableau of Figure~\ref{tableau}.
They each  shift elements in the
rows of a single copy of the matrix $H$. 
Each implementation
eliminates inessential lags from the autoregressive representation before
constructing the state space transition matrix for invariant space 
calculations.  



The most widely used version is written in MATLAB. The MATLAB version
has a convenient modeling language front end for specifying the model
equations and generating the $H_i$ matrices. 

The ``C'' version, designed especially for solving large scale
models is by far the fastest implementation and most frugal with memory.
It uses sparse linear algebra routines from SPARSKIT\cite{saad94} and
HARWELL\cite{nag95} to economize on
memory. It avoids costly eigenvector calculations by
computing vectors spanning the left invariant space using ARPACK\cite{lehoucq96}.

For small models, one can employ a symbolic algebra version of the
algorithms written in Mathematica. On present day computers this,
code can easily construct symbolic state space transition matrices and compute 
symbolic expressions for eigenvalues for models with 5 to 10 equations.
The code can often obtain symbolic expressions for the invariant space
vectors when the transition matrix is of dimension 10 or less.




\section{Other Useful Rational Expectations Solution Calculations}
\label{sec:useful}


Economists use linear rational expectations models in a wide array of
application. 
The following sections describe calculations which are 
useful for optimal policy design, model simulation and estimation exercises.




\subsection{Observable Structure:  $S$}

\label{sec:applications}



\label{sec:errorCalc}


To compute the error term for estimation of the coefficients of these models,
one must commit to a particular information set.
Two typical choices are t and t-1 period expectations.





Given structural model matrices, $H_i, i=-\tau,\ldots,\theta$ 
and 
 convergent autoregression matrices $B_i,i=-\tau,-1$
there exists an 
{ observable structure matrix,\/} $S$




\begin{gather}
  \epsilon_t= S \begin{bmatrix}
 x_{t-\tau}\\\vdots\\x_{t}
\end{bmatrix}
\end{gather}

See Section~\ref{sec:obstrucproof} for a derivation and formula for $S$.
Algorithm \ref{alg:obs} provides pseudo code for computing S for a given lag, $k^\ast$ in the availability of information.
\begin{algrthm}
\label{alg:obs}
\begin{program}
\mbox{Given $B,k^\ast$}
\FUNCT \mathcal{F}_{\ref{alg:obs}} (B,k^\ast)
  \tilde{B}=  \begin{bmatrix}    \begin{matrix}0&I      
    \end{matrix}
B  \end{bmatrix}
S=   \begin{bmatrix}0_{L\times L\max (0,k^\ast-1)}&    H_{-\tau}&\ldots&H_0  \end{bmatrix} +
  \begin{bmatrix}  \begin{bmatrix}    H_{1}\ldots H_\theta  \end{bmatrix} \begin{bmatrix}      B
\vdots
\ugB{\theta}    \end{bmatrix} \tilde{B}^{k^\ast}   & 0_{L\times L\max (0,k^\ast-1)}  \end{bmatrix}
|return| (S)
\ENDFUNCT
\end{program}
\end{algrthm}


\subsection{Stochastic Transition Matrices:  $\sTransA, \sTransB$}
\label{sec:stochtrans}
To compute covariances, practitioners will find it useful to  construct
the stochastic transition matrices $\sTransA$ and $\sTransB$. 

Given structural model matrices, $H_i, i=-\tau,\ldots,\theta$ 
and 
 convergent autoregression matrices $B_i,i=-\tau,-1$
there exist  
{\em stochastic transition
matrices\/} $\sTransB$, $\sTransA$ such that




  

\begin{gather}
\begin{bmatrix}
 x_{t-\tau+1}\\\vdots\\x_{t}
\end{bmatrix}= 
\sTransA%
\begin{bmatrix}
 x_{t-\tau}\\\vdots\\x_{t-1}
\end{bmatrix} +
\sTransB%
\begin{bmatrix}\epsilon_{t} +
\Psi  (E[z_t|I_t]-E[z_t|I_{t-1}])
\end{bmatrix}
\end{gather}


See Section~\ref{sec:stochtransproof} for  derivations and formulae for $\sTransA$
and $\sTransB$.
Algorithm \ref{alg:trans} provides pseudo code for computing $\sTransA$ and $\sTransB$
\begin{algrthm}
\label{alg:trans}
\begin{program}
\mbox{Given $H,\Psi,S$}
\FUNCT \mathcal{F}_{\ref{alg:trans}} (H,S)
\sTransA=\begin{bmatrix}  0 &I&&
\vdots&&\ddots&
0&&&I
S_0^{-1} S_{t-\tau-\max  (k^\ast-1,0)+1} &\dots &\dots&S_0^{-1} S_{-1} \end{bmatrix}
\sTransB=\begin{bmatrix}  0
\vdots 
 0 
 S_0^{-1} \end{bmatrix}
|return| (\sTransA,\sTransB)
\ENDFUNCT
\end{program}
\end{algrthm}







\section{Conclusions}
\label{sec:conc}

This paper describes a set of algorithms that have proved very durable and 
useful for staff at the central bank.
The most distinctive features of this approach are:
\begin{itemize}
\item its algorithm for computing the minimal dimension state space 
transition matrix 
\item its
use of bi-orthogonality to characterize the asymptotic constraints 
that guarantee stability.
\item It's reliance on QR Decomposition and the real Schur Decomposition
for speed and accuracy.
\end{itemize}

The unique combination of features 
makes the algorithm more efficient than all the alternatives---especially
for large models. 
Staff at the 
 Federal Reserves have developed a large scale linear rational
expectations model consisting of 421 equations with one lead and one lag.
This model provides an extreme example of the speed advantage of the 
Anderson-Moore Algorithm(AMA).
On an Intel(R) Xeon 2.80GHz CPU running Linux
the MATLAB version of AMA
computes the rational expectations 
solution in 21 seconds while the the MATLAB version of gensys, a popular
alternative procedure, requires 16,953 seconds.
See~\cite{anderson06} for a systematic comparison of this algorithm with
the alternative procedures.
The code is available for download at
\begin{verbatim}
http://www.federalreserve.gov/Pubs/oss/oss4/aimindex.html
\end{verbatim}


\bibliography{files,anderson,paperstoget,aimUsers}
%\bibliography{files,anderson,paperstoget}

\clearpage
\newpage




\appendix

\section{Proofs}
\label{sec:proofs}


\subsection{Unconstrained Autoregression}
\label{sec:uarproof}

{%\color{anewcolor}
\begin{thrm}
Let 
{\small
\begin{gather} \mathcal{H}=\left .
  \begin{bmatrix}
\hmats\\
&\hmats\\
&&\ddots\\
&&&\hmats\\
  \end{bmatrix} \right \} \text{${\scriptstyle\tau+\theta+1}$} 
\end{gather}}
There are two cases:
\begin{itemize}
\item When $\mathcal{H}$ is full rank the algorithm terminates with 
 $Z^{\sharp\ast}$ ($Z^{\flat\ast}$) and non-singular
 $H^{\sharp\ast}_{\theta}$ ($H^{\flat\ast}_{\tau}$)
\item When $\mathcal{H}$ is not full rank the algorithm terminates when
some row of $
\begin{bmatrix}
\mathcal{H}^k_{-\tau}\ldots\mathcal{H}^k_\theta 
\end{bmatrix}$ is zero.
\end{itemize}

\end{thrm}
}
\begin{prf}
Consider the case when $\mathcal{H}$ is full rank.
  Each step of the algorithm applies a  rank preserving
pre-multiplication by a non singular matrix. Each step of the algorithm
where $H^{\sharp,k}_\theta$ is singular, increases the row rank of
$Z^{\sharp,k}$ by at least one.  At each step $\mathcal{Z}^{\sharp,k}$ are full rank. The rank of $Z^{\sharp,k}$ cannot exceed $L (\tau + \theta)$.
\end{prf}


The following corollary indicates that models 
with unique steady-states always terminate with non singular $\mathcal{H}^{\sharp,\ast}_{\theta}$.

\begin{crrlry}
If $  ( {\sum_{i= - \tau}^\theta { H_i}} )$ is non singular 
then
\begin{enumerate} 
\item $\mathcal{H}$ is full rank.
\item The origin is the unique steady state of equation \ref{eq:canonical}.
\item there exists a sequence of elementary row
operations that transforms $\mathcal{H}$ into $\mathcal{H}^\ast$
\end{enumerate} 
\end{crrlry}
\begin{prf}
  Suppose
$\mathcal{H}$ is not full rank.
 Then there is
a non zero vector $V \text{ such that } V \mathcal{H}  =0$. Consequently,
\begin{gather}
  \begin{bmatrix}
    V_{-\tau}\ldots V_{\theta}
  \end{bmatrix}
 \mathcal{H}  
 \begin{bmatrix}
   I&\ldots&I\\
\vdots&\ldots&\vdots\\
I&\ldots&I&\\
 \end{bmatrix}=0\\ \intertext{ and }
V_i  ({\sum_{j= - \tau}^\theta { H_j}})=0 \, \forall i
\end{gather}
So that $ ({\sum_{i= - \tau}^\theta { H_i}})$ must be singular.
\end{prf}




\subsection{Asymptotic Constraints}
\label{sec:asymproof}


\begin{thrm}\label{leftInv}
  Consider a left invariant space and a right invariant space
with no eigenvalues in common. Suppose
$V_1$ spans the left invariant space and $W_2$ spans the right invariant space.
  \begin{gather}
V_1 A = T_1 V_1\\
A W_2=  W_2 T_2
  \end{gather}
With eigenvalues of $T_1 \neq T_2$.
Then  $V_1 W_2=0$
\end{thrm}
\begin{prf}
A right eigenvector $x_i$ and a left-eigenvector $y_j$ corresponding to distinct
eigenvalues $\lambda_i$ and $\lambda_j$ are orthogonal.\cite{NOBLE}
Finite dimensional matrices have finite dimensional 
Jordan blocks.
Raising a given matrix to a power produces a matrix with
smaller Jordan blocks. Raising the matrix to a high enough power
ultimately eliminates all nontrivial Jordan Blocks.
Consequently, the left invariant space vectors are linear combination of the
left eigenvectors and the right invariant space vectors are a linear combination
of the right eigenvectors of the transition matrix raised to some finite power.
\begin{gather}
  V_1 = \beta_{1} 
  \begin{bmatrix}
y_1\\ \vdots\\ y_J
  \end{bmatrix}\\
  W_2 = 
  \begin{bmatrix}
x_1& \dots &x_K
  \end{bmatrix}
\alpha_{2}\\
V_1 W_2 =  \beta_{1} 
  \begin{bmatrix}
y_1\\ \vdots \\ y_J
  \end{bmatrix}
  \begin{bmatrix}
x_1& \dots &x_K
  \end{bmatrix}
\alpha_{2} = 0
\end{gather}
\end{prf}

\begin{thrm}\label{nonExp}
Let $\{x^{conv}_t\}$, $t= -\tau,\ldots,\infty$ be a non explosive solution satisfying
equation~\ref{eq:canonical}. Let $A$ be the state space transition matrix
for equation
\ref{eq:canonical} and $V$ be a set of
invariant space vectors spanning the invariant space
associated with roots of
$A$ of magnitude bigger than $1$. Then for $t= 0,\ldots,\infty$
\begin{gather}
V 
\begin{bmatrix}
  x^{conv}_{t-\tau}\\
\vdots\\
  x^{conv}_{t+\theta-1}
\end{bmatrix}=0
\end{gather}
\end{thrm}
\begin{prf}
Using  $W$, the left generalized eigenvectors of $A$, one can employ the
 Jordan Canonical Form of A to write
\begin{gather}
  W^H A  = J W^H \\
\intertext{ so that}
A^t =  (W^H)^{-1} J^t W^H\\
y_t = A^t y_0\\
W^H y_t = J^t W^H y_0\\
\lim_{t\rightarrow \infty} y_t =0 \Rightarrow
\lim_{t\rightarrow \infty} W^H y_t =0 \\\intertext{Consequently,}
W_i^H y_0 =0 \, \forall i \text{ such that } |J_{ii}| > 1.\\ \intertext{ so that }
V y_0 = \alpha W^H y_0 = 0
\end{gather}
\end{prf}

\begin{crrlry}\label{cnvrg}
Let $\{x_t\}$, $t= -\tau,\ldots,\infty$ be a  solution satisfying
equation~\ref{eq:homo}.  
If $A$ has no roots with magnitude $1$ then the 
path converges to the
origin if and only if
\begin{gather}
V 
\begin{bmatrix}
  x_{t-\tau}\\
\vdots\\
  x_{t+\theta-1}
\end{bmatrix}=0
\end{gather}
for some $t$.
\end{crrlry}
\begin{prf}
  \begin{gather}
W_i^H y_0 =0 \, \forall i \text{ such that } |J_{ii}| > 1.\\ \intertext{ means $y_t$ }
    J_{ii} \ne 1.\\
y_t = A^t y_0
  \end{gather}
\end{prf}


\subsection{Existence and Uniqueness}
\label{sec:existunique}

\begin{thrm}
  \label{qzthm}
Identify $Q_L, Q_R$ from
\begin{gather}
  Q= 
  \begin{bmatrix}
    Z^{\sharp}\\V
  \end{bmatrix}
= 
  \begin{bmatrix}
    Q_L&Q_R
  \end{bmatrix} 
\end{gather}
{with $Q_R$ an $(\eta \times L\theta)$ matrix, 
$Q_L$ an $(\eta \times L\tau)$ matrix, where 
$\eta$ represent the number of rows in the matrix $Q$.}

The existence of convergent solutions depends on the magnitudes of the
ranks of the augmented
matrix 
\begin{gather}
r_1=rank \left (\begin{bmatrix}
Q_R&-Q_L x_{data}
  \end{bmatrix} 
\right)
\\ 
\intertext{ and  }
r_2=rank(Q_R).
\end{gather}
By construction,  $r_1 \ge r_2$ and $r_2 \le L\theta$.
There are three cases.
\begin{enumerate}
\item If $r_1 > r_2$ there is no nontrivial convergent solution
\item If $r_1 = r_2 = L  \theta$ there is a unique convergent solution
\item If $r_1 = r_2 < L  \theta$ the system has an infinity of convergent 
solutions
\end{enumerate}

\end{thrm}



\begin{crrlry}
  When $\eta =L \theta$, $Q_R$ is square.
If  $Q_R$ is non-singular, the system has a unique solution
\begin{gather}
    \begin{bmatrix}
    B\\\underset{2}{B}\\ \vdots \\ \underset{\theta}{B}  
  \end{bmatrix}
= Q_R^{-1} Q_L \intertext{ and solutions are of the form}
\begin{matrix}
x_t=B 
\begin{bmatrix}
  x_{t-\tau}\\
\vdots\\
  x_{t-1}
\end{bmatrix},&
x_{t+1}=\underset{2}{B} 
\begin{bmatrix}
  x_{t-\tau}\\
\vdots\\
  x_{t-1}
\end{bmatrix},&\,\,\dots\,\,
x_{t+\theta}=\underset{\theta}{B}
\begin{bmatrix}
  x_{t-\tau}\\
\vdots\\
  x_{t-1}
\end{bmatrix}
\end{matrix} 
\end{gather}
  If $Q_R$ is singular, the system has an infinity of solutions.

  When $\eta <L \theta$,
The system has an infinity of solutions.

  When $Q$ has more than $L \theta$ rows,
The system has a unique nontrivial 
solution only for specific values of $x_{data}$
\end{crrlry}


% \begin{thrm}
  
% Let
% \begin{gather}
%   Q= 
%   \begin{bmatrix}
%     Z^{\sharp}\\V
%   \end{bmatrix}
% = 
%   \begin{bmatrix}
%     Q_L&Q_R
%   \end{bmatrix}
% \end{gather}
% The existence of convergent solutions depends on the magnitude of the
% rank of the augmented
% matrix 
% \begin{gather}
% r_1=rank \left (\begin{bmatrix}
%     I&0&x_{data}\\
% Q_L&Q_R&0
%   \end{bmatrix} 
% \right)
% \\ 
% \intertext{ and  }
% r_2=rank \left  (\begin{bmatrix}
%     I&0\\
% Q_L&Q_R
%   \end{bmatrix} \right)
% \end{gather}
% and  $L (\tau+\theta)$, the number of unknowns.


% \begin{enumerate}
% \item If $r_1 > r_2$ there is no nontrivial convergent solution
% \item If $r_1 = r_2 = L (\tau + \theta)$ there is a unique convergent solution
% \item If $r_1 = r_2 < L (\tau + \theta)$ the system has an infinity of convergent 
% solutions
% \end{enumerate}

% \end{thrm}




Proof of rank of Q
\begin{prf}
The theorem applies well known 
results on existence and uniqueness of solutions to 
linear equation systems\cite{NOBLE}.
If
$\mathcal{M}_2=\begin{bmatrix}
  x_{data}\\0
\end{bmatrix}$ 
does not lie in the column space of $\mathcal{M}_1=\begin{bmatrix}
    I&0\\
Q_L&Q_R
  \end{bmatrix}$, then there is no solution.
If $\mathcal{M}_2$ lies in the column space of  $ \mathcal{M}_1 $ and the latter matrix is full rank, then there is a unique
solution.
If $\mathcal{M}_2$ lies in the column space of  $ \mathcal{M}_1$ and the latter matrix is not full rank, then there are
multiple solutions.
\end{prf}


%\begin{thrm}
%If there exist S $\text{ such that }$
%\begin{gather}
%S A S^{-1} = 
%\begin{bmatrix}
%  A_u&0&0\\
%\Upsilon_{um}&A_m&0\\
%\Upsilon_{ul}&\Upsilon_{ml}&A_l
%\end{bmatrix}\\ \intertext{ then one can compute eigenvalues and eigenvectors in matrices of smaller dimension. If }
%x A_u = M_u x\\
%y A_m = M_m y\\
%z A_l = M_l z\\
%S 
%\begin{bmatrix}
%  x & 0&0
%\end{bmatrix} A = M_u S 
%\begin{bmatrix}
%  x&0&0
%\end{bmatrix}\\
%S 
%\begin{bmatrix}
%  x_y &y&0
%\end{bmatrix} A = M_m S 
%\begin{bmatrix}
%  x_y&y&0
%\end{bmatrix}\\ \intertext{ with }
%vec (x_y)=-  ( (A^T_u \otimes I) -  (M_m \otimes I))^{-1}vec (y \Upsilon_{um})\\
%S 
%\begin{bmatrix}
%  x_z &y_z&z
%\end{bmatrix} A = M_l S 
%\begin{bmatrix}
%  x_z&y_z&z
%\end{bmatrix}\\ \intertext{ with }
%vec (y_z)=-  ( (A^T_m \otimes I) -  (M_m \otimes I))^{-1}vec (z \Upsilon_{ml})\\
%vec (x_z)=-  ( (A^T_u \otimes I) -  (M_l \otimes I))^{-1}vec (y \Upsilon_{ul})
%\end{gather}

%\end{thrm}
%\begin{prf}
%  Under construction.
%\end{prf}
%\begin{thrm}
%We can solve models with unit roots by computing the solution with no unit
%roots and then inferring the result of including a unit root.
%\end{thrm}
%\begin{prf}
%  Under construction.
%\end{prf}


\subsection{Proof of  Theorem \ref{inHomo}}
\label{sec:inhomoproof}



%\section{Proofs}

\begin{prf}
%\begin{multicols}{1}
Construct the $L\tau{} \times L\tau$ matrix:
\begin{gather}
  \tilde{B}=
  \begin{bmatrix}
    \begin{matrix}
0&I\\      
    \end{matrix}\\
B
  \end{bmatrix}\\ \intertext{Define}
\ugB{k+1} = \ugB{k} \tilde{B}
\end{gather}

Applying equation \ref{eq:homo} to the unique convergent 
solution, it follows that
\begin{gather}
  \begin{bmatrix}
    H_{-}&H_{0}&H_{+}
  \end{bmatrix}
  \begin{bmatrix}
    I\\
\begin{matrix}
      B\\\vdots\\\ugB{\theta+1}
    \end{matrix}
  \end{bmatrix}=0\intertext{ where }
H_-=
\begin{bmatrix}
  H_{-\tau} \ldots H_{-1}
\end{bmatrix}
H_+=
\begin{bmatrix}
  H_{1} \ldots H_{\theta}
\end{bmatrix}
\end{gather}

Which can also be written as:
\begin{gather}
  \begin{bmatrix}
    H_{-}&H_{0}&H_{+}
  \end{bmatrix}
  \begin{bmatrix}
    I\\
\begin{bmatrix}
  \begin{matrix}
0&I    
  \end{matrix}\\
      B\\\vdots\\\ugB{\theta}
    \end{bmatrix}\tilde{B}
  \end{bmatrix}=0
\end{gather}

So that:

\begin{gather}
  H_{-} + 
 (  \begin{bmatrix}
    0&H_0
  \end{bmatrix} + H_+ 
  \begin{bmatrix}
    B\\\vdots\\ \ugB{\theta}
  \end{bmatrix})\tilde{B} = 0\\
  H_{-} + 
 (  \begin{bmatrix}
    0&H_0
  \end{bmatrix} + H_+ 
  \begin{bmatrix}
    B_L&B_R\\\vdots&\vdots\\ \ugBL{\theta}&\ugBR{\theta}
  \end{bmatrix})\tilde{B} = 0
\end{gather}
where the $B_R^k$ matrices are $L \times{} L$.

Note that
\begin{gather}
    \begin{bmatrix}
    B_L&B_R\\\vdots&\vdots\\ \ugBL{\theta}&\ugBR{\theta}
  \end{bmatrix}\tilde{B} = 
  \begin{bmatrix}
    0&B_L\\\vdots&\vdots\\0&\ugBL{\theta}
  \end{bmatrix}+
  \begin{bmatrix}
    B_R\\\vdots\\\ugBR{\theta}
  \end{bmatrix}B
\end{gather}
Now
\begin{gather}
  H_- + H_+  \begin{bmatrix}
    0&B_L\\\vdots&\vdots\\0&\ugBL{\theta}
  \end{bmatrix}    +  (H_0 + H_+  \begin{bmatrix}
    B_R\\\vdots\\\ugBR{\theta}
  \end{bmatrix})B=0
\end{gather}

Define

\begin{gather}
  \phi=  (H_0 + H_+  \begin{bmatrix}\label{phiDef}
    B_R\\\vdots\\\ugBR{\theta}
  \end{bmatrix})^{-1}\\ \intertext{So that}
\phi H_- + \phi H_+  \begin{bmatrix}
    0&B_L\\\vdots&\vdots\\0&\ugBL{\theta}
  \end{bmatrix}  +B = 0\label{phiUse}
\end{gather}

Consider the impact that the time $t+s$ value $z_{t+s}$ has on the
value of $x_{t+s}$
We can write
\begin{gather}
  \begin{bmatrix}
    H_{-}&H_{0}&H_{+}
  \end{bmatrix}
  \begin{bmatrix}
    I&&&0\\&\ddots&&\vdots&\\&&I&0\\
0&\dots&0&I\\
0&&
\begin{matrix}
      B_L\\\vdots\\B_L^\theta
    \end{matrix}&
\begin{matrix}
      B_R\\\vdots\\B_R^\theta
    \end{matrix}
  \end{bmatrix}
  \begin{bmatrix}
    x_{t+s-\tau}\\\vdots\\x_{t+s}
  \end{bmatrix}= \Psi z_{t+s}
\end{gather}
or equivalently,
\begin{gather}
  \begin{split}
\phi\Psi z_{t+s}=&\\
&\phi   (\begin{bmatrix}
H_-&0
  \end{bmatrix} + 
  \begin{bmatrix}
    0&H_0
  \end{bmatrix} + \\
&  \begin{bmatrix}
    H_+ \begin{bmatrix}
    0&B_L\\\vdots&\vdots\\0&\ugBL{\theta}
  \end{bmatrix}& H_+\begin{bmatrix}
    B_R\\\vdots\\\ugBR{\theta}
  \end{bmatrix}
  \end{bmatrix})  \begin{bmatrix}
    x_{t+s-\tau}\\\vdots\\x_{t+s}
  \end{bmatrix}
  \end{split}
  \end{gather}
or
\begin{gather}
  \begin{split}
\phi\Psi z_{t+s}=&\\
&\phi   (\begin{bmatrix}
H_-&0
  \end{bmatrix} + \\
&  \begin{bmatrix}
    H_+ \begin{bmatrix}
    0&B_L\\\vdots&\vdots\\0&\ugBL{\theta}
  \end{bmatrix}& 0 
  \end{bmatrix})  \begin{bmatrix}
    x_{t+s-\tau}\\\vdots\\x_{t+s}
  \end{bmatrix}+\\
&\phi   (  \begin{bmatrix}
    0&H_0
  \end{bmatrix} + \\
&  \begin{bmatrix}
0& H_+\begin{bmatrix}
    B_R\\\vdots\\\ugBR{\theta}
  \end{bmatrix}
  \end{bmatrix})  \begin{bmatrix}
    x_{t+s-\tau}\\\vdots\\x_{t+s}
  \end{bmatrix}
  \end{split}
  \end{gather}
Which by equations~\ref{phiDef}~and~\ref{phiUse} can be written
\begin{gather}
  \begin{split}
\phi\Psi z_{t+s}=&\\
&\begin{bmatrix} -B&0\end{bmatrix}  \begin{bmatrix}
    x_{t+s-\tau}\\\vdots\\x_{t+s}
  \end{bmatrix}+\\
& \begin{bmatrix}
    0&I  \end{bmatrix}  \begin{bmatrix}
    x_{t+s-\tau}\\\vdots\\x_{t+s}
  \end{bmatrix}
  \end{split}
  \end{gather}
So we have
  \begin{gather}
  \begin{bmatrix}
    -B&I
  \end{bmatrix}
\begin{bmatrix}
    x_{t+s-\tau}\\\vdots\\x_{t+s}
  \end{bmatrix}= \phi\Psi z_{t+s}\\
x_{t+s} = B \begin{bmatrix}
    x_{t+s-\tau}\\\vdots\\x_{t+s-1} 
  \end{bmatrix}+\phi\Psi z_{t+s}
\end{gather}

Now consider the impact of $z_{t+s}$ on $x_{t+s-1}$.
We can write
\begin{multline*}
\phi   (\begin{bmatrix}
H_-&0
  \end{bmatrix} + 
  \begin{bmatrix}
    0&H_0
  \end{bmatrix} + \\
  \begin{bmatrix}
    H_+ \begin{bmatrix}
    0&B_L\\\vdots&\vdots\\0&\ugBL{\theta}
  \end{bmatrix}& H_+\begin{bmatrix}
    B_R\\\vdots\\\ugB{\theta}
  \end{bmatrix}
  \end{bmatrix})  \begin{bmatrix}
    x_{t+s-\tau-1}\\\vdots\\x_{t+s-1}
  \end{bmatrix}+ \\ \phi H_+\begin{bmatrix}
I\\B_R\\\vdots\\\ugB{\theta-1}
  \end{bmatrix} \phi \Psi z_{t+s}=0
\end{multline*}
where the last term captures the impact $z_{t+s}$ has on values of x $t+s$
and later.
Using equations~\ref{phiDef}~and~\ref{phiUse} we can write 
\begin{gather}
    \begin{bmatrix}
    -B&I
  \end{bmatrix}
\begin{bmatrix}
    x_{t+s-\tau-1}\\\vdots\\x_{t+s-1}
  \end{bmatrix}+  \phi H_+\begin{bmatrix}
I\\B_R\\\vdots\\\ugB{\theta-1}
  \end{bmatrix} \phi\Psi z_{t+s}=0\\
x_{t+s-1} = B \begin{bmatrix}
    x_{t+s-\tau-1}\\\vdots\\x_{t+s-2} 
  \end{bmatrix}  -\phi H_+\begin{bmatrix}
I\\B_R\\\vdots\\\ugB{\theta-1}
  \end{bmatrix} \phi\Psi z_{t+s}\\ \intertext{ and more generally}
x_{t+s-i} = B \begin{bmatrix}
    x_{t+s-\tau-1}\\\vdots\\x_{t+s-2} 
  \end{bmatrix} +  (-\phi H_+\begin{bmatrix}
I\\B_R\\\vdots\\\ugBR{\theta-1}
  \end{bmatrix})^{i} \phi\Psi z_{t+s}
\end{gather}




To accommodate lagged expectations,
suppose that information on all the endogenous variables becomes available with
the same lag ($D^\ast$) in time: 
$\exists K^\ast \text{ such that } x_{t-k} \in I_t, \forall K \ge K^\ast\\$ then set,
\begin{gather}
    \begin{bmatrix}
    E [x_{t+1} |I_t] \\
\vdots \\
    E [x_{t+\theta}|I_t]
  \end{bmatrix}  =
\begin{bmatrix}
      B\\\vdots\\\ugB{\theta}
    \end{bmatrix}
 \tilde{B}^{k^\ast}
  \begin{bmatrix}
    x^{data}_{t-\tau+1-k^\ast}\\
\vdots \\ 
   x^{data}_{t-k^\ast}
  \end{bmatrix} +\\
  \begin{bmatrix}
\sum_{s=0}^\infty  (  (-\phi H_+\begin{bmatrix}
I\\B_R\\\vdots\\\ugBR{\theta-1}
  \end{bmatrix})^{s} \phi\Psi z_{t+s+1})\\
  \vdots\\
\sum_{s=0}^\infty  (  (-\phi H_+\begin{bmatrix}
I\\B_R\\\vdots\\\ugBR{\theta-1}
  \end{bmatrix})^{s} \phi\Psi z_{t+s+\theta})
  \end{bmatrix}
\end{gather}


So that 

\begin{equation*}
  \begin{split}
x_{t} = &\\
&\begin{bmatrix}
B_L& B_R  
\end{bmatrix}
\begin{bmatrix}
     x_{t-\tau}\\ \vdots\\ x_{t-1}
  \end{bmatrix} + \\
&  \begin{bmatrix}
  0&\dots 0&I
  \end{bmatrix}
\sum_{s=0}^\infty  ( F^{s} 
\begin{bmatrix}
0\\
\phi \Psi z_{t+s}  
\end{bmatrix})\label{basic}
  \end{split}
\end{equation*}

Where
\begin{gather}
B=  \begin{bmatrix}
B_L&B_R\\
\vdots&\vdots\\
B_L^\theta&B_R^\theta
  \end{bmatrix}
= Q_R^{-1} Q_L
\end{gather}


\begin{gather}
  \phi=  (H_0 + H_+  \begin{bmatrix}
    B_R\\\vdots\\\ugBR{\theta}
  \end{bmatrix})^{-1}\\
F=
\begin{bmatrix}
0&I\\
\vdots&&\ddots\\
0&&&I\\
-\phi H_+\begin{bmatrix}
0\\ \vdots \\ 0\\I
  \end{bmatrix}&
-\phi H_+\begin{bmatrix}
0\\ \vdots\\I\\B_R
  \end{bmatrix}&\ldots&
-\phi H_+\begin{bmatrix}
I\\B_R\\\vdots\\\ugBR{\theta-1}
  \end{bmatrix}
\end{bmatrix}
\end{gather}

\begin{gather}
x_{t} = B \begin{bmatrix}
    x_{t-\tau}\\\vdots\\x_{t-1} 
  \end{bmatrix} +\sum_{s=0}^\infty  (  (-\phi H_+\begin{bmatrix}
I\\B_R\\\vdots\\\ugBR{\theta-1}
  \end{bmatrix})^{s} \phi\Psi z_{t+s})
\end{gather}
%\end{multicols}
  
\end{prf}
  
%\begin{multicols}{2}
\ 
%\end{multicols}
\clearpage




\subsection{Proof of Theorem \ref{var}}
\label{sec:exogproof}


\begin{prf}
%\begin{multicols}{2}


\begin{gather}
\sum_{s=0}^\infty%
 (F^{s} 
\begin{bmatrix}
0\\
\phi\Psi{}z_{t+s}  
\end{bmatrix}) 
=\sum_{s=0}^\infty  ( F^s 
  \begin{bmatrix}
0\\    \phi \Psi
  \end{bmatrix}
\Upsilon^{s} )z_{t}=\vartheta{} z_t
\intertext{ where }
vec(\vartheta{}) =   \begin{bmatrix}
  0&\dots 0&I
  \end{bmatrix}
 (I - \Upsilon^T\otimes{} F)^{-1} vec
\begin{bmatrix}
0\\ \vdots \\ 0 \\
\phi\Psi%
\end{bmatrix}
\end{gather}
%\end{multicols}
\end{prf}  



\subsection{Observable Structure}
\label{sec:obstrucproof}

Since one can write
\begin{gather}
  \epsilon_t =
  \begin{bmatrix}
    H_{-\tau}\ldots H_0
  \end{bmatrix} 
  \begin{bmatrix}
    x^{data}_{t-\tau}\\
\vdots \\ 
   x^{data}_{t}
  \end{bmatrix} +
  \begin{bmatrix}
    H_{1}\ldots H_\theta
  \end{bmatrix} 
  \begin{bmatrix}
    E [x_{t+1} |I_t] \\
\vdots \\
    E [x_{t+\theta}|I_t]
  \end{bmatrix} - \Psi z_t
\end{gather}
We find that

\begin{gather}
  \epsilon_t= S   \begin{bmatrix}
    x^{data}_{t-\tau+1-\max (1,k^\ast)}\\
\vdots \\ 
   x^{data}_{t}
  \end{bmatrix} - \Psi z_t\\ \intertext{ where }
S=   \begin{bmatrix}
0_{L\times L\max (0,k^\ast-1)}&    H_{-\tau}&\ldots&H_0
  \end{bmatrix} +\\
  \begin{bmatrix}
  \begin{bmatrix}
    H_{1}\ldots H_\theta
  \end{bmatrix} 
\begin{bmatrix}
      B\\\vdots\\\ugB{\theta}
    \end{bmatrix}
 \tilde{B}^{k^\ast}
   & 0_{L\times L\max (0,k^\ast-1)}
  \end{bmatrix}
\end{gather}

Note that for $k^\ast \ge 1$
\begin{gather}
  \frac{\partial \epsilon_t}{\partial x^{data}_t} = H_0
\end{gather}
and for $k^\ast = 0$
\begin{gather}
  \frac{\partial \epsilon_t}{\partial x^{data}_t} = H_0 +   \begin{bmatrix}
    H_{1}\ldots H_\theta
  \end{bmatrix} \begin{bmatrix}
      B\\\vdots\\\ugB{\theta}
    \end{bmatrix} = \phi^{-1}
\end{gather}



\subsection{Stochastic Transition Matrices}
\label{sec:stochtransproof}


One can write
\begin{gather}
  \begin{bmatrix}
    x_{t-\tau-\max  (k^\ast-1,0)+1}\\ \vdots \\ x_{t}
  \end{bmatrix}=
\sTransA   \begin{bmatrix}
    x_{t-\tau-\max  (k^\ast-1,0)}\\ \vdots \\ x_{t-1}
  \end{bmatrix}  + 
\sTransB \epsilon_t\intertext{ where }
\sTransA=
\begin{bmatrix}
  0 &I&&\\
\vdots&&\ddots&\\
0&&&I\\
S_0^{-1} S_{t-\tau-\max  (k^\ast-1,0)+1} &\dots &\dots&S_0^{-1} S_{-1} 
\end{bmatrix}\\
\sTransB=
\begin{bmatrix}
  0\\\vdots \\ 0 \\ S_0^{-1} 
\end{bmatrix}
\end{gather}




%\newpage

% \section{The Algorithm Pseudo Code and Flow Charts}
% \label{sec:fcrt}
% \label{sec:pseudocode}


%\subsection{Unconstrained Autoregression}

% \begin{figure*}[htbp]
%   \begin{center}
%     \leavevmode
% %\fbox{
% \begin{picture} (20,24)
%   \begin{picture}(20.000000,30.000000)(-6.000000,-28.000000)
% picture environment flowchart generated by flow 0.99b
\put(0.0000,-2.0000){\oval(5.0000,2.0000)}
\put(-2.5000,-3.0000){\makebox(5.0000,2.0000)[c]{\shortstack[c]{
Begin
}}}
\put(0.0000,-3.0000){\vector(0,-1){1.0000}}
\put(-2.0000,-6.0000){\framebox(4.0000,2.0000)[c]{\shortstack[c]{
$k:=0$
}}}
\put(0.0000,-6.0000){\vector(0,-1){1.0000}}
\put(-5.0000,-9.5000){\framebox(10.0000,2.5000)[c]{\shortstack[c]{
Initialize Auxiliary Initial Conditions:\\
$\mZ^0:=\varnothing$\\
Initialize Equivalent Linear Sytem\\
$\mH^0:=H$
}}}
\put(0.0000,-9.5000){\vector(0,-1){1.0000}}
\put(-4.0000,-12.5000){\line(2,1){4.0000}}
\put(-4.0000,-12.5000){\line(2,-1){4.0000}}
\put(4.0000,-12.5000){\line(-2,-1){4.0000}}
\put(4.0000,-12.5000){\line(-2,1){4.0000}}
\put(-4.0000,-14.5000){\makebox(8.0000,4.0000)[c]{\shortstack[c]{
$\mH^k_\theta$ Full Rank \\
or\\
$rows(\mZ^{k}) \ge L(\tau+\theta)$
}}}
\put(4.0000,-11.9000){\makebox(0,0)[lt]{Yes}}
\put(1.2000,-14.5000){\makebox(0,0)[lb]{No}}
\put(4.0000,-12.5000){\line(1,0){1.0000}}
\put(5.0000,-12.5000){\vector(1,0){1.0000}}
\put(10.0000,-12.5000){\oval(8.0000,2.0000)}
\put(6.0000,-13.5000){\makebox(8.0000,2.0000)[c]{\shortstack[c]{
Return\\
$\{\begin{bmatrix}\mH^k_{-\tau}&\ldots&\mH^k_{\theta}\end{bmatrix},(\Gamma,\text{ or }\varnothing),\mZ^k\}$
}}}
%\put(0.0000,-14.5000){\vector(0,-1){1.0000}}
\put(-5.0000,-18.0000){\framebox(10.0000,3.5000)[c]{\shortstack[c]{
{\small Determine a Non-singular matrix}\\
$U^k:=\begin{bmatrix}U^k_Z\\U^k_N\end{bmatrix}$\\
that annihilates $L-r(\mH^k_\theta)$ Rows of $\mH^k_\theta$
}}}
\put(0.0000,-18.0000){\vector(0,-1){1.0000}}
\put(-5.0000,-21.5000){\framebox(10.0000,2.5000)[c]{\shortstack[c]{
Shift Rows Right\\
Set $\mH^{k+1}:= \longExpH$
}}}
\put(0.0000,-21.5000){\vector(0,-1){1.0000}}
\put(-5.0000,-25.0000){\framebox(10.0000,2.5000)[c]{\shortstack[c]{
Augment Auxiliary Initial Conditions\\
Set $\mZ^{k+1}:= \longExpQ$
}}}
\put(0.0000,-25.0000){\vector(0,-1){1.0000}}
\put(-2.0000,-28.0000){\framebox(4.0000,2.0000)[c]{\shortstack[c]{
$k:=k+1$
}}}
\put(-2.0000,-27.0000){\line(-1,0){4.0000}}
\put(-6.0000,-27.0000){\line(0,1){14.5000}}
\put(-6.0000,-12.5000){\vector(1,0){2.0000}}
\end{picture}

% \end{picture}
% %}
%     \caption{Algorithm to Compute Unconstrained Auto-regression: Algorithm~\ref{alg:unconstrainedAR}:$\mathcal{F}_1 (\mathcal{H})$}
%     \label{fig:unconstrained}
%   \end{center}
% \end{figure*}

%\newpage


%\subsection{Asymptotic Constraints}
%\label{sec:asympCode}


%\subsection{Existence and Uniqueness}
%\label{sec:existpCode}


  

%\newpage

%\subsection{Inhomogeneous Solution}
%\label{sec:inhomopcode}

%\newpage

%\subsection{Stochastic Transition Matrices}
%\label{sec:stochtranspcode}


%\newpage

\section{An Example}
\label{sec:exampleapp}

  

  

This section applies algorithms 1-3  to 
 the following model with N=2.\cite{taylor79:stagg}

\begin{gather}
  w_t=\frac{1}{N}E_t[\sum_{i=0}^{N-1}W_{t+i}] - \alpha (u_t-u_n) + \nu_t\\
W_t=\frac{1}{N}\sum_{i=0}^{N-1}w_{t-i} \\
u_t=\vartheta u_{t-1}+\gamma W_t +\mu+\epsilon_t\\
E_t[\nu_{t+i}]=E_t[\epsilon_{t+i}]=0 \,\forall i\ge0
\end{gather}




%  \input{betterSimsExample.tex}


The initial matrix$H^{\sharp,0}$, for  the example model
with variable order $\{ \epsilon,\nu,u,w,W\} $ is 
%\Mvariable{Null}
%\begin{figure*}[htbp]
%\begin{center}
%  \leavevmode
  \begin{gather*}
H^{\sharp,0}=  \begin{bmatrix}
0 \anAmp 0 \anAmp 0 \anAmp 0 \anAmp 0 \anAmp 0 \anAmp -1 \anAmp \alpha \anAmp
  1 \anAmp -{\frac{1}{2}} \anAmp 0 \anAmp 0 \anAmp 0 \anAmp 0 \anAmp
  -{\frac{1}{2}}\\
0 \anAmp 0 \anAmp 0 \anAmp -{\frac{1}{2}} \anAmp 0 \anAmp 0 \anAmp 0 \anAmp 0
  \anAmp -{\frac{1}{2}} \anAmp 1 \anAmp 0 \anAmp 0 \anAmp 0 \anAmp 0 \anAmp 0\\
0 \anAmp 0 \anAmp -\vartheta \anAmp 0 \anAmp 0 \anAmp -1 \anAmp 0 \anAmp 1 \anAmp
  0 \anAmp -\gamma \anAmp 0 \anAmp 0 \anAmp 0 \anAmp 0 \anAmp 0\\
0 \anAmp 0 \anAmp 0 \anAmp 0 \anAmp 0 \anAmp 1 \anAmp 0 \anAmp 0 \anAmp 0
  \anAmp 0 \anAmp 0 \anAmp 0 \anAmp 0 \anAmp 0 \anAmp 0\\
0 \anAmp 0 \anAmp 0 \anAmp 0 \anAmp 0 \anAmp 0 \anAmp 1 \anAmp 0 \anAmp 0
  \anAmp 0 \anAmp 0 \anAmp 0 \anAmp 0 \anAmp 0 \anAmp 0
  \end{bmatrix}
  \end{gather*}
%  \caption{$H^{\sharp,0}$ for Example Model}
%  \label{h0Example}
%\end{center}
%\end{figure*}
%\Mvariable{Null}
We would like to express the time $t+1$ variables in terms of the time $t$ and
$t-1$ variables. 
If the sub-matrix corresponding to the $t+1$ variables were non singular we
could immediately write:
\begin{gather*}
\begin{bmatrix}
{{\epsilon}\unds {t + 1}} \\ {{\nu}\unds {t + 1}} \\ {u\unds {t + 1}} \\
  {w\unds {t + 1}} \\ {W\unds {t + 1}}  
\end{bmatrix}
 = (H^{\sharp,0}_\theta)^{-1} 
\begin{bmatrix}
  H^{\sharp,0}_{-1} &   H^{\sharp,0}_0
\end{bmatrix}
\begin{bmatrix}
{{\epsilon}\unds {t - 1}} \\ {{\nu}\unds {t - 1}} \\ {u\unds {t - 1}} \\
  {w\unds {t - 1}} \\ {W\unds {t - 1}}\\
{{\epsilon}\unds t} \\ {{\nu}\unds t} \\ {u\unds t} \\ {w\unds t} \\ {W\unds
  t}
\end{bmatrix}
\end{gather*}


Since $(H^{\sharp,0}_\theta)$ is singular,
 we use equations dated subsequent to the present 
time period to construct a set of linear constraints where the leading block
is non singular.

%\begin{figure*}[htbp]
%\begin{center}
%  \leavevmode
  \begin{gather}
H^{\sharp,\star}=  \begin{bmatrix}
0 \anAmp 0 \anAmp 0 \anAmp 0 \anAmp 0 \anAmp 0 \anAmp 0 \anAmp 0 \anAmp
  -{\frac{1}{2}} \anAmp 0 \anAmp 0 \anAmp 0 \anAmp 0 \anAmp -{\frac{1}{2}}
  \anAmp 1\\ \nonumber
0 \anAmp 0 \anAmp 0 \anAmp 0 \anAmp 0 \anAmp 0 \anAmp 0 \anAmp -\vartheta \anAmp
  0 \anAmp 0 \anAmp -1 \anAmp 0 \anAmp 1 \anAmp 0 \anAmp -\gamma\\ \nonumber
0 \anAmp 0 \anAmp 0 \anAmp 0 \anAmp 0 \anAmp 0 \anAmp 0 \anAmp 0 \anAmp 0
  \anAmp 0 \anAmp 1 \anAmp 0 \anAmp 0 \anAmp 0 \anAmp 0\\ 
0 \anAmp 0 \anAmp 0 \anAmp 0 \anAmp 0 \anAmp 0 \anAmp 0 \anAmp 0 \anAmp 0
  \anAmp 0 \anAmp 0 \anAmp 1 \anAmp 0 \anAmp 0 \anAmp 0\\ \nonumber
0 \anAmp 0 \anAmp 0 \anAmp 0 \anAmp 0 \anAmp 0 \anAmp -1 \anAmp \alpha \anAmp
  1 \anAmp -{\frac{1}{2}} \anAmp 0 \anAmp 0 \anAmp 0 \anAmp 0 \anAmp
  -{\frac{1}{2}} \nonumber
  \end{bmatrix}
  \end{gather}\label{h0StarExample}
%  \caption{$H^{\sharp,\star}$ for Example Model}
%  
%\end{center}
%\end{figure*}

So that


%\begin{figure*}[htbp]
%\begin{center}
%  \leavevmode
  \begin{gather}
A=  \begin{bmatrix}
0 \anAmp 0 \anAmp 0 \anAmp 0 \anAmp 0 \anAmp 1 \anAmp 0 \anAmp 0 \anAmp 0
  \anAmp 0\\
0 \anAmp 0 \anAmp 0 \anAmp 0 \anAmp 0 \anAmp 0 \anAmp 1 \anAmp 0 \anAmp 0
  \anAmp 0\\
0 \anAmp 0 \anAmp 0 \anAmp 0 \anAmp 0 \anAmp 0 \anAmp 0 \anAmp 1 \anAmp 0
  \anAmp 0\\
0 \anAmp 0 \anAmp 0 \anAmp 0 \anAmp 0 \anAmp 0 \anAmp 0 \anAmp 0 \anAmp 1
  \anAmp 0\\
0 \anAmp 0 \anAmp 0 \anAmp 0 \anAmp 0 \anAmp 0 \anAmp 0 \anAmp 0 \anAmp 0
  \anAmp 1\\
0 \anAmp 0 \anAmp 0 \anAmp 0 \anAmp 0 \anAmp 0 \anAmp 0 \anAmp 0 \anAmp 0
  \anAmp 0\\
0 \anAmp 0 \anAmp 0 \anAmp 0 \anAmp 0 \anAmp 0 \anAmp 0 \anAmp 0 \anAmp 0
  \anAmp 0\\
0 \anAmp 0 \anAmp 0 \anAmp 0 \anAmp 0 \anAmp 0 \anAmp -2\gamma \anAmp \rho
  \anAmp 2\gamma \anAmp -\gamma\\
0 \anAmp 0 \anAmp 0 \anAmp 0 \anAmp 0 \anAmp 0 \anAmp -4 \anAmp 4\alpha \anAmp
  3 \anAmp -2\\
0 \anAmp 0 \anAmp 0 \anAmp 0 \anAmp 0 \anAmp 0 \anAmp -2 \anAmp 2\alpha \anAmp
  2 \anAmp -1
  \end{bmatrix}\label{AExample}\intertext{where}
\rho= \theta + 2 \gamma \alpha
  \end{gather}
%  
%\end{center}
%\begin{center}
%  \caption{$H^\sharp_\star$ for Example Model}  
%\end{center}

%\end{figure*}



For the example model
  \begin{gather*}
    Z^\sharp=  \begin{bmatrix}
0 \anAmp 0 \anAmp 0 \anAmp -{\frac{1}{2}} \anAmp 0 \anAmp 0 \anAmp 0 \anAmp 0
  \anAmp -{\frac{1}{2}} \anAmp 1\\
0 \anAmp 0 \anAmp -\vartheta \anAmp 0 \anAmp 0 \anAmp -1 \anAmp 0 \anAmp 1 \anAmp
  0 \anAmp -\gamma\\
0 \anAmp 0 \anAmp 0 \anAmp 0 \anAmp 0 \anAmp 1 \anAmp 0 \anAmp 0 \anAmp 0
  \anAmp 0\\
0 \anAmp 0 \anAmp 0 \anAmp 0 \anAmp 0 \anAmp 0 \anAmp 1 \anAmp 0 \anAmp 0
  \anAmp 0
  \end{bmatrix}
\end{gather*}


  \begin{gather*}
    Q=  \begin{bmatrix}
0 \anAmp 0 \anAmp 0 \anAmp -{\frac{1}{2}} \anAmp 0 \anAmp 0 \anAmp 0 \anAmp 0
  \anAmp -{\frac{1}{2}} \anAmp 1\\
0 \anAmp 0 \anAmp -\vartheta \anAmp 0 \anAmp 0 \anAmp -1 \anAmp 0 \anAmp 1 \anAmp
  0 \anAmp -\gamma\\
0 \anAmp 0 \anAmp 0 \anAmp 0 \anAmp 0 \anAmp 1 \anAmp 0 \anAmp 0 \anAmp 0
  \anAmp 0\\
0 \anAmp 0 \anAmp 0 \anAmp 0 \anAmp 0 \anAmp 0 \anAmp 1 \anAmp 0 \anAmp 0
  \anAmp 0\\
0 \anAmp 0 \anAmp {{\phi} \us 4} \anAmp 0 \anAmp 0 \anAmp 0 \anAmp 2 \anAmp
  {{\phi} \us 5} \anAmp {{\phi} \us 6} \anAmp 1
  \end{bmatrix}
\intertext{
 where  the $\phi$'s are }
\phi_7=1 + \gamma\,{{\phi}_5} + 2\,{{\phi}_6}\\
\phi_6={\frac{-2\,\left( 18\,\alpha\,\gamma + {{\left( -1 + \rho \right) }^2} + 
       \left( 2 + \rho \right) \,{{{{\phi}_2}}^{{\frac{1}{3}}}} + 
       {{{{\phi}_2}}^{{\frac{2}{3}}}} \right) }{{{\phi}_3}}}\\
{\scriptstyle \phi_5=-{\frac{-1 - 3\,\alpha\,\gamma + 3\,\rho + 21\,\alpha\,\gamma\,\rho - 
      3\,{{\rho}^2} + {{\rho}^3} + {{\phi}_1} + 
      {\frac{108\,{{\alpha}^2}\,{{\gamma}^2} + 
          {{\left( -1 + \rho \right) }^4} + 
          6\,\alpha\,\gamma\,\left( -1 + \rho \right) \,
           \left( 1 + 5\,\rho \right)  + 
          2\,\left( -1 + \rho \right) \,{{\phi}_1}}{{{{{\phi}_2}}^
           {{\frac{1}{3}}}}}} + 
      \left( 12\,\alpha\,\gamma + {{\left( -1 + \rho \right) }^2} \right) \,
       {{{{\phi}_2}}^{{\frac{1}{3}}}}}{\gamma\,{{\phi}_3}}}}\\
\phi_4=-{\frac{3\,\alpha\,\gamma\,\left( 216\,{{\alpha}^2}\,{{\gamma}^2} - 
         4\,{{\left( -1 + \rho \right) }^3} + 
         9\,\alpha\,\gamma\,\left( 1 + \left( -14 + \rho \right) \,\rho
            \right)  \right)  + {{{{\phi}_1}}^2}}{\gamma\,
      {{{{\phi}_2}}^{{\frac{2}{3}}}}\,{{\phi}_3}}}\\
\phi_3=18\,\alpha\,\gamma + {{\left( -1 + \rho \right) }^2} + 
  \left( 5 + \rho \right) \,{{{{\phi}_2}}^{{\frac{1}{3}}}} + 
  {{{{\phi}_2}}^{{\frac{2}{3}}}}\\
\phi_2={{\left( -1 + \rho \right) }^3} + 
  27\,\alpha\,\gamma\,\left( 1 + \rho \right)  + 3\,{{\phi}_1}\\
\phi_1={\sqrt{-\left( \alpha\,\gamma\,\left( 216\,{{\alpha}^2}\,{{\gamma}^2} - 
         4\,{{\left( -1 + \rho \right) }^3} + 
         9\,\alpha\,\gamma\,\left( 1 - 14\,\rho + {{\rho}^2} \right)  \right) 
        \right) }}
  \end{gather*}



%\begin{figure*}[htbp]
%\begin{center}
%  \leavevmode
  \begin{gather*}
    B= \frac{1}{\phi_7} \begin{bmatrix}
0 \anAmp 0 \anAmp 0 \anAmp 0 \anAmp 0\\
0 \anAmp 0 \anAmp 0 \anAmp 0 \anAmp 0\\
0 \anAmp 0 \anAmp \vartheta - \gamma{{\phi} \us 4} + 2\vartheta{{\phi} \us 6} \anAmp
  \gamma{{\phi} \us 6} \anAmp 0\\
0 \anAmp 0 \anAmp -2\left( {{\phi} \us 4} + \vartheta{{\phi} \us 5} \right) 
  \anAmp -1 - \gamma{{\phi} \us 5} \anAmp 0\\
0 \anAmp 0 \anAmp -{{\phi} \us 4} - \vartheta{{\phi} \us 5} \anAmp {{\phi} \us 6}
  \anAmp 0
  \end{bmatrix}
  \end{gather*}
%  \caption{B Matrix}
%  \label{bmatrix}
%\end{center}
%\end{figure*}

%\begin{exmpl}
  

%\input{simsUnit.tex}
%\end{exmpl}
% \begin{exmpl}
  

% Collecting the  auxiliary constraints generated by the 
auto regression phase of the algorithm for the example model one has:
%\begin{figure*}[htbp]
%\begin{center}
%  \leavevmode
  \begin{gather*}
    \begin{bmatrix}
Z^\sharp_\ast\\Z^\flat_\ast
    \end{bmatrix}
=  \begin{bmatrix}
0 \anAmp 0 \anAmp 0 \anAmp -{\frac{1}{2}} \anAmp 0 \anAmp 0 \anAmp 0 \anAmp 0
  \anAmp -{\frac{1}{2}} \anAmp 1\\
0 \anAmp 0 \anAmp -\theta \anAmp 0 \anAmp 0 \anAmp -1 \anAmp 0 \anAmp 1 \anAmp
  0 \anAmp -\gamma\\
0 \anAmp 0 \anAmp 0 \anAmp 0 \anAmp 0 \anAmp 1 \anAmp 0 \anAmp 0 \anAmp 0
  \anAmp 0\\
0 \anAmp 0 \anAmp 0 \anAmp 0 \anAmp 0 \anAmp 0 \anAmp 1 \anAmp 0 \anAmp 0
  \anAmp 0\\
\hline
0 \anAmp -1 \anAmp \alpha \anAmp 1 \anAmp -{\frac{1}{2}} \anAmp 0 \anAmp 0
  \anAmp 0 \anAmp 0 \anAmp -{\frac{1}{2}}\\
1 \anAmp 0 \anAmp 0 \anAmp 0 \anAmp 0 \anAmp 0 \anAmp 0 \anAmp 0 \anAmp 0
  \anAmp 0\\
0 \anAmp 1 \anAmp 0 \anAmp 0 \anAmp 0 \anAmp 0 \anAmp 0 \anAmp 0 \anAmp 0
  \anAmp 0
  \end{bmatrix}
  \end{gather*}
%  \caption{Auxiliary Initial Conditions}
%  \label{zrows}
%\end{center}
%\end{figure*}
One can extend  the basis to get a non singular
matrix.\footnote{The top rows are just the row-echelon form of the $Z^{\sharp,\ast},Z^{\flat,\ast}$ vectors.}
%\begin{figure*}[htbp]
%\begin{center}
%  \leavevmode
  \begin{gather*}
    \begin{bmatrix}
Z\\\bar{Z}
    \end{bmatrix}
=  \begin{bmatrix}
1 \anAmp 0 \anAmp 0 \anAmp 0 \anAmp 0 \anAmp 0 \anAmp 0 \anAmp 0 \anAmp 0
  \anAmp 0\\
0 \anAmp 1 \anAmp 0 \anAmp 0 \anAmp 0 \anAmp 0 \anAmp 0 \anAmp 0 \anAmp 0
  \anAmp 0\\
0 \anAmp 0 \anAmp 1 \anAmp 0 \anAmp 0 \anAmp 0 \anAmp 0 \anAmp
  -{\frac{1}{\theta}} \anAmp 0 \anAmp {\frac{\gamma}{\theta}}\\
0 \anAmp 0 \anAmp 0 \anAmp 1 \anAmp 0 \anAmp 0 \anAmp 0 \anAmp 0 \anAmp 1
  \anAmp -2\\
0 \anAmp 0 \anAmp 0 \anAmp 0 \anAmp 1 \anAmp 0 \anAmp 0 \anAmp
  {\frac{-2\alpha}{\theta}} \anAmp 2 \anAmp {\frac{2\alpha\gamma -
  3\theta}{\theta}}\\
0 \anAmp 0 \anAmp 0 \anAmp 0 \anAmp 0 \anAmp 1 \anAmp 0 \anAmp 0 \anAmp 0
  \anAmp 0\\
0 \anAmp 0 \anAmp 0 \anAmp 0 \anAmp 0 \anAmp 0 \anAmp 1 \anAmp 0 \anAmp 0
  \anAmp 0\\
\hline
0 \anAmp 0 \anAmp 0 \anAmp 0 \anAmp 0 \anAmp 0 \anAmp 0 \anAmp 1 \anAmp 0
  \anAmp 0\\
0 \anAmp 0 \anAmp 0 \anAmp 0 \anAmp 0 \anAmp 0 \anAmp 0 \anAmp 0 \anAmp 1
  \anAmp 0\\
0 \anAmp 0 \anAmp 0 \anAmp 0 \anAmp 0 \anAmp 0 \anAmp 0 \anAmp 0 \anAmp 0
  \anAmp 1
  \end{bmatrix}
  \end{gather*}
%  \caption{$Z$}
%  \label{zmat}
%\end{center}
%\end{figure*}
%\begin{figure*}[htbp]
%\begin{center}
%  \leavevmode
  \begin{gather*}
    A=  \begin{bmatrix}
\rho \anAmp 2\gamma \anAmp -\gamma\\
4\alpha \anAmp 3 \anAmp -2\\
2\alpha \anAmp 2 \anAmp -1
  \end{bmatrix}
  \end{gather*}
%  \caption{Minimal Dimension Transition Matrix}
%  \label{lila}
%\end{center}
%\end{figure*}
For this model, we expect one large root. Consequently, $M= 
\begin{bmatrix}
  \lambda_L
\end{bmatrix}$, a $1\times 1$ matrix.
%\begin{figure*}[htbp]
%\begin{center}
%  \leavevmode
  \begin{gather*}
    \Pi=  \begin{bmatrix}
0 \anAmp 0 \anAmp 0 \anAmp 0 \anAmp 0 \anAmp 0 \anAmp -2\gamma\\
0 \anAmp 0 \anAmp 0 \anAmp 0 \anAmp 0 \anAmp 0 \anAmp -4\\
0 \anAmp 0 \anAmp 0 \anAmp 0 \anAmp 0 \anAmp 0 \anAmp -2
  \end{bmatrix}
  \end{gather*}
%  \caption{$\Pi$}
%  \label{pimat}
%\end{center}
%\end{figure*}
%\begin{figure*}[htbp]
%\begin{center}
%  \leavevmode
  \begin{gather*}
    J_0=  \begin{bmatrix}
0 \anAmp 0 \anAmp 0 \anAmp 0 \anAmp 0 \anAmp 1 \anAmp 0\\
0 \anAmp 0 \anAmp 0 \anAmp 0 \anAmp 0 \anAmp 0 \anAmp 1\\
0 \anAmp 0 \anAmp 0 \anAmp 0 \anAmp 0 \anAmp 0 \anAmp 0\\
0 \anAmp 0 \anAmp 0 \anAmp 0 \anAmp 0 \anAmp 0 \anAmp 0\\
0 \anAmp 0 \anAmp 0 \anAmp 0 \anAmp 0 \anAmp 0 \anAmp -8 +
  {\frac{4\alpha\gamma}{\theta}} - {\frac{2\left( 2\alpha\gamma - 3\theta
  \right) }{\theta}}\\
0 \anAmp 0 \anAmp 0 \anAmp 0 \anAmp 0 \anAmp 0 \anAmp 0\\
0 \anAmp 0 \anAmp 0 \anAmp 0 \anAmp 0 \anAmp 0 \anAmp 0
  \end{bmatrix}
  \end{gather*}
%  \caption{$J_0$}
%  \label{j0mat}
%\end{center}
%\end{figure*}
%\begin{figure*}[htbp]
%\begin{center}
%  \leavevmode
  \begin{gather*}
    ((I \otimes M) - (J_0^T \otimes I))^{-1}(\Pi^T \otimes I)=  \begin{bmatrix}
0 \anAmp 0 \anAmp 0\\
0 \anAmp 0 \anAmp 0\\
0 \anAmp 0 \anAmp 0\\
0 \anAmp 0 \anAmp 0\\
0 \anAmp 0 \anAmp 0\\
0 \anAmp 0 \anAmp 0\\
{\frac{-2\gamma}{{{\lambda} \us L}}} \anAmp {\frac{-4}{{{\lambda} \us L}}}
  \anAmp {\frac{-2}{{{\lambda} \us L}}}
  \end{bmatrix}
  \end{gather*}
%  \caption{$((I \otimes M) - (J_0^T \otimes I))^{-1}(\Pi^T \otimes I)$}
%  \label{mapper}
%\end{center}
%\end{figure*}


% \end{exmpl}


% \section{Condition Numbers}
% \label{sec:conditionNo}

% \begin{gather}
%    (H_- + \epsilon_1 F_1) + 
%    (H_0 + \epsilon_2 F_2)B (\epsilon_1,\epsilon_2,\epsilon_3) + 
%    (H_+ + \epsilon_3 F_3)B (\epsilon_1,\epsilon_2,\epsilon_3)^2 = 0\\
% \parallel F_i \parallel_2 =1\\
% \frac{\partial}{\partial \epsilon_i}
% \end{gather}


\setcounter{thrm}{0}
\setcounter{crrlry}{0}









\end{document}
