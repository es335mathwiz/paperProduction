\documentclass[12pt]{article}
\usepackage[margin=1.2in]{geometry}
\usepackage{amsmath}
\usepackage{natbib}
\usepackage{mathtools}
\usepackage{listings}
\usepackage{hyperref}
\author{Alena Audzeyeva and Gary S Anderson}
\title{Support Vector Machine Regression:   Revisiting Some of the Econometrics in ``Predictability of Emering Market Soverign Credit Spreads''}

\newcommand{\iv}{{(i:v)}}
\newcommand{\pihv}{{(-i:h,v)}}
\newcommand{\mihv}{{(-i:h,v)}}



\begin{document}

\maketitle

\begin{abstract}
In \cite{audzeyeva15:predic}, the
  authors have applied standard time series regression analysis
 to construct a  collection of models for forecasting
  credit spreads of soverign debt
  issued by Brazil, Mexico, Phillippines and Turkey for the period
  just before and just after the Lehman crisis.
I have begun an investigation
using Support Vector Machine Regression (SVMR) in place of the
traditional regression techniques they used. 
I have implemented SVMR in a mathematica
program to estimate the family of time series models and I
implemented a cross-validation technique appropriate for treating
time series.\cite{racine00:_consis}
\end{abstract}
\newpage


\section{Support Vector Machine Regression}


Consider models of the form 
\begin{gather}
y_t=E[y|x_t]+ \epsilon_t\\
E[y|x_t]= \sum_{i=1}^L\theta_ik(x_i,x)+b
\end{gather}
where $k(x_i,x)$ is some kernel that can be written as a dot product.\cite{hofmann06kernelreview}.  This ``kernel trick'' provides a very flexible array of
functional forms for fitting the data.
We minimize a loss function of the form
\begin{gather}
L=\frac{1}{2}||w||^2 + C \sum_{i=1}^L (\xi_i+\xi_i^\ast) - \sum_{i=1}^L (\eta_i\xi_i+\eta_i^\ast \xi_i^\ast) -\\
\sum_{i=1}^L \alpha_i(\epsilon_i +\xi_i - y_i + k(x_i,x) +b)- \nonumber\\
\sum_{i=1}^L \alpha_i^\ast(\epsilon_i +\xi_i^\ast + y_i - k(x_i,x) -b)\nonumber
\end{gather}

As a result, analagous to a  hinge loss function, 
errors near the predicted value are ``costless'' while those further away
impact cost in proportion to their linear distance from the prediction.

I have written a set of Mathematica functions that 
construct and solve the related quadratic programming problem.
The code is written to be easily extendable.
I have implemented several widely used kernels including dot product, 
polynomial, radial basis function.
described widely in the literature.\cite{palancz05:suppor,smola2004tutorial,Alpaydin2004}. In addition there are important implementation insights in\cite{platt98sequential}

\section{Performance Metrics}

\cite{e.h01:_applic}

{\large
\begin{tabular}{|l|l|}
  \hline
\multicolumn{1}{|c|}{Metric}&\multicolumn{1}{|c|}{Formula}\\
  \hline
MAE&$\frac{1}{n}\sum_{i=1}^n | a_i-p_i|$\\
  \hline
NMSE&$\frac{1}{\delta^2n}\sum_{i=1}^n (a_i-p_i)^2, \delta^2=\frac{1}{n-1}\sum_{i=1}^n(a_i-\bar{a})^2$\\
  \hline
DS&{\Large $\frac{100}{n \sum_{i=1}^n d_i}$}\\
& $d_i=
    \begin{cases}
      1&(a_i-a_{i-1})(p_i-p_{i-1}) \ge 0\\
      0&\text{otherwise}
    \end{cases}$\\
  \hline
WDS&{\Large $\frac{\sum_{i=1}^n d^\prime_i|a_i-p_i|}{\sum_{i=1}^n d_i|a_i-p_i|}$}\\
&$ d_i=
    \begin{cases}
      1&(a_i-a_{i-1})(p_i-p_{i-1}) \ge 0\\
      0&\text{otherwise}
    \end{cases}$\\
&$ d_i^\prime=
    \begin{cases}
      0&(a_i-a_{i-1})(p_i-p_{i-1}) \ge 0\\
      1&\text{otherwise}
    \end{cases}$\\
  \hline
\end{tabular}
}
\href{http://numericalmethod.com/blog/2011/07/28/on-whites-2000-reality-check/}{reality check}

\section{Cross Validation (CV)}

\cite{bergmeir15:_note} provide a justification for using a simplified version of cross-validation.


\cite{conf/icann/LendasseWV03}  \cite{kreiss12}

dependent case \cite{racine00:_consis}


Time series estimation can present special problems 
when applying cross-validation techniques for  constructing and
assessing models. 
Econometricians typically construct 
models that arguably should include
lags of both the dependent and independent variables.
In order to conduct  useful and statistically valid assessment
with cross validation, 
care must be taken due to the correlation present between a 
single variable and lags of the variables in the training and the test
data sets.
Fortunately,  there are many straigtforward techniques for conducting CV 
with time series.

I have applied hv-block cross-validation\cite{racine00:_consis}.
This technique is a generalization of the well known 
h-block approach. Authors have shown that
the technique can produce statistically consistent results for model and 
variable selection.
As in the usual h-block approach, the 
technique trains a proposed model on a set of 
size $n_c$ test on set of size $n_v$ while 
 maintining ``near'' independence of the 
validation and training sets by h-blocking.



To use the technique:

 \begin{enumerate}
 \item remove $v$ observations on either side of an observation $z_t=\begin{matrix}x_t\\y_t\end{matrix}$ leading to a validation set of size $n_v=2v+1$  then
 \item remove another $h$ observations from either side of the validation set. 
 \item The remaining $n-2v-2h-1$ observations will form the training set.  
 \end{enumerate}

The technique encompasses some widely used CV techniques: 
\begin{enumerate}
\item $h=v=0$ ''leave-one-out''
\item $h=0, v>0$ ''leave-$n_v$ out'' see notes page 4
\item $h=>, v=0$ h block CV
\end{enumerate}


We don't require complete independence to work see \cite{burman94}.
For the technique to work, we need 
$E(\epsilon_t\epsilon_j|X_1, \ldots X_j)\approx0 \, for \, i<j$ otherwise biased
The author's paper argues for 
a rule of thumb for h $\gamma= \frac{h}{n} =0.25$.


Denote training matrix by $Z_\mihv=(Y_\mihv,X_\mihv)$ matrix of removed  is $Z_\pihv=(Y_\pihv,X_\pihv)$ matrix of validation  $Z_\iv=(Y_\iv,X_\iv)$


  \begin{gather}
    CV_{hv}=\frac{1}{n-2v)n_v}\sum_{i-v+1}^{n-v} || Y_\iv-\hat{Y}_{\iv\mihv}||^2\\
    CV_{hv}=\frac{1}{n-2v)n_v}\sum_{i-v+1}^{n-v} || Y_\iv-X_\iv\hat{\beta}_\mihv||^2
  \end{gather}



Let $n_c= floor*(N^\delta)$ for oisutuve degrees if freedom $\frac{\log(p)}{\log(n)}<\delta<1, v=\frac{(n-n^\delta-2h-1)}{2}$


So far  I have
 obtained the data and programs from the authors and can reproduce 
their results, but I have not had 
time to conduct the comparable cross-validation exercise.

In the future I plan to complete the cross-validation,
improve the speed of the code,
implement the code in another language -- probably Julia,
implement asymmetric loss functions,
explore the use of 632+ bootstrap cross validation in conjuction with 
SVMR.

\href{http://www.wildml.com/2015/09/recurrent-neural-networks-tutorial-part-1-introduction-to-rnns/}{recurrent neural networks}

\section{Future Directions}
\label{sec:future-directions}

\cite{journals/ijon/Cao03,conf/icml/TaiebH14}

\href{https://www.researchgate.net/figure/51762586_fig2_Fig-2-Cross-validation-error-estimate-obtained-by-25-fold-cross-validation-with-Ridge}{25 fold CV with hv}
\href{http://www.svms.org/regression/}{references}
\begin{itemize}
\item data visualization
\end{itemize}

\bibliographystyle{plainnat}
\bibliography{../../bibFiles/files}

\end{document}
