\newcommand{\sForSum}{{\nu}}
\newcommand{\rcpC}{{\mathbf{c}}}

\newcommand{\xtVec}{  \begin{bmatrix}
    q_t\\r_{t}\\r_{ut}
  \end{bmatrix}
}
\newcommand{\xtPVec}{  \begin{bmatrix}
    q_{t+1}\\r_{t+1}\\r_{ut+1}
  \end{bmatrix}
}
\newcommand{\xtMVec}{  \begin{bmatrix}
    q_{t-1}\\r_{t-1}\\r_{ut-1}
  \end{bmatrix}
}
\newcommand{\expct}[1]{E_{t} \left [#1 \right ]}
\newcommand{\expc}[1]{\mathcal{E} \left [#1 \right ]}
\newcommand{\expcK}[2]{\mathcal{E}^{#1} \left [#2 \right ]}
\newcommand{\xsln}[1]{\mathbb{X} \left [#1 \right ]}
\newcommand{\xslnK}[2]{\hat{\mathbb{X}}^{#1} \left [#2 \right ]}
\newcommand{\xpth}[2]{\mathfrak{X}_{#1} \left [#2 \right ]}
\newcommand\infNorm[1]{\left\lVert#1\right\rVert_\infty}
\newcommand\twoNorm[1]{\left\lVert#1\right\rVert_2}
%\newcommand{\inorm}[1]{\left\lVert#1\right\rVert_\infty}

% \newcommand{\forPhi}{\begin{bmatrix}
% \psi_\epsilon&\psi_z
% \end{bmatrix}}
% \newcommand{\phiMult}{\phi \psi_\epsilon}
% \newcommand{\bMult}{B x_{-1} + \phiMult}
\newcommand{\phiMultBoth}[1]{
	\phi (\psi_\epsilon \epsilon_t +\psi_z z_0^#1(x_{-1},\epsilon_t))}
\newcommand{\bMultBoth}[1]{B x_{-1} + \phiMultBoth{#1}}


\newcommand{\bForOne}{\bMultBoth{1}
}

% \newcommand{\bForTwo}{\bMultBoth{2}+
% F \phi  \psi_z  
% Z_0^1(x_0^2(x_{-1}))   
% }



\newcommand{\compSlack}{z_0^1(x_{-1},\epsilon_t) \left ( \bar{x} -x\right )=0\\ z_0^1(x_{-1},\epsilon_t)> 0}
% \begin{gather*}
% 0= x_t-(B x_{t-1}+ \phi \psi_\epsilon\epsilon_t + \phi \psi_z 
% \xpt{z_{t}(x_{t-1},\epsilon_t)    } )
% \end{gather*}

\newcommand{\xpt}[1]{#1}


\makeatletter
\@ifundefined{newblock}{%
 \def\newblock{\hskip .11em plus .33em minus .07em} % important line
}
\makeatother


\makeatletter
\pgfplotsset{
    /pgfplots/table/omit header/.style={%
        /pgfplots/table/typeset cell/.append code={%
            \ifnum\c@pgfplotstable@rowindex=-1
                \pgfkeyslet{/pgfplots/table/@cell content}\pgfutil@empty%
            \fi
        }
    }
}
\makeatother
 

\makeatletter
\newcommand*\ExpandableInput[1]{\@@input#1 }
\makeatother


\newcommand{\tpExp}[1]{\phi^e_{#1}(x_{t-1})}
\newcommand{\lRat}[2]{\log \left( \frac{z_{#1}}{z_{#2}}\right)}


\newcommand{\anEdit}[1]
{

{\color{blue}
\begin{quote}
#1  
\end{quote}}

}
