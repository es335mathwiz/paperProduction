\usepackage{amsmath}
\usepackage{pgfplotstable}%for generating latex from csv files
\usepackage{mathtools}
\usepackage{datetime}
\usepackage{hyperref}
\usepackage{graphicx}

\newcommand{\lucaArgs}{{ \left (\lucaX{t-1},\lucaX{t},\epsilon_t  \right )}}
\newcommand{\lucaX}[1]{
  \begin{bmatrix}
    q_{#1}\\r_{#1}\\r^u_{#1}
  \end{bmatrix}}


\newcommand{\detF}[1]{\mathcal{D}^{\{#1\}}}
\newcommand{\stochF}[1]{\mathcal{S}^{\{#1\}}}

\newcommand{\onlEqnLHS}[4]{
  h_i(#1,#2,E_0(#3),\epsilon_#4)}

\newcommand{\onlEqn}[4]{
h^{det}_{io}(#1,#2,\epsilon_#4)+
\sum_{j=1}^{p_i} [h^{det}_{ij}(#1,#2,\epsilon_#4)h^{nondet}_{ij}(E_#4(#3))]
}



\newcommand{\nlEqnLHS}[4]{
  h_\varpi(#1,#2,E_t(#3),\epsilon_#4)}

\newcommand{\nlEqn}[4]{
h_\varpi^{det}(#1,#2,\epsilon_#4)+
H_\varpi^{det}(#1,#2,\epsilon_#4)E_#4(#3)
}

\newcommand{\nlEqnSelLHS}[4]{
 \varpi= m_\varpi(#1,#2,E_t(#3),\epsilon_#4)}

\newcommand{\nlEqnSel}[4]{
\varpi=m_\varpi^{det}(#1,#2,\epsilon_#4)+
M_\varpi^{det}(#1,#2,\epsilon_#4)E_#4(#3)
}


% \newcommand{\nlEqnLHS}[4]{
%   h(#1,#2,E_t(#3),\epsilon_#4)}

% \newcommand{\nlEqn}[4]{
% h^{det}(#1,#2,\epsilon_#4)+
% H^{det}(#1,#2,\epsilon_#4)E_#4(#3)
% }


\newcommand*\xoverline[2][0.75]{%
    \sbox{\myboxA}{$\m@th#2$}%
    \setbox\myboxB\null% Phantom box
    \ht\myboxB=\ht\myboxA%
    \dp\myboxB=\dp\myboxA%
    \wd\myboxB=#1\wd\myboxA% Scale phantom
    \sbox\myboxB{$\m@th\overline{\copy\myboxB}$}%  Overlined phantom
    \setlength\mylenA{\the\wd\myboxA}%   calc width diff
    \addtolength\mylenA{-\the\wd\myboxB}%
    \ifdim\wd\myboxB<\wd\myboxA%
       \rlap{\hskip 0.5\mylenA\usebox\myboxB}{\usebox\myboxA}%
    \else
        \hskip -0.5\mylenA\rlap{\usebox\myboxA}{\hskip 0.5\mylenA\usebox\myboxB}%
    \fi}
\makeatother

\mathtoolsset{showonlyrefs}
\usepackage{pseudocode}

%\newcommand{\detComp}{{\mathbf{detComp}}}
\newcommand{\detComp}[1]{{\detF{K-\nu}( \cdots \detF{K-1}(\detF{K}(#1))))}}
\newcommand{\itSup}[1]{{\{#1\}}}
\newcommand{\initXEN}{(x_,\epsilon_t)}
\newcommand{\initXE}{(x_{-1},\epsilon_0)}
\newcommand{\initX}{(x)}
\newcommand{\xtFuncTI}{\mathcal{X}(x,\epsilon)}
\newcommand{\XtFuncTI}{\mathbf{X}(x)}

\newcommand{\xsubtFunc}[2]{\mathcal{X}_{#1}{#2}}
\newcommand{\xtFunc}[1]{\mathcal{X}{#1}}
\newcommand{\XtFunc}[1]{\mathbf{X}{#1}}
\newcommand{\fMap}[3]{\left ( #2 \rightarrow #1 \rightarrow #3 \right )}

\newcommand{\linMod}{{\mathcal{L}}}
\newcommand{\linModMats}{{\{H,B,\phi,F,\psi_\epsilon,\psi_c,\psi_z\}}}
\newcommand{\discr}[1]{\mathcal{D}^{#1}(x_{t-1},\epsilon_t)}

\newcommand{\XZPair}[1]{(\mathcal{X}^{#1},\mathcal{Z}^{#1})}
\newcommand{\XZPairG}[1]{(\mathcal{X}^{#1}(x^g),\mathcal{Z}^{#1}(x^g))}
\newcommand{\xIter}[2]{\mathcal{X}^{#1}(#2)}
%\newcommand{\zNow}[1]{z^{#1}_0(x_{t-1},\epsilon_t)}
%\newcommand{\ZNow}[3]{\mathcal{Z}^{#1}_{#2}(x_{#3})}
\newcommand{\zNow}[1]{z^{#1}(x_{t-1},\epsilon_t)}
\newcommand{\ZNow}[3]{\mathcal{Z}^{#1}(x_{#3})}

\newcommand{\xNow}[1]{x^{#1}_t(x_{t-1},\epsilon_t)}
\newcommand{\xNowtp}[1]{x^{#1}_{t+1}(x_{t-1},\epsilon_t)}
\newcommand{\XNow}[3]{\mathcal{X}^{#1}_{#2}(x_{#3})}
\newcommand{\tArg}{(x_{t-1},\epsilon_t)}
\newcommand{\tArgZ}{(x_{t-1},\epsilon_t,z_t)}

\newcommand{\modEqns}{\mathcal{M}}

\newcommand{\modEqnsMap}
{
\fMap{\modEqns}{
\begin{bmatrix}
x_{t-1}\\x_{t}\\x_{t+1}\\ \epsilon_t
\end{bmatrix}}{
\begin{bmatrix}
m_e
\end{bmatrix}}
}
\newcommand{\xzFunc}{{\varphi^g}}
\newcommand{\xzFuncMap}
{
\fMap{\xzFunc}{
\begin{bmatrix}
x_{t-1}\\ \epsilon_t\\z_t
\end{bmatrix}}
{\begin{bmatrix}
x_{t-1}\\  x_t\\x_{t+1}\\ \epsilon_t
\end{bmatrix}}
}

\newcommand{\frFunc}{{\Lambda^g}}
\newcommand{\frFuncMap}{
\fMap{ \frFunc}{
 \begin{bmatrix}
 x_{t-1}\\ \epsilon_t
 \end{bmatrix}
}{
 \begin{bmatrix}
 x_t\\z_t
 \end{bmatrix} }
}

\newcommand{\fpFunc}{{\protect \xzSolnFunc{k+1}}}
\newcommand{\fpFuncMap}{
\fMap{ \fpFunc}{
 \begin{bmatrix}
 x_{t-1}\\ \epsilon_t
 \end{bmatrix}
}{
 \begin{bmatrix}
 x_t\\z_t
 \end{bmatrix} }
}

\newcommand{\XZExp}[1]{\Upsilon^{#1}}
\newcommand{\xzSolnFunc}[1]{\Omega^{#1}}
\newcommand{\epsDist}{\mathcal{F}}
\newcommand{\FF}{\aleph}

\newcommand{\xtGuess}{x^g}

\newcommand{\sForSum}{{\nu}}
\newcommand{\rcpC}{{\mathbf{N}}}

\newcommand{\xtVec}{  \begin{bmatrix}
    q_t\\r_{t}\\r_{ut}
  \end{bmatrix}
}
\newcommand{\xtPVec}{  \begin{bmatrix}
    q_{t+1}\\r_{t+1}\\r_{ut+1}
  \end{bmatrix}
}
\newcommand{\xtMVec}{  \begin{bmatrix}
    q_{t-1}\\r_{t-1}\\r_{ut-1}
  \end{bmatrix}
}
\newcommand{\expctEps}[1]{\mathcal{E}_{\epsilon} \left [#1 \right ]}
\newcommand{\expct}[2]{E_{#1} \left [#2 \right ]}
\newcommand{\expc}[1]{\mathcal{E} \left [#1 \right ]}
\newcommand{\expcK}[2]{\mathcal{E}^{#1} \left [#2 \right ]}
\newcommand{\xsln}[1]{\mathbb{X} \left [#1 \right ]}
\newcommand{\xslnK}[2]{\hat{\mathbb{X}}^{#1} \left [#2 \right ]}
\newcommand{\xpth}[2]{\mathfrak{X}_{#1} \left [#2 \right ]}
\newcommand\infNorm[1]{\left\lVert#1\right\rVert_\infty}
\newcommand\twoNorm[1]{\left\lVert#1\right\rVert_2}
%\newcommand{\inorm}[1]{\left\lVert#1\right\rVert_\infty}

% \newcommand{\forPhi}{\begin{bmatrix}
% \psi_\epsilon&\psi_z
% \end{bmatrix}}
% \newcommand{\phiMult}{\phi \psi_\epsilon}
% \newcommand{\bMult}{B x_{-1} + \phiMult}
\newcommand{\phiMultBoth}[1]{
	\phi (\psi_\epsilon \epsilon_t +\psi_z z_0^#1(x_{-1},\epsilon_t))}
\newcommand{\bMultBoth}[1]{B x_{-1} + \phiMultBoth{#1}}


\newcommand{\bForOne}{\bMultBoth{1}
}

% \newcommand{\bForTwo}{\bMultBoth{2}+
% F \phi  \psi_z  
% Z_0^1(x_0^2(x_{-1}))   
% }



\newcommand{\compSlack}{z_0^1(x_{-1},\epsilon_t) \left ( \bar{x} -x\right )=0\\ z_0^1(x_{-1},\epsilon_t)> 0}
% \begin{gather*}
% 0= x_t-(B x_{t-1}+ \phi \psi_\epsilon\epsilon_t + \phi \psi_z 
% \xpt{z_{t}(x_{t-1},\epsilon_t)    } )
% \end{gather*}

\newcommand{\xpt}[1]{#1}


\makeatletter
\@ifundefined{newblock}{%
 \def\newblock{\hskip .11em plus .33em minus .07em} % important line
}
\makeatother


\makeatletter
\pgfplotsset{
    /pgfplots/table/omit header/.style={%
        /pgfplots/table/typeset cell/.append code={%
            \ifnum\c@pgfplotstable@rowindex=-1
                \pgfkeyslet{/pgfplots/table/@cell content}\pgfutil@empty%
            \fi
        }
    }
}
\makeatother
 

\makeatletter
\newcommand*\ExpandableInput[1]{\@@input#1 }
\makeatother


\newcommand{\tpExp}[1]{\phi^e_{#1}(x_{t-1})}
\newcommand{\lRat}[2]{\log \left( \frac{z_{#1}}{z_{#2}}\right)}


\newcommand{\anEdit}[1]
{

{\color{blue}
\begin{quote}
#1  
\end{quote}}

}
