\documentclass{beamer}
\usepackage{endnotes}
\usepackage{moreverb}
\usepackage{rotating}
\usepackage{siunitx}
\usepackage[authoryear]{natbib}
\usepackage{datetime}
\usepackage{hyperref}
\usepackage{amsmath}
\usepackage{graphicx}


\title{``A Solution Strategy for Occasionally Binding Constraints in Otherwise
Linear Rational Expectations Models''}
\date{\today}
\author{Gary S. Anderson}



\makeatletter
\newcommand*\ExpandableInput[1]{\@@input#1 }
\makeatother


\newcommand{\xtFuncTI}{\mathcal{X}(x,\epsilon)}
\newcommand{\XtFuncTI}{\mathbf{X}(x)}

\newcommand{\xtFunc}[1]{\mathcal{X}{#1}}
\newcommand{\XtFunc}[1]{\mathbf{X}{#1}}



\newcommand{\discr}[1]{\mathcal{D}^{#1}(x_{t-1},\epsilon_t)}

\newcommand{\XZPair}[1]{(\mathcal{X}^{#1},\mathcal{Z}^{#1})}
\newcommand{\XZPairG}[1]{(\mathcal{X}^{#1}(x^g),\mathcal{Z}^{#1}(x^g))}
\newcommand{\xIter}[2]{\mathcal{X}^{#1}(#2)}
%\newcommand{\zNow}[1]{z^{#1}_0(x_{t-1},\epsilon_t)}
%\newcommand{\ZNow}[3]{\mathcal{Z}^{#1}_{#2}(x_{#3})}
\newcommand{\zNow}[1]{z^{#1}(x_{t-1},\epsilon_t)}
\newcommand{\ZNow}[3]{\mathcal{Z}^{#1}(x_{#3})}

\newcommand{\xNow}[1]{x^{#1}_t(x_{t-1},\epsilon_t)}
\newcommand{\xNowtp}[1]{x^{#1}_{t+1}(x_{t-1},\epsilon_t)}
\newcommand{\XNow}[3]{\mathcal{X}^{#1}_{#2}(x_{#3})}








\newcommand{\sForSum}{{\nu}}
\newcommand{\rcpC}{{\mathbf{c}}}

\newcommand{\xtVec}{  \begin{bmatrix}
    q_t\\r_{t}\\r_{ut}
  \end{bmatrix}
}
\newcommand{\xtPVec}{  \begin{bmatrix}
    q_{t+1}\\r_{t+1}\\r_{ut+1}
  \end{bmatrix}
}
\newcommand{\xtMVec}{  \begin{bmatrix}
    q_{t-1}\\r_{t-1}\\r_{ut-1}
  \end{bmatrix}
}
\newcommand{\expctEps}[1]{\mathcal{E}_{\epsilon} \left [#1 \right ]}
\newcommand{\expct}[2]{E_{#1} \left [#2 \right ]}
\newcommand{\expc}[1]{\mathcal{E} \left [#1 \right ]}
\newcommand{\expcK}[2]{\mathcal{E}^{#1} \left [#2 \right ]}
\newcommand{\xsln}[1]{\mathbb{X} \left [#1 \right ]}
\newcommand{\xslnK}[2]{\hat{\mathbb{X}}^{#1} \left [#2 \right ]}
\newcommand{\xpth}[2]{\mathfrak{X}_{#1} \left [#2 \right ]}
\newcommand\infNorm[1]{\left\lVert#1\right\rVert_\infty}
\newcommand\twoNorm[1]{\left\lVert#1\right\rVert_2}
%\newcommand{\inorm}[1]{\left\lVert#1\right\rVert_\infty}

% \newcommand{\forPhi}{\begin{bmatrix}
% \psi_\epsilon&\psi_z
% \end{bmatrix}}
% \newcommand{\phiMult}{\phi \psi_\epsilon}
% \newcommand{\bMult}{B x_{-1} + \phiMult}
\newcommand{\phiMultBoth}[1]{
	\phi (\psi_\epsilon \epsilon_t +\psi_z z_0^#1(x_{-1},\epsilon_t))}
\newcommand{\bMultBoth}[1]{B x_{-1} + \phiMultBoth{#1}}


\newcommand{\bForOne}{\bMultBoth{1}
}

% \newcommand{\bForTwo}{\bMultBoth{2}+
% F \phi  \psi_z  
% Z_0^1(x_0^2(x_{-1}))   
% }



\newcommand{\compSlack}{z_0^1(x_{-1},\epsilon_t) \left ( \bar{x} -x\right )=0\\ z_0^1(x_{-1},\epsilon_t)> 0}
% \begin{gather*}
% 0= x_t-(B x_{t-1}+ \phi \psi_\epsilon\epsilon_t + \phi \psi_z 
% \xpt{z_{t}(x_{t-1},\epsilon_t)    } )
% \end{gather*}

\newcommand{\xpt}[1]{#1}


\makeatletter
\@ifundefined{newblock}{%
 \def\newblock{\hskip .11em plus .33em minus .07em} % important line
}
\makeatother


\makeatletter
\pgfplotsset{
    /pgfplots/table/omit header/.style={%
        /pgfplots/table/typeset cell/.append code={%
            \ifnum\c@pgfplotstable@rowindex=-1
                \pgfkeyslet{/pgfplots/table/@cell content}\pgfutil@empty%
            \fi
        }
    }
}
\makeatother
 

\makeatletter
\newcommand*\ExpandableInput[1]{\@@input#1 }
\makeatother


\newcommand{\tpExp}[1]{\phi^e_{#1}(x_{t-1})}
\newcommand{\lRat}[2]{\log \left( \frac{z_{#1}}{z_{#2}}\right)}


\newcommand{\anEdit}[1]
{

{\color{blue}
\begin{quote}
#1  
\end{quote}}

}


\begin{document}


\frame{\titlepage}
\frame{\frametitle{Table of Contents}\tableofcontents}

\section{Problem Statement and Solution}
\label{sec:probl-stat-solut}

\newcommand{\xpt}[1]{#1}

   \begin{frame}
     \frametitle{Problem Statement}
    

 Consider systems of the form
 \begin{gather}
% \sum_{i=-\tau}^\theta H_i x_{t+i} 
H_{-1} x_{t-1} + H_0 x_t + H_1 \expct{x_{t+1}}=
 \psi_\epsilon 
 \epsilon_t 
   \,\,\forall t.  \label{myEqn}
\intertext{We wish to accommodate inequality constraints of the form}
  x_t \ge \bar{x}    \,\,\forall t.
\end{gather}


 % \begin{bmatrix}
 % \psi_\epsilon & \psi_z  
 % \end{bmatrix}
 %   \begin{bmatrix}
 % \epsilon_t \\\xpt{z_{t}(x_{t-1},\epsilon_t) }   
 %   \end{bmatrix}

We will determine a sequence of functions $\xpt{z_{t+s}(x_{t+s-1},\epsilon_t)} $ that 
 will allow us to  impose the occasionally binding constraints over successively longer time periods.
These solutions will take the form\citep{anderson10}
\begin{gather*}
   x_t=B x_{t-1}+ \phi \psi_\epsilon\epsilon_t + \sum_{s=0}^\infty F^s \phi \psi_z
 \xpt{z_{t+s}(x_{t+s-1},\epsilon_t)    }  
\end{gather*}


   \end{frame}
\section{A Simple Example}

   \begin{frame}
     \frametitle{A Simple Example}
     
For example, consider the simple model


\begin{gather*}
q_{t} +\beta_p(1 - \rho_p)q_{t + 1} + \rho_pq_{t - 1} - \sigma_pr_{t} +
     r_{ut}=0\\
 r_{t} = \max (\bar{r}, \phi_pq_{t}) \\
 r_{ut} = \rho_{ru} r_{ut - 1} + \eta \epsilon_{y,t}
\end{gather*}

Then 
\newcommand{\xtmVec}{  \begin{bmatrix}
    q_t\\r_{t}\\r_{ut}
  \end{bmatrix}
}


\begin{gather*}
  x_t=\xtmVec
\end{gather*}

Some typical values of the parameters are:

\begin{gather*}
  \beta_p = 0.99, \phi_p = 1, 
\rho_p = 0.5, \sigma_p = 1, \rho_{ru} = 0.5,
  \bar{r} = 0.02 \\
% q^L = -.5, q^H = .5, 
%   r_u^L = -4 \sigma_u/(1 - \rho_{ru}), r_u^H=  4\sigma_u/(1 - \rho_{ru}),
%    I_N = {10}, \sigma_u = 0.02,\\
%  \mu = {0},
%    \eta = 1
\end{gather*}



\begin{gather*}
  H= \vcenter{\hbox{\includegraphics{/msu/home/m1gsa00/git/ProjectionMethodTools/ProjectionMethodToolsJava/code/prettyHmat.pdf}}}\\
\psi_z=   \vcenter{\hbox{\includegraphics{/msu/home/m1gsa00/git/ProjectionMethodTools/ProjectionMethodToolsJava/code/prettyPsiZ.pdf}}}\\
\psi_\epsilon=   \vcenter{\hbox{\includegraphics{/msu/home/m1gsa00/git/ProjectionMethodTools/ProjectionMethodToolsJava/code/prettyPsiEps.pdf}}}\\
\end{gather*}




 \begin{gather*}
B=   \vcenter{\hbox{\includegraphics{/msu/home/m1gsa00/git/ProjectionMethodTools/ProjectionMethodToolsJava/code/prettyBmat.pdf}}}\\
\phi=   \vcenter{\hbox{\includegraphics{/msu/home/m1gsa00/git/ProjectionMethodTools/ProjectionMethodToolsJava/code/prettyPhimat.pdf}}}\\
F=   \vcenter{\hbox{\includegraphics{/msu/home/m1gsa00/git/ProjectionMethodTools/ProjectionMethodToolsJava/code/prettyFmat.pdf}}}
 \end{gather*}



   \end{frame}


   \begin{frame}
     \frametitle{The Solution Matrices}
     
{\tiny
\begin{gather*}
\intertext{Algorithm Inputs:}
% H=
% \begin{bmatrix}
% H_{-1} &H_0&H_{t+1}  
% \end{bmatrix}\\
  H= \vcenter{\hbox{\includegraphics{../../../ProjectionMethodTools/ProjectionMethodToolsJava/code/prettyHmat.pdf}}}\\
\psi_z=   \vcenter{\hbox{\includegraphics{../../../ProjectionMethodTools/ProjectionMethodToolsJava/code/prettyPsiZ.pdf}}}
\psi_\epsilon=   \vcenter{\hbox{\includegraphics{../../../ProjectionMethodTools/ProjectionMethodToolsJava/code/prettyPsiEps.pdf}}}\\
\intertext{Algorithm Outputs:}
B=   \vcenter{\hbox{\includegraphics{../../../ProjectionMethodTools/ProjectionMethodToolsJava/code/prettyBmat.pdf}}}\\
\phi=   \vcenter{\hbox{\includegraphics{../../../ProjectionMethodTools/ProjectionMethodToolsJava/code/prettyPhimat.pdf}}}\\
F=   \vcenter{\hbox{\includegraphics{../../../ProjectionMethodTools/ProjectionMethodToolsJava/code/prettyFmat.pdf}}}
 \end{gather*}

}

   \end{frame}

% \newcommand{\forPhi}{\begin{bmatrix}
% \psi_\epsilon&\psi_z
% \end{bmatrix}}
% \newcommand{\phiMult}{\phi \psi_\epsilon}
% \newcommand{\bMult}{B x_{-1} + \phiMult}
 \newcommand{\phiMultBoth}[1]{
 \phi (\psi_\epsilon \epsilon_0 +\psi_z z_0^#1(x_{-1},\epsilon_0))}
 \newcommand{\bMultBoth}[1]{B x_{-1} + \phiMultBoth{#1}}


 \newcommand{\bForOne}{\bMultBoth{1}
 }

% \newcommand{\bForTwo}{\bMultBoth{2}+
% F \phi  \psi_z  
% Z_0^1(x_0^2(x_{-1}))   
% }



\newcommand{\compSlack}{z_0^1(x_{-1},\epsilon_0) \left ( \bar{r} -\phi_p q_t\right )=0\\ z_0^1(x_{-1},\epsilon_0)> 0}
% \begin{gather*}
% 0= x_t-(B x_{t-1}+ \phi \psi_\epsilon\epsilon_t + \phi \psi_z 
% \xpt{z_{t}(x_{t-1},\epsilon_t)    } )
% \end{gather*}



    \begin{frame}


      \frametitle{Honoring the Constraint For One Period}


 We begin by solving a system where the constraints are honored only at time
  $t=0$.  
 Using equation \ref{myEqn}, we can construct the system which imposes the 
 constraint for time 0 alone.
To do this we set $\xpt{z_{t+s}}=0\, \forall s>0$
and now have
\begin{gather*}
x_0^1(x_{-1},\epsilon_0)-
\left ( \bForOne \right )=0\intertext{ with the complimentary slackness conditions}
\compSlack
\end{gather*}

     
    \end{frame}
    \begin{frame}
      \frametitle{For One Period (continued)}
      

We set 
\begin{gather}\label{firstIneq}
z_0^1(x_{-1},\epsilon_0)=
\begin{cases}
0&  B x_{-1} +\epsilon_0 \ge \bar{x}  \\
\bar{x}-(B x_{-1}+\epsilon_0) & B x_{-1}+\epsilon_0 < \bar{x}  
\end{cases}
\end{gather}


    \end{frame}


    \begin{frame}
      \frametitle{For the Simple Example}
     
{\tiny
       \begin{gather*}
 z_0^1(\xtMVec,\epsilon_0)=
 \begin{cases}
0&\eta >0.032399 \\
0.043056 - 1.328904 \epsilon_0 - 0.576390 q_{t-1} - 
      0.664452 r_{ut-1}& \eta \le 0.032399  
 \end{cases}\intertext{where}
\eta=\epsilon_0 + 0.433734 q_{t-1} + 0.5 r_{ut-1}\\
\intertext{and}
 q_t(\xtMVec,\epsilon_0)=
 \begin{cases}
0.617296 \epsilon_0 + 0.267742 q_{t-1} + 0.308648 r_{ut-1}&\eta >0.032399 \\
-0.0230556 + 1.3289 \epsilon_0 + 
        0.57639 q_{t-1} + 0.664452 r_{ut-1}&\eta \le 0.032399 \\
 \end{cases}\\
 r_t(\xtMVec,\epsilon_0)=
 \begin{cases}
0.617296 \epsilon_0 + 0.267742 q_{t-1} + 0.308648 r_{ut-1}&\eta >0.032399 \\
0.02&\eta \le 0.032399 \\
 \end{cases}\\
       \end{gather*}
}
    \end{frame}


    \begin{frame}
      \frametitle{Two Period Perfect Foresight Solution}
We can apply equation \ref{myEqn} to compute the $\xtVec$ values along a solution path where all future shocks are zero:
{\tiny
      \begin{gather*}
           \left(
   \begin{array}{c}
    q_{t-1} \\
    r_{t-1} \\
    ru_{t-1} \\
 0.617296 \epsilon_0+0.267742 _{t-1}+0.308648 ru_{t-1}-0.141938
       z_0^1(\xtVec,\epsilon_0)-0.535485 z_0^2(\xtMVec,\epsilon_0) \\
    0.617296 \epsilon_0+0.267742 q_{t-1}+0.308648 ru_{t-1}-0.141938
      z_0^1(\xtVec,\epsilon_0)+0.464515 z_0^2(\xtMVec,\epsilon_0) \\
    \epsilon_0+0.5 ru_{t-1} \\
    0.473924 \epsilon_0+0.0716859 q_{t-1}+0.236962 ru_{t-1}-0.573487
      z_0^1(\xtVec,\epsilon_0)-0.143372 z_0^2(\xtMVec,\epsilon_0) \\
    0.473924 \epsilon_0+0.0716859 q_{t-1}+0.236962 ru_{t-1}+0.426513
      z_0^1(\xtVec,\epsilon_0)-0.143372 z_0^2(\xtMVec,\epsilon_0) \\
    0.5 \epsilon_0+0.25 ru_{t-1} 
   \end{array}
   \right)
      \end{gather*}
}

    \end{frame}

    \begin{frame}
\frametitle{Exact Two Period Perfect Foresight Solutions  Available}

We will need the function {\small $z_0^2(\xtMVec,\epsilon_0)$ }that guarantees:
{\tiny 
\begin{gather*}
    0.617296 \epsilon_0+0.267742 q_{t-1}+0.308648 ru_{t-1}-0.141938
      z_0^1(\xtVec,\epsilon_0)+0.464515 z_0^2(\xtMVec,\epsilon_0) \ge 0.02\\  
    0.473924 \epsilon_0+0.0716859 q_{t-1}+0.236962 ru_{t-1}+0.426513
      z_0^1(\xtVec,\epsilon_0)-0.143372 z_0^2(\xtMVec,\epsilon_0)  \ge 0.02 \\
\end{gather*}
}
\begin{itemize}
\item {\small $z_0^1(\xtVec,\epsilon_0)$ } is a known function
\item Even though $q_t$ depends {\small $z_0^1(\xtVec,\epsilon_0)$ } Mathematica requires about 6 seconds to compute this solution for arbitrary initial conditions
\end{itemize}
    \end{frame}

    \begin{frame}
      \frametitle{Two Period Rational Expectation Solution}
We can apply equation \ref{myEqn} to compute the $\xtVec$ values along a solution path where only one future shock is non zero:
{\tiny
      \begin{gather*}
           \left(
   \begin{array}{c}
    q_{t-1} \\
    r_{t-1} \\
    ru_{t-1} \\
  0.163624 \epsilon_{1}+0.617296 \epsilon_0+0.267742 _{t-1}+0.308648 ru_{t-1}-0.141938
       z_0^1(\xtVec,\epsilon_0)-0.535485 z_0^2(\xtMVec,\epsilon_0) \\
    0.163624 \epsilon_{1}+0.617296 \epsilon_0+0.267742 q_{t-1}+0.308648 ru_{t-1}-0.141938
      z_0^1(\xtVec,\epsilon_0)+0.464515 z_0^2(\xtMVec,\epsilon_0) \\
    \epsilon_0+0.5 ru_{t-1} \\
    0.661105 \epsilon_{1}+0.473924 \epsilon_0+0.0716859 q_{t-1}+0.236962 ru_{t-1}-0.573487
      z_0^1(\xtVec,\epsilon_0)-0.143372 z_0^2(\xtMVec,\epsilon_0) \\
   0.661105 \epsilon_{1}+ 0.473924 \epsilon_0+0.0716859 q_{t-1}+0.236962 ru_{t-1}+0.426513
      z_0^1(\xtVec,\epsilon_0)-0.143372 z_0^2(\xtMVec,\epsilon_0) \\
    0.5 \epsilon_0+0.25 ru_{t-1} 
   \end{array}
   \right)
      \end{gather*}
}

    \end{frame}

    \begin{frame}
\frametitle{Exact Two Period Rational Expectations Solutions May Be Available}

We will need the function {\small $z_0^2(\xtMVec,\epsilon_0)$ }that guarantees:
{\tiny 
\begin{gather*}
 0.163624 \epsilon_{1}+  0.617296 \epsilon_0+0.267742 q_{t-1}+0.308648 ru_{t-1}-0.141938
      z_0^1(\xtVec,\epsilon_0)+0.464515 z_0^2(\xtMVec,\epsilon_0) \ge 0.02\\  
    0.661105 \epsilon_{1}+    0.473924 \epsilon_0+0.0716859 q_{t-1}+0.236962 ru_{t-1}+0.426513
      z_0^1(\xtVec,\epsilon_0)-0.143372 z_0^2(\xtMVec,\epsilon_0)  \ge 0.02 \\
\end{gather*}
}
\begin{itemize}
\item {\small $z_0^1(\xtVec,\epsilon_0)$ } is a known function
\item Even though $q_t$ depends {\small $z_0^1(\xtVec,\epsilon_0)$ } Mathematica requires about 6 seconds to compute this solution for arbitrary initial conditions
\end{itemize}
    \end{frame}
    \begin{frame}
      \frametitle{Computations Will Require Approximation and
        Recursion}
      \begin{itemize}
      \item Exact perfect foresight solutions for 3 periods require more than 900 seconds
      \item A Recursive Formulation Precludes Recomputing Approximation Shorter Each Horizons
      \item Low Order Projection and Interpolation Approximants are Inexpensive to Compute
      \item It appears that Interpolation Works Better for this example
      \end{itemize}
    \end{frame}
\input{../../../ProjectionMethodTools/ProjectionMethodToolsJava/code/interpOneCalcs}


   \begin{frame}
     \frametitle{Computation Time}


{\tiny
     \begin{minipage}{1.0\linewidth}
     \begin{tabular}{|l|l|
S[table-format=3.6]|
S[table-format=3.6]|
S[table-format=3.6]|
S[table-format=3.6]|
S[table-format=3.6]|
}
\hline
\multicolumn{7}{|c|}{Time in seconds for evaluation at a single initial state ( except where noted )\footnote{See the endnotes Section~\ref{sec:endnotes}.}}\\
\hline
\multicolumn{1}{|c|}{Periods}&
\multicolumn{1}{|c|}{Recursive?}&
\multicolumn{1}{|c|}{Analytic First}&
\multicolumn{1}{|c|}{Analytic Second}&
&&\\
% \multicolumn{1}{|c|}{Semi-Analytic}&
% \multicolumn{1}{|c|}{Projection}&
% \multicolumn{1}{|c|}{Interpolation}\\
\hline
\hline
1&NA\endnote{Algebraic expression for all initial states.\label{allStates}}&\ExpandableInput{/msu/home/m1gsa00/git/ProjectionMethodTools/ProjectionMethodToolsJava/code/symb01Secs}&&&\\
\hline
\hline
2&No\textsuperscript{\ref{allStates}}&\ExpandableInput{/msu/home/m1gsa00/git/ProjectionMethodTools/ProjectionMethodToolsJava/code/symb02Secs}&&&\\
\hline
2&Yes\textsuperscript{\ref{allStates}}&\ExpandableInput{/msu/home/m1gsa00/git/ProjectionMethodTools/ProjectionMethodToolsJava/code/symb02RecSecs}&&&\\
\hline
2&Yes&\ExpandableInput{/msu/home/m1gsa00/git/ProjectionMethodTools/ProjectionMethodToolsJava/code/symb02ValsRecSecs}&&&\\
\hline
2&Expectation non recursive&\ExpandableInput{/msu/home/m1gsa00/git/ProjectionMethodTools/ProjectionMethodToolsJava/code/symb02ValsExpSecs}&&&\\
\hline
\hline
3&No\textsuperscript{\ref{allStates}}&\ExpandableInput{/msu/home/m1gsa00/git/ProjectionMethodTools/ProjectionMethodToolsJava/code/symb03Secs}&&&\\
\hline
3&Yes\textsuperscript{\ref{allStates}}&\ExpandableInput{/msu/home/m1gsa00/git/ProjectionMethodTools/ProjectionMethodToolsJava/code/symb03RecSecs}&&&\\
\hline
3&No&\ExpandableInput{/msu/home/m1gsa00/git/ProjectionMethodTools/ProjectionMethodToolsJava/code/symb03ValsSecs}&&&\\
\hline
3&Yes&\ExpandableInput{/msu/home/m1gsa00/git/ProjectionMethodTools/ProjectionMethodToolsJava/code/symb03ValsRecSecs}&&&\\
\hline
3&Expectation non recursive&\ExpandableInput{/msu/home/m1gsa00/git/ProjectionMethodTools/ProjectionMethodToolsJava/code/symb03ValsExpSecs}&&&\\
\hline
\hline
4&No&\ExpandableInput{/msu/home/m1gsa00/git/ProjectionMethodTools/ProjectionMethodToolsJava/code/symb04ValsSecs}&&&\\
\hline
4&Yes&\ExpandableInput{/msu/home/m1gsa00/git/ProjectionMethodTools/ProjectionMethodToolsJava/code/symb04ValsRecSecs}&&&\\
\hline
\hline
5&No&\ExpandableInput{/msu/home/m1gsa00/git/ProjectionMethodTools/ProjectionMethodToolsJava/code/symb05ValsSecs}&&&\\
\hline
5&Yes&\ExpandableInput{/msu/home/m1gsa00/git/ProjectionMethodTools/ProjectionMethodToolsJava/code/symb05ValsRecSecs}&&&\\
\hline
\hline
6&No&\ExpandableInput{/msu/home/m1gsa00/git/ProjectionMethodTools/ProjectionMethodToolsJava/code/symb06ValsSecs}&&&\\
\hline
\hline
7&No&\ExpandableInput{/msu/home/m1gsa00/git/ProjectionMethodTools/ProjectionMethodToolsJava/code/symb07ValsSecs}&&&\\
\hline
\hline
\end{tabular}

     \end{minipage}
}
   \end{frame}

   \begin{frame}
     \frametitle{Clean up}
     \begin{itemize}
\item Redo numbering for zzz
\item memoize
\item cleanup namespace
\item limiting cases, $\bar{r}$ never, always  stochastic 0 to large
\item compare interpolation at a fixed set of points
\item measure accuracy of components using exact, interpolation, projection and by applying the original system
\item Better computationally to get close to expectations solution then iterate for fixed point of function?
     \end{itemize}
   \end{frame}


   \begin{frame}
     \frametitle{Future Directions}
     \begin{itemize}
     \item Should also consider other behavior besides absorbing barriers change in parameters
     \item  \begin{gather*}
 M x_t \ge m \\
 \end{gather*} 
\item  Smolyak Points \\
\item Parallelize
\item compute border and use piecewise
\item strategically place points for accuracy
\item use plot to get better points
\item uniqueness via continuity in homotopy of variance and constraints
\item possible to fool algorithm
\item compile
\item generic eval of complimentary slackness
\item derivatives
\item condition numbers
\item tracking border shifts
\item projection versus interpolation
\item interpretation as ``perturbation'' or other approximation method
\item how general
     \end{itemize}
   \end{frame}


   \begin{frame}
 \bibliographystyle{plainnat}
 \bibliography{anderson,files}
     
   \end{frame}


     \section{Endnotes}
{\tiny
     \theendnotes
}




%\appendix

\section{Code}
\begin{frame}
\label{sec:code}
\end{frame}
\subsection{symb01.m}

\begin{frame}
\label{sec:symb01.m}
\end{frame}



\newpage
\subsection{symb02.m}

\begin{frame}
\label{sec:symb02.m}  
\end{frame}


\newpage
\subsection{symb02Rec.m}
\begin{frame}
\label{sec:symb02Rec.m}  
\end{frame}



\newpage
\subsection{symb02ValsExp.m}
\begin{frame}
\label{sec:symb02ValsExp.m}
\end{frame}





\end{document}
