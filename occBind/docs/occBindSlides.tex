\documentclass{beamer}
\usepackage{longtable}
\usepackage{txfonts}
\usepackage{eufrak}
\usepackage{endnotes}
\usepackage{moreverb}
\usepackage{rotating}
\usepackage{siunitx}
\usepackage[authoryear]{natbib}
\usepackage{pseudocode}
\usepackage{datetime}\usepackage{hyperref}
\usepackage{amsmath}
\usepackage{graphicx}
\usepackage{multirow}
\usepackage{pgfplotstable}%for generating latex from csv files
\usepackage{booktabs}

\title{``A Solution Strategy for Occasionally Binding Constraints in Otherwise
Linear Rational Expectations Models''}
\date{\today}
\author{Gary S. Anderson}



\makeatletter
\newcommand*\ExpandableInput[1]{\@@input#1 }
\makeatother


\newcommand{\xtFuncTI}{\mathcal{X}(x,\epsilon)}
\newcommand{\XtFuncTI}{\mathbf{X}(x)}

\newcommand{\xtFunc}[1]{\mathcal{X}{#1}}
\newcommand{\XtFunc}[1]{\mathbf{X}{#1}}



\newcommand{\discr}[1]{\mathcal{D}^{#1}(x_{t-1},\epsilon_t)}

\newcommand{\XZPair}[1]{(\mathcal{X}^{#1},\mathcal{Z}^{#1})}
\newcommand{\XZPairG}[1]{(\mathcal{X}^{#1}(x^g),\mathcal{Z}^{#1}(x^g))}
\newcommand{\xIter}[2]{\mathcal{X}^{#1}(#2)}
%\newcommand{\zNow}[1]{z^{#1}_0(x_{t-1},\epsilon_t)}
%\newcommand{\ZNow}[3]{\mathcal{Z}^{#1}_{#2}(x_{#3})}
\newcommand{\zNow}[1]{z^{#1}(x_{t-1},\epsilon_t)}
\newcommand{\ZNow}[3]{\mathcal{Z}^{#1}(x_{#3})}

\newcommand{\xNow}[1]{x^{#1}_t(x_{t-1},\epsilon_t)}
\newcommand{\xNowtp}[1]{x^{#1}_{t+1}(x_{t-1},\epsilon_t)}
\newcommand{\XNow}[3]{\mathcal{X}^{#1}_{#2}(x_{#3})}








\newcommand{\sForSum}{{\nu}}
\newcommand{\rcpC}{{\mathbf{c}}}

\newcommand{\xtVec}{  \begin{bmatrix}
    q_t\\r_{t}\\r_{ut}
  \end{bmatrix}
}
\newcommand{\xtPVec}{  \begin{bmatrix}
    q_{t+1}\\r_{t+1}\\r_{ut+1}
  \end{bmatrix}
}
\newcommand{\xtMVec}{  \begin{bmatrix}
    q_{t-1}\\r_{t-1}\\r_{ut-1}
  \end{bmatrix}
}
\newcommand{\expctEps}[1]{\mathcal{E}_{\epsilon} \left [#1 \right ]}
\newcommand{\expct}[2]{E_{#1} \left [#2 \right ]}
\newcommand{\expc}[1]{\mathcal{E} \left [#1 \right ]}
\newcommand{\expcK}[2]{\mathcal{E}^{#1} \left [#2 \right ]}
\newcommand{\xsln}[1]{\mathbb{X} \left [#1 \right ]}
\newcommand{\xslnK}[2]{\hat{\mathbb{X}}^{#1} \left [#2 \right ]}
\newcommand{\xpth}[2]{\mathfrak{X}_{#1} \left [#2 \right ]}
\newcommand\infNorm[1]{\left\lVert#1\right\rVert_\infty}
\newcommand\twoNorm[1]{\left\lVert#1\right\rVert_2}
%\newcommand{\inorm}[1]{\left\lVert#1\right\rVert_\infty}

% \newcommand{\forPhi}{\begin{bmatrix}
% \psi_\epsilon&\psi_z
% \end{bmatrix}}
% \newcommand{\phiMult}{\phi \psi_\epsilon}
% \newcommand{\bMult}{B x_{-1} + \phiMult}
\newcommand{\phiMultBoth}[1]{
	\phi (\psi_\epsilon \epsilon_t +\psi_z z_0^#1(x_{-1},\epsilon_t))}
\newcommand{\bMultBoth}[1]{B x_{-1} + \phiMultBoth{#1}}


\newcommand{\bForOne}{\bMultBoth{1}
}

% \newcommand{\bForTwo}{\bMultBoth{2}+
% F \phi  \psi_z  
% Z_0^1(x_0^2(x_{-1}))   
% }



\newcommand{\compSlack}{z_0^1(x_{-1},\epsilon_t) \left ( \bar{x} -x\right )=0\\ z_0^1(x_{-1},\epsilon_t)> 0}
% \begin{gather*}
% 0= x_t-(B x_{t-1}+ \phi \psi_\epsilon\epsilon_t + \phi \psi_z 
% \xpt{z_{t}(x_{t-1},\epsilon_t)    } )
% \end{gather*}

\newcommand{\xpt}[1]{#1}


\makeatletter
\@ifundefined{newblock}{%
 \def\newblock{\hskip .11em plus .33em minus .07em} % important line
}
\makeatother


\makeatletter
\pgfplotsset{
    /pgfplots/table/omit header/.style={%
        /pgfplots/table/typeset cell/.append code={%
            \ifnum\c@pgfplotstable@rowindex=-1
                \pgfkeyslet{/pgfplots/table/@cell content}\pgfutil@empty%
            \fi
        }
    }
}
\makeatother
 

\makeatletter
\newcommand*\ExpandableInput[1]{\@@input#1 }
\makeatother


\newcommand{\tpExp}[1]{\phi^e_{#1}(x_{t-1})}
\newcommand{\lRat}[2]{\log \left( \frac{z_{#1}}{z_{#2}}\right)}


\newcommand{\anEdit}[1]
{

{\color{blue}
\begin{quote}
#1  
\end{quote}}

}


\pgfplotsset{compat=newest}
\begin{document}


\frame{\titlepage}
\frame{\frametitle{Table of Contents}\tableofcontents}

\section{Problem Statement and Solution}
\label{sec:probl-stat-solut}

\newcommand{\xpt}[1]{#1}

   \begin{frame}
     \frametitle{Problem Statement}
    

 Consider systems of the form
 \begin{gather}
% \sum_{i=-\tau}^\theta H_i x_{t+i} 
H_{-1} x_{t-1} + H_0 x_t + H_1 \expct{x_{t+1}}=
 \psi_\epsilon 
 \epsilon_t 
   \,\,\forall t.  \label{myEqn}
\intertext{We wish to accommodate inequality constraints of the form}
  x_t \ge \bar{x} \label{myCnstrnt}   \,\,\forall t. \intertext{the literature has proposed a number of alternative time invariant 
``decision rules'' solutions\cite{hintermaier10,Christiano2000} that can be written in the form
}
x_t=\xsln{x_{t-1} , \epsilon_t} \label{mySltn}
\end{gather}
   \end{frame}

   \begin{frame}
     \frametitle{A Useful Characterization}
     
     \begin{itemize}
     \item There are a number of approaches authors have proposed for producing  non-linear impulse response functions.\cite{Karame2012,DeFigueiredo2006,Potter2000,Koop1996a}
\item These solutions can be written in the form\citep{anderson10}
\begin{gather*}
\xpth{t}{x_{t-1} , \epsilon_t} =B x_{t-1}+ \phi \psi_\epsilon\epsilon_t + \sum_{s=0}^\infty F^s \phi \psi_z {z_{t+s}(x_{t+s-1},\epsilon_t)    }  \intertext{in such a way that}
H_{-1} x_{t-1} + H_0 x_t + H_1 \expct{x_{t+1}}=
 \psi_\epsilon 
\end{gather*}
     \item We will determine a sequence of functions $\xpt{z_{t+s}(x_{t+s-1},\epsilon_t)} $ that 
 will allow us to  impose the occasionally binding constraints over successively longer time periods.
     \end{itemize}


   \end{frame}


\section{A Simple Example}

   \begin{frame}
     \frametitle{A Simple Example}
     
For example, consider the simple model


\begin{gather*}
q_{t} +\beta_p(1 - \rho_p)q_{t + 1} + \rho_pq_{t - 1} - \sigma_pr_{t} +
     r_{ut}=0\\
 r_{t} = \max (\bar{r}, \phi_pq_{t}) \\
 r_{ut} = \rho_{ru} r_{ut - 1} + \eta \epsilon_{y,t}
\end{gather*}

Then 
\newcommand{\xtmVec}{  \begin{bmatrix}
    q_t\\r_{t}\\r_{ut}
  \end{bmatrix}
}


\begin{gather*}
  x_t=\xtmVec
\end{gather*}

Some typical values of the parameters are:

\begin{gather*}
  \beta_p = 0.99, \phi_p = 1, 
\rho_p = 0.5, \sigma_p = 1, \rho_{ru} = 0.5,
  \bar{r} = 0.02 \\
% q^L = -.5, q^H = .5, 
%   r_u^L = -4 \sigma_u/(1 - \rho_{ru}), r_u^H=  4\sigma_u/(1 - \rho_{ru}),
%    I_N = {10}, \sigma_u = 0.02,\\
%  \mu = {0},
%    \eta = 1
\end{gather*}



\begin{gather*}
  H= \vcenter{\hbox{\includegraphics{/msu/home/m1gsa00/git/ProjectionMethodTools/ProjectionMethodToolsJava/code/prettyHmat.pdf}}}\\
\psi_z=   \vcenter{\hbox{\includegraphics{/msu/home/m1gsa00/git/ProjectionMethodTools/ProjectionMethodToolsJava/code/prettyPsiZ.pdf}}}\\
\psi_\epsilon=   \vcenter{\hbox{\includegraphics{/msu/home/m1gsa00/git/ProjectionMethodTools/ProjectionMethodToolsJava/code/prettyPsiEps.pdf}}}\\
\end{gather*}




 \begin{gather*}
B=   \vcenter{\hbox{\includegraphics{/msu/home/m1gsa00/git/ProjectionMethodTools/ProjectionMethodToolsJava/code/prettyBmat.pdf}}}\\
\phi=   \vcenter{\hbox{\includegraphics{/msu/home/m1gsa00/git/ProjectionMethodTools/ProjectionMethodToolsJava/code/prettyPhimat.pdf}}}\\
F=   \vcenter{\hbox{\includegraphics{/msu/home/m1gsa00/git/ProjectionMethodTools/ProjectionMethodToolsJava/code/prettyFmat.pdf}}}
 \end{gather*}



   \end{frame}


   \begin{frame}
     \frametitle{The Solution Matrices}
     
{\tiny
\begin{gather*}
\intertext{Algorithm Inputs:}
% H=
% \begin{bmatrix}
% H_{-1} &H_0&H_{t+1}  
% \end{bmatrix}\\
  H= \vcenter{\hbox{\includegraphics{../../../ProjectionMethodTools/ProjectionMethodToolsJava/code/prettyHmat.pdf}}}\\
\psi_z=   \vcenter{\hbox{\includegraphics{../../../ProjectionMethodTools/ProjectionMethodToolsJava/code/prettyPsiZ.pdf}}}
\psi_\epsilon=   \vcenter{\hbox{\includegraphics{../../../ProjectionMethodTools/ProjectionMethodToolsJava/code/prettyPsiEps.pdf}}}\\
\intertext{Algorithm Outputs:}
B=   \vcenter{\hbox{\includegraphics{../../../ProjectionMethodTools/ProjectionMethodToolsJava/code/prettyBmat.pdf}}}\\
\phi=   \vcenter{\hbox{\includegraphics{../../../ProjectionMethodTools/ProjectionMethodToolsJava/code/prettyPhimat.pdf}}}\\
F=   \vcenter{\hbox{\includegraphics{../../../ProjectionMethodTools/ProjectionMethodToolsJava/code/prettyFmat.pdf}}}
 \end{gather*}

}

   \end{frame}

% \newcommand{\forPhi}{\begin{bmatrix}
% \psi_\epsilon&\psi_z
% \end{bmatrix}}
% \newcommand{\phiMult}{\phi \psi_\epsilon}
% \newcommand{\bMult}{B x_{-1} + \phiMult}
 \newcommand{\phiMultBoth}[1]{
 \phi (\psi_\epsilon \epsilon_t +\psi_z z_0^#1(x_{-1},\epsilon_t))}
 \newcommand{\bMultBoth}[1]{B x_{-1} + \phiMultBoth{#1}}


 \newcommand{\bForOne}{\bMultBoth{1}
 }

% \newcommand{\bForTwo}{\bMultBoth{2}+
% F \phi  \psi_z  
% Z_0^1(x_0^2(x_{-1}))   
% }



\newcommand{\compSlack}{z_0^1(x_{-1},\epsilon_t) \left ( \bar{r} -\phi_p q_t\right )=0\\ z_0^1(x_{-1},\epsilon_t)> 0}
% \begin{gather*}
% 0= x_t-(B x_{t-1}+ \phi \psi_\epsilon\epsilon_t + \phi \psi_z 
% \xpt{z_{t}(x_{t-1},\epsilon_t)    } )
% \end{gather*}



    \begin{frame}


      \frametitle{Honoring the Constraint For One Period}


 We begin by solving a system where the constraints are honored only at time
  $t=0$.  
 Using equation \ref{myEqn}, we can construct the system which imposes the 
 constraint for time 0 alone.
To do this we set $\xpt{z_{t+s}}=0\, \forall s>0$
and now have
\begin{gather*}
x_0^1(x_{-1},\epsilon_t)-
\left ( \bForOne \right )=0\intertext{with the complimentary slackness conditions}
\compSlack
\intertext{We set }
z_0^1(x_{-1},\epsilon_t)=
\begin{cases}
0&  B x_{-1} +\epsilon_t \ge \bar{x}  \\
\bar{x}-(B x_{-1}+\epsilon_t) & B x_{-1}+\epsilon_t < \bar{x}  
\end{cases}
\end{gather*}


     
    \end{frame}

    \begin{frame}
      \frametitle{For the Simple Example}
     
{\tiny
       \begin{gather*}
 z_0^1(\xtMVec,\epsilon_t)=
 \begin{cases}
0&\eta >0.032399 \\
0.043056 - 1.328904 \epsilon_t - 0.576390 q_{t-1} - 
      0.664452 r_{ut-1}& \eta \le 0.032399  
 \end{cases}\intertext{where}
\eta=\epsilon_t + 0.433734 q_{t-1} + 0.5 r_{ut-1}\\
\intertext{and}
 q_t(\xtMVec,\epsilon_t)=
 \begin{cases}
0.617296 \epsilon_t + 0.267742 q_{t-1} + 0.308648 r_{ut-1}&\eta >0.032399 \\
-0.0230556 + 1.3289 \epsilon_t + 
        0.57639 q_{t-1} + 0.664452 r_{ut-1}&\eta \le 0.032399 \\
 \end{cases}\\
 r_t(\xtMVec,\epsilon_t)=
 \begin{cases}
0.617296 \epsilon_t + 0.267742 q_{t-1} + 0.308648 r_{ut-1}&\eta >0.032399 \\
0.02&\eta \le 0.032399 \\
 \end{cases}\\
       \end{gather*}
}
    \end{frame}


    \begin{frame}
      \frametitle{Two Period Perfect Foresight Solution}
We can apply equation \ref{myEqn} to compute the $\xtVec$ values along a solution path where all future shocks are zero:
{\tiny
      \begin{gather*}
           \left(
   \begin{array}{c}
    q_{t-1} \\
    r_{t-1} \\
    ru_{t-1} \\
 0.617296 \epsilon_t+0.267742 _{t-1}+0.308648 ru_{t-1}-0.141938
       z_0^1(\xtVec,\epsilon_t)-0.535485 z_0^2(\xtMVec,\epsilon_t) \\
    0.617296 \epsilon_t+0.267742 q_{t-1}+0.308648 ru_{t-1}-0.141938
      z_0^1(\xtVec,\epsilon_t)+0.464515 z_0^2(\xtMVec,\epsilon_t) \\
    \epsilon_t+0.5 ru_{t-1} \\
    0.473924 \epsilon_t+0.0716859 q_{t-1}+0.236962 ru_{t-1}-0.573487
      z_0^1(\xtVec,\epsilon_t)-0.143372 z_0^2(\xtMVec,\epsilon_t) \\
    0.473924 \epsilon_t+0.0716859 q_{t-1}+0.236962 ru_{t-1}+0.426513
      z_0^1(\xtVec,\epsilon_t)-0.143372 z_0^2(\xtMVec,\epsilon_t) \\
    0.5 \epsilon_t+0.25 ru_{t-1} 
   \end{array}
   \right)
      \end{gather*}
}

    \end{frame}

    \begin{frame}
\frametitle{Exact Two Period Perfect Foresight Solutions  Available}

We will need the function {\small $z_0^2(\xtMVec,\epsilon_t)$ }that guarantees:
{\tiny 
\begin{gather*}
    0.617296 \epsilon_t+0.267742 q_{t-1}+0.308648 ru_{t-1}-0.141938
      z_0^1(\xtVec,\epsilon_t)+0.464515 z_0^2(\xtMVec,\epsilon_t) \ge 0.02\\  
    0.473924 \epsilon_t+0.0716859 q_{t-1}+0.236962 ru_{t-1}+0.426513
      z_0^1(\xtVec,\epsilon_t)-0.143372 z_0^2(\xtMVec,\epsilon_t)  \ge 0.02 \\
\end{gather*}
}
\begin{itemize}
\item {\small $z_0^1(\xtVec,\epsilon_t)$ } is a known function
\item Even though $q_t$ depends {\small $z_0^1(\xtVec,\epsilon_t)$ } Mathematica requires about 6 seconds to compute this solution for arbitrary initial conditions
\end{itemize}
    \end{frame}

    \begin{frame}
      \frametitle{Two Period Rational Expectation Solution}
We can apply equation \ref{myEqn} to compute the $\xtVec$ values along a solution path where only one future shock is non zero:
{\tiny
      \begin{gather*}
           \left(
   \begin{array}{c}
    q_{t-1} \\
    r_{t-1} \\
    ru_{t-1} \\
  0.163624 \epsilon_{1}+0.617296 \epsilon_t+0.267742 _{t-1}+0.308648 ru_{t-1}-0.141938
       z_0^1(\xtVec,\epsilon_t)-0.535485 z_0^2(\xtMVec,\epsilon_t) \\
    0.163624 \epsilon_{1}+0.617296 \epsilon_t+0.267742 q_{t-1}+0.308648 ru_{t-1}-0.141938
      z_0^1(\xtVec,\epsilon_t)+0.464515 z_0^2(\xtMVec,\epsilon_t) \\
    \epsilon_t+0.5 ru_{t-1} \\
    0.661105 \epsilon_{1}+0.473924 \epsilon_t+0.0716859 q_{t-1}+0.236962 ru_{t-1}-0.573487
      z_0^1(\xtVec,\epsilon_t)-0.143372 z_0^2(\xtMVec,\epsilon_t) \\
   0.661105 \epsilon_{1}+ 0.473924 \epsilon_t+0.0716859 q_{t-1}+0.236962 ru_{t-1}+0.426513
      z_0^1(\xtVec,\epsilon_t)-0.143372 z_0^2(\xtMVec,\epsilon_t) \\
    0.5 \epsilon_t+0.25 ru_{t-1} 
   \end{array}
   \right)
      \end{gather*}
}

    \end{frame}

    \begin{frame}
\frametitle{Exact Two Period Rational Expectations Solutions May Be Available}

We will need the function {\small $z_0^2(\xtMVec,\epsilon_t)$ }that guarantees:
{\tiny 
\begin{gather*}
 0.163624 \epsilon_{1}+  0.617296 \epsilon_t+0.267742 q_{t-1}+0.308648 ru_{t-1}-0.141938
      z_0^1(\xtVec,\epsilon_t)+0.464515 z_0^2(\xtMVec,\epsilon_t) \ge 0.02\\  
    0.661105 \epsilon_{1}+    0.473924 \epsilon_t+0.0716859 q_{t-1}+0.236962 ru_{t-1}+0.426513
      z_0^1(\xtVec,\epsilon_t)-0.143372 z_0^2(\xtMVec,\epsilon_t)  \ge 0.02 \\
\end{gather*}
}
\begin{itemize}
\item {\small $\expct{z_0^1(\xtVec,\epsilon_t)}$ } is a known function
\end{itemize}
    \end{frame}
    \begin{frame}
      \frametitle{Computations Will Require Approximation and
        Recursion}
      \begin{itemize}
      \item Exact perfect foresight solutions for 3 periods require more than 900 seconds
      \item A Recursive Formulation Precludes Recomputing Approximation Shorter Each Horizons
      \item Low Order Projection and Interpolation Approximates are Inexpensive to Compute
      \item It appears that Interpolation Works Better for this example
      \end{itemize}
    \end{frame}

    \begin{itemize}
    \item relative to complete rational expectations solution, 
errors of approximation enter from 
\begin{itemize}
\item interpolation to compute z's for the future
\item truncation errors
\item using perfect foresight instead of ratex
\end{itemize}
\item simulations iterated forward will differ from the path constructed with a short horizon for applying the constraint
  \item
    \end{itemize}
%\input{../../../ProjectionMethodTools/ProjectionMethodToolsJava/code/interpOneCalcs}


%    \begin{frame}
%      \frametitle{Computation Time}


% {\tiny
%      \begin{minipage}{1.0\linewidth}
%      \begin{tabular}{|l|l|
S[table-format=3.6]|
S[table-format=3.6]|
S[table-format=3.6]|
S[table-format=3.6]|
S[table-format=3.6]|
}
\hline
\multicolumn{7}{|c|}{Time in seconds for evaluation at a single initial state ( except where noted )\footnote{See the endnotes Section~\ref{sec:endnotes}.}}\\
\hline
\multicolumn{1}{|c|}{Periods}&
\multicolumn{1}{|c|}{Recursive?}&
\multicolumn{1}{|c|}{Analytic First}&
\multicolumn{1}{|c|}{Analytic Second}&
&&\\
% \multicolumn{1}{|c|}{Semi-Analytic}&
% \multicolumn{1}{|c|}{Projection}&
% \multicolumn{1}{|c|}{Interpolation}\\
\hline
\hline
1&NA\endnote{Algebraic expression for all initial states.\label{allStates}}&\ExpandableInput{/msu/home/m1gsa00/git/ProjectionMethodTools/ProjectionMethodToolsJava/code/symb01Secs}&&&\\
\hline
\hline
2&No\textsuperscript{\ref{allStates}}&\ExpandableInput{/msu/home/m1gsa00/git/ProjectionMethodTools/ProjectionMethodToolsJava/code/symb02Secs}&&&\\
\hline
2&Yes\textsuperscript{\ref{allStates}}&\ExpandableInput{/msu/home/m1gsa00/git/ProjectionMethodTools/ProjectionMethodToolsJava/code/symb02RecSecs}&&&\\
\hline
2&Yes&\ExpandableInput{/msu/home/m1gsa00/git/ProjectionMethodTools/ProjectionMethodToolsJava/code/symb02ValsRecSecs}&&&\\
\hline
2&Expectation non recursive&\ExpandableInput{/msu/home/m1gsa00/git/ProjectionMethodTools/ProjectionMethodToolsJava/code/symb02ValsExpSecs}&&&\\
\hline
\hline
3&No\textsuperscript{\ref{allStates}}&\ExpandableInput{/msu/home/m1gsa00/git/ProjectionMethodTools/ProjectionMethodToolsJava/code/symb03Secs}&&&\\
\hline
3&Yes\textsuperscript{\ref{allStates}}&\ExpandableInput{/msu/home/m1gsa00/git/ProjectionMethodTools/ProjectionMethodToolsJava/code/symb03RecSecs}&&&\\
\hline
3&No&\ExpandableInput{/msu/home/m1gsa00/git/ProjectionMethodTools/ProjectionMethodToolsJava/code/symb03ValsSecs}&&&\\
\hline
3&Yes&\ExpandableInput{/msu/home/m1gsa00/git/ProjectionMethodTools/ProjectionMethodToolsJava/code/symb03ValsRecSecs}&&&\\
\hline
3&Expectation non recursive&\ExpandableInput{/msu/home/m1gsa00/git/ProjectionMethodTools/ProjectionMethodToolsJava/code/symb03ValsExpSecs}&&&\\
\hline
\hline
4&No&\ExpandableInput{/msu/home/m1gsa00/git/ProjectionMethodTools/ProjectionMethodToolsJava/code/symb04ValsSecs}&&&\\
\hline
4&Yes&\ExpandableInput{/msu/home/m1gsa00/git/ProjectionMethodTools/ProjectionMethodToolsJava/code/symb04ValsRecSecs}&&&\\
\hline
\hline
5&No&\ExpandableInput{/msu/home/m1gsa00/git/ProjectionMethodTools/ProjectionMethodToolsJava/code/symb05ValsSecs}&&&\\
\hline
5&Yes&\ExpandableInput{/msu/home/m1gsa00/git/ProjectionMethodTools/ProjectionMethodToolsJava/code/symb05ValsRecSecs}&&&\\
\hline
\hline
6&No&\ExpandableInput{/msu/home/m1gsa00/git/ProjectionMethodTools/ProjectionMethodToolsJava/code/symb06ValsSecs}&&&\\
\hline
\hline
7&No&\ExpandableInput{/msu/home/m1gsa00/git/ProjectionMethodTools/ProjectionMethodToolsJava/code/symb07ValsSecs}&&&\\
\hline
\hline
\end{tabular}

%      \end{minipage}
% }
%    \end{frame}
    \begin{frame}
      \frametitle{Accuracy Components}
      \begin{itemize}
      \item For a given path lenth how well are system equations satisfied
      \item Compared to the best solution how far apart are solutions
      \item how far perfect foresight from rational expections
      \item compute infinity norm and the worst violation for r
      \item what does it mean to ``simulate'' a path
      \item my algorithm computes machine tolerance over the modelled future so that next period solution uses a shorter modelled future
      \item if iterating forward will get a different path presumable ``better,'' but this changes the expectations for the whole path. in the limit, the
changes don't matter but significant for short horizons can quantify what is short
\item distinuguish interpolation error from truncation error even for ratex, iterating should use the shorter horizon z functions along the path
      \end{itemize}
    \end{frame}
    \begin{frame}
      \frametitle{Perfect Foresight Solutions}
      \begin{itemize}
\item For a given solution path  from some set of initial conditions 
I compute the product of the discrepancy in the H equations multiplied by 
the distance from the constraint -- a complementary slackness conditions
      \item For a given path lenth, the procedure computes the solution to within machine precision for all values asserted to accomodate the constraint
      \item Apparently for this particular problem  even 0 order interpolation produces
are 
      \end{itemize}
    \end{frame}

    \begin{frame}
      \frametitle{Impact of Future Expectational Errors}
      
{\small
\input{../../../ProjectionMethodTools/ProjectionMethodToolsJava/code/impactFuture}
}
    \end{frame}

%     \begin{frame}
%       \frametitle{Accuracy Tables}
  
% {\tiny
%   \input{../../../ProjectionMethodTools/ProjectionMethodToolsJava/code/accurRERecur3642257967}  
% }
%     \end{frame}



%     \begin{frame}
%       \frametitle{Accuracy Tables}
  
% {\tiny
%   \input{../../../ProjectionMethodTools/ProjectionMethodToolsJava/code/accurRERecur3642265782}  
% }
%     \end{frame}

%\input{../../../ProjectionMethodTools/ProjectionMethodToolsJava/code/timePFNullRecurNow}  


    \begin{frame}
      \frametitle{Computation Time Tables}



  
{\tiny
\input{../../../ProjectionMethodTools/ProjectionMethodToolsJava/code/timePFNullRecurNow}  
}
    \end{frame}


    \begin{frame}
      \frametitle{Computation Time Tables}
  
 {\tiny
   \input{../../../ProjectionMethodTools/ProjectionMethodToolsJava/code/timeRENullRecurNow}  
 }
    \end{frame}




    \begin{frame}
      \frametitle{Computation Time Tables}
  
 {\tiny
   \input{../../../ProjectionMethodTools/ProjectionMethodToolsJava/code/timePFExactRecurNow}  
 }
    \end{frame}


    \begin{frame}
      \frametitle{Computation Time Tables}
  
 {\tiny
   \input{../../../ProjectionMethodTools/ProjectionMethodToolsJava/code/timeREExactRecurNow}  
 }
    \end{frame}




    \begin{frame}
      \frametitle{Path Simulation to Assess Accuracy}
      

      \begin{itemize}
      \item May need to simulate long paths to assess the accuracy
      \item errors in future expectations filter back to present
      \item estimate  impact via the formulae
      \item try and choose parameters to make the eigenvalues of F close to one
      \end{itemize}
    \end{frame}

    \begin{frame}
      \frametitle{Convergence Criteria}
      \begin{tabular}{|l|p{4cm}|c|}
\hline
        FPForInitState&FixedPoint of a vector function&Mathematica \$MachinePrecision\\
\hline
construcInterpPF&Interpolation Polynomial:  interpolation order (iOrd) and number of grid points. (nPts)&iOrd=\{1,2,3\}, nPts=\{3,5,10,20\}\\
\hline
construcInterpRE&Interpolation Polynomial:  interpolation order (iOrd) and number of grid points. (nPts), Numerical Integration 
AccuracyGoal,2, Compiled,Automatic,  PrecisionGoal, 2, WorkingPrecision ,2&\\
\hline
iterPF,iterRE convergence&Infinity Norm& \{$10^{-4},10^{-6},10^{-8}$\}\\
\hline
      \end{tabular}
    \end{frame}
\begin{frame}
	\frametitle{A General Model Class}
\end{frame}

   \begin{frame}
     \frametitle{Clean up first}
     \begin{itemize}
     \item Relate to other approaches.  all doing extend path? compute some relevant path lengths
     \item other models linearized compute how far out need to solve
     \item mathematica presentation manipulate function
     \item alternative to Judd renormalization strategy for assessing models
     \item minimizing errors determines algorithm. formulat applicable for many solution strategies.  including perfect foresight rational expectations or penalty function or invariant solution
     \item Incorporate regime switching characterization.  What happens after constraint.
     \item report max constraint violation and number?
     \item possible to do shorter horizon calculations for completeness
     \item how good is using a truncated sum of constant errors in formula to predict truncation error
     \item keep getting same value for hmatapp when only 1 iteration of assessPF$0.00245262, {qq -> 0.5, ru -> 0.08, Private`ep -> 0.02}$
     \item time evaluating final interpolation function versus constructing
     \item deal with and eliminate mat<hematica warning messages
       \item hmatapp should delete evaluating point where constraint not enforced
       \item how to measure discrepancy.  if ignore constraint comp slack condition misleadingly small if mult by satisfied equation
     \item aPath[] still uses fpForInitStateFunc[].  Is that okay? what about an approximation here
     \item identify memoize conflicts when changing parameters or model
\item identify describe algorithm components
     \item Infinity norm convergence fixedpoint convergence
\item computations with asymptotic at and away from constraint limiting cases, $\bar{r}$ never, always  stochastic 0 to large
\item difference between simulating forward and computing aPath ever for perfect foresight
\item difference for getting interpolating polynomial for whole path as opposed to just the time t value for iterating
\item necessary or better to carry around the family of zFuncs?
\item can weight errors in sytem equation and in violating constraint
     \end{itemize}
   \end{frame}


   \begin{frame}
     \frametitle{Clean up}
     \begin{itemize}
     \item ``when not stochastic the problem goes away''  how to choose a decision rule under perfect foresight since all complimentary slackness conditions easy to accomodate. seems as though always resort to contradictions at ``end of path''
     \item timing for why recursive necessary
       \item eliminate old gridPts
\item identify and itemize measures of accuracy of components using exact, interpolation, projection and by applying the original system
\item Different measures of accuracy: 
  \begin{itemize}
  \item for given horizon of constraints
  \item relative to converged solution
  \item vary fixedpoint convergence criteria
  \end{itemize}
\item different dimensions of time and accuracy for perfect foresight versus rational expectations
\item different solution concepts PF, RE others?
     \item highlight when PF and RE differ
     \end{itemize}
   \end{frame}
   % \begin{frame}
   %   \frametitle{Clean up}
   %   \begin{itemize}
   %   \end{itemize}
   % \end{frame}
   \begin{frame}
     \frametitle{Clean up}
     \begin{itemize}
     \item simulate paths compute standard error of weighted comp slac discrepancies discounted back using formula to get impact on variables.  any model possible including nonlinear  (possible machine learning algorithm)
\item refactor for ease of implementing other models
\item make all computations lists of scalar functions and memoize the vector computation
\item Other models
\item compare interpolation at a fixed set of points
\item metric for other users to see how much future expectations matter for current period
\item construct composition functions for leapfrogging
\item leapfrog by computing compositions of functions
     \item path doubling or just one iteration produce same  -- convergence  issue?
\item compute expected value for powers for confidence intervals and for skewness kurtosis etc
     \end{itemize}
   \end{frame}
   \begin{frame}
     \frametitle{Clean up}
     \begin{itemize}
\item Different scenarios can leave constraint and return
\item Probablity of escape function of state  ( sustained escape concept )
\item measure probablity can stay away from constraint  ( zero when there, but not one when close. )
\item compute ergodic distribution
\item indicator functions to measure probability of events
     \end{itemize}
   \end{frame}
   \begin{frame}
     \frametitle{Clean up last}
     \begin{itemize}
     \item relate to \cite{hintermaier10,Christiano2000} impulse response function
\item Redo numbering for zzz
\item Better computationally to get close to expectations solution then iterate for fixed point of function?
     \end{itemize}
   \end{frame}

   \begin{frame}
     \frametitle{Future Directions}
     \begin{itemize}
     \item richardson extrapolation
     \item Should also consider other behavior besides absorbing barriers change in parameters
     \item  \begin{gather*}
 M x_t \ge m \\
 \end{gather*}
\item Fold future z f b call into interpolation recursion
\item Include  powers in order to compute confidence intervals for impulse response
\item  more symbolic solutions, iterate recursively
\item likely that can skimp on accuracy for distant future expectations and focus on getting near term most accurate use formula to inform choice of precision
\item  Smolyak Points 
\item Parallelize
\item compute border and use piecewise
\item strategically place points for accuracy
  \item for non linear have to choose linearization point.  probably do unconstrained perturbation and find stochastic steady state
     \end{itemize}
   \end{frame}
   \begin{frame}
     \frametitle{Future Directions ( continued )}
     \begin{itemize}
\item use plot to get better points
\item uniqueness via continuity in homotopy of variance and constraints
\item possible to fool algorithm
\item compile
\item  use Java for integration
     \end{itemize}
   \end{frame}
   \begin{frame}
     \frametitle{Future Directions ( continued )}
     \begin{itemize}
\item generic eval of complimentary slackness
\item derivatives
\item condition numbers
\item tracking border shifts
\item projection versus interpolation
\item Just simulate and keep good trajectories and update the z function how to distinguish per foresight and stochastic?
\item Multivariate version of formula implementation
     \end{itemize}
   \end{frame}
   \begin{frame}
     \frametitle{Future Directions ( continued )}
     \begin{itemize}
\item interpretation as ``perturbation'' or other approximation method
\item how general
\item Stochastically imposed constraints ( future scenarios )
\item regime switching applications
\item impulse response functions and error bounds
\item trade-offs computation time accuracy given interpolation and length of time imposing constraints.  Given solutions can Store in the z's a sort of special basis. One correct path implies many others if solutions invariant.   Algebraic idea of group applied to dynamic systems
\item Volterra Series
\item detecting asymmetries
     \end{itemize}
   \end{frame}


   \begin{frame}





 \bibliographystyle{plainnat}
 \bibliography{../../bibFiles/anderson,../../bibFiles/files}
     
   \end{frame}


%      \section{Endnotes}
% \label{sec:endnotes}
% {\tiny
%      \theendnotes
% }




%\appendix
\newpage
\section{Formula Applied to a Constant}
{\tiny
\begin{gather*}
\phi \psi_c=  \sum_{s=0}^\infty F^s \phi \psi_c  -   F \sum_{s=0}^\infty F^s \phi \psi_c \\
(I-F) \left (\sum_{s=0}^\infty F^s \phi \psi_c \right ) =\phi \psi_c\\
\sum_{s=0}^\infty F^s \phi \psi_c=(I - F)^{-1}\phi \psi_c\\
\sum_{s=0}^{k+1} F^s \phi \psi_c=F \sum_{s=0}^{k} F^s \phi \psi_c + \phi \psi_c\\
F^{k+1} \phi \psi_c +\sum_{s=0}^{k} F^s \phi \psi_c=F \sum_{s=0}^{k} F^s \phi \psi_c + \phi \psi_c\\
(I -F)\sum_{s=0}^{k} F^s  = (I- F^{k+1}) \phi \psi_c\\
\sum_{s=0}^{k} F^s  = (I -F)^{-1}(I- F^{k+1}) \phi \psi_c\\
\sum_{s=k+1}^{\infty} F^s  = (I -F)^{-1} F^{k+1}\phi \psi_c
\end{gather*}
}
\section{Code}
\begin{frame}
\frametitle{FPForInitState}
\label{sec:code}

\begin{pseudocode}{FPForInitState}{xtm1,zFuncs}
pathLength\GETS Length[zFuncs]-2\\
xTExpression\GETS constructXtExpression(pathLength)\\
csExpression\GETS constructCompSlackExpression(pathLength)\\
zFuncAppExpressions\GETS constructCompSlackExpression(zFuncs)\\
theSystem\GETS {xtExpression,csExpression,zFuncAppExpressions}\\
initXGuess\GETS zFuncs[1][xtm1,0]\\
xtZVNxt\GETS FixedPoint[xtZVNxt\GETS Solve[theSystem,xtZVNxt],initXGuess]\\
\RETURN{xtZVNxt}
\end{pseudocode}
\end{frame}


\begin{frame}
\frametitle{construcInterpPF}
\label{sec:code}

\begin{pseudocode}{constructInterpPF}{systemFPFuncs,iOrder,iPts}
theGrid\GETS constructRectangularGrid[iPts]\\
theFVals \GETS systemFPFuncs[theGrid]\\
interpFuncs\GETS constructInterpPoly(theFVals,iOrder)\\
\RETURN{interpFuncs}
\end{pseudocode}
\end{frame}


\begin{frame}
\frametitle{construcInterpRE}
\label{sec:code}

\begin{pseudocode}{constructInterpRE}{systemFPFuncs,iOrder,iPts}
theGrid\GETS constructRectangularGrid[iPts]\\
theFVals \GETS Expectation[systemFPFuncs,theGrid]\\
interpFuncs\GETS constructInterpPoly(theFVals,iOrder)\\
\RETURN{interpFuncs}
\end{pseudocode}
\end{frame}



\begin{frame}
\frametitle{iterFP iterRE}
\label{sec:code}


\begin{pseudocode}{iterPF}{iOrder,iPts.zFuncs}
theSystemFuncs \GETS FPForInitState\\
interpFuncs \GETS constructInterpPF(theFVals,iOrder)\\
\RETURN{interpFuncs}
\end{pseudocode}
\begin{pseudocode}{iterRE}{iOrder,iPts.zFuncs}
theSystemFuncs \GETS FPForInitState\\
interpFuncs \GETS constructInterpRE(theFVals,iOrder)\\
\RETURN{interpFuncs}
\end{pseudocode}
\end{frame}
\subsection{symb01.m}

\begin{frame}
\label{sec:symb01.m}
\end{frame}



\newpage
\subsection{symb02.m}

\begin{frame}
\label{sec:symb02.m}  
\end{frame}


\newpage
\subsection{symb02Rec.m}
\begin{frame}
\label{sec:symb02Rec.m}  
\end{frame}



\newpage
\subsection{symb02ValsExp.m}
\begin{frame}
\label{sec:symb02ValsExp.m}
\end{frame}





\end{document}
