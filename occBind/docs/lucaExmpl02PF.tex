
Consider a perfect foresight solution where the constraints are honored only 
for $t=0,1$.\footnote{ Perfect foresight solutions impose the constraint that future shocks are identically zero.}

Since now a future value of z is non zero, formula \ref{myEqn}  will have additional non-zero terms.


\begin{gather*}
  \vcenter{\hbox{\includegraphics{/msu/home/m1gsa00/git/ProjectionMethodTools/ProjectionMethodToolsJava/code/prettyEqns02A.pdf}}}\\
  \vcenter{\hbox{\includegraphics{/msu/home/m1gsa00/git/ProjectionMethodTools/ProjectionMethodToolsJava/code/prettyEqns02B.pdf}}}\\
\end{gather*}
Mathematica is unable to compute this solution exactly in a reasonable amount
of time.\footnote{ 5 hours.  One could substitute the solution for the one period function to simplify the system, but Mathematica was still unable to solve the simplified system.}  However we can easily obtain numerical solutions for the z functions.  The z01 function applies to the time t variables and 
threrefore we must solve a nonlinear system that captures the fact that
the time t value of the state variables depends on this future value of z01.


Note the following graph of $r_t$ when imposing constraint for only two periods:
\begin{gather*}
\includegraphics{/msu/home/m1gsa00/git/ProjectionMethodTools/ProjectionMethodToolsJava/code/prettyrr02.pdf}
\end{gather*}



\newcommand{\lucaXt}{
   \begin{bmatrix}
    q_{t-1}\\r_{ut-1}\\r_{t-1}\\
     q_{t}\\r_{ut}\\r_{t}\\
     q_{t+1}\\r_{ut+1}\\r_{t+1}
   \end{bmatrix}}

 \newcommand{\lucaXtpOne}{
   \begin{bmatrix}
     q_{t}\\r_{ut}\\r_{t}\\
     q_{t+1}\\r_{ut+1}\\r_{t+1}\\
     q_{t+2}\\r_{ut+2}\\r_{t+2}
   \end{bmatrix}}


 Note the following graph of 

 \begin{gather*}
 H_{2}\lucaXt ( r_t-0.02)
 \end{gather*}

  when imposing constraint for only two periods:
 \begin{gather*}
 \includegraphics{/msu/home/m1gsa00/git/ProjectionMethodTools/ProjectionMethodToolsJava/code/prettyhapp02A.pdf}
 \end{gather*}


 Note the following graph of 

 \begin{gather*}
 H_{5}\lucaXt( r_{t+1}-0.02)
 \end{gather*}

  when imposing constraint for only two periods:
 \begin{gather*}
 \includegraphics{/msu/home/m1gsa00/git/ProjectionMethodTools/ProjectionMethodToolsJava/code/prettyhapp02B.pdf}
 \end{gather*}


Both have infinity norms near machine precision for the pecified range of values of the state variables.  There are comparable values for non zero $\epsilon$.


We can use the one period solution to aid in solving the two period problem.
Complicated recursion since z, though known, is recursivly applied to itself


\begin{gather*}
  \vcenter{\hbox{\includegraphics{/msu/home/m1gsa00/git/ProjectionMethodTools/ProjectionMethodToolsJava/code/prettyEqns02RecA.pdf}}}\\
  \vcenter{\hbox{\includegraphics{/msu/home/m1gsa00/git/ProjectionMethodTools/ProjectionMethodToolsJava/code/prettyEqns02RecB.pdf}}}\\
\end{gather*}

The second equation comes from applying the $z_{0t}$ function to the time t variables.


The solution obtained is essentially identical as can be seen by evaluating the
same complementary slackness conditions above.\footnote{The Piecewise solutions now appear to have four branches.  But, this
 is must be an artifact since the time t solutions only have two branches.}


 Note the following graph of 

 \begin{gather*}
 H_{2}\lucaXt ( r_t-0.02)
 \end{gather*}

  when imposing constraint for only two periods:
 \begin{gather*}
 \includegraphics{/msu/home/m1gsa00/git/ProjectionMethodTools/ProjectionMethodToolsJava/code/prettyhapp02RecA.pdf}
 \end{gather*}


 Note the following graph of 

 \begin{gather*}
 H_{5}\lucaXt( r_{t+1}-0.02)
 \end{gather*}

  when imposing constraint for only two periods:
 \begin{gather*}
 \includegraphics{/msu/home/m1gsa00/git/ProjectionMethodTools/ProjectionMethodToolsJava/code/prettyhapp02RecB.pdf}
 \end{gather*}




There are four regions.  There are regions for each combination of zero/nonzero z values for two periods.



\begin{gather*}
  0=\vcenter{\hbox{\includegraphics{/msu/home/m1gsa00/git/ProjectionMethodTools/ProjectionMethodToolsJava/code/prettyreg02pltA.pdf}}}
\end{gather*}


\begin{gather*}
  0=\vcenter{\hbox{\includegraphics{/msu/home/m1gsa00/git/ProjectionMethodTools/ProjectionMethodToolsJava/code/prettyreg02pltB.pdf}}}
\end{gather*}


\begin{gather*}
  0=\vcenter{\hbox{\includegraphics{/msu/home/m1gsa00/git/ProjectionMethodTools/ProjectionMethodToolsJava/code/prettyreg02pltC.pdf}}}
\end{gather*}


\begin{gather*}
  0=\vcenter{\hbox{\includegraphics{/msu/home/m1gsa00/git/ProjectionMethodTools/ProjectionMethodToolsJava/code/prettyreg02pltD.pdf}}}
\end{gather*}




It will be useful to solve for the z's recursively, using the results from shorter periods to compute solutions for longer period.

Suppose we have computed $z^0_t(x_{t-1},\epsilon_t)$ imposing the consraint for time $t$ alone.  We can use this function to impose the constraint at time $t+1$ by applying the function to the later, time $t$ values   $z^0_t(x_{t},0)$ with perfect foresight solution $\epsilon_{t+1}=0$


\begin{gather*}
     \left(
   \begin{array}{cc}
    \{ & 
   \begin{array}{cc}
    0 & 0.767742 \text{eps}+0.116129 \text{qtm1}+0.383871 \text{rutm1}-0.0615634
      \text{zzz$\$$0$\$$1}(t)-0.232258 \text{zzz$\$$1$\$$1}(t)\geq 0.0323994 \\
    -1.02026 \text{eps}-0.154324 \text{qtm1}-0.510128 \text{rutm1}+0.0818118
      \text{zzz$\$$0$\$$1}(t)+0.308648 \text{zzz$\$$1$\$$1}(t)+0.0430556 &
      \text{True} \\
   \end{array}
    \\
   \end{array}
   \right)=\text{zzz$\$$0$\$$1}(t)
\end{gather*}
%https://github.com/es335mathwiz/paperProduction.git

% Global`zzz$0$1[Global`t]==
% Global`base[aPath02ValsRec[[4,1]],aPath02ValsRec[[6,1]],0]/.
% dSub/.{Global`qtm1->Global`qqVar,Global`rutm1->Global`ruVar,Global`eps->Global`epsVar}),
% {Global`zzz$0$1[Global`t],0}]],dSub/.{Global`qtm1->Global`qqVar,Global`rutm1->Global`ruVar,Global`eps->Global`epsVar}]}
