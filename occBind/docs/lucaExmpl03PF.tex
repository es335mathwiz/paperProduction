
Consider a perfect foresight solution where the constraints are honored only 
for $t=0,1$.\footnote{ Perfect foresight solutions impose the constraint that future shocks are identically zero.}

Since now a future value of z is non zero, formula \ref{myEqn}  will have additional non-zero terms.


\begin{gather*}
  \vcenter{\hbox{\includegraphics{/msu/home/m1gsa00/git/ProjectionMethodTools/ProjectionMethodToolsJava/code/prettyEqns03RecA.pdf}}}\\
  \vcenter{\hbox{\includegraphics{/msu/home/m1gsa00/git/ProjectionMethodTools/ProjectionMethodToolsJava/code/prettyEqns03RecB.pdf}}}\\
  \vcenter{\hbox{\includegraphics{/msu/home/m1gsa00/git/ProjectionMethodTools/ProjectionMethodToolsJava/code/prettyEqns03RecC.pdf}}}\\
\end{gather*}
Mathematica is unable to compute this solution exactly in a reasonable amount
of time.\footnote{ 5 hours.  One could substitute the solution for the one period function to simplify the system, but Mathematica was still unable to solve the simplified system.}  However we can easily obtain numerical solutions for the z functions.  The z01 function applies to the time t variables and 
threrefore we must solve a nonlinear system that captures the fact that
the time t value of the state variables depends on this future value of z01.


Note the following graph of $r_t$ when imposing constraint for only two periods:
\begin{gather*}
\includegraphics{/msu/home/m1gsa00/git/ProjectionMethodTools/ProjectionMethodToolsJava/code/prettyrr03.pdf}
\end{gather*}





 Note the following graph of 

 \begin{gather*}
 H_{2}\lucaXt ( r_t-0.02)
 \end{gather*}

  when imposing constraint for only two periods:
 \begin{gather*}
 \includegraphics{/msu/home/m1gsa00/git/ProjectionMethodTools/ProjectionMethodToolsJava/code/prettyhapp03A.pdf}
 \end{gather*}


 Note the following graph of 

 \begin{gather*}
 H_{5}\lucaXt( r_{t+1}-0.02)
 \end{gather*}

  when imposing constraint for only two periods:
 \begin{gather*}
 \includegraphics{/msu/home/m1gsa00/git/ProjectionMethodTools/ProjectionMethodToolsJava/code/prettyhapp03B.pdf}
 \end{gather*}


Both have infinity norms near machine precision for the pecified range of values of the state variables.  There are comparable values for non zero $\epsilon$.


We can use the one and two period solutions to aid in solving the three period problem.
Complicated recursion since z, though known, is recursivly applied to itself


\begin{gather*}
  \vcenter{\hbox{\includegraphics{/msu/home/m1gsa00/git/ProjectionMethodTools/ProjectionMethodToolsJava/code/prettyEqns03RecA.pdf}}}\\
  \vcenter{\hbox{\includegraphics{/msu/home/m1gsa00/git/ProjectionMethodTools/ProjectionMethodToolsJava/code/prettyEqns03RecB.pdf}}}\\
\end{gather*}

The second equation comes from applying the $z_{0t}$ function to the time t variables.


The solution obtained is essentially identical as can be seen by evaluating the
same complementary slackness conditions above.\footnote{The Piecewise solutions now appear to have four branches.  But, this
 is must be an artifact since the time t solutions only have two branches.}


 Note the following graph of 

 \begin{gather*}
 H_{2}\lucaXt ( r_t-0.02)
 \end{gather*}

  when imposing constraint for only two periods:
 \begin{gather*}
 \includegraphics{/msu/home/m1gsa00/git/ProjectionMethodTools/ProjectionMethodToolsJava/code/prettyhapp03RecA.pdf}
 \end{gather*}


 Note the following graph of 

 \begin{gather*}
 H_{5}\lucaXt( r_{t+1}-0.02)
 \end{gather*}

  when imposing constraint for only two periods:
 \begin{gather*}
 \includegraphics{/msu/home/m1gsa00/git/ProjectionMethodTools/ProjectionMethodToolsJava/code/prettyhapp03RecB.pdf}
 \end{gather*}
