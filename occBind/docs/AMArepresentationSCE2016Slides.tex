\documentclass[tikz]{beamer}

\usepackage{tikz}
\usetikzlibrary{shapes,arrows}
\mode<presentation>{}
\usepackage{beamerthemeshadow}

%\usepackage{pseudocode}
%\usepackage{dirtree}
% \usepackage[utf8]{inputenc}
% \usepackage{listings,xcolor}

% \usepackage{soul}
% \usepackage{ulem}



\usepackage{amsmath}
\usepackage{algorithm2e}
\usepackage{algorithmicx}
\usepackage{algpseudocode}

\usepackage[round,authoryear]{natbib}
%\usepackage[margin=1.0in]{geometry}
\usepackage{graphicx}
\usepackage{moreverb}
\usepackage{hyperref}
\newcommand\infNorm[1]{\left\lVert#1\right\rVert_\infty}

\usepackage{mathtools}
\mathtoolsset{showonlyrefs}
\usepackage{datetime}
\usepackage{amsfonts}

\newcommand{\xgusst}[1]{x^{guess}_{#1}}
\newcommand{\xnxt}[1]{x^{next}_{#1}}
\newcommand{\zgusst}[1]{z^{guess}_{#1}}
\newcommand{\xguss}{x^{tguess}}
\newcommand{\xtpguss}{x^{tp1guess}}



\newcommand{\Rn}[1]{{\mathcal{R}^{#1}}}
\newcommand{\numX}{{N_x}}
\newcommand{\numZ}{{N_z}}
\newcommand{\numEps}{{N_\epsilon}}
\newcommand{\numR}{{N_r}}
\newcommand{\numIters}{{K}}
\newcommand{\numTerms}{{N_{terms}}}


\newcommand{\ADRUE}{{\bf ADRUEF}}
\newcommand{\ADR}{{\bf ADRF}}

\newcommand{\xtmEpsArg}{{(x_{t-1},\epsilon_t)}} 
\newcommand{\xtArg}{{(x_{t}) }}


\newcommand{\sumLinPart}{{
B x_{t-1}+ \phi \psi_\epsilon\epsilon + (I - F)^{-1} \phi \psi_c 
}}
\newcommand{\sumZPart}{{
 \sum_{\sForSum=0}^\infty F^s \phi z_{t+\sForSum}(x_{t-1},\epsilon) 
}}
\newcommand{\sumZPartZero}{{
 \phi z_{t}(x_{t-1},\epsilon) 
}}
\newcommand{\sumZPartPos}{{
 \sum_{\sForSum=1}^\infty F^s \phi Z_{t+\sForSum}(x_{t-1},\epsilon) 
}}

\newcommand{\EsumLinPart}{{
	 B x_{t-1}+ (I - F)^{-1} \phi \psi_c 
}}
\newcommand{\EsumZPartZero}{{
	  \sum_{\sForSum=0}^\infty F^s \phi Z^{PF}_{t+\sForSum}(x_{t-1},0) 
}}
\newcommand{\EsumZPartEpsilon}{{
	  \sum_{\sForSum=0}^\infty F^s \phi \expct{z_{t+\sForSum}(x_{t-1},\epsilon)}
}}
\newcommand{\EsumCapZPart}[1]{{
	  \sum_{\sForSum=0}^\infty F^s \phi {Z^{#1}_{t+\sForSum}(x_{t-1})}
}}
\newcommand{\uncnxpt}[2]{{\mathcal{E}\left [ #1 \left | #2 \right . \right ]}}
\newcommand{\expct}[1]{{\mathcal{E}\left [ #1 xs\right ]}}

\newcommand{\xzFuncGuess}{{\mathbb{g}}}
\newcommand{\xzFuncGuessSig}{{(\Rn{\numX+\numEps+\numZ}\times 1)\rightarrow
(\Rn{({3\numX+\numEps+\numZ})}\times 1)}}


\newcommand{\xzFunc}{{{\gamma}}}
\newcommand{\xzFuncSig}{{(\Rn{\numX+\numEps+\numZ}\times 1)\rightarrow
(\Rn{({3\numX+\numEps+\numZ})}\times 1)}}

\newcommand{\XZFunc}{{\mathbb{G}}}
\newcommand{\bigXFuncSig}{{(\Rn{\numX}\times 1)\rightarrow
(\Rn{({\numX+\numZ})}\times 1)}}


\newcommand{\genXGFP}{{\mathcal{U}}}
\newcommand{\xgFP}{{\mathcal{V}}}
\newcommand{\genSlvr}{{\mathcal{T}}}
\newcommand{\slvr}{{\mathcal{S}}}
\newcommand{\solverSig}{{
\frfpnsRegimeFuncHelp
}}

\newcommand{\eqnFunc}{\mathbb{M}}
\newcommand{\eqnFuncSig}{{\Rn{3\numX+\numEps+\numZ}\rightarrow\Rn{\numX}}}


\newcommand{\frfpnsFuncSig}{{\Rn{\numX+\numEps}\rightarrow\Rn{\numX+\numZ}}}

\newcommand{\bigXRegimeFuncSig}{{(\Rn{\numX+1}\times 1)\rightarrow
(\Rn{({\numX+\numZ})}\times 1)}}
\newcommand{\lilXRegimeFuncSig}{{(\Rn{\numX+1+\numEps+\numZ}\times 1)\rightarrow
(\Rn{({3(\numX+1)+\numEps+\numZ})}\times 1)}}
\newcommand{\eqnRegimeFuncSig}{{
\{\Rn{3(\numX+1)+\numEps+\numZ}\rightarrow\Rn{\numX},
\ldots,
\Rn{3(\numX+1)+\numEps+\numZ}\rightarrow\Rn{\numX}\}
}}
\newcommand{\frfpnsRegimeFuncHelp}{{\Rn{\numX+\numEps}\rightarrow\Rn{\numX+\numZ}}}

\newcommand{\frfpnsRegimeFuncSig}{{
\{ (\frfpnsRegimeFuncHelp{1}), \ldots ,  (\frfpnsRegimeFuncHelp{\numR})\}
}}

\newcommand{\dstSpec}{{\mathbf{distSpec}}}
\newcommand{\expctSpec}{{\mathbf{expctSpec}}}
\newcommand{\grdSpec}{{\mathbf{grdSpec}}}


\newcommand{\linMod}{{\mathcal{L}}}
\newcommand{\linModMats}{{\{H,\psi_\epsilon,\psi_c;B,\phi,F\}}}
\newcommand{\sForSum}{{\nu}}

\newcommand{\xWarg}{   \mathcal{X}_{t}(x_{-1},\epsilon)}
\newcommand{\xWargK}{   \hat{\mathcal{X}}_{t}(x_{-1},\epsilon,k)}


\newcommand{\xIter}[2]{\mathcal{X}^{#1}(#2)}
\newcommand{\xNow}[1]{x^{#1}_t(x_{t-1},\epsilon_t)}
\newcommand{\zNow}[1]{z^{#1}(x_{t-1},\epsilon_t)}
\newcommand{\xNowtp}[1]{x^{#1}_{t+1}(x_{t-1},\epsilon_t)}
\newcommand{\XNow}[3]{\mathcal{X}^{#1}_{#2}(x_{#3})}
\newcommand{\ZNow}[3]{\mathcal{Z}^{#1}(x_{#3})}


\newcommand{\rcpC}{{\mathbf{N}}}

\begin{document}
\title[A Series Representation  for Solving  Models]{A Series Representation for Dynamic Economic Model Solutions: Regime Switching DSGE Models with Occasionally Binding Constraints }
%\subtitle{this is a subtitle}


\author{Gary S. Anderson}
\date{June 26, 2016} 


\frame{\titlepage}

\section{Introduction and Summary}

 \begin{itemize}
 \item Initial Diversion but Useful Payoffs
 \item Likely Useful for Wide class Dynamic Economic Models
   \begin{itemize}
   \item structure the solution allows attack many difficult problems
   \item bounded solution paths
   \item today focus on time invariant decision rules 
   \item splits problem into two phases
     \begin{itemize}
     \item solving a potentially difficult deterministic problem given a guess for conditional expectations
     \item updating the conditional expectations
   \end{itemize}
   \begin{itemize}
     \item occassionally binding constraints
     \item regime switching
   \end{itemize}
\end{itemize}
\end{itemize}

\section{The Series Representation}


\subsection{Linear Rational Expectation Solution Preliminaries}

\begin{frame}
  \frametitle{A {\em Linear Reference Model}}
For any linear homogeneous 
$L$ dimensional 
deterministic 
system 
\begin{gather}
  	 H_{-1} x_{t-1} + H_0 x_t + H_1 x_{t+1}=0\label{hSystem}
\end{gather}
with a unique stable solution\citep{anderson10}
\begin{gather}
	 H_{-1} x_{t-1} + H_0 x_t + H_1 x_{t+1}=\psi_\epsilon \epsilon +\psi_{c}\\
x_t=B x_{t-1} + \phi \psi_\epsilon \epsilon + (I - F)^{-1} \phi \psi_c
\intertext{where}
\phi= (H_0 +H_1 B)^{-1}  \text{ and } \,\,F=-\phi H_1 
\end{gather}
Define $\linMod \equiv \linModMats$.
\end{frame}

\begin{frame}
  \frametitle{Bounded Deterministic Paths}

Consider a family of functions:
 \begin{gather}
   \xWarg \in{R^L}\,\,\infNorm{\xWarg}  \le \bar{\mathcal{X}}\,\,\forall t\ > 0 \label{fFamily}.
 \end{gather}
 \begin{itemize}
 \item The $x_{-1}$ is an  $L$ dimensional state vector
 \item $\epsilon$ is a $K$ dimensional ``shock'' vector
 \item  together index individual trajectories for future state vectors.  
 \item No continuity or smoothness required  -- just boundedness
 \end{itemize}

\end{frame}

\begin{frame}
  
{\small
Define 
$  z_{t}(x_{t-1},\epsilon)$ as  %\footnote{These $z$ functions will soon prove useful in an algorithm for computing unknown trajectories like \refeq{fFamily}.}:
{

  \begin{align}
  z_{t}(x_{t-1},\epsilon) \equiv& H_{-1} \mathcal{X}_{t-1}(x_{t-1},\epsilon) + \nonumber\\
& H_0 \mathcal{X}_{t}(x_{t-1},\epsilon) +  \label{defZ} \\
& H_1 \mathcal{X}_{t+1}(x_{t-1},\epsilon). \nonumber
  \end{align}
}}
Then,
{\small
	 \begin{gather}
	 \mathcal{X}_{t}(x_{t-1},\epsilon) =B x_{t-1}+ \phi \psi_\epsilon\epsilon + (I - F)^{-1} \phi \psi_c +\\ \sum_{\sForSum=0}^\infty F^s \phi z_{t+\sForSum}(x_{t-1},\epsilon) \label{theSeries}
\intertext{and}
	 \mathcal{X}_{t+1}(x_{t-1},\epsilon) =B \mathcal{X}_{t} + \sum_{\sForSum =0}^\infty F^\sForSum \phi z_{t+1+\sForSum}(x_{t-1},\epsilon) + (I - F)^{-1} \phi \psi_c \,\,\,\forall t \ge  0.
	 \end{gather}
}

\end{frame}

\begin{frame}
\frametitle{Approximating $\mathcal{X}_t(x_{t-1},\epsilon)$} 

 	 \begin{gather}
 	 \xWargK \equiv B x_{t-1}+ \phi \psi_\epsilon\epsilon + \sum_{s=0}^k F^s \phi z_{t}(x_{t-1},\epsilon) + (I - F)^{-1} \phi \psi_c \label{theTruncSeries}
 \end{gather}

Since
{\small
    \begin{gather}
      \label{eq:1}
\sum_{s=k+1}^{\infty} F^s \phi \psi_z = (I -F)^{-1} F^{k+1}\phi \psi_z       \\
\infNorm{\xWarg-\xWargK} \le\\ \infNorm{(I -F)^{-1} F^{k+1}\phi \psi_z} \left ( \infNorm{H_{-1} }+ \infNorm{H_{0} }+ \infNorm{H_{1} } \right )\bar{\mathcal{X}}
    \end{gather}

}
\end{frame}

\begin{frame}
  
\subsection{A Simple Example: An ``Almost'' Arbitrary Linear Model and an ``Almost'' Arbitrary Family of Solution Paths}
\label{sec:almostarbitrary}


\frametitle{An ``Almost'' Arbitrary Linear Model}
\begin{gather}
  \begin{bmatrix}
H_{-1}&H_{0}&H_{1} 
  \end{bmatrix}=
\vcenter{\hbox{\includegraphics{refHmat.pdf}}}\intertext{with $\psi_c=\psi_\epsilon=0, \,\,  \psi_z=I$.
the series representation requires that the linear model
have a unique stable solution.}
  B=
\vcenter{\hbox{\includegraphics{refBmat.pdf}}}\\
\phi=
\vcenter{\hbox{\includegraphics{refPhimat.pdf}}}\\
F=
\vcenter{\hbox{\includegraphics{refFmat.pdf}}}
\end{gather} 

\end{frame}
\begin{frame}
  

\begin{figure}
  \centering
\includegraphics[width=1.1in]{piPath.pdf}
\includegraphics[width=1.1in]{oscillPath.pdf}
\includegraphics[width=1.1in]{pseudoPath.pdf}

\includegraphics[width=2in]{theZs.pdf}
\includegraphics[width=2in]{arbTruncErr.pdf}  
\caption{RBC Truncation Error Bound Versus Actual}
  \caption{State Variables and the  z's Corresponding to  $x_{-1}=(1,2,3),\epsilon=(2,1,2)$} \label{arbFig}
\end{figure}

\end{frame}

\begin{frame}
  
\begin{figure}
  \centering
  \caption{Error Bound Versus Actual Error} \label{figArbTrunc}
\end{figure}



\end{frame}


  \begin{frame}
    
\frametitle{ RBC Model Example}
  See for example\cite{Maliar2005}
 \begin{gather*}
   \max\left \{  u(c_t^t) + E_t \sum_{\tau=t}^\infty \beta \delta^{\tau+1-t}u(c_{\tau+1}^t)\right \}\\
c_\tau^t + k_\tau^{t+1}=(1-d)k_\tau^{t-1} + \theta_\tau f(k_\tau^{t-1})\\
f(k_\tau^{t-1})= k_\tau^\alpha
\intertext{with first order conditions}
\frac{1}{c_t^{\eta}}=\alpha \delta k_{t}^{\alpha-1} E_t \left (\frac{\theta_{t}}{c_{t+1}^\eta} \right ) \\
c_t + k_t=\theta_{t-1}k_{t-1}^\alpha \\
 \theta_t =\theta_{t-1}^\rho e^{\epsilon_t}\label{rbcSys}
 \end{gather*}

  \end{frame}
\begin{frame}
\frametitle{for $\eta=\delta=1$}


\begin{gather}
\frac{1}{c_t}=\alpha \delta k_{t}^{\alpha-1} E_t \left (\frac{\theta_{t}}{c_{t+1}} \right ) \\
c_t + k_t=\theta_{t-1}k_{t-1}^\alpha \\
\theta_t =\theta_{t-1}^\rho e^{\epsilon_t}\label{rbcSys}
\intertext{and there is a closed form solution}
  k_{t}= \alpha \delta \theta_{t} k_{t-1}^\alpha.\label{soln}\\
c_t=  (1-\alpha \delta) \theta_{t} k_{t-1}^\alpha
\end{gather}
  \end{frame}
\begin{frame}


For mean zero iid $\epsilon_t$ we can easily compute a family of trajectories like \refeq{fFamily}
\begin{gather}
  \begin{bmatrix}
c_{t+s}(k_{t-1},\theta_t,\epsilon_t)\\k_{t+s}(k_{t-1},\theta_t,\epsilon_t)    \\ \theta_{t+s}(\theta_{t-1},\theta_t,\epsilon_t)    
  \end{bmatrix}
\intertext{with conditional mean converging over time to }
  \begin{bmatrix}
    c_{ss}\\k_{ss}
  \end{bmatrix}=
  \begin{bmatrix}
\nu^\alpha-\nu\\ \nu
  \end{bmatrix}\intertext{where}
\nu= \alpha ^{\frac{1}{1-\alpha }} \delta ^{\frac{1}{1-\alpha }}
\end{gather}

\end{frame}


\begin{frame}
\frametitle{Consider  models that can be written in  the following form}


\begin{gather}
  h_i(x_{t-1},x_{t},x_{t+1},\epsilon_t)=h^{det}_{io}(x_{t-1},x_{t},\epsilon_t)+\\ \sum_{j=1}^{p_i} [h^{det}_{ij}(x_{t-1},x_{t},\epsilon_t)h^{nondet}_{ij}(x_{t},x_{t+1},\epsilon_t)]=0
\end{gather}

\begin{itemize}
\item models where expectations are computed at time t, $\epsilon_t$  known
\item this specification allows use of auxiliary variables for 
accurately computing expected values of nonlinear quatities.
\item if expected lagged values shocks known then deterministic system in L variables
\end{itemize}

\end{frame}


\begin{frame}
\frametitle{the example  model }
\label{sec:simple-rbc-model-ext} can be written as
\begin{gather}
h_{10^{det}}(\cdot)=\frac{1}{c_t},\,\,
h_{11}^{det}()=\alpha \delta k_{t}^{\alpha-1} ,\,\,
h_{11}^{nondet}(\cdot)=E_t \left (\frac{\theta_{t+1}}{c_{t+1}} \right )\\
h_{20}^{det}(\cdot)=c_t + k_t-\theta_tk_{t-1}^\alpha,\,\,
h_{21}^{det}(\cdot)=0\\
h_{30}^{det}(\cdot)=\ln \theta_t -(\rho \ln \theta_{t-1} + \epsilon_t),\,\,
h_{31}^{det}(\cdot)=0
\end{gather}

\end{frame}
\begin{frame}
  

\begin{description}
\item[Perfect Foresight]
\begin{gather}
     \mathcal{H}^{PF}[g^{k}(x,\epsilon_{t+T-k+1})]=
g^{k}(x,0)\\
\end{gather}


\item[Rational Expectations] 
\begin{gather}
     \mathcal{H}^{RE}[g^{k}(x,\epsilon_{t+T-k+1})]=
\mathcal{E}_t[g^{k}(x,\epsilon_{t+T-k+1})|x]\\
\end{gather}

 \end{description}


\end{frame}



\begin{frame}
\frametitle{Program Signatures}
\label{sec:program-listings}
\begin{tabular}{|l|c|}
\hline
Number of $x_t$ variables&$\numX$\\
\hline
Number of $z_t$ variables&$\numZ$\\
\hline
Number of $\epsilon_t$ variables&$\numEps$\\
\hline
Number of regimes variables&$\numR$\\
\hline
Number of recursive iterations&$\numIters$\\
\hline
Interpolation Grid Specification&$\dstSpec$\\
\hline
Shock Distributions  Specification&$\dstSpec$\\
\hline
The Linear Reference Model&$\linMod\equiv\linModMats$\\  
\hline
Model Equation Function&$\eqnFuncSig$\\
\hline
\end{tabular}

\end{frame}

\begin{frame}
  \begin{gather}
  g^k(x_{t-1},\epsilon_t)=
  \begin{bmatrix}
    x_{t-1}\\ x_t\\ \xtpguss
  \end{bmatrix}\\
x_t= \sum stuff,\,\,\,
\xtpguss=G^{k-1}(\xguss) = \sum stuff(\xguss)
  \end{gather}
\end{frame}

\begin{frame}
\frametitle{Solvers}
  \begin{gather}
    \slvr : \solverSig
  \end{gather}
\end{frame}


\begin{frame}
\frametitle{genLilXkZkFunc}
\label{sec:genlilxkzkfunc}
\begin{gather}
g^k(x_{t-1},\epsilon_t):\lilXFuncSig\\
G^k(x_{t-1})=\int g^k(x_{t-1},\epsilon_t):\lilXFuncSig\\
G^k(x_{t-1}):\bigXFuncSig\\
g^{k+1}(x_{t-1},\epsilon_t):\lilXFuncSig\\
  \Gamma_1(\linMod,\{G^k,\numTerms\},\xguss)
\end{gather}
{\small
\begin{gather*}
\linMod \times \{  \bigXFuncSig, k \} \times \Rn{(\numX+\numEps)} \rightarrow
\lilXFuncSig
\end{gather*}
}
\end{frame}

\begin{frame}
\frametitle{genFRFunc,genNSFunc}
\label{sec:genfrfunc}
  \begin{itemize}
  \item use accuracy requirement and truncation error formula to determine number of terms
  \item begin with the linear reference model to construct $xz^0$ and $XZ^0$
  \item construct an initial guess for a next $xz^{k+1}$ func which embeds
for any given $(x_{t-1},\epsilon_t)$ a guess for $x_t$ 
  \item use the solver to solve for tentative $z_t$ and consequently 
a tentative $x_t$ use this as a new guess.  
\item Repeat until guess doesn't change producing  a new $xz^{k+1}$
\item Compute $XZ^{k+1}$
\item Repeat above process until $xz$ and consequently $XZ$ don't change
  \end{itemize}
\end{frame}

\begin{frame}
  
\end{frame}

\begin{frame}
\frametitle{genFRFunc,genNSFunc}
\begin{gather*}
\{\numX,\numEps,\numZ\}\times(\lilXFuncSig)\times (\eqnFuncSig)    \rightarrow\\
\frfpnsFuncSig
\end{gather*}
\end{frame}

\begin{frame}
\frametitle{genFPFunc}
\label{sec:genfpfunc}
\begin{gather*}
\linMod \times \{  \bigXFuncSig, k \} \times (\eqnFuncSig)    \rightarrow\\ 
\frfpnsFuncSig
\end{gather*}

\end{frame}

\begin{frame}
\frametitle{genXZREInterpFunc}
\label{sec:genfpfunc}
\begin{gather*}
\{\numX,\numEps,\numZ\}\times(\lilXFuncSig)\times \grdSpec \times  \dstSpec   \rightarrow\\
\bigXFuncSig
\end{gather*}



\end{frame}
\begin{frame}
\frametitle{doIterREInterp}
\label{sec:doiterreinterp}

\begin{gather*}
  \linMod \times 
w\{(\lilXFuncSig)_1,\ldots,(\lilXFuncSig)_{\numR}\}  \\
 \times (\eqnFuncSig ) \times \grdSpec \times \dstSpec \rightarrow\\
\{\frfpnsFuncSig, \{(\lilXFuncSig)_1,\ldots,(\lilXFuncSig)_{\numR}\}\}
\end{gather*}



\end{frame}
\begin{frame}
\frametitle{nestIterREInterp}
\label{sec:nestiterreinterp}



\begin{gather*}
  \linMod \times 
\{(\lilXFuncSig)_1,\ldots,(\lilXFuncSig)_{\numR}\}  \\
 \times (\eqnFuncSig ) \times \grdSpec \times \dstSpec \times \numIters \rightarrow\\
\{\frfpnsFuncSig, \{(\lilXFuncSig)_1,\ldots,(\lilXFuncSig)_{\numR}\}\}
\end{gather*}



\end{frame}
\begin{frame}
\frametitle{genLilXkZkRegimeFuncs}
\label{sec:genlilxkzkregimefunc}
{\small
\begin{gather*}
\linMod \times \{(\{  \bigXFuncSig, k_i \})_{k_i=1},\ldots,(\{  \bigXFuncSig, k_i \})_{k_i=\numR}\} \times\\ \Rn{(\numX+\numEps)} \rightarrow\\
\{(\lilXRegimeFuncSig)_1,\ldots,(\lilXFuncSig)_{\numR}\}
\end{gather*}
}





\end{frame}
\begin{frame}
\frametitle{genFRRegimeFuncs, genNSRegimeFuncs}
\label{sec:genfrregimefunc}



{\small
\begin{gather*}
\{\numX,\numEps,\numZ\}\times\\
\{(\lilXRegimeFuncSig)_1,\ldots,(\lilXFuncSig)_{\numR}\}  \\
 \times (\eqnRegimeFuncSig )\rightarrow\\
\frfpnsRegimeFuncSig
\end{gather*}
}








\end{frame}
\begin{frame}
\frametitle{genFPRegimeFuncs}
\label{sec:genfpregimefunc}




{\small
\begin{gather*}
\{\numX,\numEps,\numZ\}\times\\
\{(\lilXRegimeFuncSig)_1,\ldots,(\lilXFuncSig)_{\numR}\}  \\
 \times (\eqnRegimeFuncSig )\rightarrow\\
\frfpnsRegimeFuncSig
\end{gather*}
}




\end{frame}
\begin{frame}
\frametitle{genXZREInterpRegimeFuncs}
\label{sec:genfpfunc}
\begin{gather*}
\{\numX,\numEps,\numZ\}\times(\lilXFuncSig)\times \dstSpec \times  \expctSpec   \rightarrow\\
\frfpnsFuncSig
\end{gather*}



\end{frame}
\begin{frame}
\frametitle{doIterREInterpRegime}
\label{sec:doiterreinterp}

\begin{gather*}
  \linMod \times 
w\{(\lilXRegimeFuncSig)_1,\ldots,(\lilXFuncSig)_{\numR}\}  \\
 \times (\eqnRegimeFuncSig ) \times \grdSpec \times \dstSpec \rightarrow\\
\{\frfpnsRegimeFuncSig, \{(\lilXRegimeFuncSig)_1,\ldots,(\lilXFuncSig)_{\numR}\}\}
\end{gather*}



\end{frame}
\begin{frame}
\frametitle{nestIterREInterpRegime}
\label{sec:nestiterreinterp}



\begin{gather*}
  \linMod \times 
w\{(\lilXRegimeFuncSig)_1,\ldots,(\lilXFuncSig)_{\numR}\}  \\
 \times (\eqnRegimeFuncSig ) \times \grdSpec \times \dstSpec  \times \numIters \rightarrow\\
\{\frfpnsRegimeFuncSig, \{(\lilXRegimeFuncSig)_1,\ldots,(\lilXFuncSig)_{\numR}\}\}
\end{gather*}


\end{frame}
\begin{frame}
\frametitle{Representing a Path}

\end{frame}
\begin{frame}
\frametitle{Representing a Family of Paths}

\end{frame}
\begin{frame}
\frametitle{Useful Properties of Series Solution}

\begin{itemize}
\item linear sum of functions
\item far away points matter less
\item Bounds on errors
\end{itemize}



\end{frame}
\begin{frame}
\frametitle{An RBC Model Example}
\end{frame}
\begin{frame}
\frametitle{Known Solution: Conditional Expectations Path}
\begin{itemize}
\item can easily compute series
\item algorithm can recover known solution
\end{itemize}


\end{frame}
\begin{frame}
\frametitle{UnKnown Solution: Conditional Expectations Path}
\begin{itemize}
\item can discover unknown solutions
\end{itemize}



\end{frame}
\begin{frame}
\frametitle{Occassionally Binding Constraints}
\begin{itemize}
\item can discover unknown solutions
\end{itemize}



\end{frame}
\begin{frame}
\frametitle{Regime Switching}
\begin{itemize}
\item can discover unknown solutions
\end{itemize}


\end{frame}
\begin{frame}
\frametitle{Regime Switching with Occassionally Binding Constraints}
\begin{itemize}
\item can discover unknown solutions
\end{itemize}

\end{frame}
\begin{frame}
\frametitle{Future Directions}
\end{frame}
\begin{frame}
  \frametitle{Future Directions}
  \begin{itemize}
  \item Heterogeneous Agents Problems
  \item Convergence Properties
    \begin{itemize}
    \item no solution
    \item multiple solutions
    \end{itemize}
\item algorithmic details
  \begin{itemize}
 \item Smolyak nodes
 \item Projection Methods
 \item Perturbation Methods for initial guess
  \end{itemize}
\item Insights for ``pruning''
  \end{itemize}
\end{frame}

\begin{frame}
  \frametitle{Bibliography}
  \bibliographystyle{plainnat}
\bibliography{anderson,files}

\end{frame}



\end{document}
