\documentclass[tikz]{beamer}

\usepackage{tikz}
\usetikzlibrary{shapes,arrows}
\mode<presentation>{}
\usepackage{beamerthemeshadow}

%\usepackage{pseudocode}
%\usepackage{dirtree}
% \usepackage[utf8]{inputenc}
% \usepackage{listings,xcolor}

% \usepackage{soul}
% \usepackage{ulem}



\usepackage{amsmath}
\usepackage{algorithm2e}
\usepackage{algorithmicx}
\usepackage{algpseudocode}

\usepackage[round,authoryear]{natbib}
%\usepackage[margin=1.0in]{geometry}
\usepackage{graphicx}
\usepackage{moreverb}
\usepackage{hyperref}
\newcommand\infNorm[1]{\left\lVert#1\right\rVert_\infty}

\usepackage{mathtools}
\mathtoolsset{showonlyrefs}
\usepackage{datetime}
\usepackage{amsfonts}

\newcommand{\xgusst}[1]{x^{guess}_{#1}}
\newcommand{\xnxt}[1]{x^{next}_{#1}}
\newcommand{\zgusst}[1]{z^{guess}_{#1}}
\newcommand{\xguss}{x^{tguess}}
\newcommand{\xtpguss}{x^{tp1guess}}



\newcommand{\Rn}[1]{{\mathcal{R}^{#1}}}
\newcommand{\numX}{{N_x}}
\newcommand{\numZ}{{N_z}}
\newcommand{\numEps}{{N_\epsilon}}
\newcommand{\numR}{{N_r}}
\newcommand{\numIters}{{K}}
\newcommand{\numTerms}{{N_{terms}}}


\newcommand{\ADRUE}{{\bf ADRUEF}}
\newcommand{\ADR}{{\bf ADRF}}

\newcommand{\xtmEpsArg}{{(x_{t-1},\epsilon_t)}} 
\newcommand{\xtArg}{{(x_{t}) }}


\newcommand{\sumLinPart}{{
B x_{t-1}+ \phi \psi_\epsilon\epsilon + (I - F)^{-1} \phi \psi_c 
}}
\newcommand{\sumZPart}{{
 \sum_{\sForSum=0}^\infty F^s \phi z_{t+\sForSum}(x_{t-1},\epsilon) 
}}
\newcommand{\sumZPartZero}{{
 \phi z_{t}(x_{t-1},\epsilon) 
}}
\newcommand{\sumZPartPos}{{
 \sum_{\sForSum=1}^\infty F^s \phi Z_{t+\sForSum}(x_{t-1},\epsilon) 
}}

\newcommand{\EsumLinPart}{{
	 B x_{t-1}+ (I - F)^{-1} \phi \psi_c 
}}
\newcommand{\EsumZPartZero}{{
	  \sum_{\sForSum=0}^\infty F^s \phi Z^{PF}_{t+\sForSum}(x_{t-1},0) 
}}
\newcommand{\EsumZPartEpsilon}{{
	  \sum_{\sForSum=0}^\infty F^s \phi \expct{z_{t+\sForSum}(x_{t-1},\epsilon)}
}}
\newcommand{\EsumCapZPart}[1]{{
	  \sum_{\sForSum=0}^\infty F^s \phi {Z^{#1}_{t+\sForSum}(x_{t-1})}
}}
\newcommand{\uncnxpt}[2]{{\mathcal{E}\left [ #1 \left | #2 \right . \right ]}}
\newcommand{\expct}[1]{{\mathcal{E}\left [ #1 xs\right ]}}

\newcommand{\xzFuncGuess}{{\mathbb{g}}}
\newcommand{\xzFuncGuessSig}{{(\Rn{\numX+\numEps+\numZ}\times 1)\rightarrow
(\Rn{({3\numX+\numEps+\numZ})}\times 1)}}


\newcommand{\xzFunc}{{{\gamma}}}
\newcommand{\xzFuncSig}{{(\Rn{\numX+\numEps+\numZ}\times 1)\rightarrow
(\Rn{({3\numX+\numEps+\numZ})}\times 1)}}

\newcommand{\XZFunc}{{\mathbb{G}}}
\newcommand{\bigXFuncSig}{{(\Rn{\numX}\times 1)\rightarrow
(\Rn{({\numX+\numZ})}\times 1)}}


\newcommand{\genXGFP}{{\mathcal{U}}}
\newcommand{\xgFP}{{\mathcal{V}}}
\newcommand{\genSlvr}{{\mathcal{T}}}
\newcommand{\slvr}{{\mathcal{S}}}
\newcommand{\solverSig}{{
\frfpnsRegimeFuncHelp
}}

\newcommand{\eqnFunc}{\mathbb{M}}
\newcommand{\eqnFuncSig}{{\Rn{3\numX+\numEps+\numZ}\rightarrow\Rn{\numX}}}


\newcommand{\frfpnsFuncSig}{{\Rn{\numX+\numEps}\rightarrow\Rn{\numX+\numZ}}}

\newcommand{\bigXRegimeFuncSig}{{(\Rn{\numX+1}\times 1)\rightarrow
(\Rn{({\numX+\numZ})}\times 1)}}
\newcommand{\lilXRegimeFuncSig}{{(\Rn{\numX+1+\numEps+\numZ}\times 1)\rightarrow
(\Rn{({3(\numX+1)+\numEps+\numZ})}\times 1)}}
\newcommand{\eqnRegimeFuncSig}{{
\{\Rn{3(\numX+1)+\numEps+\numZ}\rightarrow\Rn{\numX},
\ldots,
\Rn{3(\numX+1)+\numEps+\numZ}\rightarrow\Rn{\numX}\}
}}
\newcommand{\frfpnsRegimeFuncHelp}{{\Rn{\numX+\numEps}\rightarrow\Rn{\numX+\numZ}}}

\newcommand{\frfpnsRegimeFuncSig}{{
\{ (\frfpnsRegimeFuncHelp{1}), \ldots ,  (\frfpnsRegimeFuncHelp{\numR})\}
}}

\newcommand{\dstSpec}{{\mathbf{distSpec}}}
\newcommand{\expctSpec}{{\mathbf{expctSpec}}}
\newcommand{\grdSpec}{{\mathbf{grdSpec}}}


\newcommand{\linMod}{{\mathcal{L}}}
\newcommand{\linModMats}{{\{H,\psi_\epsilon,\psi_c;B,\phi,F\}}}
\newcommand{\sForSum}{{\nu}}

\newcommand{\xWarg}{   \mathcal{X}_{t}(x_{-1},\epsilon)}
\newcommand{\xWargK}{   \hat{\mathcal{X}}_{t}(x_{-1},\epsilon,k)}


\newcommand{\xIter}[2]{\mathcal{X}^{#1}(#2)}
\newcommand{\xNow}[1]{x^{#1}_t(x_{t-1},\epsilon_t)}
\newcommand{\zNow}[1]{z^{#1}(x_{t-1},\epsilon_t)}
\newcommand{\xNowtp}[1]{x^{#1}_{t+1}(x_{t-1},\epsilon_t)}
\newcommand{\XNow}[3]{\mathcal{X}^{#1}_{#2}(x_{#3})}
\newcommand{\ZNow}[3]{\mathcal{Z}^{#1}(x_{#3})}


\newcommand{\rcpC}{{\mathbf{N}}}

\begin{document}
\title[A Series Representation  for Solving  Models]{A Series Representation for Dynamic Economic Model Solutions: Regime Switching DSGE Models with Occasionally Binding Constraints }
%\subtitle{this is a subtitle}


\author{Gary S. Anderson}
\date{June 26, 2016} 


\frame{\titlepage}

\section{Introduction and Summary}

 \begin{itemize}
 \item Initial Diversion but Useful Payoffs
 \item Likely Useful for Wide class Dynamic Economic Models
   \begin{itemize}
   \item structure the solution allows attack many difficult problems
   \item bounded solution paths
   \item today focus on time invariant decision rules 
   \item splits problem into two phases
     \begin{itemize}
     \item solving a potentially difficult deterministic problem given a guess for conditional expectations
     \item updating the conditional expectations
   \end{itemize}
   \begin{itemize}
     \item occassionally binding constraints
     \item regime switching
   \end{itemize}
\end{itemize}
\end{itemize}

\section{The Series Representation}


\subsection{Linear Rational Expectation Solution Preliminaries}

\begin{frame}
  \frametitle{A {\em Linear Reference Model}}
For any linear homogeneous 
$L$ dimensional 
deterministic 
system 
\begin{gather}
  	 H_{-1} x_{t-1} + H_0 x_t + H_1 x_{t+1}=0\label{hSystem}
\end{gather}
with a unique stable solution\citep{anderson10}
\begin{gather}
	 H_{-1} x_{t-1} + H_0 x_t + H_1 x_{t+1}=\psi_\epsilon \epsilon +\psi_{c}\\
x_t=B x_{t-1} + \phi \psi_\epsilon \epsilon + (I - F)^{-1} \phi \psi_c
\intertext{where}
\phi= (H_0 +H_1 B)^{-1}  \text{ and } \,\,F=-\phi H_1 
\end{gather}
Define $\linMod \equiv \linModMats$.
\end{frame}

\begin{frame}
  \frametitle{Bounded Deterministic Paths}

Consider a family of functions:
 \begin{gather}
   \xWarg \in{R^L}\,\,\infNorm{\xWarg}  \le \bar{\mathcal{X}}\,\,\forall t\ > 0 \label{fFamily}.
 \end{gather}
 \begin{itemize}
 \item The $x_{-1}$ is an  $L$ dimensional state vector
 \item $\epsilon$ is a $K$ dimensional ``shock'' vector
 \item  together index individual trajectories for future state vectors.  
 \item No continuity or smoothness required  -- just boundedness
 \end{itemize}

\end{frame}

\begin{frame}
  
{\small
Define 
$  z_{t}(x_{t-1},\epsilon)$ as  %\footnote{These $z$ functions will soon prove useful in an algorithm for computing unknown trajectories like \refeq{fFamily}.}:
{

  \begin{align}
  z_{t}(x_{t-1},\epsilon) \equiv& H_{-1} \mathcal{X}_{t-1}(x_{t-1},\epsilon) + \nonumber\\
& H_0 \mathcal{X}_{t}(x_{t-1},\epsilon) +  \label{defZ} \\
& H_1 \mathcal{X}_{t+1}(x_{t-1},\epsilon). \nonumber
  \end{align}
}}
Then,
{\small
	 \begin{gather}
	 \mathcal{X}_{t}(x_{t-1},\epsilon) =B x_{t-1}+ \phi \psi_\epsilon\epsilon + (I - F)^{-1} \phi \psi_c +\\ \sum_{\sForSum=0}^\infty F^s \phi z_{t+\sForSum}(x_{t-1},\epsilon) \label{theSeries}
\intertext{and}
	 \mathcal{X}_{t+1}(x_{t-1},\epsilon) =B \mathcal{X}_{t} + \sum_{\sForSum =0}^\infty F^\sForSum \phi z_{t+1+\sForSum}(x_{t-1},\epsilon) + (I - F)^{-1} \phi \psi_c \,\,\,\forall t \ge  0.
	 \end{gather}
}

\end{frame}

\begin{frame}
\frametitle{Approximating $\mathcal{X}_t(x_{t-1},\epsilon)$} 

 	 \begin{gather}
 	 \xWargK \equiv B x_{t-1}+ \phi \psi_\epsilon\epsilon + \sum_{s=0}^k F^s \phi z_{t}(x_{t-1},\epsilon) + (I - F)^{-1} \phi \psi_c \label{theTruncSeries}
 \end{gather}

Since
{\small
    \begin{gather}
      \label{eq:1}
\sum_{s=k+1}^{\infty} F^s \phi \psi_z = (I -F)^{-1} F^{k+1}\phi \psi_z       \\
\infNorm{\xWarg-\xWargK} \le\\ \infNorm{(I -F)^{-1} F^{k+1}\phi \psi_z} \left ( \infNorm{H_{-1} }+ \infNorm{H_{0} }+ \infNorm{H_{1} } \right )\bar{\mathcal{X}}
    \end{gather}

}
\end{frame}

\begin{frame}
  
\subsection{A Simple Example: An ``Almost'' Arbitrary Linear Model and an ``Almost'' Arbitrary Family of Solution Paths}
\label{sec:almostarbitrary}


\frametitle{An ``Almost'' Arbitrary Linear Model}
\begin{gather}
  \begin{bmatrix}
H_{-1}&H_{0}&H_{1} 
  \end{bmatrix}=
\vcenter{\hbox{\includegraphics{refHmat.pdf}}}\intertext{with $\psi_c=\psi_\epsilon=0, \,\,  \psi_z=I$.
the series representation requires that the linear model
have a unique stable solution.}
  B=
\vcenter{\hbox{\includegraphics{refBmat.pdf}}}\\
\phi=
\vcenter{\hbox{\includegraphics{refPhimat.pdf}}}\\
F=
\vcenter{\hbox{\includegraphics{refFmat.pdf}}}
\end{gather} 

\end{frame}
\subsection{Representing a Path}

\subsection{Representing a Family of Paths}

\subsection{Useful Properties of Series Solution}

\begin{itemize}
\item linear sum of functions
\item far away points matter less
\item Bounds on errors
\end{itemize}


\section{An RBC Model Example}
\subsection{Known Solution: Conditional Expectations Path}
\begin{itemize}
\item can easily compute series
\item algorithm can recover known solution
\end{itemize}


\subsection{UnKnown Solution: Conditional Expectations Path}
\begin{itemize}
\item can discover unknown solutions
\end{itemize}



\subsection{Occassionally Binding Constraints}
\begin{itemize}
\item can discover unknown solutions
\end{itemize}



\subsection{Regime Switching}
\begin{itemize}
\item can discover unknown solutions
\end{itemize}


\subsection{Regime Switching with Occassionally Binding Constraints}
\begin{itemize}
\item can discover unknown solutions
\end{itemize}

\section{Future Directions}

\begin{frame}
  \frametitle{Future Directions}
  \begin{itemize}
  \item Heterogeneous Agents Problems
  \item Convergence Properties
    \begin{itemize}
    \item no solution
    \item multiple solutions
    \end{itemize}
\item algorithmic details
  \begin{itemize}
 \item Smolyak nodes
 \item Projection Methods
 \item Perturbation Methods for initial guess
  \end{itemize}
\item Insights for ``pruning''
  \end{itemize}
\end{frame}

\begin{frame}
  \frametitle{Bibliography}
  \bibliographystyle{plainnat}
\bibliography{anderson,files}

\end{frame}



\end{document}
