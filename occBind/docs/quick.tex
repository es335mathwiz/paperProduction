\documentclass[letter]{beamer}
\newcommand{\xtFuncTI}{\mathcal{X}(x,\epsilon)}
\newcommand{\XtFuncTI}{\mathbf{X}(x)}

\newcommand{\xtFunc}[1]{\mathcal{X}{#1}}
\newcommand{\XtFunc}[1]{\mathbf{X}{#1}}



\newcommand{\discr}[1]{\mathcal{D}^{#1}(x_{t-1},\epsilon_t)}

\newcommand{\XZPair}[1]{(\mathcal{X}^{#1},\mathcal{Z}^{#1})}
\newcommand{\XZPairG}[1]{(\mathcal{X}^{#1}(x^g),\mathcal{Z}^{#1}(x^g))}
\newcommand{\xIter}[2]{\mathcal{X}^{#1}(#2)}
%\newcommand{\zNow}[1]{z^{#1}_0(x_{t-1},\epsilon_t)}
%\newcommand{\ZNow}[3]{\mathcal{Z}^{#1}_{#2}(x_{#3})}
\newcommand{\zNow}[1]{z^{#1}(x_{t-1},\epsilon_t)}
\newcommand{\ZNow}[3]{\mathcal{Z}^{#1}(x_{#3})}

\newcommand{\xNow}[1]{x^{#1}_t(x_{t-1},\epsilon_t)}
\newcommand{\xNowtp}[1]{x^{#1}_{t+1}(x_{t-1},\epsilon_t)}
\newcommand{\XNow}[3]{\mathcal{X}^{#1}_{#2}(x_{#3})}








\newcommand{\sForSum}{{\nu}}
\newcommand{\rcpC}{{\mathbf{c}}}

\newcommand{\xtVec}{  \begin{bmatrix}
    q_t\\r_{t}\\r_{ut}
  \end{bmatrix}
}
\newcommand{\xtPVec}{  \begin{bmatrix}
    q_{t+1}\\r_{t+1}\\r_{ut+1}
  \end{bmatrix}
}
\newcommand{\xtMVec}{  \begin{bmatrix}
    q_{t-1}\\r_{t-1}\\r_{ut-1}
  \end{bmatrix}
}
\newcommand{\expctEps}[1]{\mathcal{E}_{\epsilon} \left [#1 \right ]}
\newcommand{\expct}[2]{E_{#1} \left [#2 \right ]}
\newcommand{\expc}[1]{\mathcal{E} \left [#1 \right ]}
\newcommand{\expcK}[2]{\mathcal{E}^{#1} \left [#2 \right ]}
\newcommand{\xsln}[1]{\mathbb{X} \left [#1 \right ]}
\newcommand{\xslnK}[2]{\hat{\mathbb{X}}^{#1} \left [#2 \right ]}
\newcommand{\xpth}[2]{\mathfrak{X}_{#1} \left [#2 \right ]}
\newcommand\infNorm[1]{\left\lVert#1\right\rVert_\infty}
\newcommand\twoNorm[1]{\left\lVert#1\right\rVert_2}
%\newcommand{\inorm}[1]{\left\lVert#1\right\rVert_\infty}

% \newcommand{\forPhi}{\begin{bmatrix}
% \psi_\epsilon&\psi_z
% \end{bmatrix}}
% \newcommand{\phiMult}{\phi \psi_\epsilon}
% \newcommand{\bMult}{B x_{-1} + \phiMult}
\newcommand{\phiMultBoth}[1]{
	\phi (\psi_\epsilon \epsilon_t +\psi_z z_0^#1(x_{-1},\epsilon_t))}
\newcommand{\bMultBoth}[1]{B x_{-1} + \phiMultBoth{#1}}


\newcommand{\bForOne}{\bMultBoth{1}
}

% \newcommand{\bForTwo}{\bMultBoth{2}+
% F \phi  \psi_z  
% Z_0^1(x_0^2(x_{-1}))   
% }



\newcommand{\compSlack}{z_0^1(x_{-1},\epsilon_t) \left ( \bar{x} -x\right )=0\\ z_0^1(x_{-1},\epsilon_t)> 0}
% \begin{gather*}
% 0= x_t-(B x_{t-1}+ \phi \psi_\epsilon\epsilon_t + \phi \psi_z 
% \xpt{z_{t}(x_{t-1},\epsilon_t)    } )
% \end{gather*}

\newcommand{\xpt}[1]{#1}


\makeatletter
\@ifundefined{newblock}{%
 \def\newblock{\hskip .11em plus .33em minus .07em} % important line
}
\makeatother


\makeatletter
\pgfplotsset{
    /pgfplots/table/omit header/.style={%
        /pgfplots/table/typeset cell/.append code={%
            \ifnum\c@pgfplotstable@rowindex=-1
                \pgfkeyslet{/pgfplots/table/@cell content}\pgfutil@empty%
            \fi
        }
    }
}
\makeatother
 

\makeatletter
\newcommand*\ExpandableInput[1]{\@@input#1 }
\makeatother


\newcommand{\tpExp}[1]{\phi^e_{#1}(x_{t-1})}
\newcommand{\lRat}[2]{\log \left( \frac{z_{#1}}{z_{#2}}\right)}


\newcommand{\anEdit}[1]
{

{\color{blue}
\begin{quote}
#1  
\end{quote}}

}

\begin{document}
\begin{frame}
  \frametitle{My algorithm is based on basic properties of deterministic maps.}
  \begin{itemize}
  \item     The $\epsilon$'s are causing 
confusion because they play two distinct roles.
To see the two roles, consider a family of stochastic functions 
$\{\stochF{T},\ldots,\stochF{0}\}$ with 
$\stochF{T-t}(x,\epsilon_t): {\mathcal{R}}^{L+1} \rightarrow \mathcal{R}^L$ along with $\epsilon_t \sim $ iid.
    \begin{enumerate}
\item The distribution of the $\epsilon_t$ makes it possible to compute 
a family of deterministic functions $\detF{k}(x)=E( \stochF{k}(x,\epsilon_t)|x) $
    \item at some initial time, say t=0, an $\epsilon_0$ 
along with an $x_{-1}$, determines a unique $x_0=\stochF{T}(x_{-1},\epsilon_0)$.
Subsequently, $x_{t}=\detF{{T-t}}(x_{t-1})\,\, \forall t>0$
    \end{enumerate}
 \item The series representation requires a unique trajectory beginning with $x_0=\stochF{T}(x_{-1},\epsilon_0)$ and 
evolving forward from any given $x_0$:  $x_t=\detF{T-t}(x_{t-1})\,\, \forall t>0$. 
  \item The algorithm for discovering unknown solutions does not rely on time invariance. 
\item However, when it converges, the algorithm constructs a time invariant function.
  \end{itemize}
    
  \end{frame}

%\footnote{We set $\psi_c=0$  without loss of generality 
%in the expressions that follow merely to save on notation.}

\begin{frame}
\frametitle{The reference model and the target model}
\begin{itemize}
\item The target model is of the form
\begin{gather*}
\nlEqnLHS{x_{t-1}}{x_t}{x_{t+1}}{t}=\\
\nlEqn{x_{t-1}}{x_t}{x_{t+1}}{t}=0\\
 (L \times 1) +  (L \times L ) \cdot (L \times 1)
\end{gather*}
\item The algorithm begins with solutions satisfying 
the reference model and
constructs a sequence of functions
 honoring the target model equations for successively 
longer solution horizons.

\item It will be useful to express the sequence of stochastic 
functions using
  \begin{gather*}
	 \stochF{k}(x_{t-1},\epsilon_t) =B x_{t-1}+ \phi \psi_\epsilon\epsilon_t + \\ \sum_{\sForSum=0}^\infty F^s \phi z_{t+\sForSum}(x_{t-1},\epsilon_t) + (I - F)^{-1} \phi \psi_c
\label{theSeries}
  \end{gather*}
  \end{itemize}
\end{frame}

\begin{frame}
\frametitle{$k=0$ completely ignore the target model}
  \begin{itemize}
\item 
  \begin{gather*}
\stochF{0}()=B x_{-1} +\phi \psi_\epsilon \epsilon_0 + \phi z_0^0()\\
x_{1}^0=B x_0^0+\phi \psi_\epsilon \epsilon_1 \intertext{so that since $E_t \epsilon_{t+s}=0 \,\,\forall s>0$}
E_0x_{1}^0=B x_0^0\intertext{ can compute conditional expectations.}
M_0(x)=B x_0^0, \, 
Z_0(x)=0
  \end{gather*}
  \end{itemize}
\end{frame}



  \begin{frame}

{\small
 
  \begin{gather*}
	 \mathcal{X}_{t}(x_{t-1},\epsilon_t) =B x_{t-1}+ \phi \psi_\epsilon\epsilon_t + \sum_{\sForSum=0}^\infty F^s \phi z_{t+\sForSum}(x_{t-1},\epsilon_t) + (I - F)^{-1} \phi \psi_c
\label{theSeries}\intertext{and}
	 \mathcal{X}_{t+s+1}(x_{t-1},\epsilon_t) =B \mathcal{X}_{t+s} + \sum_{\sForSum =0}^\infty F^\sForSum \phi z_{t+s+\sForSum}(x_{t-1},\epsilon_t) + (I - F)^{-1} \phi \psi_c \,\,\,\forall s \ge  0
  \end{gather*}

Must write equations in the form: 

\begin{gather*}
\nlEqnLHS{x_{t-1}}{x_t}{x_{t+1}}{t}=\\
\nlEqn{x_{t-1}}{x_t}{x_{t+1}}{t}=0\\
 (L \times 1) +  (L \times L ) \cdot (L \times 1)
\end{gather*}
}

\end{frame}

\begin{frame}

{\tiny

  \begin{itemize}
  \item $k=1$ assume $z_k()=0, \, \forall k>0$
  \begin{gather*}
x_0^1()=m_1()=B x_{-1} +\phi \psi_\epsilon \epsilon_0 + \phi z_0^1()\\
\intertext{ we use the $k=0$ solution in effect imposing the linear model for the evolution of the model equations }
E_0(x_1^1)=M_0(x_0^1())=B x_0^1()\intertext{We use the deterministic equation}
\nlEqn{x_{-1}}{x_0^1}{x_{1}^1}{0} 
\intertext{to solve for } 
z_0^1() \text{ and }  x_0^1()\intertext{We can compute  the deterministic 
conditional expectations functions 
for $x_0$ and $z_0$ given some $x$ but prior to knowlege of $\epsilon_0$.}
X_1(x) \text{ and }  Z_1(x)
  \end{gather*}
  \end{itemize}
}
\end{frame}



\begin{frame}

{\tiny

  \begin{itemize}
  \item $k=2$ assume $z_k()=0, \, \forall k>1$.
We can use our $k=1$ solution to write:
  \begin{gather*}
x_0^2()=m_2()=B x_{-1} +\phi \psi_\epsilon \epsilon_0 + \phi z_0^2()+
 F \phi Z_1(x_0^2())\\
E_0(x_1^2())=M_1(x_0^2()) \\
\nlEqn{x_{-1}}{x_0^2}{M_1(x_0^2())}{0} \\
  \end{gather*}
  \item $k=3$ assume $z_k()=0, \, \forall k>2$.
We can use our $k=2$ solution to write:
  \begin{gather*}
x_0^3()=m_2()=B x_{-1} +\phi \psi_\epsilon \epsilon_0 + \phi z_0^3()+
 F \phi Z_2(x_0^3())+
 F^2 \phi Z_1(x_1^3())\\
E_0(x_1^3())=M_1(x_0^3()) \\
\nlEqn{x_{-1}}{x_0^3}{M_1(x_0^3())}{0} \\
  \end{gather*}
  \end{itemize}
}
\end{frame}



\begin{frame}

{\tiny

  \begin{itemize}
  \item $k=k^\ast$ assume $z_k()=0, \, \forall k>k^\ast$.
We can use our $k^\ast$ solution to write:
  \begin{gather*}
x_0^{k^\ast+1}()=m_{k^\ast+1}()=B x_{-1} +\phi \psi_\epsilon \epsilon_0 + \phi z_0^{k^\ast+1}()+
\sum_{s=1}^{k^\ast}  F^s \phi Z_s(x_s^{k^\ast+1}())\\
E_0(x_1^{k^\ast + 1}())=M_{k^\ast}(x_0^{k^\ast}()) \\
\nlEqn{x_{-1}}{x_0^2}{M_{k^\ast}(x_0^{k^\ast+1}())}{0} \\ \intertext{where the $Z$ are evaluated along the  conditional expectations path}
x_1^{k^\ast+1}=  X_{k^\ast}(x_0^{k\ast+1})\\ x_2^{k^\ast+1}=  X_{k^\ast-1}(X_{k^\ast}(x_0^{k\ast+1}))\\ \vdots \\x_{k^\ast}^{k^\ast+1}= X_{1}(X_{2}(\ldots( X_{k^\ast-1}(X_{k^\ast}(x_0^{k\ast+1}))))
  \end{gather*}
  \end{itemize}
}
\end{frame}



\begin{frame}

{\tiny

  \begin{itemize}
  \item $k=1$ assume $z_k()=0, \, \forall k>1$
  \begin{gather*}
x_0^1()=m_1()=B x_{-1} +\phi \psi_\epsilon \epsilon_0 + \phi z_0^1()+ F \phi z_1^1()\\
x_1^1()=B x_0^1()+\phi \psi_\epsilon \epsilon_1+\phi z_1^1()\\
x_2^1()=B x_1^1()+\phi \psi_\epsilon \epsilon_2\intertext{solves}
\nlEqn{x_{-1}}{x_0^1}{x_{1}^1}{0} \\
\nlEqn{x_0}{x_1^1}{x_2^1}{1}
\intertext{we can use our $k=0$ solution in the second equation}
x_1^1(x_0,\epsilon_1) = x_0^0(x_0,\epsilon_1),\,\,
z_1^1(x_0,\epsilon_1) = z_0^0(x_0,\epsilon_1) \intertext{and}
E_0(x_1^1(x_0,\epsilon_1)) = E_0( m_0(x_0,\epsilon_1))=
E_0(B x_0 +\phi \psi_\epsilon \epsilon_1 + \phi z_0^0(x_0,\epsilon_1))\\
E_0(x_1^1(x_0,\epsilon_1)) =B x_0 + \phi E_0(z_0^0(x_0,\epsilon_1))=\\
B x_0^1() + \phi Z_0^0(x_0)=
B x_0^1() + \phi Z_0^0(x_0^1()) \intertext{leaving  a deterministic equation to solve for $z_0^1()$ (and consequently $x_0^1()$)}
  \end{gather*}
  \end{itemize}
}
\end{frame}



\begin{frame}

 Denote  $f(x_{t-1},\epsilon_t) \equiv f()$
Construct a sequence of function $(m_k(),z_k())$ That compute conditional expectations paths honoring successively longer soluiton horizons:\footnote{We set $\psi_c=0$  without loss of generality 
in the expressions that follow merely to save on notation.}

  \begin{itemize}
  \item $k=0$ assume $z_k()=0, \, \forall k>0$
  \begin{gather*}
x_0^0()=m_0()=B x_{-1} +\phi \psi_\epsilon \epsilon_0 + \phi z_0^0()\\
x_{1}^0=B x_0^0+\phi \psi_\epsilon \epsilon_1 \intertext{so that since $E_t \epsilon_{t+s}=0 \,\,\forall s>0$}
E_0x_{1}^0=B x_0^0()\intertext{compute $z_0^0()$ ( 
and consequently $x_0^0$ ) that solves the deterministic equation}
\nlEqn{x_{-1}}{x_0^0}{x_{1}^0}{0}=0 
  \end{gather*}
  \end{itemize}
% \forall s>0$,

\end{frame}



\begin{frame}

{\tiny

  \begin{itemize}
  \item $k=2$ assume $z_k()=0, \, \forall k>2$
  \begin{gather*}
x_0^2()=B x_{-1} +\phi \psi_\epsilon \epsilon_0 + \phi z_0^2()+ F \phi z_1^2()+ F \phi z_2^2()\\
x_1^2()=B x_0^2()+\phi \psi_\epsilon \epsilon_1+\phi z_1^2()+F\phi z_2^2()\\
x_2^2()=B x_0^2()+\phi \psi_\epsilon \epsilon_2+\phi z_2^2()\\
x_3^2()=B x_0^2()+\phi \psi_\epsilon \epsilon_3
\intertext{solves}
\nlEqn{x_{-1}}{x_0^2}{x_{1}^2}{0} \\
\nlEqn{x_0}{x_1^2}{x_2^2}{1}\\
\nlEqn{x_1}{x_2^2}{x_3^2}{2}
\intertext{so that}
x_1^2(x_0,\epsilon_1) = x_0^0(x_0,\epsilon_1),\,\,
z_1^2(x_0,\epsilon_1) = z_0^0(x_0,\epsilon_1) \\
x_2^2(x_1,\epsilon_2) = x_0^0(x_1,\epsilon_2),\,\,
z_2^2(x_1,\epsilon_2) = z_0^0(x_1,\epsilon_2) \intertext{and}
E_0(x_1^2(x_0,\epsilon_1)) = E_0( x_0^0(x_0,\epsilon_1))=
E_0(B x_0 +\phi \psi_\epsilon \epsilon_1 + \phi z_0^0(x_0,\epsilon_1))\\
E_0(x_1^2(x_0,\epsilon_1)) =B x_0 + \phi E_0(z_0^0(x_0,\epsilon_1))=\\
B x_0^2() + \phi Z_0^0(x_0)=
B x_0^2() + \phi Z_0^0(x_0^2())
  \end{gather*}
  \end{itemize}
}
\end{frame}



\begin{frame}

{\small

  \begin{itemize}
  \item Given $k$
  \begin{gather*}
x_0^k()=B x_{-1} +\phi \psi_\epsilon \epsilon_0 + \sum_{s=0}^k F^s \phi z_s^k()\\
x_1^k()=B x_0()+\phi \psi_\epsilon \epsilon_1+ \sum_{s=1}^k F^{s-1} \phi z_s^k()\\
\vdots\\
x_k^k()=B x_{k-1}()+\phi \psi_\epsilon \epsilon_k+ \phi z_k^k()\intertext{solves}
\nlEqn{x_{-1}}{x_0^k}{x_{1}^k}{0} \\
\nlEqn{x_0}{x_1^k}{x_2^k}{1}\\
\vdots \\
\nlEqn{x_{k-1}}{x_k^k}{x_{k+1}^k}{k}
  \end{gather*}
  \end{itemize}
}
\end{frame}

\begin{frame}


  \begin{gather*}
    x_k^k()=x_0^0(x_{k-1}^k()),     z_k^k()=z_0^0(x_{k-1}^k())\\
  \end{gather*}

  Construct

  \begin{gather*}
x_0^{k+1}()=B x_{-1} +\phi \psi_\epsilon \epsilon_0 + \sum_{s=0}^{k+1} F^{s} \phi z_s^{k}()\\
  \end{gather*}
\end{frame}
% \begin{frame}
%   \frametitle{The law I don't use}
%    \begin{itemize}
%  \item Law of Iterated Expectations applies
%    \begin{gather*}
%      E_t(\mathcal{X}(x_{t+k-1},\epsilon_{t+k}))=
%      E_t(E_{t+k}(\mathcal{X}(x_{t+k-1},\epsilon_{t+k})))
%    \end{gather*}
%   \item I don't not need or use the ``Law of Iterated Expectations'' to construct solutions.  One could interpret  result of the computations produces as obeying a version of the Law, but. in hindsight,
%  it does not appear to be a useful concept for 
% developing or explaining the algorithm
%   \item My series representation is not connected to time invariance

%  \end{itemize}

% \end{frame}



\end{document}
