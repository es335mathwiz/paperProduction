\documentclass[letter]{beamer}
\newcommand{\xtFuncTI}{\mathcal{X}(x,\epsilon)}
\newcommand{\XtFuncTI}{\mathbf{X}(x)}

\newcommand{\xtFunc}[1]{\mathcal{X}{#1}}
\newcommand{\XtFunc}[1]{\mathbf{X}{#1}}



\newcommand{\discr}[1]{\mathcal{D}^{#1}(x_{t-1},\epsilon_t)}

\newcommand{\XZPair}[1]{(\mathcal{X}^{#1},\mathcal{Z}^{#1})}
\newcommand{\XZPairG}[1]{(\mathcal{X}^{#1}(x^g),\mathcal{Z}^{#1}(x^g))}
\newcommand{\xIter}[2]{\mathcal{X}^{#1}(#2)}
%\newcommand{\zNow}[1]{z^{#1}_0(x_{t-1},\epsilon_t)}
%\newcommand{\ZNow}[3]{\mathcal{Z}^{#1}_{#2}(x_{#3})}
\newcommand{\zNow}[1]{z^{#1}(x_{t-1},\epsilon_t)}
\newcommand{\ZNow}[3]{\mathcal{Z}^{#1}(x_{#3})}

\newcommand{\xNow}[1]{x^{#1}_t(x_{t-1},\epsilon_t)}
\newcommand{\xNowtp}[1]{x^{#1}_{t+1}(x_{t-1},\epsilon_t)}
\newcommand{\XNow}[3]{\mathcal{X}^{#1}_{#2}(x_{#3})}








\newcommand{\sForSum}{{\nu}}
\newcommand{\rcpC}{{\mathbf{c}}}

\newcommand{\xtVec}{  \begin{bmatrix}
    q_t\\r_{t}\\r_{ut}
  \end{bmatrix}
}
\newcommand{\xtPVec}{  \begin{bmatrix}
    q_{t+1}\\r_{t+1}\\r_{ut+1}
  \end{bmatrix}
}
\newcommand{\xtMVec}{  \begin{bmatrix}
    q_{t-1}\\r_{t-1}\\r_{ut-1}
  \end{bmatrix}
}
\newcommand{\expctEps}[1]{\mathcal{E}_{\epsilon} \left [#1 \right ]}
\newcommand{\expct}[2]{E_{#1} \left [#2 \right ]}
\newcommand{\expc}[1]{\mathcal{E} \left [#1 \right ]}
\newcommand{\expcK}[2]{\mathcal{E}^{#1} \left [#2 \right ]}
\newcommand{\xsln}[1]{\mathbb{X} \left [#1 \right ]}
\newcommand{\xslnK}[2]{\hat{\mathbb{X}}^{#1} \left [#2 \right ]}
\newcommand{\xpth}[2]{\mathfrak{X}_{#1} \left [#2 \right ]}
\newcommand\infNorm[1]{\left\lVert#1\right\rVert_\infty}
\newcommand\twoNorm[1]{\left\lVert#1\right\rVert_2}
%\newcommand{\inorm}[1]{\left\lVert#1\right\rVert_\infty}

% \newcommand{\forPhi}{\begin{bmatrix}
% \psi_\epsilon&\psi_z
% \end{bmatrix}}
% \newcommand{\phiMult}{\phi \psi_\epsilon}
% \newcommand{\bMult}{B x_{-1} + \phiMult}
\newcommand{\phiMultBoth}[1]{
	\phi (\psi_\epsilon \epsilon_t +\psi_z z_0^#1(x_{-1},\epsilon_t))}
\newcommand{\bMultBoth}[1]{B x_{-1} + \phiMultBoth{#1}}


\newcommand{\bForOne}{\bMultBoth{1}
}

% \newcommand{\bForTwo}{\bMultBoth{2}+
% F \phi  \psi_z  
% Z_0^1(x_0^2(x_{-1}))   
% }



\newcommand{\compSlack}{z_0^1(x_{-1},\epsilon_t) \left ( \bar{x} -x\right )=0\\ z_0^1(x_{-1},\epsilon_t)> 0}
% \begin{gather*}
% 0= x_t-(B x_{t-1}+ \phi \psi_\epsilon\epsilon_t + \phi \psi_z 
% \xpt{z_{t}(x_{t-1},\epsilon_t)    } )
% \end{gather*}

\newcommand{\xpt}[1]{#1}


\makeatletter
\@ifundefined{newblock}{%
 \def\newblock{\hskip .11em plus .33em minus .07em} % important line
}
\makeatother


\makeatletter
\pgfplotsset{
    /pgfplots/table/omit header/.style={%
        /pgfplots/table/typeset cell/.append code={%
            \ifnum\c@pgfplotstable@rowindex=-1
                \pgfkeyslet{/pgfplots/table/@cell content}\pgfutil@empty%
            \fi
        }
    }
}
\makeatother
 

\makeatletter
\newcommand*\ExpandableInput[1]{\@@input#1 }
\makeatother


\newcommand{\tpExp}[1]{\phi^e_{#1}(x_{t-1})}
\newcommand{\lRat}[2]{\log \left( \frac{z_{#1}}{z_{#2}}\right)}


\newcommand{\anEdit}[1]
{

{\color{blue}
\begin{quote}
#1  
\end{quote}}

}

\begin{document}


\begin{frame}
  \frametitle{Algorithm based on basic properties of deterministic maps.}
  \begin{itemize}
  \item     The $\epsilon$'s  play two distinct roles in the algorithm.
To see the two roles, consider a family of stochastic functions 
$\{\stochF{T},\ldots,\stochF{0}\}$ with 
$\stochF{k}(x,\epsilon_t): {\mathcal{R}}^{L+1} \rightarrow \mathcal{R}^L$ 
    \begin{enumerate}
\item I assume  $\epsilon_t \sim $ iid. Knowing the distribution 
of the $\epsilon_t$ makes it possible to compute 
a corresponding family of deterministic functions $\detF{k}(x)=E( \stochF{k}(x,\epsilon_t)|x) $
    \item I assume that at some initial time, say t=0, a particular 
 $\epsilon_0$ draw 
along with an $x_{-1}$, determines a unique $x_0=\stochF{T}\initXE$.
Subsequently, $x_{t}=\detF{{T-t}}(x_{t-1})\,\, \forall t>0$
    \end{enumerate}
 \item The series representation requires a unique trajectory beginning with $x_0=\stochF{T}\initXE$ and 
evolving forward from any $x_0$ given by, a potentially time varying deterministic map,  $x_t=\detF{T-t}(x_{t-1})\,\, \forall t>0$. 
  \item The algorithm for discovering unknown solutions does not rely on time invariance.
  \end{itemize}
    
  \end{frame}

%\footnote{We set $\psi_c=0$  without loss of generality 
%in the expressions that follow merely to save on notation.}

\begin{frame}
\frametitle{The model of interest and a linear reference model}
\begin{itemize}
\item The target model is of the form
\begin{gather}
\nlEqnLHS{x_{t-1}}{x_t}{x_{t+1}}{t}=\\
\nlEqn{x_{t-1}}{x_t}{x_{t+1}}{t}=0\\ \label{refMod}
 (L \times 1) +  (L \times L ) \cdot (L \times 1)\\
\nlEqnSel{x_{t-1}}{x_t}{x_{t+1}}{t}\\ 
\end{gather}



\item The algorithm constructs a sequence of stochastic 
functions 
  \begin{gather}
	 \stochF{k}\initXE =B x_{t-1}+ (I - F)^{-1} \phi \psi_c+  \phi \psi_\epsilon\epsilon_0  +\\ \phi z^{\{k\}}\initXE+ \sum_{\sForSum=1}^{k-1} F^s \phi z^{\{k-\nu\}}_{\sForSum}\initXE 
\label{theSeries}
  \end{gather}

\item One can show that the algorithm begins with solutions satisfying 
the linear reference model $\linModMats$ and
constructs a sequence of functions
 honoring the target model equations for successively 
longer solution horizons.
  \end{itemize}
\end{frame}
\begin{frame}
  \frametitle{Recursive updating}
{\small
  \begin{itemize}
  \item Given a linear reference model,$\linModMats$,  any bounded 
sequence of deterministic maps $\{\detF{T},\ldots,\detF{0}\}, \,\, T>0$ generates 
a series representation
  \begin{gather}
     \label{eq:2}
	 \mathcal{X}_{t}(x_{t-1}) =B x_{t-1}+  (I - F)^{-1} \phi \psi_c +\\ \sum_{\sForSum=0}^{T-1} F^s \phi z_{T-\sForSum}(x_{t-1}) \intertext{it is straightforward to 
update to a new series representation for the map trajectories resulting from prepending an additional deterministic map to the sequence $\detF{T+1},\ldots,\detF{0}\}$}
	 \mathcal{X}_{t}(x_{t-1}) =B x_{t-1}+  (I - F)^{-1} \phi \psi_c +\\ 
\phi z_{T+1}(x_{t-1}) + \sum_{\sForSum=1}^{T} F^s \phi z_{T+1-\sForSum}(x_{t-1}) 
  \end{gather}
\item 
  \end{itemize}
}
\end{frame}

\begin{frame}
\frametitle{Early iterations}
{\small
  \begin{itemize}
  \item $k=0$: Completely ignore the target model
  \begin{gather}
z^{\itSup{0}}\initXE=0 \intertext{ so that}
\stochF{0}\initXE=B x_{-1} + (I - F)^{-1} \phi \psi_c +\phi \psi_\epsilon \epsilon_0  \intertext{initialize the sequence of deterministic maps with }
\{\detF{0}\initX=B x \}
  \end{gather}
  \item $k=1$: Employ the target model for one period. Use
  \begin{gather}
x_0\initXE=B x_{-1}+ (I - F)^{-1} \phi \psi_c +\phi \psi_\epsilon \epsilon_0 + \phi z^{\itSup{1}}\initXE\\
E(x_1\initXE)=\detF{0}(x_0\initXE) \intertext{in the deterministic equation 
\refeq{refMod} to determine $z^{\itSup{1}}\initXE$ so}
\stochF{1}\initXE = B x_{-1}+ (I - F)^{-1} \phi \psi_c +\phi \psi_\epsilon \epsilon_0 + 
\phi z^{\itSup{1}}\initXE \intertext{%Define $Z^{\itSup{1}}(x)\equiv E \left [ z^{\itSup{1}}\initXEN | x \right ]$ and 
 augment the sequence}
\{\detF{1}\initX=B x  + \phi Z^{\itSup{1}}(x),\detF{0}\}
  \end{gather}
  \end{itemize}
}
\end{frame}

\begin{frame}
\frametitle{Early iterations}
{\small
  \begin{itemize}
  \item $k=2$: Employ the target model for two periods. Use
  \begin{gather}
x_0\initXE=B x_{-1}+ (I - F)^{-1} \phi \psi_c +\phi \psi_\epsilon \epsilon_0 +\\ \phi z^{\itSup{2}}\initXE + F \phi Z^{\itSup{1}}(x_0\initXE)\\
E(x_1\initXE)=\detF{1}(x_0\initXE) \intertext{in the deterministic equation 
\refeq{refMod} to determine $z^{\itSup{2}}\initXE$}
\stochF{2}\initXE = B x_{-1}+ (I - F)^{-1} \phi \psi_c +\phi \psi_\epsilon \epsilon_0 +\\ \phi z^{\itSup{2}}\initXE + F \phi Z^{\itSup{1}}(x_0\initXE)\intertext{
Define $Z^{\itSup{2}}(x)\equiv 
E \left [ z^{\itSup{2}}\initXEN | x \right ]$}
\{\detF{2}\initX=B x  + \phi  Z^{\itSup{2}}(x) + F \phi Z^{\itSup{1}}(x_0\initXE),
\detF{1}\initX,\detF{0}\initX\}
  \end{gather}
  \end{itemize}
}
\end{frame}


\begin{frame}
\frametitle{Early iterations}
{\small
  \begin{itemize}
  \item $k=3$: Employ the target model for two periods. Use
  \begin{gather}
x_0\initXE=B x_{-1}+ (I - F)^{-1} \phi \psi_c +\phi \psi_\epsilon \epsilon_0 + \phi z^{\itSup{3}}\initXE +\\ F \phi Z^{\itSup{2}}(x_0\initXE) + F^2 \phi Z^{\itSup{1}}(\detF{2}(x_0\initXE))\\
E(x_1\initXE)=\detF{2}(x_0\initXE) \intertext{in the deterministic equation 
\refeq{refMod} to determine $z^{\itSup{3}}\initXE$}
\stochF{3}\initXE = B x_{-1}+ (I - F)^{-1} \phi \psi_c +\phi \psi_\epsilon \epsilon_0 + \phi z^{\itSup{3}}\initXE + \\F \phi Z^{\itSup{2}}(x_0\initXE) +F^2 \phi Z^{\itSup{1}}(\detF{2}(x_0\initXE))\intertext{
Define $Z^{\itSup{3}}(x)\equiv 
E \left [ z^{\itSup{3}}\initXEN | x \right ]$}
\detF{2}\initX=B x  + \phi  Z^{\itSup{3}}(x) + 
F \phi Z^{\itSup{2}}(x_0\initXE) + \\F^2 \phi Z^{\itSup{1}}(\detF{1}(x_0\initXE))
  \end{gather}
  \end{itemize}
}
\end{frame}

\begin{frame}
\frametitle{General iterations}
{\small
  \begin{itemize}
  \item $k=K+1$: Employ the target model for $K+1$ periods. Use
  \begin{gather}
x_0\initXE=B x_{-1}+ (I - F)^{-1} \phi \psi_c +\phi \psi_\epsilon \epsilon_0 + \phi z^{\itSup{K+1}}\initXE +\\ \sum_\sForSum^K F^\nu \phi Z^{\itSup{K+1-s}}\detComp{x_0\initXE} \\
E(x_1\initXE)=\detF{K}(x_0\initXE) \intertext{in the deterministic equation 
\refeq{refMod} to determine $z^{\itSup{K+1}}\initXE$}
\stochF{K+1}\initXE = B x_{-1}+ (I - F)^{-1} \phi \psi_c +\phi \psi_\epsilon \epsilon_0 + \phi z^{\itSup{K+1}}\initXE + \\
\sum_\sForSum^K F^\nu \phi Z^{\itSup{K+1-s}}\detComp{x_0\initXE} 
  \end{gather}
  \end{itemize}
}
\end{frame}

\begin{frame}
  \frametitle{Barthelemy and Marx  Model 1
\cite{marxbarthelemy2012}}

\begin{gather}
  \label{eq:3}
a_t - a^{\rho_0}_{t-1} \exp(\epsilon_t)\\
  \lambda_t-U^\prime(c_t)\\
\beta E_t[ \alpha a_{t+1} k_t^{\alpha-1} - (1-\delta)\lambda_{t+1}] -\lambda_t \\
c_t+k_t-a_t k_t^\alpha - (1-\delta)k_{t-1}
\end{gather}

\end{frame}

\begin{frame}
  \frametitle{Barthelemy and Marx  Model 2: Regime Switching
\cite{marxbarthelemy2012}}


\cite{troy2007}
\begin{gather}
  \label{eq:4}
  i_t =E_t \pi_{t+1} + r_t\\
r_t= \rho r_{t-1} +u_t\\
i_t=\alpha_{s_t} \pi_t
\end{gather}

Bounded solutions if and only if all eigenvalues of 
\begin{gather}
  \label{eq:5}
  \begin{bmatrix}
    \frac{1}{|a_1|}&0\\
0&    \frac{1}{|a_2|}
  \end{bmatrix}
  \begin{bmatrix}
    p_{11}&p_{12}\\p_{21}&p_{22}
  \end{bmatrix}
\end{gather}
 are inside unit circle

\end{frame}

\begin{frame}
  \frametitle{Barthelemy and Marx  Model 3: ZLB
\cite{marxbarthelemy2012}}

\begin{gather}
x_t=E_t x_{t+1} - \sigma (i_t - E_t \pi_{t+1} - r_t^n)\\
\pi_t = \beta E_t\pi_{t+1} + \kappa x_t + u_t\\
i_t\ge -r^\ast\\
i_t= \max ( -r^\ast,\alpha \pi_t)
\end{gather}

\end{frame}

\begin{frame}
  \frametitle{\cite{Guerrieri2015}}
{\small
  \begin{gather}
    \label{eq:6}
h_0 \lucaArgs =
\begin{bmatrix}
q_t -( \rho q_{t-1} - \sigma r_t + r^u_t)\\
  (r^u_t-r^\ast) = \rho_u (r^u_{t-1}-r^\ast) +  \epsilon^u_t\\
r_t-  \gamma q_t \\
\end{bmatrix}\\
h_1 \lucaArgs =
\begin{bmatrix}
q_t -( \rho q_{t-1} - \sigma r_t + r^u_t)\\
  (r^u_t-r^\ast) = \rho_u (r^u_{t-1}-r^\ast) +  \epsilon^u_t\\
r_t-  \bar{r} \\
\end{bmatrix}\\
 H_{0,1}\lucaArgs=
 \begin{bmatrix}
   -\beta (1 - \rho)&0&0\\0&0&0\\0&0&0
 \end{bmatrix}\\
% q_t =(\beta (1 - \rho) q_{t+1} + \rho q_{t-1} - 
%       \sigma r_t + r^u_t)\\
%   (r^u_t-r^\ast) = \rho_u (r^u_{t-1}-r^\ast) +  \epsilon^u_t\\
% r_t-   \begin{cases}
%  \gamma q_t & \gamma q_t \ge  \bar{r}\\
%  \bar{r}&\gamma q_t < \bar{r}
%    \end{cases}=0
\varpi=%m\lucaArgs=
\begin{cases}
  0& \gamma q_t \ge  \bar{r}\\
  1& \gamma q_t <  \bar{r}
\end{cases}
\end{gather}
}
\end{frame}

\begin{frame}
  \frametitle{Liu, Waggoner and Zha \cite{liu11:_sourc}}
  \begin{itemize}
  \item moderate-sized nonlinear model DSGE
  \item regime switching
  \item non linear and log linearized versions
  \end{itemize}
\end{frame}

\begin{frame}
  \frametitle{\cite{werner13,shi12:_liquid}}
  
\end{frame}

 \bibliographystyle{plainnat}
 \bibliography{../../bibFiles/anderson,../../bibFiles/files}



\end{document}
