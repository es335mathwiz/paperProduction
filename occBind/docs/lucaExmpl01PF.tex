We can ompute exact solutions for this case using
Mathematica, a symbolic algebra program.  These exact 
solutions will be useful for
characterizing the accuracy of the numerical procedure that I will outline
below.

We will require $zzz\$0\$1[t]$ to satisfy\footnote{Generated by symb01.m}:
 \begin{gather*}
\vcenter{\hbox{\includegraphics{/msu/home/m1gsa00/git/ProjectionMethodTools/ProjectionMethodToolsJava/code/try01A.pdf}}}\\
or\\
\vcenter{\hbox{\includegraphics{/msu/home/m1gsa00/git/ProjectionMethodTools/ProjectionMethodToolsJava/code/try01B.pdf}}}\\
 \end{gather*}


Which leads to\footnote{Generated by symb01.m}:
 \begin{gather*}
\vcenter{\hbox{\includegraphics{/msu/home/m1gsa00/git/ProjectionMethodTools/ProjectionMethodToolsJava/code/red01A.pdf}}}\\
and\\
\vcenter{\hbox{\includegraphics{/msu/home/m1gsa00/git/ProjectionMethodTools/ProjectionMethodToolsJava/code/red01B.pdf}}}\\
or\\
\vcenter{\hbox{\includegraphics{/msu/home/m1gsa00/git/ProjectionMethodTools/ProjectionMethodToolsJava/code/red01C.pdf}}}
 \end{gather*}



To numerically compute the solution for the model variables, we use\footnote{Generated by symb01.m}:

\begin{gather*}
  0=\vcenter{\hbox{\includegraphics{/msu/home/m1gsa00/git/ProjectionMethodTools/ProjectionMethodToolsJava/code/prettyEqns01.pdf}}}
\end{gather*}

The first and second equations correspond to the values for $q_t,r_{ut}$.
The third equation defines $\delta_t$, the discrepancy between the value for $r_t$ and the constraint.    The fourth equation provides the values for $r_t$.  This equation is not really used here, because we actually impose the inequality
constraint in the final equation. The final equation
 sets $z_t$ to a value that depends on 
whether or not the equation for $r_t$ would produce a value that violates the
constraint.  It sets $z_t=0$ if the constraint would  not be violated ( $\delta_t\ge0$) ,  or
to a value that would set $r_t=\bar{r}$ ( $\delta_t<0$).


We can explicitly solve this system to obtain  expressions for $q_t, r_t, z_{t}(x_{t-1},\epsilon_t) $:\footnote{Need to cleanup Mathematica solution details showing erroneous default values for Piecewise}\footnote{Generated by symb01.m}

\begin{gather*}
  qq_t=\vcenter{\hbox{\includegraphics{/msu/home/m1gsa00/git/ProjectionMethodTools/ProjectionMethodToolsJava/code/prettySoln01Q.pdf}}}
\end{gather*}

\begin{gather*}
  r_t=\vcenter{\hbox{\includegraphics{/msu/home/m1gsa00/git/ProjectionMethodTools/ProjectionMethodToolsJava/code/prettySoln01R.pdf}}}
\end{gather*}

\begin{gather*}
  z_t=\vcenter{\hbox{\includegraphics{/msu/home/m1gsa00/git/ProjectionMethodTools/ProjectionMethodToolsJava/code/prettySoln01Z.pdf}}}
\end{gather*}

This is the exact solution if the constraints are no longer relevant for 
$t >0$.

To verify this is true, compute the time t and t+1 values using the expression for z.

\begin{gather*}
  0=\vcenter{\hbox{\includegraphics{/msu/home/m1gsa00/git/ProjectionMethodTools/ProjectionMethodToolsJava/code/prettyPath01.pdf}}}
\end{gather*}

Premultiply by $H$ to get

\begin{gather*}
  0=\vcenter{\hbox{\includegraphics{/msu/home/m1gsa00/git/ProjectionMethodTools/ProjectionMethodToolsJava/code/prettyhmatApp01.pdf}}}
\end{gather*}


This one period solution does not impose the constraint at time t+1.  For some values of the state variables and the shocks, the expected value 
of $r_{t+1}$ will violate the constraint, even assuming that future shocks are all zero -- a so called ``perfect foresight solution.''


There are two regions.  One has $z=0$, the constraint non-binding the other 
$z\ne 0$ constraint binding.


\begin{gather*}
  0=\vcenter{\hbox{\includegraphics{/msu/home/m1gsa00/git/ProjectionMethodTools/ProjectionMethodToolsJava/code/prettyreg01pltA.pdf}}}
\end{gather*}


\begin{gather*}
  0=\vcenter{\hbox{\includegraphics{/msu/home/m1gsa00/git/ProjectionMethodTools/ProjectionMethodToolsJava/code/prettyreg01pltB.pdf}}}
\end{gather*}


