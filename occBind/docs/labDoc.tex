\documentclass[12pt]{article}
%http://mathematica.stackexchange.com/questions/1542/exporting-graphics-to-pdf-huge-file huge image solutions

\usepackage{endnotes}
\usepackage{moreverb}
\usepackage{rotating}
\usepackage{siunitx}
\usepackage[authoryear]{natbib}
\title{Lab Notes Version for ``A Solution Strategy for Occasionally Binding Constraints in Otherwise
Linear Rational Expectations Models''}
\author{Gary S. Anderson}
\usepackage{datetime}
\usepackage{hyperref}
\usepackage{amsmath}
\usepackage{graphicx}
\date{\today \   at \currenttime}




\makeatletter
\newcommand*\ExpandableInput[1]{\@@input#1 }
\makeatother
\newcommand{\xtFuncTI}{\mathcal{X}(x,\epsilon)}
\newcommand{\XtFuncTI}{\mathbf{X}(x)}

\newcommand{\xtFunc}[1]{\mathcal{X}{#1}}
\newcommand{\XtFunc}[1]{\mathbf{X}{#1}}



\newcommand{\discr}[1]{\mathcal{D}^{#1}(x_{t-1},\epsilon_t)}

\newcommand{\XZPair}[1]{(\mathcal{X}^{#1},\mathcal{Z}^{#1})}
\newcommand{\XZPairG}[1]{(\mathcal{X}^{#1}(x^g),\mathcal{Z}^{#1}(x^g))}
\newcommand{\xIter}[2]{\mathcal{X}^{#1}(#2)}
%\newcommand{\zNow}[1]{z^{#1}_0(x_{t-1},\epsilon_t)}
%\newcommand{\ZNow}[3]{\mathcal{Z}^{#1}_{#2}(x_{#3})}
\newcommand{\zNow}[1]{z^{#1}(x_{t-1},\epsilon_t)}
\newcommand{\ZNow}[3]{\mathcal{Z}^{#1}(x_{#3})}

\newcommand{\xNow}[1]{x^{#1}_t(x_{t-1},\epsilon_t)}
\newcommand{\xNowtp}[1]{x^{#1}_{t+1}(x_{t-1},\epsilon_t)}
\newcommand{\XNow}[3]{\mathcal{X}^{#1}_{#2}(x_{#3})}








\newcommand{\sForSum}{{\nu}}
\newcommand{\rcpC}{{\mathbf{c}}}

\newcommand{\xtVec}{  \begin{bmatrix}
    q_t\\r_{t}\\r_{ut}
  \end{bmatrix}
}
\newcommand{\xtPVec}{  \begin{bmatrix}
    q_{t+1}\\r_{t+1}\\r_{ut+1}
  \end{bmatrix}
}
\newcommand{\xtMVec}{  \begin{bmatrix}
    q_{t-1}\\r_{t-1}\\r_{ut-1}
  \end{bmatrix}
}
\newcommand{\expctEps}[1]{\mathcal{E}_{\epsilon} \left [#1 \right ]}
\newcommand{\expct}[2]{E_{#1} \left [#2 \right ]}
\newcommand{\expc}[1]{\mathcal{E} \left [#1 \right ]}
\newcommand{\expcK}[2]{\mathcal{E}^{#1} \left [#2 \right ]}
\newcommand{\xsln}[1]{\mathbb{X} \left [#1 \right ]}
\newcommand{\xslnK}[2]{\hat{\mathbb{X}}^{#1} \left [#2 \right ]}
\newcommand{\xpth}[2]{\mathfrak{X}_{#1} \left [#2 \right ]}
\newcommand\infNorm[1]{\left\lVert#1\right\rVert_\infty}
\newcommand\twoNorm[1]{\left\lVert#1\right\rVert_2}
%\newcommand{\inorm}[1]{\left\lVert#1\right\rVert_\infty}

% \newcommand{\forPhi}{\begin{bmatrix}
% \psi_\epsilon&\psi_z
% \end{bmatrix}}
% \newcommand{\phiMult}{\phi \psi_\epsilon}
% \newcommand{\bMult}{B x_{-1} + \phiMult}
\newcommand{\phiMultBoth}[1]{
	\phi (\psi_\epsilon \epsilon_t +\psi_z z_0^#1(x_{-1},\epsilon_t))}
\newcommand{\bMultBoth}[1]{B x_{-1} + \phiMultBoth{#1}}


\newcommand{\bForOne}{\bMultBoth{1}
}

% \newcommand{\bForTwo}{\bMultBoth{2}+
% F \phi  \psi_z  
% Z_0^1(x_0^2(x_{-1}))   
% }



\newcommand{\compSlack}{z_0^1(x_{-1},\epsilon_t) \left ( \bar{x} -x\right )=0\\ z_0^1(x_{-1},\epsilon_t)> 0}
% \begin{gather*}
% 0= x_t-(B x_{t-1}+ \phi \psi_\epsilon\epsilon_t + \phi \psi_z 
% \xpt{z_{t}(x_{t-1},\epsilon_t)    } )
% \end{gather*}

\newcommand{\xpt}[1]{#1}


\makeatletter
\@ifundefined{newblock}{%
 \def\newblock{\hskip .11em plus .33em minus .07em} % important line
}
\makeatother


\makeatletter
\pgfplotsset{
    /pgfplots/table/omit header/.style={%
        /pgfplots/table/typeset cell/.append code={%
            \ifnum\c@pgfplotstable@rowindex=-1
                \pgfkeyslet{/pgfplots/table/@cell content}\pgfutil@empty%
            \fi
        }
    }
}
\makeatother
 

\makeatletter
\newcommand*\ExpandableInput[1]{\@@input#1 }
\makeatother


\newcommand{\tpExp}[1]{\phi^e_{#1}(x_{t-1})}
\newcommand{\lRat}[2]{\log \left( \frac{z_{#1}}{z_{#2}}\right)}


\newcommand{\anEdit}[1]
{

{\color{blue}
\begin{quote}
#1  
\end{quote}}

}

\begin{document}
\maketitle



\begin{abstract}
This paper shows how to apply the formulae in\citep{anderson10} to compute 
rational expectations solutions for models linear except for occasionally binding constraints.  The formulae facilitate the recursive 
computation of solutions that honor the 
constraints for successively 
longer horizons. The solutions thus computed
accommodate the possibility that model
trajectories may depart from and re-engage the constraints.
The technique is applicable for nonlinear inequality constraints.

\end{abstract}




\section{Introduction and Summary}
\label{sec:introduction-summary}

This paper shows how to apply the formulae in\citep{anderson10} to compute 
rational expectations solutions for models linear except for occasionally binding constraints.

\section{Problem Statement and Solution}
\label{sec:probl-stat-solut}






Consider systems of the form
\newcommand{\xpt}[1]{\mathcal{E}_t \left ( #1 \right ) }

\begin{gather}
\sum_{i=-\tau}^\theta H_i x_{t+i} =
\begin{bmatrix}
\psi_\epsilon & \psi_z  
\end{bmatrix}
  \begin{bmatrix}
\epsilon_t \\\xpt{z_{t}(x_{t-1},\epsilon_t) }   
  \end{bmatrix}
  \,\,\forall t. \intertext{Solutions will take the form\citep{anderson10}}
  x_t=B x_{t-1}+ \phi \psi_\epsilon\epsilon_t + \sum_{s=0}^\infty F^s \phi \psi_z 
\xpt{z_{t+s}(x_{t+s-1},\epsilon_t)    } \label{myEqn}
% \intertext{ if $z_{t+s}=z^\ast, \forall s$}
%   x_t=B x_{t-1}+(z_t(x_{t-1},\epsilon_t) - \xpt{z^\ast(x_{t-1},\epsilon_{t+s})}(I-F)^{-1} \phi \psi
%   \begin{bmatrix}
% \epsilon_t \\z^\ast(x_{t-1},\epsilon_t)    
%   \end{bmatrix}\intertext{ if $z_{t+s}=z^\ast, \forall s$}
\end{gather}
To accommodate inequality constraints of the form
\begin{gather*}
  x_t \ge \bar{x}
\end{gather*}
we will determine a sequence of $z^i_t$ that 
will allow us to
 impose the constraints over successively longer time periods.  
We begin by solving a system where the constraints are honored only at time
 $t=0$.  To do this we set $\xpt{z_{t+s}}=0\, \forall s>0$\footnote{Generalize to linear combinations of variables. \begin{gather*}
M x_t \ge m  
\end{gather*} 

Should also consider other behavior besides absorbing barriers.
}
\newcommand{\forPhi}{\begin{bmatrix}
\psi_\epsilon&\psi_z
\end{bmatrix}}
\newcommand{\phiMult}{\phi \psi_\epsilon}
\newcommand{\bMult}{B x_{-1} + \phiMult}
\newcommand{\phiMultBoth}[1]{
\phi (\psi_\epsilon \epsilon_0 +\psi_z z_0^#1(x_{-1},\epsilon_0))}
\newcommand{\bMultBoth}[1]{B x_{-1} + \phiMultBoth{#1}}


\newcommand{\bForOne}{\bMultBoth{1}
}

\newcommand{\bForTwo}{\bMultBoth{2}+
F \phi  \psi_z  
Z_0^1(x_0^2(x_{-1}))   
}



\newcommand{\compSlack}{z_0^1(x_{-1},\epsilon_0) \left ( \bar{r} -\phi_p q_t\right )=0\\ z_0^1(x_{-1},\epsilon_0)> 0}
We now have
\begin{gather*}
x_0^1(x_{-1},\epsilon_0)-
\left ( \bForOne \right )=0\intertext{ with the complimentary slackness condition}
\compSlack
\end{gather*}




We set \footnote{use logs or squares?}
\begin{gather}\label{firstIneq}
z_0^1(x_{-1},\epsilon_0)=
\begin{cases}
0&  B x_{-1} +\epsilon_0 \ge \bar{x}  \\
\bar{x}-(B x_{-1}+\epsilon_0) & B x_{-1}+\epsilon_0 < \bar{x}  
\end{cases}
\end{gather}




\subsection{A Simple Example}
\label{sec:simple-example}


First call time much less than second call time.



\href{http://www.tecmint.com/command-line-tools-to-monitor-linux-performance/}{This is a link for measuring performance.}


    

%from will
%cat /proc/cpuinfo | grep processor | wc -l
%uptime | tr -s ' ' ' ' | cut -d' ' -f11-
%vmstat | tail -n 1 | tr -s ' ' ' ' | cut -d' ' -f5

%\renewcommand{\thempfootnote}{\arabic{mpfootnote}}

\begin{table}
%\begin{sidewaystable}
%  \begin{minipage}{1.0\linewidth}

{\tiny     
\caption{Computation Time for Imposing Constraint for Various Lengths of Time}
\label{tab:varLen}
  \centering



\begin{tabular}{|l|l|
S[table-format=3.6]|
S[table-format=3.6]|
S[table-format=3.6]|
S[table-format=3.6]|
S[table-format=3.6]|
}
\hline
\multicolumn{7}{|c|}{Time in seconds for evaluation at a single initial state ( except where noted )\footnote{See the endnotes Section~\ref{sec:endnotes}.}}\\
\hline
\multicolumn{1}{|c|}{Periods}&
\multicolumn{1}{|c|}{Recursive?}&
\multicolumn{1}{|c|}{Analytic First}&
\multicolumn{1}{|c|}{Analytic Second}&
&&\\
% \multicolumn{1}{|c|}{Semi-Analytic}&
% \multicolumn{1}{|c|}{Projection}&
% \multicolumn{1}{|c|}{Interpolation}\\
\hline
\hline
1&NA\endnote{Algebraic expression for all initial states.\label{allStates}}&\ExpandableInput{/msu/home/m1gsa00/git/ProjectionMethodTools/ProjectionMethodToolsJava/code/symb01Secs}&&&\\
\hline
\hline
2&No\textsuperscript{\ref{allStates}}&\ExpandableInput{/msu/home/m1gsa00/git/ProjectionMethodTools/ProjectionMethodToolsJava/code/symb02Secs}&&&\\
\hline
2&Yes\textsuperscript{\ref{allStates}}&\ExpandableInput{/msu/home/m1gsa00/git/ProjectionMethodTools/ProjectionMethodToolsJava/code/symb02RecSecs}&&&\\
\hline
2&Yes&\ExpandableInput{/msu/home/m1gsa00/git/ProjectionMethodTools/ProjectionMethodToolsJava/code/symb02ValsRecSecs}&&&\\
\hline
2&Expectation non recursive&\ExpandableInput{/msu/home/m1gsa00/git/ProjectionMethodTools/ProjectionMethodToolsJava/code/symb02ValsExpSecs}&&&\\
\hline
\hline
3&No\textsuperscript{\ref{allStates}}&\ExpandableInput{/msu/home/m1gsa00/git/ProjectionMethodTools/ProjectionMethodToolsJava/code/symb03Secs}&&&\\
\hline
3&Yes\textsuperscript{\ref{allStates}}&\ExpandableInput{/msu/home/m1gsa00/git/ProjectionMethodTools/ProjectionMethodToolsJava/code/symb03RecSecs}&&&\\
\hline
3&No&\ExpandableInput{/msu/home/m1gsa00/git/ProjectionMethodTools/ProjectionMethodToolsJava/code/symb03ValsSecs}&&&\\
\hline
3&Yes&\ExpandableInput{/msu/home/m1gsa00/git/ProjectionMethodTools/ProjectionMethodToolsJava/code/symb03ValsRecSecs}&&&\\
\hline
3&Expectation non recursive&\ExpandableInput{/msu/home/m1gsa00/git/ProjectionMethodTools/ProjectionMethodToolsJava/code/symb03ValsExpSecs}&&&\\
\hline
\hline
4&No&\ExpandableInput{/msu/home/m1gsa00/git/ProjectionMethodTools/ProjectionMethodToolsJava/code/symb04ValsSecs}&&&\\
\hline
4&Yes&\ExpandableInput{/msu/home/m1gsa00/git/ProjectionMethodTools/ProjectionMethodToolsJava/code/symb04ValsRecSecs}&&&\\
\hline
\hline
5&No&\ExpandableInput{/msu/home/m1gsa00/git/ProjectionMethodTools/ProjectionMethodToolsJava/code/symb05ValsSecs}&&&\\
\hline
5&Yes&\ExpandableInput{/msu/home/m1gsa00/git/ProjectionMethodTools/ProjectionMethodToolsJava/code/symb05ValsRecSecs}&&&\\
\hline
\hline
6&No&\ExpandableInput{/msu/home/m1gsa00/git/ProjectionMethodTools/ProjectionMethodToolsJava/code/symb06ValsSecs}&&&\\
\hline
\hline
7&No&\ExpandableInput{/msu/home/m1gsa00/git/ProjectionMethodTools/ProjectionMethodToolsJava/code/symb07ValsSecs}&&&\\
\hline
\hline
\end{tabular}

}
%  \end{minipage}
%\end{sidewaystable}

\end{table}




Constraints for region where have gone out far enough.


For example, consider the simple model


\begin{gather*}
q_{t} +\beta_p(1 - \rho_p)q_{t + 1} + \rho_pq_{t - 1} - \sigma_pr_{t} +
     r_{ut}=0\\
 r_{t} = \max (\bar{r}, \phi_pq_{t}) \\
 r_{ut} = \rho_{ru} r_{ut - 1} + \eta \epsilon_{y,t}
\end{gather*}

Then 
\newcommand{\xtmVec}{  \begin{bmatrix}
    q_t\\r_{t}\\r_{ut}
  \end{bmatrix}
}


\begin{gather*}
  x_t=\xtmVec
\end{gather*}

Some typical values of the parameters are:

\begin{gather*}
  \beta_p = 0.99, \phi_p = 1, 
\rho_p = 0.5, \sigma_p = 1, \rho_{ru} = 0.5,
  \bar{r} = 0.02 \\
% q^L = -.5, q^H = .5, 
%   r_u^L = -4 \sigma_u/(1 - \rho_{ru}), r_u^H=  4\sigma_u/(1 - \rho_{ru}),
%    I_N = {10}, \sigma_u = 0.02,\\
%  \mu = {0},
%    \eta = 1
\end{gather*}



\begin{gather*}
  H= \vcenter{\hbox{\includegraphics{/msu/home/m1gsa00/git/ProjectionMethodTools/ProjectionMethodToolsJava/code/prettyHmat.pdf}}}\\
\psi_z=   \vcenter{\hbox{\includegraphics{/msu/home/m1gsa00/git/ProjectionMethodTools/ProjectionMethodToolsJava/code/prettyPsiZ.pdf}}}\\
\psi_\epsilon=   \vcenter{\hbox{\includegraphics{/msu/home/m1gsa00/git/ProjectionMethodTools/ProjectionMethodToolsJava/code/prettyPsiEps.pdf}}}\\
\end{gather*}




 \begin{gather*}
B=   \vcenter{\hbox{\includegraphics{/msu/home/m1gsa00/git/ProjectionMethodTools/ProjectionMethodToolsJava/code/prettyBmat.pdf}}}\\
\phi=   \vcenter{\hbox{\includegraphics{/msu/home/m1gsa00/git/ProjectionMethodTools/ProjectionMethodToolsJava/code/prettyPhimat.pdf}}}\\
F=   \vcenter{\hbox{\includegraphics{/msu/home/m1gsa00/git/ProjectionMethodTools/ProjectionMethodToolsJava/code/prettyFmat.pdf}}}
 \end{gather*}




{\tiny
\begin{gather*}
\intertext{Algorithm Inputs:}
% H=
% \begin{bmatrix}
% H_{-1} &H_0&H_{t+1}  
% \end{bmatrix}\\
  H= \vcenter{\hbox{\includegraphics{../../../ProjectionMethodTools/ProjectionMethodToolsJava/code/prettyHmat.pdf}}}\\
\psi_z=   \vcenter{\hbox{\includegraphics{../../../ProjectionMethodTools/ProjectionMethodToolsJava/code/prettyPsiZ.pdf}}}
\psi_\epsilon=   \vcenter{\hbox{\includegraphics{../../../ProjectionMethodTools/ProjectionMethodToolsJava/code/prettyPsiEps.pdf}}}\\
\intertext{Algorithm Outputs:}
B=   \vcenter{\hbox{\includegraphics{../../../ProjectionMethodTools/ProjectionMethodToolsJava/code/prettyBmat.pdf}}}\\
\phi=   \vcenter{\hbox{\includegraphics{../../../ProjectionMethodTools/ProjectionMethodToolsJava/code/prettyPhimat.pdf}}}\\
F=   \vcenter{\hbox{\includegraphics{../../../ProjectionMethodTools/ProjectionMethodToolsJava/code/prettyFmat.pdf}}}
 \end{gather*}

}




\subsection{Solution for One Period}
\label{sec:solution-one-period}



We begin by solving a system where the constraints are honored only at time
 $t=0$.  
Using equation \ref{myEqn}, we can construct the system which imposes the 
constraint for time 0 alone
To do this we set $\xpt{z_{t+s}}=0\, \forall s>0$\footnote{Generalize to linear combinations of variables. \begin{gather*}
M x_t \ge m  
\end{gather*} 

Should also consider other behavior besides absorbing barriers.
}
\newcommand{\forPhi}{\begin{bmatrix}
\psi_\epsilon&\psi_z
\end{bmatrix}}
\newcommand{\phiMult}{\phi \psi_\epsilon}
\newcommand{\bMult}{B x_{-1} + \phiMult}
\newcommand{\phiMultBoth}[1]{
\phi (\psi_\epsilon \epsilon_0 +\psi_z z_0^#1(x_{-1},\epsilon_0))}
\newcommand{\bMultBoth}[1]{B x_{-1} + \phiMultBoth{#1}}


\newcommand{\bForOne}{\bMultBoth{1}
}

\newcommand{\bForTwo}{\bMultBoth{2}+
F \phi  \psi_z  
Z_0^1(x_0^2(x_{-1}))   
}



\newcommand{\compSlack}{z_0^1(x_{-1},\epsilon_0) \left ( \bar{r} -\phi_p q_t\right )=0\\ z_0^1(x_{-1},\epsilon_0)> 0}
% \begin{gather*}
% 0= x_t-(B x_{t-1}+ \phi \psi_\epsilon\epsilon_t + \phi \psi_z 
% \xpt{z_{t}(x_{t-1},\epsilon_t)    } )
% \end{gather*}
We now have
\begin{gather*}
x_0^1(x_{-1},\epsilon_0)-
\left ( \bForOne \right )=0\intertext{ with the complimentary slackness condition}
\compSlack
\end{gather*}




We set \footnote{use logs or squares?}
\begin{gather}\label{firstIneq}
z_0^1(x_{-1},\epsilon_0)=
\begin{cases}
0&  B x_{-1} +\epsilon_0 \ge \bar{x}  \\
\bar{x}-(B x_{-1}+\epsilon_0) & B x_{-1}+\epsilon_0 < \bar{x}  
\end{cases}
\end{gather}



We can ompute exact solutions for this case using
Mathematica, a symbolic algebra program.  These exact 
solutions will be useful for
characterizing the accuracy of the numerical procedure that I will outline
below.

We will require $zzz\$0\$1[t]$ to satisfy\footnote{Generated by symb01.m}:
 \begin{gather*}
\vcenter{\hbox{\includegraphics{../../../ProjectionMethodTools/ProjectionMethodToolsJava/code/try01A.pdf}}}\\
or\\
\vcenter{\hbox{\includegraphics{../../../ProjectionMethodTools/ProjectionMethodToolsJava/code/try01B.pdf}}}\\
 \end{gather*}


Which leads to\footnote{Generated by symb01.m}:
 \begin{gather*}
\vcenter{\hbox{\includegraphics{../../../ProjectionMethodTools/ProjectionMethodToolsJava/code/red01A.pdf}}}\\
and\\
\vcenter{\hbox{\includegraphics{../../../ProjectionMethodTools/ProjectionMethodToolsJava/code/red01B.pdf}}}\\
or\\
\vcenter{\hbox{\includegraphics{../../../ProjectionMethodTools/ProjectionMethodToolsJava/code/red01C.pdf}}}
 \end{gather*}



To numerically compute the solution for the model variables, we use\footnote{Generated by symb01.m}:

\begin{gather*}
  0=\vcenter{\hbox{\includegraphics{../../../ProjectionMethodTools/ProjectionMethodToolsJava/code/prettyEqns01.pdf}}}
\end{gather*}

The first and second equations correspond to the values for $q_t,r_{ut}$.
The third equation defines $\delta_t$, the discrepancy between the value for $r_t$ and the constraint.    The fourth equation provides the values for $r_t$.  This equation is not really used here, because we actually impose the inequality
constraint in the final equation. The final equation
 sets $z_t$ to a value that depends on 
whether or not the equation for $r_t$ would produce a value that violates the
constraint.  It sets $z_t=0$ if the constraint would  not be violated ( $\delta_t\ge0$) ,  or
to a value that would set $r_t=\bar{r}$ ( $\delta_t<0$).


We can explicitly solve this system to obtain  expressions for $q_t, r_t, z_{t}(x_{t-1},\epsilon_t) $:\footnote{Need to cleanup Mathematica solution details showing erroneous default values for Piecewise}\footnote{Generated by symb01.m}

\begin{gather*}
  qq_t=\vcenter{\hbox{\includegraphics{../../../ProjectionMethodTools/ProjectionMethodToolsJava/code/prettySoln01Q.pdf}}}
\end{gather*}

\begin{gather*}
  r_t=\vcenter{\hbox{\includegraphics{../../../ProjectionMethodTools/ProjectionMethodToolsJava/code/prettySoln01R.pdf}}}
\end{gather*}

\begin{gather*}
  z_t=\vcenter{\hbox{\includegraphics{../../../ProjectionMethodTools/ProjectionMethodToolsJava/code/prettySoln01Z.pdf}}}
\end{gather*}

This is the exact solution if the constraints are no longer relevant for 
$t >0$.

To verify this is true, compute the time t and t+1 values using the expression for z.

\begin{gather*}
  0=\vcenter{\hbox{\includegraphics{../../../ProjectionMethodTools/ProjectionMethodToolsJava/code/prettyPath01.pdf}}}
\end{gather*}

Premultiply by $H$ to get

\begin{gather*}
  0=\vcenter{\hbox{\includegraphics{../../../ProjectionMethodTools/ProjectionMethodToolsJava/code/prettyhmatApp01.pdf}}}
\end{gather*}


This one period solution does not impose the constraint at time t+1.  For some values of the state variables and the shocks, the expected value 
of $r_{t+1}$ will violate the constraint, even assuming that future shocks are all zero -- a so called ``perfect foresight solution.''


There are two regions.  One has $z=0$, the constraint non-binding the other 
$z\ne 0$ constraint binding.


\begin{gather*}
  0=\vcenter{\hbox{\includegraphics{../../../ProjectionMethodTools/ProjectionMethodToolsJava/code/prettyreg01pltA.pdf}}}
\end{gather*}


\begin{gather*}
  0=\vcenter{\hbox{\includegraphics{../../../ProjectionMethodTools/ProjectionMethodToolsJava/code/prettyreg01pltB.pdf}}}
\end{gather*}




We will use an approximation for the z function in our recursive computation.
This is a brief assessmt of the accuracy and the cost of computing the approximations.


\input{/msu/home/m1gsa00/git/ProjectionMethodTools/ProjectionMethodToolsJava/code/interpOneCalcs}


\subsection{Solution for Two Periods}

\begin{itemize}
\item Solve doesn't work on composition of interpolating functions
\end{itemize}


Consider a perfect foresight solution where the constraints are honored only 
for $t=0,1$.\footnote{ Perfect foresight solutions impose the constraint that future shocks are identically zero.}

Since now a future value of z is non zero, formula \ref{myEqn}  will have additional non-zero terms.


\begin{gather*}
  \vcenter{\hbox{\includegraphics{/msu/home/m1gsa00/git/ProjectionMethodTools/ProjectionMethodToolsJava/code/prettyEqns02A.pdf}}}\\
  \vcenter{\hbox{\includegraphics{/msu/home/m1gsa00/git/ProjectionMethodTools/ProjectionMethodToolsJava/code/prettyEqns02B.pdf}}}\\
\end{gather*}
Mathematica is unable to compute this solution exactly in a reasonable amount
of time.\footnote{ 5 hours.  One could substitute the solution for the one period function to simplify the system, but Mathematica was still unable to solve the simplified system.}  However we can easily obtain numerical solutions for the z functions.  The z01 function applies to the time t variables and 
threrefore we must solve a nonlinear system that captures the fact that
the time t value of the state variables depends on this future value of z01.


Note the following graph of $r_t$ when imposing constraint for only two periods:
\begin{gather*}
\includegraphics{/msu/home/m1gsa00/git/ProjectionMethodTools/ProjectionMethodToolsJava/code/prettyrr02.pdf}
\end{gather*}



\newcommand{\lucaXt}{
   \begin{bmatrix}
    q_{t-1}\\r_{ut-1}\\r_{t-1}\\
     q_{t}\\r_{ut}\\r_{t}\\
     q_{t+1}\\r_{ut+1}\\r_{t+1}
   \end{bmatrix}}

 \newcommand{\lucaXtpOne}{
   \begin{bmatrix}
     q_{t}\\r_{ut}\\r_{t}\\
     q_{t+1}\\r_{ut+1}\\r_{t+1}\\
     q_{t+2}\\r_{ut+2}\\r_{t+2}
   \end{bmatrix}}


 Note the following graph of 

 \begin{gather*}
 H_{2}\lucaXt ( r_t-0.02)
 \end{gather*}

  when imposing constraint for only two periods:
 \begin{gather*}
 \includegraphics{/msu/home/m1gsa00/git/ProjectionMethodTools/ProjectionMethodToolsJava/code/prettyhapp02A.pdf}
 \end{gather*}


 Note the following graph of 

 \begin{gather*}
 H_{5}\lucaXt( r_{t+1}-0.02)
 \end{gather*}

  when imposing constraint for only two periods:
 \begin{gather*}
 \includegraphics{/msu/home/m1gsa00/git/ProjectionMethodTools/ProjectionMethodToolsJava/code/prettyhapp02B.pdf}
 \end{gather*}


Both have infinity norms near machine precision for the pecified range of values of the state variables.  There are comparable values for non zero $\epsilon$.


We can use the one period solution to aid in solving the two period problem.
Complicated recursion since z, though known, is recursivly applied to itself


\begin{gather*}
  \vcenter{\hbox{\includegraphics{/msu/home/m1gsa00/git/ProjectionMethodTools/ProjectionMethodToolsJava/code/prettyEqns02RecA.pdf}}}\\
  \vcenter{\hbox{\includegraphics{/msu/home/m1gsa00/git/ProjectionMethodTools/ProjectionMethodToolsJava/code/prettyEqns02RecB.pdf}}}\\
\end{gather*}

The second equation comes from applying the $z_{0t}$ function to the time t variables.


The solution obtained is essentially identical as can be seen by evaluating the
same complementary slackness conditions above.\footnote{The Piecewise solutions now appear to have four branches.  But, this
 is must be an artifact since the time t solutions only have two branches.}


 Note the following graph of 

 \begin{gather*}
 H_{2}\lucaXt ( r_t-0.02)
 \end{gather*}

  when imposing constraint for only two periods:
 \begin{gather*}
 \includegraphics{/msu/home/m1gsa00/git/ProjectionMethodTools/ProjectionMethodToolsJava/code/prettyhapp02RecA.pdf}
 \end{gather*}


 Note the following graph of 

 \begin{gather*}
 H_{5}\lucaXt( r_{t+1}-0.02)
 \end{gather*}

  when imposing constraint for only two periods:
 \begin{gather*}
 \includegraphics{/msu/home/m1gsa00/git/ProjectionMethodTools/ProjectionMethodToolsJava/code/prettyhapp02RecB.pdf}
 \end{gather*}




There are four regions.  There are regions for each combination of zero/nonzero z values for two periods.



\begin{gather*}
  0=\vcenter{\hbox{\includegraphics{/msu/home/m1gsa00/git/ProjectionMethodTools/ProjectionMethodToolsJava/code/prettyreg02pltA.pdf}}}
\end{gather*}


\begin{gather*}
  0=\vcenter{\hbox{\includegraphics{/msu/home/m1gsa00/git/ProjectionMethodTools/ProjectionMethodToolsJava/code/prettyreg02pltB.pdf}}}
\end{gather*}


\begin{gather*}
  0=\vcenter{\hbox{\includegraphics{/msu/home/m1gsa00/git/ProjectionMethodTools/ProjectionMethodToolsJava/code/prettyreg02pltC.pdf}}}
\end{gather*}


\begin{gather*}
  0=\vcenter{\hbox{\includegraphics{/msu/home/m1gsa00/git/ProjectionMethodTools/ProjectionMethodToolsJava/code/prettyreg02pltD.pdf}}}
\end{gather*}




It will be useful to solve for the z's recursively, using the results from shorter periods to compute solutions for longer period.

Suppose we have computed $z^0_t(x_{t-1},\epsilon_t)$ imposing the consraint for time $t$ alone.  We can use this function to impose the constraint at time $t+1$ by applying the function to the later, time $t$ values   $z^0_t(x_{t},0)$ with perfect foresight solution $\epsilon_{t+1}=0$


\begin{gather*}
   \begin{array}{cc}
    \{ & 
   \begin{array}{cc}
    0 & 0.767742 \text{eps}+0.116129 \text{qtm1}+0.383871 \text{rutm1}-0.0615634
      \text{zzz$\$$0$\$$1}(t)-0.232258 \text{zzz$\$$1$\$$1}(t)\geq 0.0323994 \\
    -1.02026 \text{eps}-0.154324 \text{qtm1}-0.510128 \text{rutm1}+0.0818118
      \text{zzz$\$$0$\$$1}(t)+0.308648 \text{zzz$\$$1$\$$1}(t)+0.0430556 &
      \text{True} \\
   \end{array}
    \\
   \end{array}
\end{gather*}
%https://github.com/es335mathwiz/paperProduction.git

% Global`zzz$0$1[Global`t]==
% Global`base[aPath02ValsRec[[4,1]],aPath02ValsRec[[6,1]],0]/.
% dSub/.{Global`qtm1->Global`qqVar,Global`rutm1->Global`ruVar,Global`eps->Global`epsVar}),
% {Global`zzz$0$1[Global`t],0}]],dSub/.{Global`qtm1->Global`qqVar,Global`rutm1->Global`ruVar,Global`eps->Global`epsVar}]}






\subsection{Solution for Three Periods}


Consider a perfect foresight solution where the constraints are honored only 
for $t=0,1$.\footnote{ Perfect foresight solutions impose the constraint that future shocks are identically zero.}

Since now a future value of z is non zero, formula \ref{myEqn}  will have additional non-zero terms.


\begin{gather*}
  \vcenter{\hbox{\includegraphics{/msu/home/m1gsa00/git/ProjectionMethodTools/ProjectionMethodToolsJava/code/prettyEqns03RecA.pdf}}}\\
  \vcenter{\hbox{\includegraphics{/msu/home/m1gsa00/git/ProjectionMethodTools/ProjectionMethodToolsJava/code/prettyEqns03RecB.pdf}}}\\
  \vcenter{\hbox{\includegraphics{/msu/home/m1gsa00/git/ProjectionMethodTools/ProjectionMethodToolsJava/code/prettyEqns03RecC.pdf}}}\\
\end{gather*}
Mathematica is unable to compute this solution exactly in a reasonable amount
of time.\footnote{ 5 hours.  One could substitute the solution for the one period function to simplify the system, but Mathematica was still unable to solve the simplified system.}  However we can easily obtain numerical solutions for the z functions.  The z01 function applies to the time t variables and 
threrefore we must solve a nonlinear system that captures the fact that
the time t value of the state variables depends on this future value of z01.


Note the following graph of $r_t$ when imposing constraint for only two periods:
\begin{gather*}
\includegraphics{/msu/home/m1gsa00/git/ProjectionMethodTools/ProjectionMethodToolsJava/code/prettyrr03.pdf}
\end{gather*}





 Note the following graph of 

 \begin{gather*}
 H_{2}\lucaXt ( r_t-0.02)
 \end{gather*}

  when imposing constraint for only two periods:
 \begin{gather*}
 \includegraphics{/msu/home/m1gsa00/git/ProjectionMethodTools/ProjectionMethodToolsJava/code/prettyhapp03A.pdf}
 \end{gather*}


 Note the following graph of 

 \begin{gather*}
 H_{5}\lucaXt( r_{t+1}-0.02)
 \end{gather*}

  when imposing constraint for only two periods:
 \begin{gather*}
 \includegraphics{/msu/home/m1gsa00/git/ProjectionMethodTools/ProjectionMethodToolsJava/code/prettyhapp03B.pdf}
 \end{gather*}


Both have infinity norms near machine precision for the pecified range of values of the state variables.  There are comparable values for non zero $\epsilon$.


We can use the one and two period solutions to aid in solving the three period problem.
Complicated recursion since z, though known, is recursivly applied to itself


\begin{gather*}
  \vcenter{\hbox{\includegraphics{/msu/home/m1gsa00/git/ProjectionMethodTools/ProjectionMethodToolsJava/code/prettyEqns03RecA.pdf}}}\\
  \vcenter{\hbox{\includegraphics{/msu/home/m1gsa00/git/ProjectionMethodTools/ProjectionMethodToolsJava/code/prettyEqns03RecB.pdf}}}\\
\end{gather*}

The second equation comes from applying the $z_{0t}$ function to the time t variables.


The solution obtained is essentially identical as can be seen by evaluating the
same complementary slackness conditions above.\footnote{The Piecewise solutions now appear to have four branches.  But, this
 is must be an artifact since the time t solutions only have two branches.}


 Note the following graph of 

 \begin{gather*}
 H_{2}\lucaXt ( r_t-0.02)
 \end{gather*}

  when imposing constraint for only two periods:
 \begin{gather*}
 \includegraphics{/msu/home/m1gsa00/git/ProjectionMethodTools/ProjectionMethodToolsJava/code/prettyhapp03RecA.pdf}
 \end{gather*}


 Note the following graph of 

 \begin{gather*}
 H_{5}\lucaXt( r_{t+1}-0.02)
 \end{gather*}

  when imposing constraint for only two periods:
 \begin{gather*}
 \includegraphics{/msu/home/m1gsa00/git/ProjectionMethodTools/ProjectionMethodToolsJava/code/prettyhapp03RecB.pdf}
 \end{gather*}


\begin{itemize}
\item Equations
\item Can't analytically solve for arbitrary state
\item Can numerically solve for given state
\item Can approximate 
\item Can do recursion
\item Time and accuracy  (Time one period)
\item even perfect foresight paths can reengage constraint
\item focus in on border graph performance perfect foresight perhaps stochastic better performance
\item Strategically use Piecewise to target tough parts
\item Hermite versus spline
\item For one period some interolation failed to converge.
\item check that recursion and simultaneous match
\item parallelization
\item errors in distant future generate/reflect errors in the interim. show they matter.
\end{itemize}

\newpage

\theendnotes
\newpage

\appendix

\section{Code}
\label{sec:code}

\subsection{symb01.m}
\label{sec:symb01.m}

\listinginput{1}{/msu/home/m1gsa00/git/ProjectionMethodTools/ProjectionMethodToolsJava/code/symb01.m}

\newpage
\subsection{symb02.m}
\label{sec:symb02.m}

\listinginput{1}{/msu/home/m1gsa00/git/ProjectionMethodTools/ProjectionMethodToolsJava/code/symb02.m}


\newpage
\subsection{symb02Rec.m}
\label{sec:symb02Rec.m}

\listinginput{1}{/msu/home/m1gsa00/git/ProjectionMethodTools/ProjectionMethodToolsJava/code/symb02Rec.m}

\newpage
\subsection{symb02ValsExp.m}
\label{sec:symb02ValsExp.m}

\listinginput{1}{/msu/home/m1gsa00/git/ProjectionMethodTools/ProjectionMethodToolsJava/code/symb02ValsExp.m}


\bibliographystyle{plainnat}
\bibliography{anderson}

\end{document}
