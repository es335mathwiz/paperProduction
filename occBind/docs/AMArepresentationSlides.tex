\documentclass{beamer}
\usepackage{mathtools}
\usepackage{pgfplotstable}%for generating latex from csv files
\usepackage{datetime}
\usepackage{hyperref}


\title{``A New Series Representation for Time Invariant Functions and
a Solution Strategy for Occasionally Binding Constraints in Rational Expectations Models''}
\date{\currenttime -- \today }



\author{Gary S. Anderson\thanks{I would like to thank Luca Guerrieri, Christopher Gust and Robert Tetlow for their comments and suggestions. }}
\newcommand{\xtFuncTI}{\mathcal{X}(x,\epsilon)}
\newcommand{\XtFuncTI}{\mathbf{X}(x)}

\newcommand{\xtFunc}[1]{\mathcal{X}{#1}}
\newcommand{\XtFunc}[1]{\mathbf{X}{#1}}



\newcommand{\discr}[1]{\mathcal{D}^{#1}(x_{t-1},\epsilon_t)}

\newcommand{\XZPair}[1]{(\mathcal{X}^{#1},\mathcal{Z}^{#1})}
\newcommand{\XZPairG}[1]{(\mathcal{X}^{#1}(x^g),\mathcal{Z}^{#1}(x^g))}
\newcommand{\xIter}[2]{\mathcal{X}^{#1}(#2)}
%\newcommand{\zNow}[1]{z^{#1}_0(x_{t-1},\epsilon_t)}
%\newcommand{\ZNow}[3]{\mathcal{Z}^{#1}_{#2}(x_{#3})}
\newcommand{\zNow}[1]{z^{#1}(x_{t-1},\epsilon_t)}
\newcommand{\ZNow}[3]{\mathcal{Z}^{#1}(x_{#3})}

\newcommand{\xNow}[1]{x^{#1}_t(x_{t-1},\epsilon_t)}
\newcommand{\xNowtp}[1]{x^{#1}_{t+1}(x_{t-1},\epsilon_t)}
\newcommand{\XNow}[3]{\mathcal{X}^{#1}_{#2}(x_{#3})}








\newcommand{\sForSum}{{\nu}}
\newcommand{\rcpC}{{\mathbf{c}}}

\newcommand{\xtVec}{  \begin{bmatrix}
    q_t\\r_{t}\\r_{ut}
  \end{bmatrix}
}
\newcommand{\xtPVec}{  \begin{bmatrix}
    q_{t+1}\\r_{t+1}\\r_{ut+1}
  \end{bmatrix}
}
\newcommand{\xtMVec}{  \begin{bmatrix}
    q_{t-1}\\r_{t-1}\\r_{ut-1}
  \end{bmatrix}
}
\newcommand{\expctEps}[1]{\mathcal{E}_{\epsilon} \left [#1 \right ]}
\newcommand{\expct}[2]{E_{#1} \left [#2 \right ]}
\newcommand{\expc}[1]{\mathcal{E} \left [#1 \right ]}
\newcommand{\expcK}[2]{\mathcal{E}^{#1} \left [#2 \right ]}
\newcommand{\xsln}[1]{\mathbb{X} \left [#1 \right ]}
\newcommand{\xslnK}[2]{\hat{\mathbb{X}}^{#1} \left [#2 \right ]}
\newcommand{\xpth}[2]{\mathfrak{X}_{#1} \left [#2 \right ]}
\newcommand\infNorm[1]{\left\lVert#1\right\rVert_\infty}
\newcommand\twoNorm[1]{\left\lVert#1\right\rVert_2}
%\newcommand{\inorm}[1]{\left\lVert#1\right\rVert_\infty}

% \newcommand{\forPhi}{\begin{bmatrix}
% \psi_\epsilon&\psi_z
% \end{bmatrix}}
% \newcommand{\phiMult}{\phi \psi_\epsilon}
% \newcommand{\bMult}{B x_{-1} + \phiMult}
\newcommand{\phiMultBoth}[1]{
	\phi (\psi_\epsilon \epsilon_t +\psi_z z_0^#1(x_{-1},\epsilon_t))}
\newcommand{\bMultBoth}[1]{B x_{-1} + \phiMultBoth{#1}}


\newcommand{\bForOne}{\bMultBoth{1}
}

% \newcommand{\bForTwo}{\bMultBoth{2}+
% F \phi  \psi_z  
% Z_0^1(x_0^2(x_{-1}))   
% }



\newcommand{\compSlack}{z_0^1(x_{-1},\epsilon_t) \left ( \bar{x} -x\right )=0\\ z_0^1(x_{-1},\epsilon_t)> 0}
% \begin{gather*}
% 0= x_t-(B x_{t-1}+ \phi \psi_\epsilon\epsilon_t + \phi \psi_z 
% \xpt{z_{t}(x_{t-1},\epsilon_t)    } )
% \end{gather*}

\newcommand{\xpt}[1]{#1}


\makeatletter
\@ifundefined{newblock}{%
 \def\newblock{\hskip .11em plus .33em minus .07em} % important line
}
\makeatother


\makeatletter
\pgfplotsset{
    /pgfplots/table/omit header/.style={%
        /pgfplots/table/typeset cell/.append code={%
            \ifnum\c@pgfplotstable@rowindex=-1
                \pgfkeyslet{/pgfplots/table/@cell content}\pgfutil@empty%
            \fi
        }
    }
}
\makeatother
 

\makeatletter
\newcommand*\ExpandableInput[1]{\@@input#1 }
\makeatother


\newcommand{\tpExp}[1]{\phi^e_{#1}(x_{t-1})}
\newcommand{\lRat}[2]{\log \left( \frac{z_{#1}}{z_{#2}}\right)}


\newcommand{\anEdit}[1]
{

{\color{blue}
\begin{quote}
#1  
\end{quote}}

}

\mathtoolsset{showonlyrefs}
\usepackage{pseudocode}

\begin{document}
\begin{frame}
  \titlepage
\end{frame}

\begin{frame}
  \frametitle{Overview}
  \begin{itemize}
  \item   This paper proposes a new representation for time invariant functions.

  \item This series representation 
can accurately characterize the solutions for a 
wide array of nonlinear rational expectations models
\item provides a formula
for computing accuracy bounds for any proposed time invariant model solution.
\item The series representation serves as an important component in an algorithmfor constructing approximate solutions for nonlinear rational expectations
models.
\item In this context, it  facilitates exploiting the 
``law of iterated expectations'' in computing rational expectations solutions for models with occasionally binding constraints.
  \end{itemize}
\end{frame}

\begin{frame}
  \frametitle{Time Invariant Functions}

Consider a time invariant stochastic function $\xtFuncTI$, 
\begin{itemize}
\item where $x$ is an $L$ dimensional real variable
\item $\epsilon$ a $K$ dimensional random variable.
\item time $t$ realizations of $\epsilon$, $\epsilon_t$, to be independently and identically distributed.
\end{itemize}
We define a time invariant deterministic function $\XtFuncTI\equiv \expctEps{\xtFuncTI}$ and denote
\begin{gather*}
\expct{t}{x_{t+k}}\equiv\begin{cases}
\xtFunc{(x_{t-1},\epsilon_t)} &k=0\\
\XtFunc{(\expct{t}{x_{t+k-1}})} &k>0
\end{cases}
\end{gather*}

\end{frame}

\begin{frame}
  \frametitle{Iterating Forward: Conditional Expectations}
Consider Iterating the function $\mathcal{X}$ forward by 
recursively applying $\mathcal{X}$ to compute a solution path
\begin{gather}
\underbrace{(x_{t-1},\epsilon_t)} 
\underbrace{{\mathcal{X}}(x_{t-1},\epsilon_t)}
\underbrace{\int {\mathcal{X}}({\mathcal{X}}(x_{t-1},\epsilon_t),\epsilon_{t+1})}
\underbrace{\ldots}
\intertext{Suppose this process produces bounded trajectories for $\mathcal{X}$}
   \mathcal{X}_{t+s}(x_{t-1},\epsilon_t), \,\,\mathcal{X}_{t+s} \in{R^k}\,\,\infNorm{\mathcal{X}_{t+s}}  \le \bar{\mathcal{X}}\,\,\forall s\ge 0 \label{fFamily}.
 \end{gather}

It will then be possible to write down a useful series representation for 
the function $\mathcal{X}(x_{t-1},\epsilon_t)$.

 % \begin{itemize}
 % \item Law of Iterated Expectations applies
 %   \begin{gather*}
 %     E_t(\mathcal{X}(x_{t+k-1},\epsilon_{t+k}))=
 %     E_t(E_{t+k}(\mathcal{X}(x_{t+k-1},\epsilon_{t+k})))
 %   \end{gather*}
 % \end{itemize}


\end{frame}


\begin{frame}
  \frametitle{Series Representation}

Now, for any linear homogeneous 
$k$ dimensional 
deterministic 
system, 
\begin{gather}
  	 H_{-1} x_{t-1} + H_0 x_t + H_1 x_{t+1}=0\label{hSystem}
\end{gather}
that produces  a unique stable solution, 
it is well known\ \cite{anderson10} that
  

\begin{gather}
	 H_{-1} x_{t-1} + H_0 x_t + H_1 x_{t+1}=\psi_\epsilon \epsilon_t +\psi_{c}
\intertext{as}
x_t=B x_{t-1} + \phi \psi_\epsilon \epsilon_t + (I - F)^{-1} \phi \psi_c
\intertext{where}
\phi= (H_0 +H_1 B)^{-1} %\\F=-\phi H_1 
\end{gather}


\end{frame}


\begin{frame}
  \frametitle{Series Representation}
{\small
Given the trajectories \refeq{fFamily}, define 
$  z_{t+s}(x_{t-1},\epsilon_t)$ as  %\footnote{These $z$ functions will soon prove useful in an algorithm for computing unknown trajectories like \refeq{fFamily}.}:
{

  \begin{align}
  z_{t+s}(x_{t+s-1},\epsilon_t) \equiv& H_{-1} \mathcal{X}_{t-1}(x_{t+s-1},\epsilon_t) + \nonumber\\
& H_0 \mathcal{X}_{t+s}(x_{t-1},\epsilon_t) +  \label{defZ} \\
& H_1 \mathcal{X}_{t+s+1}(x_{t-1},\epsilon_t) \nonumber
  \end{align}
}


\cite{anderson10}  demonstrates that, for 
such models,
	 \begin{gather}
	 \mathcal{X}_{t}(x_{t-1},\epsilon_t) =B x_{t-1}+ \phi \psi_\epsilon\epsilon_t + \sum_{\sForSum=0}^\infty F^s \phi z_{t+\sForSum}(x_{t-1},\epsilon_t) + (I - F)^{-1} \phi \psi_c
\label{theSeries}\intertext{and}
	 \mathcal{X}_{t+s+1}(x_{t-1},\epsilon_t) =B \mathcal{X}_{t+s} + \sum_{\sForSum =0}^\infty F^\sForSum \phi z_{t+s+\sForSum}(x_{t-1},\epsilon_t) + (I - F)^{-1} \phi \psi_c \,\,\,\forall s \ge  0\nonumber
\intertext{where}
F=-\phi H_1 
	 \end{gather}
}

\end{frame}


\begin{frame}
  \frametitle{Series Appromimation Truncation Errors}
  \begin{itemize}
  \item Can choose an ``arbitrary''  linear system unrelated (save for dimensionality) to origin of the solution trajectories
  \item can choose the number of series terms
  \item More terms more accuracy
  \item linear system with more ``similar'' dynamics more accuracy for given number of terms
  \item RBC example
    \begin{itemize}
    \item ``arbitrary'' linear system

\framebox{Table will show truncation errors for the ``Arbitrary'' Linear Model}
  \item linearized RBC  linear system

\framebox{Table will show truncation errors for the  Linearized RBC Model}
    \end{itemize}

  \end{itemize}
\end{frame}

\begin{frame}
  \frametitle{Correctly Computing Expectations of Nonlinear Functions}
{\small 
  \begin{itemize}
  \item solutions for models that can be written in  the form
\begin{gather}
  h_i(x_{t-1},x_{t},x_{t+1},\epsilon_t)=h^{det}_{io}(x_{t-1},x_{t},\epsilon_t)+\\ 
\sum_{j=1}^{p_i} [h^{det}_{ij}(x_{t-1},x_{t},\epsilon_t)h^{nondet}_{ij}(x_{t+1})]=0
\end{gather}
\item Euler equations for the  neoclassical growth  model 
\label{sec:simple-rbc-model-ext} 
\begin{gather}
h_{10}^{det}(\cdot)=\frac{1}{c_t^\eta},\,\,
h_{11}^{det}()=\alpha \delta k_{t}^{\alpha-1} ,\,\,
h_{11}^{nondet}(\cdot)=E_t \left (\frac{\theta_{t+1}}{c_{t+1}^\eta} \right )\\
h_{20}^{det}(\cdot)=c_t + k_t-\theta_tk_{t-1}^\alpha,\,\,
h_{21}^{det}(\cdot)=0\\
h_{30}^{det}(\cdot)=\ln \theta_t -(\rho \ln \theta_{t-1} + \epsilon_t),\,\,
h_{31}^{det}(\cdot)=0
\end{gather}
\item $\epsilon_t$ is known, all stochastic components have $t+1$ time subscripts . 
\item include auxiliary variables for each $h_{ij}^{nondet}$  -- text assumes models orignally given in this form
  \end{itemize}

% the conditional expectation of nonlinear expressions,  
% accurately recursively computing  the appropriate expected values.
% Below, we will consider 
% systems that augment these dynamic equations with additional constraints 
% on the evolution of the variables.



}
\end{frame}


\begin{frame}
  \frametitle{Assessing Rational Expectations Solution Accuracy}
{\small
  \begin{itemize}
  \item Use auxiliary variables for nonlinear stochastic ``chunks''
\item interested in
 finding a time invariant function $g^\ast$ that satisfies
 \begin{gather}
   \begin{split}
 h(x_{t+s-1},g^\ast(x_{t+s-1},\epsilon_{t+s}),\mathcal{H}[g^\ast(g^\ast(x_{t+s-1},\epsilon_{t+s}),\epsilon_{t+s+1})],\epsilon_{t+s}) \label{theProblem} \\
 m(x_{t+s-1},g^\ast(x_{t+s-1},\epsilon_{t+s}),\mathcal{H}[g^\ast(g^\ast(x_{t+s-},\epsilon_{t+s}),\epsilon_{t+s+1})],\epsilon_{t+s}) \ge 0  
   \end{split}
   \intertext{define} 
  \mathcal{G}^\ast(x_{t+s-1},\epsilon_{t+s})= \mathcal{H}[g^\ast(g^\ast(x_{t+s-1},\epsilon_{t+s}),\epsilon_{t+s+1})] \nonumber
  \end{gather}
\item for all $s>0$ where $\mathcal{H}$ is an operator, 
   that maps stochastic to deterministic functions
 \begin{description}
 \item[Perfect Foresight]
 \begin{gather}
      \mathcal{H}^{PF}[g^{k}(x,\epsilon_{t+T-k+1})]=
 g^{k}(x,0)\\
 \end{gather}
 \item[Rational Expectations] 
 \begin{gather}
      \mathcal{H}^{RE}[g^{k}(x,\epsilon_{t+T-k+1})]=
 \mathcal{E}_t[g^{k}(x,\epsilon_{t+T-k+1})|x]\\
 \end{gather}
 \end{description}
  \end{itemize}
}  
\end{frame}



\begin{frame}
  \frametitle{The Algorithms}


\begin{pseudocode}{doIterRE}{linMod,XZCExpKto0,modEqns,epsDist}
xzFuncKp1[xtm1,epst]=genFPFunc(linMod,XZCExpKto0,modEqns)\\
XZCExpKp1to0=genXZFuncRE(xzFuncKp1[x,eps],XZCExpKp1to0,epsDist)\\
\RETURN{xzFuncKp1[xtm1,epst],XZCExpKp1to0}  
\end{pseudocode}

\begin{pseudocode}{genFPFunc}{linMod,XZCExpKto0,modEqns}
anFRFunc[xtGuess]=\\genFRFunc[genxkzkFunc[linMod,XZCExpKto0,xtGuess]]\\
anFPFunc[xtm1,epst]=
FixedPoint[anFRFunc[xt]]\\
\RETURN{aFPFunc[xtm1,epst]}
\end{pseudocode}
\end{frame}


\begin{frame}
  \frametitle{Algorithms}
  


\begin{pseudocode}{genFRFunc}{xzFunc,modEqns}
forFRFunc[xzArgs]=modEqns[xzFunc[xzArgs]]\\
xzFuncKp1[xtm1,epst]=FindRoot[forFRFunc[xzVals]]\\
\RETURN{xzFuncKp1[xtm1,epst]}  
\end{pseudocode}

\begin{pseudocode}{genxkzkFunc}{linMod,XZCExpKto0,xtGuess}
\RETURN{\chi[xtm1,epst,zt]}  
\end{pseudocode}


\end{frame}

\begin{frame}
  \frametitle{Algorithms}
  


{\small

% Define $\XZPair{0}\equiv(B x_{t-1}+ \phi \psi_\epsilon\epsilon_{t} +
%  (I - F)^{-1} \phi \psi_c,0)$.

Now, given $x^g$, a guess for $x_t$, compute $\XZPairG{k}$ pairs:
\begin{gather}
  \{\XZPairG{0},\ldots,\XZPairG{k}\}\intertext{ so that we can compute}
% \intertext{ find }
% \zNow{k+1} \ni \xNow{k+1} \intertext{satisfies the model equations where}
  \xNow{}=B x_{t-1} + \phi \psi_\epsilon \epsilon_t + (I - F)^{-1} \phi \psi_c +\\  \phi \zNow{k+1}+
%  \begin{cases}
%0& \mbox{if }k=0\\    
\sum_{s=1}^{k} F^s \phi  \mathcal{Z}^{k+1-s}(x_{t+1}^k)%&\mbox{if } k>0
%  \end{cases}
\intertext{and}
  x_{t+1}^{}= B x_{t} + (I - F)^{-1} \phi \psi_c +\\ 
%  \begin{cases}
%0& \mbox{if }k=0\\    
\sum_{s=0}^{k-1} F^s \phi  \mathcal{Z}^{k+1-s}(x_{t+1}^k)%&\mbox{if } k>0
%  \end{cases}
\intertext{these expressions define a function $\chi$ generating}
\chi(x_{t-1},\epsilon_t,z_t) \rightarrow
\begin{bmatrix}
  x_t\\z_t\\ \epsilon_t
\end{bmatrix}\intertext{ convenient for evaluating the model function equations.}
\end{gather}
}

\end{frame}

\begin{frame}
  \frametitle{RBC Model Example}
  
\begin{gather}
\frac{1}{c_t^{\eta}}=\alpha \delta k_{t}^{\alpha-1} E_t \left (\frac{\theta_{t+1}}{c_{t+1}^\eta} \right ) \\
c_t + k_t=\theta_tk_{t-1}^\alpha \\
\ln \theta_t =\rho \ln \theta_{t-1} + \epsilon_t\label{rbcSys}
\intertext{for $\eta=1$}
\frac{1}{c_t}=\alpha \delta k_{t}^{\alpha-1} E_t \left (\frac{\theta_{t+1}}{c_{t+1}} \right ) \\
c_t + k_t=\theta_tk_{t-1}^\alpha \\
\ln \theta_t =\rho \ln \theta_{t-1} + \epsilon_t\label{rbcSys}
\intertext{and there is a closed form solution}
  k_{t}= \theta_{t} \alpha \delta k_{t-1}^\alpha.\label{soln}\\
c_t= \theta_t k_{t-1}^\alpha (1-\alpha \delta) 
\end{gather}
\end{frame}

 \bibliographystyle{plainnat}
 \bibliography{../../bibFiles/anderson,../../bibFiles/files}


\end{document}
