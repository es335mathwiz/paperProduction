\documentclass[letter]{beamer}
\newcommand{\xtVec}{  \begin{bmatrix}
    q_t\\r_{t}\\r_{ut}
  \end{bmatrix}
}
\newcommand{\xtPVec}{  \begin{bmatrix}
    q_{t+1}\\r_{t+1}\\r_{ut+1}
  \end{bmatrix}
}
\newcommand{\xtMVec}{  \begin{bmatrix}
    q_{t-1}\\r_{t-1}\\r_{ut-1}
  \end{bmatrix}
}
\newcommand{\expct}[1]{E_{t} \left [#1 \right ]}
\newcommand{\expc}[1]{\mathcal{E} \left [#1 \right ]}
\newcommand{\expcK}[2]{\mathcal{E}^{#1} \left [#2 \right ]}
\newcommand{\xsln}[1]{\mathbb{X} \left [#1 \right ]}
\newcommand{\xslnK}[2]{\hat{\mathbb{X}}^{#1} \left [#2 \right ]}
\newcommand{\xpth}[2]{\mathfrak{X}_{#1} \left [#2 \right ]}
\newcommand\infNorm[1]{\left\lVert#1\right\rVert_\infty}
\newcommand\twoNorm[1]{\left\lVert#1\right\rVert_2}
%\newcommand{\inorm}[1]{\left\lVert#1\right\rVert_\infty}

% \newcommand{\forPhi}{\begin{bmatrix}
% \psi_\epsilon&\psi_z
% \end{bmatrix}}
% \newcommand{\phiMult}{\phi \psi_\epsilon}
% \newcommand{\bMult}{B x_{-1} + \phiMult}
\newcommand{\phiMultBoth}[1]{
	\phi (\psi_\epsilon \epsilon_t +\psi_z z_0^#1(x_{-1},\epsilon_t))}
\newcommand{\bMultBoth}[1]{B x_{-1} + \phiMultBoth{#1}}


\newcommand{\bForOne}{\bMultBoth{1}
}

% \newcommand{\bForTwo}{\bMultBoth{2}+
% F \phi  \psi_z  
% Z_0^1(x_0^2(x_{-1}))   
% }



\newcommand{\compSlack}{z_0^1(x_{-1},\epsilon_t) \left ( \bar{x} -x\right )=0\\ z_0^1(x_{-1},\epsilon_t)> 0}
% \begin{gather*}
% 0= x_t-(B x_{t-1}+ \phi \psi_\epsilon\epsilon_t + \phi \psi_z 
% \xpt{z_{t}(x_{t-1},\epsilon_t)    } )
% \end{gather*}

\newcommand{\xpt}[1]{#1}


\makeatletter
\@ifundefined{newblock}{%
 \def\newblock{\hskip .11em plus .33em minus .07em} % important line
}
\makeatother


\makeatletter
\pgfplotsset{
    /pgfplots/table/omit header/.style={%
        /pgfplots/table/typeset cell/.append code={%
            \ifnum\c@pgfplotstable@rowindex=-1
                \pgfkeyslet{/pgfplots/table/@cell content}\pgfutil@empty%
            \fi
        }
    }
}
\makeatother
 

\makeatletter
\newcommand*\ExpandableInput[1]{\@@input#1 }
\makeatother


\newcommand{\tpExp}[1]{\phi^e_{#1}(x_{t-1})}
\newcommand{\lRat}[2]{\log \left( \frac{z_{#1}}{z_{#2}}\right)}


\newcommand{\anEdit}[1]
{

{\color{blue}
\begin{quote}
#1  
\end{quote}}

}

\title{``A New Series Representation for Deterministic Maps and
a Solution Strategy for Occasionally Binding Constraints in Rational Expectations Models''}
\date{\currenttime -- \today }



\author{Gary S. Anderson\thanks{The analysis and conclusions set forth are those of the author and do not indicate concurrence by other members of the research staff or the Board of Governors. I would like to thank Luca Guerrieri, Christopher Gust, Hess Chung, Benjamin Johannsen  and Robert Tetlow for their comments and suggestions. }}


\begin{document}



\begin{frame}
  \titlepage
\end{frame}



% \begin{frame}
%   \frametitle{Barthelemy and Marx  Model 1
% \cite{marxbarthelemy2012}}

% \begin{gather}
%   \label{eq:3}
% a_t - a^{\rho_0}_{t-1} \exp(\epsilon_t)\\
%   \lambda_t-U^\prime(c_t)\\
% \beta E_t[ \alpha a_{t+1} k_t^{\alpha-1} - (1-\delta)\lambda_{t+1}] -\lambda_t \\
% c_t+k_t-a_t k_t^\alpha - (1-\delta)k_{t-1}
% \end{gather}

% \end{frame}


\begin{frame}
  \frametitle{Context and Motivation}
  \begin{itemize}
  \item Non linear models more pervasive
  \item Modeling occasionally binding constraints (OBC) essential
  \item Since \cite{Christiano2000} authors have described a variety of approaches. 

\cite{holden15:_exist_dsge,guerrieri15:_occbin,benigno09,hintermaier10,brumm10,nakov08,haefke98,nakata12,gordon11,billi11,Hintermaier2010,Guerrieri2015}
\item Series representation provides  a coherent framework for attacking a wide variety of complicated nonlinear models
      \end{itemize}
    \end{frame}

\begin{frame}
\frametitle{From Models to Approximate Solutions }
  \begin{gather}
    \fbox{Nonlinear Rational Expectations Model}\\ \Downarrow\\
\fbox{Bounded Time Invariant Function Solution}\\\Downarrow\\
\fbox{Series Representation}\\\Downarrow\\
\fbox{Series Approximation}
  \end{gather}
  \begin{itemize}
  \item  How can we take the  steps?
  \item Time invariant stochastic functions 
lead naturally to an associated family of deterministic maps
\item Deterministic maps have a convenient series representation
  \end{itemize}

\end{frame}




\begin{frame}
  \frametitle{Overview}
  \begin{itemize}
\item New series representation for deterministic maps
  \begin{itemize}
\item Series representation broadly applicable for nonlinear rational expectations models 
\item Leads to a formula for accuracy bounds for any proposed solution
\item An important component in an algorithm for constructing approximate solutions 
\item Facilitates exploiting recursive computation of the solutions for complicated models.
  \end{itemize}
%\item facilitates exploiting the ``law of iterated expectations'' in computing solutions for complicated models.
%\item convergence to time invariant model solution.
\item Mathematica code implementing the algorithms
\item Applicable for nonlinear models with OBC: a reference linear model plays the role of a ``recording device''
  \end{itemize}
\end{frame}


\begin{frame}
  \frametitle{Time Invariant Functions}

For notational simplicity, consider a {\it time invariant } stochastic function $\xtFuncTI$, 
\begin{itemize}
\item where $x$ is an $L$ dimensional real variable
\item $\epsilon$ a $K$ dimensional random variable.
\item time $t$ realizations of $\epsilon$, $\epsilon_t$, are independently and identically distributed.  
\end{itemize}
We define a time invariant deterministic function $\XtFuncTI\equiv \expctEps{\xtFuncTI}$ and denote
\begin{gather*}
\xsubtFunc{t+k}{(x_{t-1},\epsilon_t)}\equiv\begin{cases}
\xtFunc{(x_{t-1},\epsilon_t)} &k=0\\
\XtFunc{(\xsubtFunc{t+k-1}{(x_{t-1},\epsilon_t)})} &k>0
\end{cases}
\end{gather*}

\end{frame}



\begin{frame}
  \frametitle{Iterating Stochastic Functions Forward: Conditional Expectations}

 \begin{itemize}
\item Many rational expectations models have solutions characterized by a stochastic function $\mathcal{X}(x_{t-1},\epsilon_t):\mathcal{R}^n \times \mathcal{R}^k \rightarrow \mathcal{R}^n$ $\epsilon_t \sim iid$
\item Consider Iterating the function $\mathcal{X}$ forward by 
recursively applying $\XtFunc{}$ to compute a solution path
\begin{gather}
\underbrace{(x_{t-1},\epsilon_t)} 
\underbrace{{\mathcal{X}}(x_{t-1},\epsilon_t)}
\underbrace{\xsubtFunc{t+1}{(x_{t-1},\epsilon_t)}}
\underbrace{\xsubtFunc{t+2}{(x_{t-1},\epsilon_t)}}
\underbrace{\ldots}
\intertext{Suppose this iteration produces bounded trajectories }
\infNorm{\xsubtFunc{t+k}{(x_{t-1},\epsilon_t)}}  \le \bar{\mathcal{X}}\,\,\forall s\ge 0 \label{fFamily}.
 \end{gather}

 \item It will then, be possible to write down a useful 
series representation for
the function $\mathcal{X}(x_{t-1},\epsilon_t)$.
\item Based on discrepancies from an ``arbitrary'' Blanchard-Kahn linear dynamic system
 \end{itemize}

 % \begin{itemize}
 % \item Law of Iterated Expectations applies
 %   \begin{gather*}
 %     E_t(\mathcal{X}(x_{t+k-1},\epsilon_{t+k}))=
 %     E_t(E_{t+k}(\mathcal{X}(x_{t+k-1},\epsilon_{t+k})))
 %   \end{gather*}
 % \end{itemize}


\end{frame}

\begin{frame}
  \frametitle{Series Representation}
  \begin{itemize}
  \item 
For any linear homogeneous 
$k$ dimensional 
deterministic 
system 
\begin{gather}
  	 H_{-1} x_{t-1} + H_0 x_t + H_1 x_{t+1}=0\label{hSystem}
\end{gather}
  
 that produces  a unique stable solution, 
it is well known\ \cite{anderson10} that  inhomogeneous solutions can be computed


\begin{gather}
	 H_{-1} x_{t-1} + H_0 x_t + H_1 x_{t+1}=\psi_\epsilon \epsilon_t +\psi_{c}
\intertext{as}
x_t=B x_{t-1} + \phi \psi_\epsilon \epsilon_t + (I - F)^{-1} \phi \psi_c
\intertext{where}
\phi= (H_0 +H_1 B)^{-1}  \text{ and } \,\,F=-\phi H_1 
\end{gather}
\item 
Define $\linMod \equiv \linModMats$ 
  \end{itemize}

\end{frame}


\begin{frame}
  \frametitle{Series Representation}
{\small
Given the trajectories \refeq{fFamily}, define 
$  z_{t+s}(x_{t-1},\epsilon_t)$ as  %\footnote{These $z$ functions will soon prove useful in an algorithm for computing unknown trajectories like \refeq{fFamily}.}:
{

  \begin{align}
  z_{t+s}(x_{t-1},\epsilon_t) \equiv& H_{-1} \mathcal{X}_{t+s-1}(x_{t-1},\epsilon_t) + \nonumber\\
& H_0 \mathcal{X}_{t+s}(x_{t-1},\epsilon_t) +  \label{defZ} \\
& H_1 \mathcal{X}_{t+s+1}(x_{t-1},\epsilon_t) \nonumber
  \end{align}
}


\cite{anderson10}  also proves that, for these linear models,
	 \begin{gather}
	 \mathcal{X}_{t}(x_{t-1},\epsilon_t) =B x_{t-1}+ \phi \psi_\epsilon\epsilon_t + (I - F)^{-1} \phi \psi_c +\\ \sum_{\sForSum=0}^\infty F^s \phi z_{t+\sForSum}(x_{t-1},\epsilon_t) 
\label{theSeries}\intertext{and}
	 \mathcal{X}_{t+s+1}(x_{t-1},\epsilon_t) =B \mathcal{X}_{t+s} + \sum_{\sForSum =0}^\infty F^\sForSum \phi z_{t+s+\sForSum}(x_{t-1},\epsilon_t) + (I - F)^{-1} \phi \psi_c \,\,\,\forall s \ge  0
%\intertext{where}
%F=-\phi H_1 
	 \end{gather}
}

\end{frame}


\begin{frame}
  \frametitle{RBC Model Example}
  
\begin{gather}
\frac{1}{c_t^{\eta}}=\alpha \delta k_{t}^{\alpha-1} E_t \left (\frac{\theta_{t}}{c_{t+1}^\eta} \right ) \\
c_t + k_t=\theta_{t-1}k_{t-1}^\alpha \\
 \theta_t =\theta_{t-1}^\rho e^{\epsilon_t}\label{rbcSys}
\intertext{for $\eta=1$}
\frac{1}{c_t}=\alpha \delta k_{t}^{\alpha-1} E_t \left (\frac{\theta_{t}}{c_{t+1}} \right ) \\
c_t + k_t=\theta_{t-1}k_{t-1}^\alpha \\
\theta_t =\theta_{t-1}^\rho e^{\epsilon_t}\label{rbcSys}
\intertext{and there is a closed form solution}
  k_{t}= \alpha \delta \theta_{t} k_{t-1}^\alpha.\label{soln}\\
c_t=  (1-\alpha \delta) \theta_{t} k_{t-1}^\alpha
\end{gather}
\end{frame}

\begin{frame}
  \frametitle{An ``Almost Arbitrary'' Linear Model}
 {\small 
Consider the following constructed from ``nearly'' arbitrary coefficients
\begin{gather}
  \begin{bmatrix}
H_{-1}&H_{0}&H_{1} 
  \end{bmatrix}=
\vcenter{\hbox{\includegraphics{refHmat.pdf}}}\intertext{with $\psi_c=\psi_\epsilon=0, \,\,  \psi_z=I$.
These coefficients  happen to produce a unique stable solution.}
  B=
\vcenter{\hbox{\includegraphics{refBmat.pdf}}}\\
\phi=
\vcenter{\hbox{\includegraphics{refPhimat.pdf}}}\\
F=
\vcenter{\hbox{\includegraphics{refFmat.pdf}}}
\end{gather} 
}
\end{frame}




\begin{frame}
  \frametitle{Series Approximation Truncation Errors}
  \begin{itemize}
  \item Can choose an ``arbitrary''  linear system unrelated (save for dimensionality) to the source of the solution trajectories
  \item can choose the number of series terms
    \begin{gather}
      \label{eq:1}
\sum_{s=k+1}^{\infty} F^s \phi \psi_z = (I -F)^{-1} F^{k+1}\phi \psi_z       
    \end{gather}
  \item More terms more accuracy
  \item linear system with more ``similar'' dynamics more accuracy for given number of terms
  \end{itemize}
\end{frame}

\begin{frame}
  \frametitle{An RBC example: Bound Versus Actual}
  \begin{itemize}
  \item $B_n$ Infinity norm of truncation error matrix times $Z_n$
  \item $Z_n$ Size of difference between truncated series and exact value
  \item Pessimistic Bounds
  \item Arbitrary Model Converges, but Linearized RBC performs better
  \end{itemize}
 \includegraphics[height=1.3in]{arbBoundsVActual.pdf}
 \includegraphics[height=1.3in]{simpBoundsVActual.pdf}
\end{frame}
\begin{frame}
  \frametitle{Assessing Rational Expectations Solution Accuracy}
{\small
  \begin{itemize}
  \item the $m$ function codifies changes in model equations due to
inequality constraints or regime switching.$(\varpi \in \{1,\ldots,n\})$
\item interested in finding a time invariant function $g^\ast$ that satisfies
 \begin{gather}
   \begin{split}
 h_{\varpi}(x_{t+s-1},g^\ast(x_{t+s-1},\epsilon_{t+s}),\mathcal{H}[g^\ast(g^\ast(x_{t+s-1},\epsilon_{t+s}),\epsilon_{t+s+1})],\epsilon_{t+s}) \label{theProblem} \\
\varpi= m(x_{t+s-1},g^\ast(x_{t+s-1},\epsilon_{t+s}),\mathcal{H}[g^\ast(g^\ast(x_{t+s-},\epsilon_{t+s}),\epsilon_{t+s+1})],\epsilon_{t+s}) 
   \end{split}\intertext{ for all $s>0$ where $\mathcal{H}$ is an operator,    that maps stochastic to deterministic functions}
  \end{gather}
 \begin{description}
 \item[Perfect Foresight]
 \begin{gather}
      \mathcal{H}^{PF}[g^{k}(x,\epsilon_{t+T-k+1})]=
 g^{k}(x,0)\\
 \end{gather}
 \item[Rational Expectations] 
 \begin{gather}
      \mathcal{H}^{RE}[g^{k}(x,\epsilon_{t+T-k+1})]=
 \mathcal{E}_t[g^{k}(x,\epsilon_{t+T-k+1})|x]\\
 \end{gather}
 \end{description}
  \end{itemize}
}  
%  \intertext{define} 
%  \mathcal{G}^\ast(x_{t+s-1},\epsilon_{t+s})= \mathcal{H}[g^\ast(g^\ast(x_{t+s-1},\epsilon_{t+s}),\epsilon_{t+s+1})] \nonumber


\end{frame}

\begin{frame}
  \frametitle{Assessing Accuracy}
{\small

  \begin{itemize}
  \item As with series approximation,
 construct a family of bounded trajectories and compute
$  z_{t+s}(x_{t-1},\epsilon_t)$ as  %\footnote{These $z$ functions will soon prove useful in an algorithm for computing unknown trajectories like \refeq{fFamily}.}:
{
\begin{gather}
  z_{t+s}(x_{t-1},\epsilon_t) \equiv\\
   \begin{split}
 h_{\varpi}(\mathcal{X}_{t+s-1},g^\ast(\mathcal{X}_{t+s-1},\epsilon_{t+s}),\mathcal{H}[g^\ast(g^\ast(\mathcal{X}_{t+s-1},\epsilon_{t+s}),\epsilon_{t+s+1})],\epsilon_{t+s}) \label{theProblem} \\
\varpi= m(\mathcal{X}_{t+s-1},g^\ast(\mathcal{X}_{t+s-1},\epsilon_{t+s}),\mathcal{H}[g^\ast(g^\ast(\mathcal{X}_{t+s-},\epsilon_{t+s}),\epsilon_{t+s+1})],\epsilon_{t+s}) 
   \end{split}
  \end{gather}
}
\item The formula \refeq{theSeries} provides information about how much $x_{t}$ would need
to change in order for the trajectory to honor the constraints along the path
\item An exact solution should produce zero for all the $z$ functions
\item One can use a truncated series to carry out the approximation for changes in $x_t$
\item Useful to have measures of accuracy that don't rely upon knowing the solution beforehand
\item can adopt Judd approach for loss function for characterizing importance of errors
  \end{itemize}
}
\end{frame}


% \begin{frame}
%   \frametitle{Exact Solution Correct to Machine Precision}
%  % \includegraphics[height=1.3in]{exactNormErrorZeroEQ1.pdf}
%  % \includegraphics[height=1.3in]{exactNormErrorZeroEQ2.pdf}

%  % \includegraphics[height=1.3in]{exactNormErrorZeroEQ3.pdf}
%  % \includegraphics[height=1.3in]{exactNormErrorZeroEQ4.pdf}
% \end{frame}

% \begin{frame}
%   \frametitle{Various Decision Rules and Error Bounds}

%   \begin{itemize}
%   \item Using the ``worst'' $\epsilon_t$
% {\tiny
%   \begin{tabular}{|l|r|r|r|r|}
% \hline
%     \multicolumn{1}{|c|}{Decision Rule}&
%     \multicolumn{1}{|c|}{Bound: Global}&
%     \multicolumn{1}{|c|}{Bound: Local}&
%     \multicolumn{1}{|c|}{Actual Max Error:Global}&
%     \multicolumn{1}{|c|}{Actual Max Error:Local}\\
% \hline
% Exact&0&0&0&0\\
% \hline
%   \end{tabular}
% }

%   \end{itemize}


% \end{frame}


% \begin{frame}
%   \frametitle{Assessing Accuracy: the RBC Model Example}
%   \begin{itemize}
%   \item Validate the exact solution

% \framebox{``Arbitrary'' Linear Model} =\framebox{linearized RBC Linear Model}
%   \item How close is the perfect foresight solution to $\eta=1$ solution
% \framebox{Table of truncation errors ``Arbitrary'' Linear Model} $\ne$\framebox{table of truncation errors linearized RBC Linear Model}
%   \item How close is the perfect foresight solution to $\eta \ne 1$ solution
% \framebox{Table of truncation errors ``Arbitrary'' Linear Model} $\ne$\framebox{table of truncation errors linearized RBC Linear Model}
%   \item How close is the exact $\eta=1$  solution to $\eta \ne 1$ solution
% \framebox{Table of truncation errors ``Arbitrary'' Linear Model} $\ne$\framebox{table of truncation errors linearized RBC Linear Model}
%   \end{itemize}
% \end{frame}



\begin{frame}
  \frametitle{Discovering Unknown Solutions}

A, conceptually, very simple solution strategy:  

  \begin{enumerate}
  \item Begin with some convergent linear model, $\linMod$, of appropriate dimension.
  \item Compute solutions honoring the model equations for time t, 
but that assume the trajectories subsequently 
evolve according to the convergent 
linear model, $\linMod$.
  \item Given solutions valid for time $t$ to $t+k$, extend the solutions to be valid for time $t$ to $t+k+1$ \label{stepNo}
\item Repeat step \ref{stepNo} unless truncation formulas indicate the solution is sufficiently accurate
  \end{enumerate}


\end{frame}


\begin{frame}
  \frametitle{Algorithm based on basic properties of deterministic maps.}
  \begin{itemize}
  \item     The $\epsilon$'s  play two distinct roles in the algorithm.
To see the two roles, consider a family of stochastic functions 
$\{\stochF{T},\ldots,\stochF{0}\}$ with 
$\stochF{k}(x,\epsilon_t): {\mathcal{R}}^{L+1} \rightarrow \mathcal{R}^L$ 
    \begin{enumerate}
\item I assume  $\epsilon_t \sim $ iid. Knowing the distribution 
of the $\epsilon_t$ makes it possible to compute 
a corresponding family of deterministic functions $\detF{k}(x)=E( \stochF{k}(x,\epsilon_t)|x) $
    \item I assume that at some initial time, say t=0, a particular 
 $\epsilon_0$ draw 
along with an $x_{-1}$, determines a unique $x_0=\stochF{T}\initXE$.
Subsequently, $x_{t}=\detF{{T-t}}(x_{t-1})\,\, \forall t>0$
    \end{enumerate}
 \item The series representation requires a unique trajectory beginning with $x_0=\stochF{T}\initXE$ and 
evolving forward from any $x_0$ given by, a potentially time varying deterministic map,  $x_t=\detF{T-t}(x_{t-1})\,\, \forall t>0$. 
  \item The algorithm for discovering unknown solutions does not rely on time invariance.
  \end{itemize}
    
  \end{frame}




\begin{frame}
  \frametitle{Recursive updating}
{\small
  \begin{itemize}
  \item Given a linear reference model,$\linModMats$,  any bounded 
sequence of deterministic maps $\{\detF{T},\ldots,\detF{0}\}, \,\, T>0$ generates 
a series representation
  \begin{gather}
     \label{eq:2}
	 \mathcal{X}_{t}(x_{t-1}) =B x_{t-1}+  (I - F)^{-1} \phi \psi_c +\\ \sum_{\sForSum=0}^{T-1} F^s \phi z_{T-\sForSum}(x_{t-1}) \intertext{it is straightforward to 
update to a new series representation for the map trajectories resulting from prepending an additional deterministic map to the sequence $\detF{T+1},\ldots,\detF{0}\}$}
	 \mathcal{X}_{t}(x_{t-1}) =B x_{t-1}+  (I - F)^{-1} \phi \psi_c +\\ 
\phi z_{T+1}(x_{t-1}) + \sum_{\sForSum=1}^{T} F^s \phi z_{T+1-\sForSum}(x_{t-1}) 
  \end{gather}
\item 
  \end{itemize}
}
\end{frame}


%\footnote{We set $\psi_c=0$  without loss of generality 
%in the expressions that follow merely to save on notation.}

\begin{frame}
\frametitle{The model of interest and a linear reference model}
\begin{itemize}
\item The target model is of the form
\begin{gather}
\nlEqnLHS{x_{t-1}}{x_t}{x_{t+1}}{t}=\\
\nlEqn{x_{t-1}}{x_t}{x_{t+1}}{t}=0\\ \label{refMod}
 (L \times 1) +  (L \times L ) \cdot (L \times 1)\\
\nlEqnSel{x_{t-1}}{x_t}{x_{t+1}}{t}\\ 
\end{gather}



\item The algorithm constructs a sequence of stochastic 
functions 
  \begin{gather}
	 \stochF{k}\initXE =B x_{t-1}+ (I - F)^{-1} \phi \psi_c+  \phi \psi_\epsilon\epsilon_0  +\\ \phi z^{\{k\}}\initXE+ \sum_{\sForSum=1}^{k-1} F^s \phi z^{\{k-\nu\}}_{\sForSum}\initXE 
\label{theSeries}
  \end{gather}

\item One can show that the algorithm begins with solutions satisfying 
the linear reference model $\linModMats$ and
constructs a sequence of functions
 honoring the target model equations for successively 
longer solution horizons.
  \end{itemize}
\end{frame}


\begin{frame}
  \frametitle{Correctly Computing Expectations of Nonlinear Functions}

  \begin{itemize}
\item Auxiliary equations for non linear ``chunks'' provide mechanism for characterizing  nonlinear expectations in the future
  \item Formula maps future equation errors  to current time changes in model
variables.
\item  $\epsilon_t$ is known so that,
given a function characterizing future expectation,  model requires
a deterministic solution at time $t$
\item The formula is linear in the potentially nonlinear $z$ functions
  \end{itemize}
\end{frame}


\begin{frame}
\frametitle{Early iterations}
{\small
  \begin{itemize}
  \item $k=0$: Completely ignore the target model
  \begin{gather}
z^{\itSup{0}}\initXE=0 \intertext{ so that}
\stochF{0}\initXE=B x_{-1} + (I - F)^{-1} \phi \psi_c +\phi \psi_\epsilon \epsilon_0  \intertext{initialize the sequence of deterministic maps with }
\{\detF{0}\initX=B x \}
  \end{gather}
  \item $k=1$: Employ the target model for one period. Use
  \begin{gather}
x_0\initXE=B x_{-1}+ (I - F)^{-1} \phi \psi_c +\phi \psi_\epsilon \epsilon_0 + \phi z^{\itSup{1}}\initXE\\
E(x_1\initXE)=\detF{0}(x_0\initXE) \intertext{in the deterministic equation 
\refeq{refMod} to determine $z^{\itSup{1}}\initXE$ so}
\stochF{1}\initXE = B x_{-1}+ (I - F)^{-1} \phi \psi_c +\phi \psi_\epsilon \epsilon_0 + 
\phi z^{\itSup{1}}\initXE \intertext{%Define $Z^{\itSup{1}}(x)\equiv E \left [ z^{\itSup{1}}\initXEN | x \right ]$ and 
 augment the sequence}
\{\detF{1}\initX=B x  + \phi Z^{\itSup{1}}(x),\detF{0}\}
  \end{gather}
  \end{itemize}
}
\end{frame}

\begin{frame}
\frametitle{Early iterations}
{\small
  \begin{itemize}
  \item $k=2$: Employ the target model for two periods. Use
  \begin{gather}
x_0\initXE=B x_{-1}+ (I - F)^{-1} \phi \psi_c +\phi \psi_\epsilon \epsilon_0 +\\ \phi z^{\itSup{2}}\initXE + F \phi Z^{\itSup{1}}(x_0\initXE)\\
E(x_1\initXE)=\detF{1}(x_0\initXE) \intertext{in the deterministic equation 
\refeq{refMod} to determine $z^{\itSup{2}}\initXE$}
\stochF{2}\initXE = B x_{-1}+ (I - F)^{-1} \phi \psi_c +\phi \psi_\epsilon \epsilon_0 +\\ \phi z^{\itSup{2}}\initXE + F \phi Z^{\itSup{1}}(x_0\initXE)\intertext{
Define $Z^{\itSup{2}}(x)\equiv 
E \left [ z^{\itSup{2}}\initXEN | x \right ]$}
\{\detF{2}\initX=B x  + \phi  Z^{\itSup{2}}(x) + F \phi Z^{\itSup{1}}(x_0\initXE),
\detF{1}\initX,\detF{0}\initX\}
  \end{gather}
  \end{itemize}
}
\end{frame}


\begin{frame}
\frametitle{Early iterations}
{\small
  \begin{itemize}
  \item $k=3$: Employ the target model for two periods. Use
  \begin{gather}
x_0\initXE=B x_{-1}+ (I - F)^{-1} \phi \psi_c +\phi \psi_\epsilon \epsilon_0 + \phi z^{\itSup{3}}\initXE +\\ F \phi Z^{\itSup{2}}(x_0\initXE) + F^2 \phi Z^{\itSup{1}}(\detF{2}(x_0\initXE))\\
E(x_1\initXE)=\detF{2}(x_0\initXE) \intertext{in the deterministic equation 
\refeq{refMod} to determine $z^{\itSup{3}}\initXE$}
\stochF{3}\initXE = B x_{-1}+ (I - F)^{-1} \phi \psi_c +\phi \psi_\epsilon \epsilon_0 + \phi z^{\itSup{3}}\initXE + \\F \phi Z^{\itSup{2}}(x_0\initXE) +F^2 \phi Z^{\itSup{1}}(\detF{2}(x_0\initXE))\intertext{
Define $Z^{\itSup{3}}(x)\equiv 
E \left [ z^{\itSup{3}}\initXEN | x \right ]$}
\detF{2}\initX=B x  + \phi  Z^{\itSup{3}}(x) + 
F \phi Z^{\itSup{2}}(x_0\initXE) + \\F^2 \phi Z^{\itSup{1}}(\detF{1}(x_0\initXE))
  \end{gather}
  \end{itemize}
}
\end{frame}

\begin{frame}
\frametitle{General iterations}
{\small
  \begin{itemize}
  \item $k=K+1$: Employ the target model for $K+1$ periods. Use
  \begin{gather}
x_0\initXE=B x_{-1}+ (I - F)^{-1} \phi \psi_c +\phi \psi_\epsilon \epsilon_0 + \phi z^{\itSup{K+1}}\initXE +\\ \sum_\sForSum^K F^\nu \phi Z^{\itSup{K+1-s}}\detComp{x_0\initXE} \\
E(x_1\initXE)=\detF{K}(x_0\initXE) \intertext{in the deterministic equation 
\refeq{refMod} to determine $z^{\itSup{K+1}}\initXE$}
\stochF{K+1}\initXE = B x_{-1}+ (I - F)^{-1} \phi \psi_c +\phi \psi_\epsilon \epsilon_0 + \phi z^{\itSup{K+1}}\initXE + \\
\sum_\sForSum^K F^\nu \phi Z^{\itSup{K+1-s}}\detComp{x_0\initXE} 
  \end{gather}
  \end{itemize}
}
\end{frame}







\begin{frame}
  \frametitle{The RBC Model ($\eta=1$): Recovering a Known Solution}
  \begin{itemize}
  \item Compute $\linMod$ for RBC Model $\eta=1,\rcpC_t=\frac{1}{c_t}$
  \end{itemize}

Applying formula \refeq{theSeries} produces:

{\tiny
\begin{gather}
  \begin{bmatrix}
c_t\\k_t\\ \rcpC_t\\\theta_t
  \end{bmatrix}=\\%paperCalcsRBCExample xt00
   \left(
   \begin{array}{c}
 0.359845 \epsilon _t+0.692632 k_{t-1}+0.341853 \theta _{t-1}-0.0442851
   \text{z1}_{t-1}+0.658 \text{z2}_{t-1}+0.359845 \text{z3}_{t-1}-0.111552 \\
 0.187032 \epsilon _t+0.36 k_{t-1}+0.17768 \theta _{t-1}+0.0442851
   \text{z1}_{t-1}+0.342 \text{z2}_{t-1}+0.187032 \text{z3}_{t-1}-0.0579799 \\
 -5.34898 k_{t-1}+0.342 \text{z1}_{t-1}-5.08153
   \text{z2}_{t-1}+\text{z4}_{t-1}+3.7794 \\
 \epsilon _t+0.95 \theta _{t-1}+\text{z3}_{t-1}+0.05 \\
   \end{array}
   \right)
\end{gather}
}

and 


{\tiny
%xt01=Private`computeNextXt[{Private`bmat,Private`phimat,Private`fmat,Private`psieps,Private`psic,Private`psiz},solnFunc00PF[[3+Range[3]]],{{cc},{kk},{tt}},{1}]//N//Expand//Simplify
\begin{gather}
  \begin{bmatrix}
c_{t+1}\\k_{t+1}\\ \rcpC_{t+1}\\\theta_{t+1}
  \end{bmatrix}=\\%paperCalcsRBCExample xt00
  \left(
   \begin{array}{c}
 0.471397 \epsilon _t+0.249347 k_{t-1}+0.447827 \theta _{t-1}+0.0306732
   \text{z1}_{t-1}+0.23688 \text{z2}_{t-1}+0.471397 \text{z3}_{t-1}-0.134618 \\
 0.245012 \epsilon_t+0.1296 k_{t-1}+0.232761 \theta _{t-1}+0.0159426
   \text{z1}_{t-1}+0.12312 \text{z2}_{t-1}+0.245012 \text{z3}_{t-1}-0.0699687 \\
 -1.00043 \epsilon _t-1.92563 k_{t-1}-0.950409 \theta _{t-1}-0.23688
   \text{z1}_{t-1}-1.82935 \text{z2}_{t-1}-1.00043 \text{z3}_{t-1}+4.08954 \\
 0.95 \epsilon _t+0.9025 \theta _{t-1}+0.95 \text{z3}_{t-1}+0.0975 \\
   \end{array}
   \right)
\end{gather}}


\end{frame}


% \begin{frame}
%   \frametitle{Solution Graphs}
  

% \includegraphics[width=2.4in]{../../../ProjectionMethodTools/ProjectionMethodToolsJava/code/exact.pdf}
% \includegraphics[width=2.4in]{../../../ProjectionMethodTools/ProjectionMethodToolsJava/code/first.pdf}
% \includegraphics[width=2.4in]{../../../ProjectionMethodTools/ProjectionMethodToolsJava/code/second.pdf}
% \includegraphics[width=2.4in]{../../../ProjectionMethodTools/ProjectionMethodToolsJava/code/third.pdf}
% \includegraphics[width=2.4in]{../../../ProjectionMethodTools/ProjectionMethodToolsJava/code/fourth.pdf}
% \includegraphics[width=2.4in]{../../../ProjectionMethodTools/ProjectionMethodToolsJava/code/fifth.pdf}

% \end{frame}


\begin{frame}
  \frametitle{Begin by Solving a Deterministic System at time $t$}
{\small

  \begin{itemize}
  \item For any given $\left (  \begin{bmatrix}
c_{t-1}\\k_{t-1}\\ \rcpC_{t-1}\\\theta_{t-1}
  \end{bmatrix}, \epsilon_t \right )=\tArg$ 
compute
  \begin{gather}
    \label{eq:3}
    x_t^1\tArg=B x_{t-1} + \phi \psi_e\epsilon_t + \phi z^1_t\tArg\\
    E_t(x^1_{t+1}\tArg)=B x^1_{t}\tArg
  \end{gather}
\item The model equations provide a deterministic system  for computing $  z^1_t=\begin{bmatrix}
    z^1_{1t}\tArg\\
    z^1_{2t}\tArg\\
    z^1_{3t}\tArg\\
    z^1_{4t}\tArg
  \end{bmatrix}$.
\item The solution satisfies the nonlinear model equations for one 
period and the stand-in linear model for subsequent periods.
\item Assess accuracy
  \end{itemize}
}

\end{frame}

% \begin{frame}
%   \frametitle{Assess One Period Solution Accuracy}
%   \begin{itemize}
%   \item Show graphs of solution
%   \item Show graphs of error
%   \item Report accuracy assessment bounds
%   \end{itemize}
% \end{frame}

\begin{frame}
  \frametitle{Nonlinear 2 Periods: Solve time $t$ Deterministic
    System }
{\small
  \begin{itemize}
  \item Compute $Z^1(x)= E_t(z^1_t(x,\epsilon_t))$
  \item For any given $\tArg$ 
compute
{\small
  \begin{gather}
    x_t^2\tArg=B x_{t-1} + \phi \psi_e\epsilon_t + \phi z^2_t\tArg + F \phi Z^1(x^2_t) \label{bothS}\\
    E_t(x^2_{t+1}\tArg)=B x^2_{t}\tArg+ \phi Z^1(x^2_t)
  \end{gather}
}
\item The model equations provide a deterministic system  for computing $  z^2_t=\begin{bmatrix}
    z^2_{1t}\tArg\\
    z^2_{2t}\tArg\\
    z^2_{3t}\tArg\\
    z^2_{4t}\tArg
  \end{bmatrix}$.
\item The solution satisfies the nonlinear model equations for two 
period and the stand-in linear model for subsequent periods.
\item unlike the first step, $x^2_t$ appears on both sides of equation \refeq{bothS}
\item  surprisingly, a simple fixed point iteration solves the nonlinear system
\item Assess accuracy
  \end{itemize}
}
\end{frame}

% \begin{frame}
%   \frametitle{Assess 2 Period Solution Accuracy}
%   \begin{itemize}
%   \item Show graphs of solution
%   \item Show graphs of error
%   \item Report accuracy assessment bounds
%   \end{itemize}
% \end{frame}



\begin{frame}
  \frametitle{Barthelemy and Marx  Model 2: Regime Switching
\cite{marxbarthelemy2012}}


\cite{troy2007}
\begin{gather}
  \label{eq:4}
  i_t =E_t \pi_{t+1} + r_t\\
r_t= \rho r_{t-1} +u_t\\
i_t=\alpha_{s_t} \pi_t
\end{gather}

Bounded solutions if and only if all eigenvalues of 
\begin{gather}
  \label{eq:5}
  \begin{bmatrix}
    \frac{1}{|a_1|}&0\\
0&    \frac{1}{|a_2|}
  \end{bmatrix}
  \begin{bmatrix}
    p_{11}&p_{12}\\p_{21}&p_{22}
  \end{bmatrix}
\end{gather}
 are inside unit circle

\end{frame}

% \begin{frame}
%   \frametitle{Barthelemy and Marx  Model 3: ZLB
% \cite{marxbarthelemy2012}}

% \begin{gather}
% x_t=E_t x_{t+1} - \sigma (i_t - E_t \pi_{t+1} - r_t^n)\\
% \pi_t = \beta E_t\pi_{t+1} + \kappa x_t + u_t\\
% i_t\ge -r^\ast\\
% i_t= \max ( -r^\ast,\alpha \pi_t)
% \end{gather}

% \end{frame}

% \begin{frame}
%   \frametitle{\cite{Guerrieri2015}}
% {\small
%   \begin{gather}
%     \label{eq:6}
% h_0 \lucaArgs =
% \begin{bmatrix}
% q_t -( \rho q_{t-1} - \sigma r_t + r^u_t)\\
%   (r^u_t-r^\ast) = \rho_u (r^u_{t-1}-r^\ast) +  \epsilon^u_t\\
% r_t-  \gamma q_t \\
% \end{bmatrix}\\
% h_1 \lucaArgs =
% \begin{bmatrix}
% q_t -( \rho q_{t-1} - \sigma r_t + r^u_t)\\
%   (r^u_t-r^\ast) = \rho_u (r^u_{t-1}-r^\ast) +  \epsilon^u_t\\
% r_t-  \bar{r} \\
% \end{bmatrix}\\
%  H_{0,1}\lucaArgs=
%  \begin{bmatrix}
%    -\beta (1 - \rho)&0&0\\0&0&0\\0&0&0
%  \end{bmatrix}\\
% % q_t =(\beta (1 - \rho) q_{t+1} + \rho q_{t-1} - 
% %       \sigma r_t + r^u_t)\\
% %   (r^u_t-r^\ast) = \rho_u (r^u_{t-1}-r^\ast) +  \epsilon^u_t\\
% % r_t-   \begin{cases}
% %  \gamma q_t & \gamma q_t \ge  \bar{r}\\
% %  \bar{r}&\gamma q_t < \bar{r}
% %    \end{cases}=0
% \varpi=%m\lucaArgs=
% \begin{cases}
%   0& \gamma q_t \ge  \bar{r}\\
%   1& \gamma q_t <  \bar{r}
% \end{cases}
% \end{gather}
% }
% \end{frame}

% \begin{frame}
%   \frametitle{Liu, Waggoner and Zha \cite{liu11:_sourc}}
%   \begin{itemize}
%   \item moderate-sized nonlinear model DSGE
%   \item regime switching
%   \item non linear and log linearized versions
%   \end{itemize}
% \end{frame}

% \begin{frame}
%   \frametitle{\cite{werner13,shi12:_liquid}}
  
% \end{frame}









% \begin{frame}
%   \frametitle{Model Definition: Functional Programming Style}
% {\small
%   \begin{itemize}
%   \item Model Definition
%     \begin{description}
%       \item[Model Equations $\modEqns()$]\ 
%    {\tiny     \begin{gather}
%           \label{eq:4}
%           \modEqnsMap
%         \end{gather}}
%       \item[Linear Reference Model: $\linMod$]$\linModMats$
%       \item[Probability Distributions: $\epsDist_i()$] 
%     \end{description}
%   \item Solution Programs/Functions
%     \begin{description}
% \item[doIterRE (See slide \ref{doIterREDef})] Increment number of periods the model equation are honored
% \item[genFPFunc (See slide \ref{genFPFuncDef})] Guess  $x^g_t$, calculate $E_t x^g_{t+1}$, use
% formula \refeq{theSeries} to compute $x_t$, set $x^g_t=x_t$, repeat until $x^g_t$ no longer changes.
% \item[genFRFunc (See slide \ref{genFRFuncDef})] Given  $x^g_t$ and corresponding conditional expectations,  use model equations  to solve for $z_t$ and $x_t$
% \item[ $\aleph$ (See slide \ref{alephDef})] Given $x^g_t$ and corresponding conditional expectations
% apply formula \refeq{theSeries} to compute $x_t$  and $x^g_{t+1}$ 
%     \end{description}
%   \end{itemize}
% }
% \end{frame}


% \begin{frame}
%   \frametitle{Convergence to Known Solution}
%   \begin{itemize}
%  \item  Machine Precision versus Approximation via  Interpolation
%   \item Show graphs of solution
%   \item Show graphs of error
%   \item Report accuracy assessment bounds
%   \end{itemize}
% \end{frame}

% \begin{frame}
%   \frametitle{Occasionally Binding Constraints}
%   \begin{itemize}
%   \item Model definition
% \begin{gather}
% \frac{1}{c_t^{\eta}}=\alpha \delta k_{t}^{\alpha-1} E_t \left (\frac{\theta_{t}}{c_{t+1}^\eta} \right ) \\
% c_t +I_t=\theta_{t-1} k^\alpha_{t-1}\\
% \ln \theta_t =\rho \ln \theta_{t-1} + \epsilon_t\label{rbcSysOBC}\\
% I \ge \bar{I}\intertext{$\varpi \in \{1,2\} $ }
% \frac{1}{c_t^{\eta}}=\alpha \delta k_{t}^{\alpha-1} E_t \left (\frac{\theta_{t}}{c_{t+1}^\eta} \right ) \\
% I_t=
% \begin{cases}
% \theta_{t-1} k^\alpha_{t-1}-c_t&\theta_{t-1} k^\alpha_{t-1}-c_t > 0,\varpi=1\\
% \bar{I}&\theta_{t-1} k^\alpha_{t-1}-c_t \le 0,\varpi=2
% \end{cases}\\
% \ln \theta_t =\rho \ln \theta_{t-1} + \epsilon_t\label{rbcSysOBC}\\
% \intertext{$\varpi \in \{1,2\} $ }
% \end{gather}

%   \item Show graphs of solution
%   \item Show graphs of error
%   \item Report accuracy assessment bounds
%   \end{itemize}
% \end{frame}


% \begin{frame}
%   \frametitle{Regime Switching }
%   \begin{itemize}
%   \item Model definition
%   \item Show graphs of solution
%   \item Show graphs of error
%   \item Report accuracy assessment bounds
%   \end{itemize}
      
%     \end{frame}
%     \begin{frame}

%     \frametitle{Algorithms}
%     \begin{itemize}
%     \item algorithm and code for assessment
%     \item layout constraint example
%     \item layout regime change example
%     \end{itemize}

% \end{frame}

% \begin{frame}
%   \begin{itemize}
%   \item Where compositions done
%     \begin{itemize}
%     \item tests
%     \item exact path tests
%     \end{itemize}
%   \item algorithm converges
%   \item generic code 
%     \begin{itemize}
%     \item assess
%   \item construct
%     \end{itemize}
%   \item rid external xguess use X0Z0
%   \item formula informative
%   \item good path iteration graphic
%   \end{itemize}
% \end{frame}




% \begin{frame}
%   \frametitle{Algorithms}
  
% \label{alephDef}
% {\small


% % Define $\XZPair{0}\equiv(B x_{t-1}+ \phi \psi_\epsilon\epsilon_{t} +
% %  (I - F)^{-1} \phi \psi_c,0)$.

% Now, given $x^g$, a guess for $x_t$, compute $\XZPairG{k}$ pairs:

% \begin{gather}
%  \XZExp{k}(x^g)= \{\XZPairG{0},\ldots,\XZPairG{k}\}\intertext{ Next, compute}
% % \intertext{ find }
% % \zNow{k+1} \ni \xNow{k+1} \intertext{satisfies the model equations where}
%   \xNow{}=B x_{t-1} + \phi \psi_\epsilon \epsilon_t + 
% (I - F)^{-1} \phi \psi_c +\\  \phi \zNow{k+1}+
% %  \begin{cases}
% %0& \mbox{if }k=0\\    
% \sum_{s=1}^{k} F^s \phi  \mathcal{Z}^{k+1-s}(x^g)%&\mbox{if } k>0
% %  \end{cases}
% \intertext{and}
%   x_{t+1}^{}= B x_{t} + (I - F)^{-1} \phi \psi_c +
% %  \begin{cases}
% %0& \mbox{if }k=0\\    
% \sum_{s=0}^{k-1} F^s \phi  \mathcal{Z}^{k+1-s}(x^g)%&\mbox{if } k>0
% %  \end{cases}
% \intertext{these expressions define a function $\xzFunc$ generating a vector convenient for evaluating the model function equations given an $\tArg$ pair, and a $z_t$}
% \fMap{\FF}
% {\begin{bmatrix}
%   \linMod\\  \XZExp{k}()\\x^g
% \end{bmatrix}}
% {
% \fMap{\xzFunc}
% {
% \begin{bmatrix}
% x_{t-1}\\ \epsilon_t\\z_t
% \end{bmatrix}
% }{
% \begin{bmatrix}
% x_{t-1}\\  x_t\\x^g_{t+1}\\ \epsilon_t
% \end{bmatrix} 
% }
% }
% \end{gather}
% }

% \end{frame}

% \begin{frame}
%   \frametitle{Algorithms}
  

% \begin{pseudocode}{$\aleph$}{\xtGuess,\XZExp{k}()|\linMod}
% \RETURN{\xzFunc\tArgZ}  
% \end{pseudocode}

% \label{genFRFuncDef}



% \begin{pseudocode}{genFRFunc}{\xzFunc()|\modEqns()}
% forFRFunc\tArgZ=\modEqns(\xzFunc\tArgZ)\\
% \frFunc\tArg=FindRoot(forFRFunc\tArgZ)\\
% \RETURN{\frFunc\tArg}  
% \end{pseudocode}
% \begin{gather}
% \left \{
% \xzFuncMap
% ,
% \modEqnsMap
%  \right \}
%  \rightarrow\\
% \frFuncMap
% \end{gather}
% %(* input   [function (xt,eps,zt)->(xtm1,xt,xtp1,eps), function (xtm1,xt,xtp1,eps)->me]*)
% %(* output   [function  (xt,eps) ->(xt,zt)] *)


% \end{frame}


%  \begin{frame}
%    \frametitle{Algorithms}
   
% \label{genFPFuncDef}

%  \begin{pseudocode}{genFPFunc}{\XZExp{k}()|\linMod,\modEqns()}
%  anFRFunc(\xtGuess)=\\genFRFunc(\aleph(\linMod,\XZExp{k}(),\xtGuess))\\
% \xzSolnFunc{k+1}\tArg=
%  FixedPoint(anFRFunc(x_t))\\
%  \RETURN{\xzSolnFunc{k+1}\tArg}
% \end{pseudocode}
% {\small
% \begin{gather}
%   \label{eq:2}
% \fMap{\Pi}
% {\left [\linMod, \XZExp{k}(),x^g,
% \modEqnsMap
% \right ] \right .}{\\ \left .
% \fpFuncMap
% }
% \end{gather}
% }

% \end{frame}
% %(* input   [linMod,XZ, xguess,function (xt,eps,zt)->(xtm1,xt,xtp1,eps), function (xtm1,xt,xtp1,eps)->me]*)
% %(* output   [function  (xt,eps) ->(xt,zt)] *)



% \begin{frame}
%   \frametitle{The Algorithms}

% \label{doIterREDef}

%  \begin{pseudocode}{doIterRE}{\XZExp{k}()|\linMod,\modEqns(),\epsDist()}
%  \xzSolnFunc{k+1}\tArg=genFPFunc(\XZExp{k}(),\linMod,\modEqns())\\
%  \XZExp{k+1}(x_t)=genXZFuncRE(\xzSolnFunc{k+1}(),\XZExp{k}(),\epsDist())\\
%  \RETURN{\xzSolnFunc{k+1}\tArg,\XZExp{k+1}()}  
%  \end{pseudocode}
%  \end{frame}



% \begin{frame}
%   \frametitle{Spanning $\epsilon$}

% Assuming that the $\epsilon_t$ are mean zero:
%   \begin{gather}
%     \label{eq:5}
% 	 \mathcal{X}_{t}(x_{t-1},\epsilon_t) -	
% \overline{\mathcal{X}_{t}(x_{t-1})} =\\
%  \sum_{\sForSum=0}^\infty F^s \phi \left \{z_{t+\sForSum}(x_{t-1},\epsilon_t)- 
% \overline{z_{t+\sForSum}(x_{t-1})} \right \}\intertext{it may be possible to solve of $\epsilon_t^{CE}(x_{t-1})$ so that}
% \mathcal{X}_{t}(x_{t-1},\epsilon_t^{CE}(x_{t-1})) =	\overline{\mathcal{X}_{t}(x_{t-1})}\intertext{and the conditional expectation along the path 
% could be obtained by using these shocks.\footnote{could use this for ergodic distribution moments}}
%   \end{gather}

  
% \end{frame}



%  \begin{frame}
%    \frametitle{To Do}
%    \begin{itemize}
%   \item Like policy iteration. What is different other than series rep
% \item Connection cond exp function  for rest of path and. Perfect foresiggt.  0 errors except first period
% \item model input code, dynare input, cloud site

%   \item may be case that ``chucks'' allow for all conditional expectations path 
% work like perfect foresight paths and satisfy the model equations

% \item put depreciation rate and other maliar features back in?

%    \item Don't interpolate or compute projections across regimes (unless ordered)
%    \item $x(x_{t-1},\epsilon_t,s_t),z(x_{t-1},\epsilon_t,s_t)$ integrate out both
% the $\epsilon_t$ and $s_t$
%        \end{itemize}
%      \end{frame}



%      \begin{frame}
%        \begin{itemize}

%    \item regimes discrete, any other difference(depend on future expectations?)
%    \item characterize as conditional expectation
% \href{https://en.wikipedia.org/wiki/Conditional_expectation}{conditional expectation wikipedia}
%  \item investigate the possiblity solving for $m_\infty,z_\infty$
%    \item flows and convergence properties: why algorithm works, guarantee convergence?
%    \item tighter bounds: flows use local value as new period less
%    \item gridspts can just add its using grid spec or smolyak or there specs add to list
%    \item Hess Sims formula with index
%   \item Only the nonlinear parts of models matter.  Counterintuitive until intuitive.  Two very diff linear models no iteration just get the Z's
%       \end{itemize}
%     \end{frame}

%     \begin{frame}
%       \frametitle{to do}
%       \begin{itemize}
%   \item this is the sense in which arbitrary is robust
%    \item op counts
%    \item demo perfrect foresight
%    \item mod input code?
%    \item closed form, one variable power series
%    \item construct power series?
%    \item factor foldlist xz
%    \item exact path tests
%    \item generic code assess and construct
%    \item split interp func so integ faster
%    \item rid xguess vestige
%    \item formulae informative
%    \item good path iter pic
%    \item test coverage
%    \item fix test anomalies
%    \item NIntegrate 0 error issue
%    \item converge f cont  small criteria involve integral approximation assessment 
%    \end{itemize}
%  \end{frame}

%  \begin{frame}
%    \frametitle{to do}
%    \begin{itemize}
% \item Should use ``worst'' $\epsilon_t$
% \item at each point can compute the value to machine precision since exact function is known
% \item Graph Generated using an interpolation of the norm of the errors for each equation
% \item Using this approximation to compute the infinity norm for the vector of
% errors produces $10^{-19}$.
% \item Can use the calculations in the formula for $x_t$ to put an upper bound
% on how much $x_t$ has to change in order to get agreement in all the equations
% \end{itemize}

%  \end{frame}

 \bibliographystyle{plainnat}
 \bibliography{../../bibFiles/anderson,../../bibFiles/files}


\end{document}
