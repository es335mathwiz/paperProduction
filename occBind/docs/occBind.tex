\documentclass[12pt]{article}
\usepackage[authoryear]{natbib}
\title{A Solution Strategy for Occasionally Binding Constraints in Otherwise
Linear Rational Expectations Models}
\author{Gary S. Anderson}
\usepackage{datetime}
\usepackage{amsmath}
\usepackage{graphicx}
\begin{document}
\maketitle
Consider systems of the form
\newcommand{\xpt}[1]{\mathcal{E}_t \left ( #1 \right ) }

\begin{gather*}
\sum_{i=-\tau}^\theta H_i x_{t+i} =
\begin{bmatrix}
\psi_\epsilon & \psi_z  
\end{bmatrix}
  \begin{bmatrix}
\epsilon_t \\\xpt{z_{t}(x_{t-1},\epsilon_t) }   
  \end{bmatrix}
  \,\,\forall t. \intertext{Solutions will take the form\citep{anderson10}}
  x_t=B x_{t-1}+\sum_{s=0}^\infty F^s \phi \psi
  \begin{bmatrix}
\epsilon_t \\\xpt{z_{t+s}(x_{t+s-1},\epsilon_t)    }
  \end{bmatrix}
% \intertext{ if $z_{t+s}=z^\ast, \forall s$}
%   x_t=B x_{t-1}+(z_t(x_{t-1},\epsilon_t) - \xpt{z^\ast(x_{t-1},\epsilon_{t+s})}(I-F)^{-1} \phi \psi
%   \begin{bmatrix}
% \epsilon_t \\z^\ast(x_{t-1},\epsilon_t)    
%   \end{bmatrix}\intertext{ if $z_{t+s}=z^\ast, \forall s$}
\end{gather*}
To accommodate inequality constraints of the form
\begin{gather*}
  x_t \ge \bar{x}
\end{gather*}
we will determine a sequence of $z^i_t$ that 
will allow us to
 impose the constraints over successively longer time periods.  
We begin by solving a system where the constraints are honored only at time
 $t=0$.  To do this we set $\xpt{z_{t+s}}=0\, \forall s>0$\footnote{Generalize to linear combinations of variables. \begin{gather*}
M x_t \ge m  
\end{gather*} 

Should also consider other behavior besides absorbing barriers.
}
\newcommand{\forPhi}{\begin{bmatrix}
\psi_\epsilon&\psi_z
\end{bmatrix}}
\newcommand{\phiMult}{\phi \psi_\epsilon}
\newcommand{\bMult}{B x_{-1} + \phiMult}
\newcommand{\phiMultBoth}[1]{
\phi (\psi_\epsilon \epsilon_0 +\psi_z z_0^#1(x_{-1},\epsilon_0))}
\newcommand{\bMultBoth}[1]{B x_{-1} + \phiMultBoth{#1}}


\newcommand{\bForOne}{\bMultBoth{1}
}

\newcommand{\bForTwo}{\bMultBoth{2}+
F \phi  \psi_z  
Z_0^1(x_0^2(x_{-1}))   
}



\newcommand{\compSlack}{z_0^1(x_{-1},\epsilon_0) \left ( \bar{r} -\phi_p q_t\right )=0\\ z_0^1(x_{-1},\epsilon_0)> 0}
We now have
\begin{gather*}
x_0^1(x_{-1},\epsilon_0)-
\left ( \bForOne \right )=0\intertext{ with the complimentary slackness condition}
\compSlack
\end{gather*}




We set 
\begin{gather}\label{firstIneq}
z_0^1(x_{-1},\epsilon_0)=
\begin{cases}
0&  B x_{-1} +\epsilon_0 \ge \bar{x}  \\
\bar{x}-(B x_{-1}+\epsilon_0) & B x_{-1}+\epsilon_0 < \bar{x}  
\end{cases}
\end{gather}

For example, consider the simple model


\begin{gather*}
q_{t} +\beta_p(1 - \rho_p)q_{t + 1} + \rho_pq_{t - 1} - \sigma_pr_{t} +
     r_{ut}=0\\
 r_{t} = \max (\bar{r}, \phi_pq_{t}) \\
 r_{ut} = \rho_{ru} r_{ut - 1} + \eta \epsilon_{y,t}
\end{gather*}

Then 

\begin{gather*}
  x_t=
  \begin{bmatrix}
    q_t\\r_{t}\\r_{ut}
  \end{bmatrix}
\end{gather*}

Some typical values of the parameters are:

\begin{gather*}
  \beta_p = 0.99, \phi_p = 1, 
\rho_p = 0.5, \sigma_p = 1, \rho_{ru} = 0.5,
  \bar{r} = 0.02 \\
% q^L = -.5, q^H = .5, 
%   r_u^L = -4 \sigma_u/(1 - \rho_{ru}), r_u^H=  4\sigma_u/(1 - \rho_{ru}),
%    I_N = {10}, \sigma_u = 0.02,\\
%  \mu = {0},
%    \eta = 1
\end{gather*}



\begin{gather*}
  H= \vcenter{\hbox{\includegraphics{/msu/home/m1gsa00/git/ProjectionMethodTools/ProjectionMethodToolsJava/code/prettyHmat.pdf}}}\\
\psi_z=   \vcenter{\hbox{\includegraphics{/msu/home/m1gsa00/git/ProjectionMethodTools/ProjectionMethodToolsJava/code/prettyPsiZ.pdf}}}\\
\psi_\epsilon=   \vcenter{\hbox{\includegraphics{/msu/home/m1gsa00/git/ProjectionMethodTools/ProjectionMethodToolsJava/code/prettyPsiEps.pdf}}}\\
\end{gather*}




We can compute 

 \begin{gather*}
B=   \vcenter{\hbox{\includegraphics{/msu/home/m1gsa00/git/ProjectionMethodTools/ProjectionMethodToolsJava/code/prettyBmat.pdf}}}\\
\phi=   \vcenter{\hbox{\includegraphics{/msu/home/m1gsa00/git/ProjectionMethodTools/ProjectionMethodToolsJava/code/prettyPhimat.pdf}}}
 \end{gather*}



Consequently we have

 \begin{gather*}
z_0^1(x_{-1},\epsilon_0)=   \vcenter{\hbox{\includegraphics[width=10.5cm]{/msu/home/m1gsa00/git/ProjectionMethodTools/ProjectionMethodToolsJava/code/prettyZ1.pdf}}}
 \end{gather*}


 \begin{gather*}
x_0^1(x_{-1},\epsilon_0)=   \vcenter{\hbox{\includegraphics[width=10.5cm]{/msu/home/m1gsa00/git/ProjectionMethodTools/ProjectionMethodToolsJava/code/prettyXVals1.pdf}}}
 \end{gather*}





We can use equations \ref{firstIneq} to compute the expected value function $Z_0^1(x_{-1})=E_{0}[z_0^1(x_{-1},\epsilon_0)]$.

 \begin{gather*}
Z_0^1(x_{-1})=   \vcenter{\hbox{\includegraphics[width=10.5cm]{/msu/home/m1gsa00/git/ProjectionMethodTools/ProjectionMethodToolsJava/code/prettyZ1EXP.pdf}}} \intertext{where}
\mu=\vcenter{\hbox{\includegraphics{/msu/home/m1gsa00/git/ProjectionMethodTools/ProjectionMethodToolsJava/code/prettyMuVal.pdf}}} 
 \end{gather*}



We will use this function  as the value for $z_{1}$ in 
our next calculation.  In so doing we will be able to
capture the impact of imposing the constraint at both time t=0 and t=1.  The next step, in general requires the solution of a nonlinear equation as $x_0$ appears in side the function $Z_0^1(x_0)$ applied at time t=1.\footnote{Ultimately, we will  need to apply a projection method to approximate the function 
that solves the nonlinear expression.  For our simple example model we can compute an analytic expresssion for $Z_0^1$.}



%\includegraphics{/msu/home/m1gsa00/git/ProjectionMethodTools/ProjectionMethodToolsJava/code/Z0EXPGraph.pdf}


%\includegraphics{/msu/home/m1gsa00/git/ProjectionMethodTools/ProjectionMethodToolsJava/code/Z1EXPGraph.pdf}


%\includegraphics{/msu/home/m1gsa00/git/ProjectionMethodTools/ProjectionMethodToolsJava/code/Z2EXPGraph.pdf}





We now have
\begin{gather*}
x_0^2(x_{-1},\epsilon_0)- \left (
\bForTwo \right )=0\intertext{ with the complimentary slackness conditions}
\compSlack
\end{gather*}




\newcommand{\zForTwo}{
\bMult
  \begin{bmatrix}
\epsilon_0 \\z_{t}(x_{t-1},\epsilon_0)    
  \end{bmatrix}+ F \phi   \begin{bmatrix}
0 \\Z_t^1(x_{t})   
  \end{bmatrix}
}

\newcommand{\bForK}{\bMult
  \begin{bmatrix}
\epsilon_0 \\0
  \end{bmatrix}+ \sum_{i=0}^{k-1} F^i \phi  \psi  \begin{bmatrix}
0 \\Z_t^{i-1}(x_{t+i}(x_{t-1}))   
  \end{bmatrix} 
}
\newcommand{\zForK}{
\bMult
  \begin{bmatrix}
\epsilon_t \\z^k_{t}(x_{t-1},\epsilon_t)    
  \end{bmatrix}+ F \phi   \begin{bmatrix}
0 \\Z_t^k(x_{t})   
  \end{bmatrix}
}

We set 
\begin{gather*}
z_t(x_{t-1},\epsilon_t)=
\begin{cases}
0&  \mathcal{M}_2 \ge \bar{x}_1  \\
\bar{x}_1-
\mathcal{M}_2 &\mathcal{M}_2 < \bar{x}_1  
\end{cases}\intertext{where}
\mathcal{M}_2= \left (
\bForTwo
\right )
\end{gather*}

\begin{gather*}
  x_t=
\left (
\zForTwo
\right )
\end{gather*}


For our example,



 \begin{gather*}
F=   \vcenter{\hbox{\includegraphics{/msu/home/m1gsa00/git/ProjectionMethodTools/ProjectionMethodToolsJava/code/prettyFmat.pdf}}}
 \end{gather*}






 \begin{gather*}
z_0^2(x_{-1},\epsilon_0)=   \vcenter{\hbox{\includegraphics[width=10.5cm]{/msu/home/m1gsa00/git/ProjectionMethodTools/ProjectionMethodToolsJava/code/prettyZ2.pdf}}}
 \end{gather*}



%  \begin{gather*}
% Z_0^2(x_{-1})=   \vcenter{\hbox{\includegraphics{/msu/home/m1gsa00/git/ProjectionMethodTools/ProjectionMethodToolsJava/code/prettyZ2EXP.pdf}}} \intertext{where}
% \mu=\vcenter{\hbox{\includegraphics{/msu/home/m1gsa00/git/ProjectionMethodTools/ProjectionMethodToolsJava/code/prettyMuVal.pdf}}} \intertext{where}
%  \end{gather*}



 \begin{gather*}
x_0^2(x_{-1},\epsilon_0)=   \vcenter{\hbox{\includegraphics[width=10.5cm]{/msu/home/m1gsa00/git/ProjectionMethodTools/ProjectionMethodToolsJava/code/prettyXVals2.pdf}}}
 \end{gather*}


The solution is nonlinear, but the nonlinear component dissipates rapidly as the values for $q_{-1+t}$ and $r_{-1+ut}$ deviate from the line 
$1.14549 -15.3348 q_{-1+t} -17.6777 r_{-1+ut} =0 $.






% We now have

% \begin{gather*}
% x_0^2(x_{-1})-
% \left (
% \bForK
% \right )
% \end{gather*}

% We set 
% \begin{gather*}
% z_t(x_{t-1},\epsilon)=
% \begin{cases}
% 0&  \mathcal{M}_k \ge \bar{x}_1  \\
% \bar{x}_1-
% \mathcal{M}_k &\mathcal{M}_k < \bar{x}_1  
% \end{cases}\intertext{where}
% \mathcal{M}_k= \left (
% \bForK
% \right )
% \end{gather*}

% \begin{gather*}
%   x_t^k=
% \left (
% \zForK
% \right )
% \end{gather*}

% Consider the equations 

% \begin{gather*}
% q_{t} +\beta_p(1 - \rho_p)q_{t + 1} + \rho_pq_{t - 1} - \sigma_pr_{t} +
%      r_{ut}=0\\
%  r_{t} = \max (\bar{r}, \phi_pq_{t}) \\
%  r_{ut} = \rho_{ru} r_{ut - 1} + \eta \epsilon_{y,t}
% \end{gather*}



% \begin{gather*}
% q_{t} +\beta_p(1 - \rho_p)q_{t + 1} + \rho_pq_{t - 1} - \sigma_pr_{t} +
%      r_{ut}=0\\
%  r_{t} = \max (\bar{r}, \phi_pq_{t}) \\
%  r_{ut} = \rho_{ru} r_{ut - 1} + \eta \epsilon_{y,t}
% \end{gather*}


% \begin{gather*}
%   B=   \left(
%    \begin{array}{ccc}
%     \frac{2 \rho _p}{\nu +\phi _p \sigma _p+1} & 0 & \frac{2 \rho _u}{\nu +2
%       \beta _p \left(\rho _p-1\right) \rho _u+\phi _p \sigma _p+1} \\
%     \frac{2 \phi _p \rho _p}{\nu +\phi _p \sigma _p+1} & 0 & \frac{2 \phi _p
%       \rho _u}{\nu +2 \beta _p \left(\rho _p-1\right) \rho _u+\phi _p \sigma
%       _p+1} \\
%     0 & 0 & \rho _u
%    \end{array}
%    \right)
% \end{gather*}

% Where
% \begin{gather*}
%   \nu=    \sqrt{4 \beta _p \left(\rho _p-1\right) \rho _p+\left(\phi _p \sigma
%     _p+1\right){}^2}
% \end{gather*}



% \newcommand{\tVec}{
%   \begin{bmatrix}
%     q_t\\r_{rt}\\r_{ut}
%   \end{bmatrix}}



% \begin{gather*}
%   \phi=   \left(
%    \begin{array}{ccc}
%     \frac{2}{\nu +\phi _p \sigma _p+1} & \frac{2}{\nu +2 \beta _p \left(\rho
%       _p-1\right) \rho _u+\phi _p \sigma _p+1} & -\frac{2 \sigma _p}{\nu +\phi
%       _p \sigma _p+1} \\
%     \frac{2 \phi _p}{\nu +\phi _p \sigma _p+1} & \frac{2 \phi _p}{\nu +2 \beta
%       _p \left(\rho _p-1\right) \rho _u+\phi _p \sigma _p+1} & \frac{2 \beta
%       _p \left(\rho _p-1\right) \rho _p+\phi _p \sigma _p \left(-\nu +\phi _p
%       \sigma _p+1\right)}{2 \beta _p \left(\rho _p-1\right) \rho _p} \\
%     0 & 1 & 0
%    \end{array}
%    \right)
% \end{gather*}

% \begin{gather*}
%   F=    \left(
%    \begin{array}{ccc}
%     \frac{\phi _p \sigma _p-\sqrt{\left(\phi _p \sigma _p+1\right){}^2+4 \beta
%       _p \left(\rho _p-1\right) \rho _p}+1}{2 \rho _p} & 0 & 0 \\
%     \frac{\phi _p \left(\phi _p \sigma _p-\sqrt{\left(\phi _p \sigma
%       _p+1\right){}^2+4 \beta _p \left(\rho _p-1\right) \rho _p}+1\right)}{2
%       \rho _p} & 0 & 0 \\
%     0 & 0 & 0
%    \end{array}
%    \right)
% \end{gather*}

% Applying the formula produces
% \begin{gather*}
% \tVec=      \left(
%    \begin{array}{c}
%     2 \left(\frac{q_{t-1} \rho _p}{\nu +\phi _p \sigma _p+1}+\frac{\epsilon
%       _t+r_{\text{ut}-1} \rho _u}{\nu +2 \beta _p \left(\rho _p-1\right) \rho
%       _u+\phi _p \sigma _p+1}\right)+\frac{\sigma _p
%       \left(\text{Exptn}\left(z_{t+1}\right) \left(\nu -\phi _p \sigma
%       _p-1\right)-2 \rho _p z_t\right)}{\rho _p \left(\nu +\phi _p \sigma
%       _p+1\right)} \\
%     2 \phi _p \left(\frac{q_{t-1} \rho _p}{\nu +\phi _p \sigma
%       _p+1}+\frac{\epsilon _t+r_{\text{ut}-1} \rho _u}{\nu +2 \beta _p
%       \left(\rho _p-1\right) \rho _u+\phi _p \sigma
%       _p+1}\right)+\frac{\text{Exptn}\left(z_{t+1}\right) \phi _p \sigma _p
%       \left(\nu -\phi _p \sigma _p-1\right)+\rho _p \left(\nu -\phi _p \sigma
%       _p+1\right) z_t}{\rho _p \left(\nu +\phi _p \sigma _p+1\right)} \\
%     \epsilon _t+r_{\text{ut}-1} \rho _u
%    \end{array}
%    \right)
% \end{gather*}

% For the numerical values above
% \begin{gather*}
% \tVec=    \left(
%    \begin{array}{c}
%     -0.141938 \text{Exptn}\left(z_{t+1}\right)+0.617296 \epsilon _t+0.267742
%       q_{t-1}+0.308648 r_{\text{ut}-1}-0.535485 z_t \\
%     -0.141938 \text{Exptn}\left(z_{t+1}\right)+0.617296 \epsilon _t+0.267742
%       q_{t-1}+0.308648 r_{\text{ut}-1}+0.464515 z_t \\
%     \epsilon _t+0.5 r_{\text{ut}-1}
%    \end{array}
%    \right)
% \end{gather*}

% Imposing constraints generates a nonlinear set of equations to solve.
% The solution, including the expectation term incorporates the impact of the
% constraint on this and the next period.

% Alternatively, for imposing the constraints throughout time one gets

% \begin{gather*}
% \tVec=    \left(
%    \begin{array}{c}
%     -0.19313 \text{Exptn}\left(z_{t+1}\right)+0.617296 \epsilon _t+0.267742
%       q_{t-1}+0.308648 r_{\text{ut}-1}-0.535485 z_t \\
%     -0.19313 \text{Exptn}\left(z_{t+1}\right)+0.617296 \epsilon _t+0.267742
%       q_{t-1}+0.308648 r_{\text{ut}-1}+0.464515 z_t \\
%     \epsilon _t+0.5 r_{\text{ut}-1}
%    \end{array}
%    \right)
% \end{gather*}



\begin{itemize}
\item timing
\item transition to projection
\item fixed point criterion
\end{itemize}
\section{Projection Method Occasionally Binding Constraints Implementation}
\label{sec:proj-meth-occass}
\begin{itemize}
\item Mathematica code generates java code that is called to interact with the java program implementing the projection method.
\item describe eqvaldrv
\item how they generate equation
\item describe if statement
\item how if statement generates equations
\item shift evaluation region to include curved part of solution
\end{itemize}


\appendix
\section{ProjectionMethodToolsJava}
\label{sec:proj}

\bibliographystyle{plainnat}
\bibliography{anderson}
\end{document}

